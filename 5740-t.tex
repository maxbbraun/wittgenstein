% %%%%%%%%%%%%%%%%%%%%%%%%%%%%%%%%%%%%%%%%%%%%%%%%%%%%%%%%%%%%%%%%%%%%%%% %
%                                                                         %
% Project Gutenberg's Tractatus Logico-Philosophicus, by Ludwig Wittgenstein
%                                                                         %
% This eBook is for the use of anyone anywhere at no cost and with        %
% almost no restrictions whatsoever.  You may copy it, give it away or    %
% re-use it under the terms of the Project Gutenberg License included     %
% with this eBook or online at www.gutenberg.org                          %
%                                                                         %
%                                                                         %
% Title: Tractatus Logico-Philosophicus                                   %
%                                                                         %
% Author: Ludwig Wittgenstein                                             %
%                                                                         %
% Contributor: Bertrand Russell                                           %
%                                                                         %
% Translator: C. K. Ogden                                                 %
%                                                                         %
% Release Date: October 22, 2010 [EBook #5740]                            %
%                                                                         %
% Language: German                                                        %
%                                                                         %
% Character set encoding: ISO-8859-1                                      %
%                                                                         %
% *** START OF THIS PROJECT GUTENBERG EBOOK TRACTATUS LOGICO-PHILOSOPHICUS ***
%                                                                         %
% %%%%%%%%%%%%%%%%%%%%%%%%%%%%%%%%%%%%%%%%%%%%%%%%%%%%%%%%%%%%%%%%%%%%%%% %

\def\ebook{5740}
%%%%%%%%%%%%%%%%%%%%%%%%%%%%%%%%%%%%%%%%%%%%%%%%%%%%%%%%%%%%%%%%%%%%%%
%%                                                                  %%
%% Packages and substitutions:                                      %%
%%                                                                  %%
%% book:     Required.                                              %%
%% inputenc: Standard DP input encoding. Required.                  %%
%% fontenc:  T1 Font encoding. Required.                            %%
%% babel:    German language hyphenation. Required.                 %%
%%                                                                  %%
%% ifthen:   Logical conditionals. Required.                        %%
%%                                                                  %%
%% amsmath:  AMS mathematics enhancements. Required.                %%
%% amssymb:  Additional mathematical symbols. Required.             %%
%%                                                                  %%
%% footmisc: Extended footnote capabilities. Required.              %%
%%                                                                  %%
%% fancyhdr: Enhanced running headers and footers. Required.        %%
%%                                                                  %%
%% graphicx: Graphics inclusion. Required.                          %%
%%                                                                  %%
%% alltt:    Fixed-width font environment. Required.                %%
%%                                                                  %%
%% enumitem: Customise the appearance of lists. Required.           %%
%%                                                                  %%
%% soul:     Gesperrt text. Optional.                               %%
%%                                                                  %%
%% geometry: Enhanced page layout package. Required.                %%
%% hyperref: Hypertext embellishments for pdf output. Required.     %%
%%                                                                  %%
%% Producer's comments:                                             %%
%%   14 illustrations from the original are provided as PDF files.  %%
%%   EPS files for use with LaTeX are in images/sources.            %%
%%                                                                  %%
%% PDF pages: 173                                                   %%
%% PDF page size: A4                                                %%
%% PDF bookmarks: created for main divisions (no ToC, no chapters)  %%
%% PDF document info: filled in                                     %%
%% Images: 14 PDF diagrams/illustrations                            %%
%%                                                                  %%
%% Compile History:                                                 %%
%%                                                                  %%
%% 2010-October-08 Frau Sma Compiled with pdflatex:                 %%
%%         [pdfTeX, Version 3.1415926-1.40.10 (TeX Live 2009)]      %%
%%                                                                  %%
%% pdflatex x2                                                      %%
%%                                                                  %%
%%                                                                  %%
%% October 2010: pglatex.                                           %%
%%   Compile this project with:                                     %%
%%   pdflatex 5740-t.tex ..... TWO times                            %%
%%                                                                  %%
%%   pdfTeXk, Version 3.141592-1.40.3 (Web2C 7.5.6)                 %%
%%                                                                  %%
%%%%%%%%%%%%%%%%%%%%%%%%%%%%%%%%%%%%%%%%%%%%%%%%%%%%%%%%%%%%%%%%%%%%%%

\listfiles
\documentclass[12pt,oneside]{book}[2007/10/19]

\usepackage[latin1]{inputenc}[2008/03/30]
\usepackage[T1]{fontenc}[2005/09/27]
\usepackage[german,english]{babel}[2008/07/06]

\usepackage{ifthen}[2001/05/26]

\usepackage{amsmath}[2000/07/18]
\usepackage{amssymb}[2009/06/22]

\usepackage[perpage,symbol*]{footmisc}[2009/09/15]

\usepackage{fancyhdr}

\usepackage{graphicx}[1999/02/16]

\usepackage{alltt}[1997/06/16]

\usepackage{enumitem}[2009/05/18]

\IfFileExists{soul.sty}{%
	\usepackage{soul}[2003/11/17]
}{}

% A4 paper with decent margins for printing
\newcommand{\Margins}{hmarginratio=1:1} % Asymmetric margins
\newcommand{\HLinkColor}{black}         % Hyperlink color
\newcommand{\PDFPageLayout}{TwoPageRight}
\setlength{\paperwidth}{21cm}
\setlength{\paperheight}{29.7cm}
\usepackage[body={13cm,20cm},asymmetric,bindingoffset=1cm,\Margins]{geometry}[2002/07/08]

% no head rule (below page header)
\setlength{\headheight}{15pt}
\renewcommand{\headrulewidth}{0pt}

\providecommand{\ebook}{00000}    % Overridden during white-washing

\usepackage[pdftex,
  hyperref,
  hyperfootnotes=false,
  pdfauthor={Ludwig Wittgenstein},
  pdftitle={The Project Gutenberg eBook \#\ebook: Tractatus Logico-Philosophicus},
  pdfkeywords={Jana Srna, Norbert H. Langkau and the
               Project Gutenberg Online Distributed Proofreading Team},
  pdfstartview=Fit,    % default value
  pdfstartpage=1,      % default value
  pdfpagemode=UseNone, % default value
  bookmarks=true,      % default value
  linktocpage=true,
  pdfpagelayout=\PDFPageLayout,
  pdfdisplaydoctitle,
  pdfpagelabels=true,
  bookmarksopen=true,
  bookmarksopenlevel=0,
  colorlinks=true,
  linkcolor=\HLinkColor]{hyperref}[2009/10/09]

%% fixed-width environment to format PG boilerplate
\newenvironment{PGtext}{%
  \begin{alltt}
  \fontsize{9}{10}\ttfamily\selectfont}%
  {\end{alltt}%
}

% make sure the next page starts on a new recto page with the
% specified page style, and if there's a blank page, its page
% style should be empty
\newcommand{\SkipToNewPage}[1]{\newpage\pagestyle{empty}
  \cleardoublepage\pagestyle{#1}
}

%% Sectioning and page layout

\newcommand{\Title}[1]{\vspace*{4.5ex}{\LARGE\bfseries #1}\vspace{9ex}}

% transcriber's note
\newcommand{\TransNoteText}{%
The original publication was a parallel translation; after the introduction,
even pages contained the German original, odd pages the English translation.
This e-book has been reformatted to contain the English translation first and
the German original after that. In the PDF file, the proposition numbers are
linked back and forth between the languages.

The original used a lower-case `v' for the \emph{logical or} operator; it has been
replaced with the correct `$\lor$' character.

In the German part of the original, variables were printed upright; they have
been italicised in this e-book.

Every effort has been made to replicate the original text as faithfully as
possible. Minor typesetting errors have been corrected; all changes are detailed
in the \LaTeX{} source code.
}

% publisher's note
\newcommand{\Note}{\newpage\null\vfill
  \begin{center}
  {\large NOTE}
  \end{center}
}

% introduction
\newcommand{\Introduction}{\cleardoublepage
  \phantomsection
  \pdfbookmark[-1]{Main Matter}{Main Matter}

  \phantomsection
  \pdfbookmark[0]{Introduction}{Introduction}

  \pagestyle{fancy}
  \fancyhf{}
  \chead{INTRODUCTION}
  \cfoot{\thepage}
  \thispagestyle{plain}

  \begin{center}
  {\Large INTRODUCTION}

  \vspace{1ex}
  \textsc{By BERTRAND RUSSELL}
  \vspace{3.5ex}
  \end{center}
}

% preface(s)
\newcommand{\Preface}[2]{%
  \SkipToNewPage{plain}
  \phantomsection
  \pdfbookmark[0]{#2}{#2}
  \begin{center}
  \Title{#1}

  {\large \MakeUppercase{#2}}
  \end{center}
}

% start of the actual text
\newcommand{\MainMatter}[1]{%
  \SkipToNewPage{fancy}
  \phantomsection
  \pdfbookmark[0]{#1}{#1}
  \vspace*{0.15\textheight}
  \chead{\MakeUppercase{#1}}
  \thispagestyle{plain}
}

% PG Boilerplate
\newcommand{\Boilerplate}{%
  \phantomsection
  \pdfbookmark[0]{PG Boilerplate}{PG Boilerplate}
}

% PG Licence
\newcommand{\Licence}{%
  \cleardoublepage
  \pagenumbering{alph}
  \phantomsection
  \pdfbookmark[-1]{Back Matter}{Back Matter}
  \phantomsection
  \pdfbookmark[0]{PG Licence}{PG Licence}
}

% an English proposition
\newcommand{\PropositionE}[2]{%
  \item[\phantomsection\label{PropE:#1}\PropGRef{#1}] #2%
}
% reference to an English proposition
\newcommand{\PropERef}[1]{\hyperref[PropE:#1]{#1}}
% a German proposition
\newcommand{\PropositionG}[2]{%
  \item[\phantomsection\label{PropG:#1}\PropERef{#1}] #2%
}
% reference to a German proposition
\newcommand{\PropGRef}[1]{\hyperref[PropG:#1]{#1}}

% convenience macro for including an image
\newcommand{\Illustration}[2][0.3\textwidth]{%
  \begin{center}
  \includegraphics*[width=#1]{images/#2.pdf}
  \end{center}
}

% a typesetting error
\newcommand{\DPtypo}[2]{#2}

% several mathematical operators specific to this book
\newcommand{\Not}[1]{\mathord{\thicksim} #1}
\newcommand{\DotOp}{\mathbin{.}}
\newcommand{\BarOp}{\mathbin{|}}
\newcommand{\Implies}{\supset}

% a citation in the main text
\newcommand{\BookTitle}[1]{\emph{#1}}

% a German word in the English introduction
\newcommand{\German}[1]{\foreignlanguage{german}{\emph{#1}}}

% if we have the soul package, emphasis in the German part should be gesperrt
\IfFileExists{soul.sty}{%
  \sodef\Emph{}{0.15em}{0.6em plus0.4em}{0.4em plus0.4em minus0.1em}%
  \newcommand{\EmphPart}[1]{\,\Emph{##1}}% emphasised partial word
}{% otherwise, emphasise using italics
  \newcommand{\Emph}[1]{\emph{#1}}%
  \newcommand{\EmphPart}[1]{\emph{#1}}%
}

% abbreviations for English part
\newcommand{\idEst}{\textit{i.e.}}
\newcommand{\IdEst}{\textit{I.e.}}
\newcommand{\exempliGratia}{\textit{e.g.}}
\newcommand{\ExempliGratia}{\textit{E.g.}}

% abbreviations for German part
\newcommand{\dasHeiszt}{d.\;h.}
\newcommand{\zumBeispiel}{z.\;B.}
\newcommand{\ZumBeispiel}{Z.\;B.}
\newcommand{\undAndere}{u.\;a.}
\newcommand{\UndSoWeiter}{U.\;s.\;w.}
\newcommand{\undSoFort}{u.\;s.\;f.}

% ditto commands for tables
\newcommand{\ditto}{\quotedblbase}

\newlength{\DittoLen}

\newcommand{\DittoInWords}{%
 \settowidth{\DittoLen}{in}%
 \makebox[\DittoLen]{\ditto}~%
 \settowidth{\DittoLen}{words:}%
 \makebox[\DittoLen]{\ditto}}

\newcommand{\DittoInWorten}{%
 \settowidth{\DittoLen}{in}%
 \makebox[\DittoLen]{\ditto}~%
 \settowidth{\DittoLen}{Worten:}%
 \makebox[\DittoLen]{\ditto}}

% F and W with spacing for better alignment in large German table
% (which uses W/F for T/F, W being wider than F)
\newlength{\WLen}     % define a length variable ...
\settowidth{\WLen}{W} % ... and set it to the width of a W
\newcommand{\False}{\makebox[\WLen]{F}} % F centered in space of the width of a W ...
\newcommand{\Wahr}{\makebox[\WLen]{W}}  % ... and W to go with it

% hack for making the table row below an \hline look less crowded
\newcommand{\Strut}[1][12pt]{\rule{0pt}{#1}}

% four- and five-dot ellipses
\newcommand{\fourdots}{\mathinner{\ldotp\ldotp\ldotp\ldotp}}
\newcommand{\fivedots}{\mathinner{\ldotp\ldotp\ldotp\ldotp\ldotp}}

% spacing fixes for paragraphs where LaTeX doesn't get the hyphenation right
\newcommand{\stretchyspace}{\spaceskip0.4em plus 0.2em minus 0.1em}
\newcommand{\verystretchyspace}{\spaceskip0.5em plus 0.5em minus 0.25em}

% allow a line break (e.g. before a dash)
\newcommand{\AllowBreak}{\discretionary{}{}{}}

% fix hyphenation for some words where LaTeX gets it wrong by default
\hyphenation{ap-pli-ca-tion atom-ic be-tween de-ter-mined ev-ery-thing ex-is-tence hap-pened iden-ti-cal in-de-pen-dent in-de-pen-dent-ly le-git-i-mate-ly neg-a-tive or-dered phys-ics prop-o-si-tion prop-o-si-tio-nal prop-o-si-tions re-sult-ing Soc-ra-tes wheth-er Witt-gen-stein}

%%%%%%%%%%%%%%%%%%%%%%%%%%%% BEGIN DOCUMENT %%%%%%%%%%%%%%%%%%%%%%%%%%
\begin{document}
\selectlanguage{english}

% define new list type for propositions
\newlist{propositions}{enumerate}{1}
\setlist[propositions,1]{label=5.47321, leftmargin=*, align=left, itemsep=4pt, parsep=2pt, listparindent=\parindent}

\pagestyle{empty}
\pagenumbering{roman}

\phantomsection
\pdfbookmark[-1]{Front Matter}{Front Matter}

%%%% PG BOILERPLATE %%%%
\Boilerplate

\begin{center}
\begin{minipage}{\textwidth}
\small
\begin{PGtext}
Project Gutenberg's Tractatus Logico-Philosophicus, by Ludwig Wittgenstein

This eBook is for the use of anyone anywhere at no cost and with
almost no restrictions whatsoever.  You may copy it, give it away or
re-use it under the terms of the Project Gutenberg License included
with this eBook or online at www.gutenberg.org


Title: Tractatus Logico-Philosophicus

Author: Ludwig Wittgenstein

Contributor: Bertrand Russell

Translator: C. K. Ogden

Release Date: October 22, 2010 [EBook #5740]

Language: German

Character set encoding: ISO-8859-1

*** START OF THIS PROJECT GUTENBERG EBOOK TRACTATUS LOGICO-PHILOSOPHICUS ***
\end{PGtext}
\end{minipage}
\end{center}

%%%% Credits %%%%
\clearpage
\begin{center}
\begin{minipage}{\textwidth}
\begin{PGtext}
Produced by Jana Srna, Norbert H. Langkau, and the Online
Distributed Proofreading Team at http://www.pgdp.net
\end{PGtext}
\end{minipage}
\end{center}
\vfill

%%%% Transcriber's Note %%%%
{\small
\phantomsection
\pdfbookmark[0]{Transcriber's Note}{Transcriber's Note}

\begin{center}
\MakeUppercase{Transcriber's Note}
\end{center}

\TransNoteText
}

%%%% FRONT MATTER %%%%
% -----File: 001.png---
\mainmatter
\begin{center}
{\LARGE International Library of Psychology\\
\vspace{0.5ex}
Philosophy and Scientific Method}

\vspace{\baselineskip}
\begin{tabular}{lc}
\textsc{General Editor:} & \hspace{2em}\textsc{C.~K. Ogden, m.a.}\\
& \hspace{2em}(\textit{Magdalene College, Cambridge})\\
\end{tabular}
% -----File: 002.png---
% -----File: 003.png---


\cleardoublepage
{\Huge\bfseries Tractatus\\
\vspace{1.2ex}
Logico-Philosophicus}

\vspace{15ex}

By\\
\vspace{0.7ex}
\textbf{LUDWIG WITTGENSTEIN}

\vspace{2\baselineskip}

With an Introduction by\\
\vspace{0.7ex}
\textbf{BERTRAND RUSSELL, F.R.S.}

\vfill

LONDON\\
\vspace{0.5ex}
{\large KEGAN PAUL, TRENCH, TRUBNER \& CO., LTD.\\}
\vspace{0.5ex}
NEW YORK: HARCOURT, BRACE \& COMPANY, INC.\\
\vspace{1ex}
1922
% -----File: 004.png---

\newpage\null\vfill
{\footnotesize PRINTED IN GREAT BRITAIN BY THE EDINBURGH PRESS,\\
9 AND 11 YOUNG STREET, EDINBURGH.}
\end{center}
% -----File: 005.png---




\Note

\textit{In rendering Mr Wittgenstein's \BookTitle{Tractatus Logico-Philosophicus}
available for English readers, the somewhat unusual course has been
adopted of printing the original side by side with the translation.
Such a method of presentation seemed desirable both on account of the
obvious difficulties raised by the vocabulary and in view of the
peculiar literary character of the whole. As a result, a certain
latitude has been possible in passages to which objection might
otherwise be taken as over-literal.}

\textit{The proofs of the translation and the version of the original
which appeared in the final number of Ostwald's \BookTitle{Annalen der
Naturphilosophie (1921)} have been very carefully revised by
the author himself; and the Editor further desires to express his
indebtedness to Mr F.~P. Ramsey, of Trinity College, Cambridge,
for assistance both with the translation and in the preparation of
the book for the press.}

\begin{flushright}
\textit{C.~K.~O.}
\end{flushright}
\vfill\vfill
% -----File: 006.png---
% -----File: 007.png---




%%%% MAIN MATTER %%%%
\Introduction

\textsc{Mr Wittgenstein's} \BookTitle{Tractatus Logico-Philosophicus}, whether
or not it prove to give the ultimate truth on the matters
with which it deals, certainly deserves, by its breadth and
scope and profundity, to be considered an important event
in the philosophical world. Starting from the principles
of Symbolism and the relations which are necessary
between words and things in any language, it applies
the result of this inquiry to various departments of traditional
philosophy, showing in each case how traditional
philosophy and traditional solutions arise out of ignorance
of the principles of Symbolism and out of misuse of
language.

The logical structure of propositions and the nature
of logical inference are first dealt with. Thence we pass
successively to Theory of Knowledge, Principles of Physics,
Ethics, and finally the Mystical (\German{das Mystische}).

In order to understand Mr Wittgenstein's book, it is
necessary to realize what is the problem with which he is
concerned. In the part of his theory which deals with
Symbolism he is concerned with the conditions which
would have to be fulfilled by a logically perfect language.
There are various problems as regards language. First,
there is the problem what actually occurs in our minds
when we use language with the intention of meaning
something by it; this problem belongs to psychology.
Secondly, there is the problem as to what is the relation
subsisting between thoughts, words, or sentences, and that
which they refer to or mean; this problem belongs to
epistemology. Thirdly, there is the problem of using
sentences so as to convey truth rather than falsehood;
% -----File: 008.png---
this belongs to the special sciences dealing with the
subject-matter of the sentences in question. Fourthly,
there is the question: what relation must one fact (such
as a sentence) have to another in order to be \emph{capable}
of being a symbol for that other? This last is a logical
question, and is the one with which Mr Wittgenstein is
concerned. He is concerned with the conditions for \emph{accurate}
Symbolism, \idEst\ for Symbolism in which a sentence
``means'' something quite definite. In practice, language
is always more or less vague, so that what we assert is
never quite precise. Thus, logic has two problems to deal
with in regard to Symbolism: (1)~the conditions for sense
rather than nonsense in combinations of symbols; (2)~the
conditions for uniqueness of meaning or reference in
symbols or combinations of symbols. A logically perfect
language has rules of syntax which prevent nonsense, and
has single symbols which always have a definite and
unique meaning. Mr Wittgenstein is concerned with the
conditions for a logically perfect language---not that any
language is logically perfect, or that we believe ourselves
capable, here and now, of constructing a logically perfect
language, but that the whole function of language is to
have meaning, and it only fulfils this function in proportion
as it approaches to the ideal language which we
postulate.

The essential business of language is to assert or
deny facts. Given the syntax of a language, the meaning
of a sentence is determinate as soon as the meaning of
the component words is known. In order that a certain
sentence should assert a certain fact there must, however
the language may be constructed, be something in common
between the structure of the sentence and the structure of
the fact. This is perhaps the most fundamental thesis
of Mr Wittgenstein's theory. That which has to be in
common between the sentence and the fact cannot, so
he contends, be itself in turn \emph{said} in language. It can,
in his phraseology, only be \emph{shown}, not said, for whatever
we may say will still need to have the same structure.
% -----File: 009.png---

The first requisite of an ideal language would be that
there should be one name for every simple, and never the
same name for two different simples. A name is a simple
symbol in the sense that it has no parts which are themselves
symbols. In a logically perfect language nothing
that is not simple will have a simple symbol. The symbol
for the whole will be a ``complex,'' containing the symbols
for the parts. In speaking of a ``complex'' we are, as
will appear later, sinning against the rules of philosophical
grammar, but this is unavoidable at the outset. ``Most
propositions and questions that have been written about
philosophical matters are not false but senseless. We
cannot, therefore, answer questions of this kind at all,
but only state their senselessness. Most questions and
propositions of the \DPtypo{philosopher}{philosophers} result from the fact that
we do not understand the logic of our language. They
are of the same kind as the question whether the Good is
more or less identical than the Beautiful'' (\PropERef{4.003}). What
is complex in the world is a fact. Facts which are not
compounded of other facts are what Mr Wittgenstein calls
\German{Sachverhalte}, whereas a fact which may consist of two
or more facts is called a \German{Tatsache}: thus, for example,
``Socrates is wise'' is a \German{Sachverhalt}, as well as a \German{Tatsache},
whereas ``Socrates is wise and Plato is his pupil'' is a
\German{Tatsache} but not a \German{Sachverhalt}.

He compares linguistic expression to projection in
geometry. A geometrical figure may be projected in
many ways: each of these ways corresponds to a different
language, but the projective properties of the original
figure remain unchanged whichever of these ways may
be adopted. These projective properties correspond to
that which in his theory the proposition and the fact
must have in common, if the proposition is to assert the
fact.

In certain elementary ways this is, of course, obvious.
It is impossible, for example, to make a statement about
two men (assuming for the moment that the men may
be treated as simples), without employing two names, and
% -----File: 010.png---
if you are going to assert a relation between the two men
it will be necessary that the sentence in which you make
the assertion shall establish a relation between the two
names. If we say ``Plato loves Socrates,'' the word
``loves'' which occurs between the word ``Plato'' and the
word ``Socrates'' establishes a certain relation between
these two words, and it is owing to this fact that our
sentence is able to assert a relation between the person's
name by the words ``Plato'' and ``Socrates.'' ``We must
not say, the complex sign `$a R b$' says `$a$ stands in a
certain relation $R$ to $b$'; but we must say, that `$a$'
stands in a certain relation to `$b$' says \emph{that $a R b$}''
(\PropERef{3.1432}).

Mr Wittgenstein begins his theory of Symbolism with
the statement (\PropERef{2.1}): ``We make to ourselves pictures of
facts.'' A picture, he says, is a model of the reality, and
to the objects in the reality correspond the elements of
the picture: the picture itself is a fact. The fact that
things have a certain relation to each other is represented
by the fact that in the picture its elements have a certain
relation to one another. ``In the picture and the pictured
there must be something identical in order that the one
can be a picture of the other at all. What the picture
must have in common with reality in order to be able
to represent it after its manner---rightly or falsely---is its
form of representation'' (\PropERef{2.161}, \PropERef{2.17}).

We speak of a logical picture of a reality when we
wish to imply only so much resemblance as is essential to
its being a picture in any sense, that is to say, when we
wish to imply no more than identity of logical form.
The logical picture of a fact, he says, is a \German{Gedanke}. A
picture can correspond or not correspond with the fact and
be accordingly true or false, but in both cases it shares the
logical form with the fact. The sense in which he speaks of
pictures is illustrated by his statement: ``The gramophone
record, the musical thought, the score, the waves of sound,
all stand to one another in that pictorial internal relation
which holds between language and the world. To all of
% -----File: 011.png---
them the logical structure is common. (Like the two
youths, their two horses and their lilies in the story.
They are all in a certain sense one)'' (\PropERef{4.014}). The
possibility of a proposition representing a fact rests upon
the fact that in it objects are represented by signs. The
so-called logical ``constants'' are not represented by signs,
but are themselves present in the proposition as in the
fact. The proposition and the fact must exhibit the same
logical ``manifold,'' and this cannot be itself represented
since it has to be in common between the fact and
the picture. Mr Wittgenstein maintains that everything
properly philosophical belongs to what can only be shown,
to what is in common between a fact and its logical
picture. It results from this view that nothing correct can
be said in philosophy. Every philosophical proposition
is bad grammar, and the best that we can hope to achieve
by philosophical discussion is to lead people to see that
philosophical discussion is a mistake. ``Philosophy is
not one of the natural sciences. (The word `philosophy'
must mean something which stands above or below, but
not beside the natural sciences.) The object of philosophy
is the logical clarification of thoughts. Philosophy is not
a theory but an activity. A philosophical work consists
essentially of elucidations. The result of philosophy is
not a number of `philosophical propositions,' but to
make propositions clear. Philosophy should make clear
and delimit sharply the thoughts which otherwise are,
as it were, opaque and blurred'' (\PropERef{4.111} and \PropERef{4.112}). In
accordance with this principle the things that have to be
said in leading the reader to understand Mr Wittgenstein's
theory are all of them things which that theory itself
condemns as meaningless. With this proviso we will
endeavour to convey the picture of the world which
seems to underlie his system.

The world consists of facts: facts cannot strictly
speaking be defined, but we can explain what we mean
by saying that facts are what make propositions true, or
false. Facts may contain parts which are facts or may
% -----File: 012.png---
contain no such parts; for example: ``Socrates was a wise
Athenian,'' consists of the two facts, ``Socrates was wise,''
and ``Socrates was an Athenian.'' A fact which has no
parts that are facts is called by Mr Wittgenstein a \German{Sachverhalt}.
This is the same thing that he calls an atomic fact.
An atomic fact, although it contains no parts that are
facts, nevertheless does contain parts. If we may regard
``Socrates is wise'' as an atomic fact we perceive that it
contains the constituents ``Socrates'' and ``wise.'' If an
atomic fact is analysed as fully as \DPtypo{possibly}{possible} (theoretical,
not practical possibility is meant) the constituents finally
reached may be called ``simples'' or ``objects.'' It is not
contended by Wittgenstein that we can actually isolate
the simple or have empirical knowledge of it. It is a
logical necessity demanded by theory, like an electron.
His ground for maintaining that there must be simples
is that every complex presupposes a fact. It is not
necessarily assumed that the complexity of facts is finite;
even if every fact consisted of an infinite number of atomic
facts and if every atomic fact consisted of an infinite
number of objects there would still be objects and atomic
facts (\PropERef{4.2211}). The assertion that there is a certain
complex reduces to the assertion that its constituents
are related in a certain way, which is the assertion of
a \emph{fact}: thus if we give a name to the complex the name
only has meaning in virtue of the truth of a certain
proposition, namely the proposition asserting the relatedness
of the constituents of the complex. Thus the naming
of complexes presupposes propositions, while propositions
presupposes the naming of simples. In this way the
naming of simples is shown to be what is logically first
in logic.

The world is fully described if all atomic facts are
known, together with the fact that these are all of them.
The world is not described by merely naming all the
objects in it; it is necessary also to know the atomic facts
of which these objects are constituents. Given this total
of atomic facts, every true proposition, however complex,
% -----File: 013.png---
can theoretically be inferred. A proposition (true or
false) asserting an atomic fact is called an atomic proposition.
All atomic propositions are logically independent
of each other. No atomic proposition implies any other
or is inconsistent with any other. Thus the whole business
of logical inference is concerned with propositions which
are not atomic. Such propositions may be called
molecular.

Wittgenstein's theory of molecular propositions turns
upon his theory of the construction of truth-functions.

A truth-function of a proposition $p$ is a proposition
containing $p$ and such that its truth or falsehood depends
only upon the truth or falsehood of $p$, and similarly a
truth-function of several propositions $p$, $q$, $r$\ldots is one
containing $p$, $q$, $r$\ldots and such that its truth or falsehood
depends only upon the truth or falsehood of
$p$, $q$, $r$\ldots\ It might seem at first sight as though there
were other functions of propositions besides truth-functions;
such, for example, would be ``A believes $p$,'' for in general
A will believe some true propositions and some false
ones: unless he is an exceptionally gifted individual, we
cannot infer that $p$ is true from the fact that he believes
it or that $p$ is false from the fact that he does not believe
it. Other apparent exceptions would be such as ``$p$ is a
very complex proposition'' or ``$p$ is a proposition about
Socrates.'' Mr Wittgenstein maintains, however, for
reasons which will appear presently, that such exceptions
are only apparent, and that every function of a proposition
is really a truth-function. It follows that if we can
define truth-functions generally, we can obtain a general
definition of all propositions in terms of the original set
of atomic propositions. This Wittgenstein proceeds to
do.

It has been shown by Dr Sheffer (\BookTitle{Trans.\ Am.\ Math.\ Soc.},
Vol.~XIV. pp.~481--488) that all truth-functions of a given
set of propositions can be constructed out of either of
the two functions ``not-$p$ or not-$q$'' or ``not-$p$ and not-$q$.''
Wittgenstein makes use of the latter, assuming a knowledge
% -----File: 014.png---
of Dr Sheffer's work. The manner in which other
truth-functions are constructed out of ``not-$p$ and not-$q$''
is easy to see. ``Not-$p$ and not-$p$'' is equivalent to
``not-$p$,'' hence we obtain a definition of negation in terms
of our primitive function: hence we can define ``$p$ or $q$,''
since this is the negation of ``not-$p$ and not-$q$,'' \idEst\ of
our primitive function. The development of other truth-functions
out of ``not-$p$'' and ``$p$ or $q$'' is given in detail
at the beginning of \BookTitle{Principia Mathematica}. This gives all
that is wanted when the propositions which are arguments
to our truth-function are given by enumeration. Wittgenstein,
however, by a very interesting analysis succeeds in
extending the process to general propositions, \idEst\ to cases
where the propositions which are arguments to our truth-function
are not given by enumeration but are given as
all those satisfying some condition. For example, let $fx$
be a propositional function (\idEst\ a function whose values
are propositions), such as ``$x$ is human''---then the various
values of $fx$ form a set of propositions. We may extend
the idea ``not-$p$ and not-$q$'' so as to apply to simultaneous
denial of all the propositions which are values of $fx$. In
this way we arrive at the proposition which is ordinarily
represented in mathematical logic by the words ``$fx$
is false for all values of $x$.'' The negation of this would
be the proposition ``there is at least one $x$ for which $fx$ is
true'' which is represented by ``$(\exists x) \DotOp fx$.'' If we had
started with not-$fx$ instead of $fx$ we should have arrived
at the proposition ``$fx$ is true for all values of $x$'' which
is represented by ``$(x) \DotOp fx$.'' Wittgenstein's method of
dealing with general propositions [\idEst\ ``$(x) \DotOp fx$'' and
``$(\exists x) \DotOp fx$''] differs from previous methods by the fact
that the generality comes only in specifying the set of
propositions concerned, and when this has been done the
building up of truth-functions proceeds exactly as it would
in the case of a finite number of enumerated arguments
\enlargethispage{9pt} % enlarge to make the last line fit
$p$, $q$, $r\fourdots$

Mr Wittgenstein's explanation of his symbolism at
this point is not quite fully given in the text. The symbol
% -----File: 015.png---
he uses is $(\overline{p}, \overline{\xi}, \DPtypo{\overline{N}}{N}(\overline{\xi}))$. The following is the explanation
of this symbol:

\begin{center}{%
\begin{minipage}{0.65\textwidth}

\noindent\hangindent 1.5em $\overline{p}$ stands for all atomic propositions.

\noindent\hangindent 1.5em $\overline{\xi}$ stands for any set of propositions.

\noindent\hangindent 1.5em $\DPtypo{\overline{N}}{N}(\overline{\xi})$ stands for the negation of all the propositions
making up $\overline{\xi}$.
\end{minipage}%
}\end{center}

The whole symbol $(\overline{p}, \overline{\xi}, \DPtypo{\overline{N}}{N}(\overline{\xi}))$ means whatever can be
obtained by taking any selection of atomic propositions,
negating them all, then taking any selection of the set
of propositions now obtained, together with any of the
originals---and so on indefinitely. This is, he says, the
general truth-function and also the general form of proposition.
What is meant is somewhat less complicated
than it sounds. The symbol is intended to describe a
process by the help of which, given the atomic propositions,
all others can be manufactured. The process depends
upon:

(\textit{a})~Sheffer's proof that all truth-functions can be
obtained out of simultaneous negation, \idEst\ out of ``not-$p$
and not-$q$'';

(\textit{b})~Mr Wittgenstein's theory of the derivation of
general propositions from conjunctions and disjunctions;

(\textit{c})~The assertion that a proposition can only occur in
another proposition as argument to a truth-function.
Given these three foundations, it follows that all propositions
which are not atomic can be derived from such
as are, by a uniform process, and it is this process which
is indicated by Mr Wittgenstein's symbol.

From this uniform method of construction we arrive
at an amazing simplification of the theory of inference,
as well as a definition of the sort of propositions that
belong to logic. The method of generation which has
just been described, enables Wittgenstein to say that all
propositions can be constructed in the above manner from
atomic propositions, and in this way the totality of propositions
is defined. (The apparent exceptions which we
mentioned above are dealt with in a manner which we
% -----File: 016.png---
shall consider later.) Wittgenstein is enabled to assert
that propositions are all that follows from the totality of
atomic propositions (together with the fact that it is the
totality of them); that a proposition is always a truth-function
of atomic propositions; and that if $p$ follows from
$q$ the meaning of $p$ is contained in the meaning of $q$, from
which of course it results that nothing can be deduced
from an atomic proposition. All the propositions of logic,
he maintains, are tautologies, such, for example, as ``$p$
or not $p$.''

The fact that nothing can be deduced from an atomic
proposition has interesting applications, for example, to
causality. There cannot, in Wittgenstein's logic, be any
such thing as a causal nexus. ``The events of the future,''
he says, ``\emph{cannot} be inferred from those of the present.
Superstition is the belief in the causal nexus.'' That the
sun will rise to-morrow is a hypothesis. We do not in
fact know whether it will rise, since there is no compulsion
according to which one thing must happen because another
happens.

Let us now take up another subject---that of names.
In Wittgenstein's theoretical logical language, names are
only given to simples. We do not give two names to
one thing, or one name to two things. There is no way
whatever, according to him, by which we can describe
the totality of things that can be named, in other words,
the totality of what there is in the world. In order to be
able to do this we should have to know of some property
which must belong to every thing by a logical necessity.
It has been sought to find such a property in self-identity,
but the conception of identity is subjected by Wittgenstein
to a destructive criticism from which there seems no escape.
The definition of identity by means of the identity of indiscernibles
is rejected, because the identity of indiscernibles
appears to be not a logically necessary principle. According
to this principle $x$ is identical with $y$ if every property
of $x$ is a property of $y$, but it would, after all, be logically
possible for two things to have exactly the same properties.
% -----File: 017.png---
If this does not in fact happen that is an accidental
characteristic of the world, not a logically necessary
characteristic, and accidental characteristics of the world
must, of course, not be admitted into the structure of
logic. Mr Wittgenstein accordingly banishes identity
and adopts the convention that different letters are to
mean different things. In practice, identity is needed as
between a name and a description or between two descriptions.
It is needed for such propositions as ``Socrates
is the philosopher who drank the hemlock,'' or ``The
even prime is the next number after 1.'' For such uses
of identity it is easy to provide on Wittgenstein's system.

The rejection of identity removes one method of
speaking of the totality of things, and it will be found
that any other method that may be suggested is equally
fallacious: so, at least, Wittgenstein contends and, I
think, rightly. This amounts to saying that ``object'' is
a pseudo-concept. To say ``$x$ is an object'' is to say
nothing. It follows from this that we cannot make such
statements as ``there are more than three objects in the
world,'' or ``there are an infinite number of objects in
the world.'' Objects can only be mentioned in connexion
with some definite property. We can say ``there are more
than three objects which are human,'' or ``there are more
than three objects which are red,'' for in these statements
the word object can be replaced by a variable in the
language of logic, the variable being one which satisfies
in the first case the function ``$x$ is human''; in the second
the function ``$x$ is red.'' But when we attempt to say
``there are more than three objects,'' this substitution of
the variable for the word ``object'' becomes impossible,
and the proposition is therefore seen to be meaningless.

We here touch one instance of Wittgenstein's fundamental
thesis, that it is impossible to say anything about
the world as a whole, and that whatever can be said has
to be about bounded portions of the world. This view
may have been originally suggested by notation, and if
so, that is much in its favour, for a good notation has
% -----File: 018.png---
a subtlety and suggestiveness which at times make it
seem almost like a live teacher. Notational irregularities
are often the first sign of philosophical errors, and
a perfect notation would be a substitute for thought.
But although notation may have first suggested to
Mr Wittgenstein the limitation of logic to things within
the world as opposed to the world as a whole, yet the
view, once suggested, is seen to have much else to
recommend it. Whether it is ultimately true I do not,
for my part, profess to know. In this Introduction I
am concerned to expound it, not to pronounce upon it.
According to this view we could only say things about
the world as a whole if we could get outside the world,
if, that is to say, it ceased to be for us the whole world.
Our world may be bounded for some superior being who
can survey it from above, but for us, however finite it
may be, it cannot have a boundary, since it has nothing
outside it. Wittgenstein uses, as an analogy, the field
of vision. Our field of vision does not, for us, have a
visual boundary, just because there is nothing outside
it, and in like manner our logical world has no logical
boundary because our logic knows of nothing outside it.
These considerations lead him to a somewhat curious
discussion of Solipsism. Logic, he says, fills the world.
The boundaries of the world are also its boundaries. In
logic, therefore, we cannot say, there is this and this in
the world, but not that, for to say so would apparently
presuppose that we exclude certain possibilities, and this
cannot be the case, since it would require that logic
should go beyond the boundaries of the world as if it
could contemplate these boundaries from the other side
also. What we cannot think we cannot think, therefore
we also cannot say what we cannot think.

This, he says, gives the key to Solipsism. What
Solipsism intends is quite correct, but this cannot be
said, it can only be shown. That the world is \emph{my} world
appears in the fact that the boundaries of language (the
only language I understand) indicate the boundaries of
% -----File: 019.png---
my world. The metaphysical subject does not belong to
the world but is a boundary of the world.

We must take up next the question of molecular propositions
which are at first sight not truth-functions,
of the propositions that they contain, such, for example,
as ``A believes $p$.''

Wittgenstein introduces this subject in the statement
of his position, namely, that all molecular functions are
truth-functions. He says (\PropERef{5.54}): ``In the general propositional
form, propositions occur in a proposition only
as bases of truth-operations.'' At first sight, he goes on
to explain, it seems as if a proposition could also occur
in other ways, \exempliGratia\ ``A believes $p$.'' Here it seems superficially
as if the proposition $p$ stood in a sort of relation
to the object A. ``But it is clear that `A believes that
$p$,' `A thinks $p$,' `A says $p$' are of the form `$p$ says $p$';
and here we have no co-ordination of a fact and an object,
but a co-ordination of facts by means of a co-ordination
of their objects'' (\PropERef{5.542}).

What Mr Wittgenstein says here is said so shortly
that its point is not likely to be clear to those who have
not in mind the controversies with which he is concerned.
The theory with which he is disagreeing will be found
in my articles on the nature of truth and falsehood in
\BookTitle{Philosophical Essays} and \BookTitle{Proceedings of the Aristotelian
Society}, 1906--7. The problem at issue is the problem of
the logical form of belief, \idEst\ what is the schema representing
what occurs when a man believes. Of course, the
problem applies not only to belief, but also to a host of
other mental phenomena which may be called propositional
attitudes: doubting, considering, desiring, etc. In all
these cases it seems natural to express the phenomenon
in the form ``A doubts $p$,'' ``A desires $p$,'' etc., which
makes it appear as though we were dealing with a relation
between a person and a proposition. This cannot, of
course, be the ultimate analysis, since persons are fictions
and so are propositions, except in the sense in which they
are facts on their own account. A proposition, considered
% -----File: 020.png---
as a fact on its own account, may be a set of words which
a man says over to himself, or a complex image, or train
of images passing through his mind, or a set of incipient
bodily movements. It may be any one of innumerable
different things. The proposition as a fact on its own
account, for example the actual set of words the man
pronounces to himself, is not relevant to logic. What is
relevant to logic is that common element among all these
facts, which enables him, as we say, to \emph{mean} the fact
which the proposition asserts. To psychology, of course,
more is relevant; for a symbol does not mean what it
symbolizes in virtue of a logical relation alone, but in
virtue also of a psychological relation of intention, or
association, or what-not. The psychological part of meaning,
however, does not concern the logician. What does
concern him in this problem of belief is the logical schema.
It is clear that, when a person believes a proposition, the
person, considered as a metaphysical subject, does not
have to be assumed in order to explain what is happening.
What has to be explained is the relation between the set
of words which is the proposition considered as a fact on
its own account, and the ``objective'' fact which makes
the proposition true or false. This reduces ultimately to
the question of the meaning of propositions, that is to
say, the meaning of propositions is the only non-psychological
portion of the problem involved in the analysis
of belief. This problem is simply one of a relation of
two facts, namely, the relation between the series of words
used by the believer and the fact which makes these
words true or false. The series of words is a fact just
as much as what makes it true or false is a fact. The
relation between these two facts is not unanalysable, since
the meaning of a proposition results from the meaning
of its constituent words. The meaning of the series of
words which is a proposition is a function of the meanings
of the separate words. Accordingly, the proposition as a
whole does not really enter into what has to be explained
in explaining the meaning of a proposition. It would
% -----File: 021.png---
perhaps help to suggest the point of view which I am
trying to indicate, to say that in the cases we have been
considering the proposition occurs as a fact, not as a
proposition. Such a statement, however, must not be
taken too literally. The real point is that in believing,
desiring, etc., what is logically fundamental is the relation
of a proposition \emph{considered as a fact}, to the fact which
makes it true or false, and that this relation of two facts
is reducible to a relation of their constituents. Thus the
proposition does not occur at all in the same sense in
which it occurs in a truth-function.

There are some respects, in which, as it seems to me,
Mr Wittgenstein's theory stands in need of greater
technical development. This applies in particular to
his theory of number (\PropERef{6.02}~ff.) which, as it stands, is only
capable of dealing with finite numbers. No logic can
be considered adequate until it has been shown to be
capable of dealing with transfinite numbers. I do not
think there is anything in Mr Wittgenstein's system to
make it impossible for him to fill this lacuna.

More interesting than such questions of comparative
detail is Mr Wittgenstein's attitude towards the mystical.
His attitude upon this grows naturally out of his doctrine
in pure logic, according to which the logical proposition
is a picture (true or false) of the fact, and has in common
with the fact a certain structure. It is this common
structure which makes it capable of being a picture of
the fact, but the structure cannot itself be put into words,
since it is a structure \emph{of} words, as well as of the facts to
which they refer. Everything, therefore, which is involved
in the very idea of the expressiveness of language must
remain incapable of being expressed in language, and is,
therefore, inexpressible in a perfectly precise sense. This
inexpressible contains, according to Mr Wittgenstein, the
whole of logic and philosophy. The right method of
teaching philosophy, he says, would be to confine oneself
to propositions of the sciences, stated with all possible
clearness and exactness, leaving philosophical assertions
% -----File: 022.png---
to the learner, and proving to him, whenever he made
them, that they are meaningless. It is true that the fate
of Socrates might befall a man who attempted this method
of teaching, but we are not to be deterred by that fear, if
it is the only right method. It is not this that causes
some hesitation in accepting Mr Wittgenstein's position,
in spite of the very powerful arguments which he brings
to its support. What causes hesitation is the fact that,
after all, Mr Wittgenstein manages to say a good deal
about what cannot be said, thus suggesting to the
sceptical reader that possibly there may be some loophole
through a hierarchy of languages, or by some other
exit. The whole subject of ethics, for example, is placed
by Mr Wittgenstein in the mystical, inexpressible region.
Nevertheless he is capable of conveying his ethical
opinions. His defence would be that what he calls the
mystical can be shown, although it cannot be said. It
may be that this defence is adequate, but, for my part,
I confess that it leaves me with a certain sense of
intellectual discomfort.

There is one purely logical problem in regard to
which these difficulties are peculiarly acute. I mean the
problem of generality. In the theory of generality it is
necessary to consider all propositions of the form $fx$ where
$fx$ is a given propositional function. This belongs to
the part of logic which can be expressed, according to
Mr Wittgenstein's system. But the totality of possible
values of $x$ which might seem to be involved in the totality
of propositions of the form $fx$ is not admitted by Mr
Wittgenstein among the things that can be spoken of,
for this is no other than the totality of things in the world,
and thus involves the attempt to conceive the world as a
whole; ``the feeling of the world as a bounded whole is
the mystical''; hence the totality of the values of $x$ is
mystical (\PropERef{6.45}). This is expressly argued when Mr
Wittgenstein denies that we can make propositions as
to how many things there are in the world, as for example,
that there are more than three.
% -----File: 023.png---

These difficulties suggest to my mind some such
possibility as this: that every language has, as Mr
Wittgenstein says, a structure concerning which, \emph{in the
language}, nothing can be said, but that there may be
another language dealing with the structure of the first
language, and having itself a new structure, and that to
this hierarchy of languages there may be no limit. Mr
Wittgenstein would of course reply that his whole theory
is applicable unchanged to the totality of such languages.
The only retort would be to deny that there is any such
totality. The totalities concerning which Mr Wittgenstein
holds that it is impossible to speak logically are nevertheless
thought by him to exist, and are the subject-matter of his
mysticism. The totality resulting from our hierarchy
would be not merely logically inexpressible, but a fiction,
a mere delusion, and in this way the supposed sphere of
the mystical would be abolished. Such an hypothesis is
very difficult, and I can see objections to it which at the
moment I do not know how to answer. Yet I do not
see how any easier hypothesis can escape from Mr
Wittgenstein's conclusions. Even if this very difficult
hypothesis should prove tenable, it would leave untouched
a very large part of Mr Wittgenstein's theory, though
possibly not the part upon which he himself would wish
to lay most stress. As one with a long experience of the
difficulties of logic and of the deceptiveness of theories
which seem irrefutable, I find myself unable to be sure
of the rightness of a theory, merely on the ground that I
cannot see any point on which it is wrong. But to have
constructed a theory of logic which is not at any point
obviously wrong is to have achieved a work of extraordinary
difficulty and importance. This merit, in my
opinion, belongs to Mr Wittgenstein's book, and makes
it one which no serious philosopher can afford to neglect.

\begin{minipage}{0.9\textwidth}
\vspace{3.5ex}
\begin{flushright}
\textsc{Bertrand Russell.}
\end{flushright}

{\small \textit{May} 1922.}
\end{minipage}
% -----File: 024.png---




\SkipToNewPage{empty}
\begin{center}
\Title{Tractatus Logico-Philosophicus}

\vfill

{\small DEDICATED\\
\vspace{0.5ex}
TO THE MEMORY OF MY FRIEND\\}
\vspace{0.5ex}
DAVID H. PINSENT

\vfill
\vfill

\begin{minipage}[c]{0.85\textwidth}
\noindent\Emph{Motto}: \ldots\ und alles, was man weiss, nicht bloss rauschen
und brausen gehört hat, lässt sich in drei Worten sagen.

\vspace{-0.5\baselineskip}
\begin{flushright}
\textsc{Kürnberger.}
\end{flushright}
\end{minipage}
\end{center}
\vfill
\vfill
\vfill
% -----File: 025.png---
% -----File: 027.png---




\Preface{Tractatus Logico-Philosophicus}{Preface}


This book will perhaps only be understood by those
who have themselves already thought the thoughts which
are expressed in it---or similar thoughts. It is therefore
not a text-book. Its object would be attained if there
were one person who read it with understanding and to
whom it afforded pleasure.

The book deals with the problems of philosophy and
shows, as I believe, that the method of formulating these
problems rests on the misunderstanding of the logic of
our language. Its whole meaning could be summed up
somewhat as follows: What can be said at all can be said
clearly; and whereof one cannot speak thereof one must
be silent.

The book will, therefore, draw a limit to thinking,
or rather---not to thinking, but to the expression of
thoughts; for, in order to draw a limit to thinking we
should have to be able to think both sides of this limit
(we should therefore have to be able to think what cannot
be thought).

The limit can, therefore, only be drawn in language
and what lies on the other side of the limit will be simply
nonsense.

How far my efforts agree with those of other philosophers
I will not decide. Indeed what I have here
written makes no claim to novelty in points of detail;
and therefore I give no sources, because it is indifferent
to me whether what I have thought has already been
thought before me by another.
% -----File: 029.png---

I will only mention that to the great works of Frege
and the writings of my friend Bertrand Russell I owe in
large measure the stimulation of my thoughts.

If this work has a value it consists in two things.
First that in it thoughts are expressed, and this value will
be the greater the better the thoughts are expressed. The
more the nail has been hit on the head.---Here I am
conscious that I have fallen far short of the possible.
Simply because my powers are insufficient to cope with
the task.---May others come and do it better.

On the other hand the \emph{truth} of the thoughts communicated
here seems to me unassailable and definitive. I
am, therefore, of the opinion that the problems have in
essentials been finally solved. And if I am not mistaken
in this, then the value of this work secondly consists in the
fact that it shows how little has been done when these
problems have been solved.
% -----File: 031.png---


\MainMatter{Tractatus logico-philosophicus}

\begin{propositions}
\PropositionE{1}
{The world is everything that is the case.\footnote{The decimal figures as numbers of the separate propositions indicate the logical
importance of the propositions, the emphasis laid upon them in my exposition.
The propositions \textit{n}.1, \textit{n}.2, \textit{n}.3, etc., are comments on proposition No.\;\textit{n}; the propositions
\textit{n}.\textit{m}1, \textit{n}.\textit{m}2, etc., are comments on the proposition No.\;\textit{n}.\textit{m}; and so on.}}


\PropositionE{1.1}
{The world is the totality of facts, not of
things.}


\PropositionE{1.11}
{The world is determined by the facts, and by
these being \emph{all} the facts.}


\PropositionE{1.12}
{For the totality of facts determines both what is
the case, and also all that is not the case.}


\PropositionE{1.13}
{The facts in logical space are the world.}


\PropositionE{1.2}
{The world divides into facts.}


\PropositionE{1.21}
{Any one can either be the case or not be the
case, and everything else remain the same.}


\PropositionE{2}
{What is the case, the fact, is the existence of
atomic facts.}


\PropositionE{2.01}
{An atomic fact is a combination of objects
(entities, things).}


\PropositionE{2.011}
{It is essential to a thing that it can be a constituent
part of an atomic fact.}


\PropositionE{2.012}
{In logic nothing is accidental: if a thing \emph{can}
occur in an atomic fact the possibility of that
atomic fact must already be prejudged in the
thing.}


\PropositionE{2.0121}
{It would, so to speak, appear as an accident, when
to a thing that could exist alone on its own account,
subsequently a state of affairs could be made to fit.

If things can occur in atomic facts, this possibility
must already lie in them.

(A logical entity cannot be merely possible.
Logic treats of every possibility, and all possibilities
are its facts.)
% -----File: 033.png---

Just as we cannot think of spatial objects at
all apart from space, or temporal objects apart
from time, so we cannot think of \emph{any} object apart
from the possibility of its connexion with other
things.

If I can think of an object in the context of an
atomic fact, I cannot think of it apart from the
\emph{possibility} of this context.}


\PropositionE{2.0122}
{The thing is independent, in so far as it can
occur in all \emph{possible} circumstances, but this form
of independence is a form of connexion with the
atomic fact, a form of dependence. (It is impossible
for words to occur in two different ways,
alone and in the proposition.)}


\PropositionE{2.0123}
{If I know an object, then I also know all the
possibilities of its occurrence in atomic facts.

(Every such possibility must lie in the nature
of the object.)

A new possibility cannot subsequently be
found.}


\PropositionE{2.01231}
{In order to know an object, I must know not
its external but all its internal qualities.}


\PropositionE{2.0124}
{If all objects are given, then thereby are all
\emph{possible} atomic facts also given.}


\PropositionE{2.013}
{Every thing is, as it were, in a space of possible
atomic facts. I can think of this space as empty,
but not of the thing without the space.}


\PropositionE{2.0131}
{A spatial object must lie in infinite space.
(A point in space is a place for an argument.)

A speck in a visual field need not be red,
but it must have a colour; it has, so to speak,
a colour space round it. A tone must have \emph{a}
% -----File: 035.png---
pitch, the object of the sense of touch \emph{a} hardness,
etc.}


\PropositionE{2.014}
{Objects contain the possibility of all states of
affairs.}


\PropositionE{2.0141}
{The possibility of its occurrence in atomic facts
is the form of the object.}


\PropositionE{2.02}
{The object is simple.}


\PropositionE{2.0201}
{Every statement about complexes can be analysed
into a statement about their constituent parts, and
into those propositions which completely describe
the complexes.}


\PropositionE{2.021}
{Objects form the substance of the world.
Therefore they cannot be compound.}


\PropositionE{2.0211}
{If the world had no substance, then whether
a proposition had sense would depend on whether
another proposition was true.}


\PropositionE{2.0212}
{It would then be impossible to form a picture
of the world (true or false).}


\PropositionE{2.022}
{It is clear that however different from the real
one an imagined world may be, it must have something---a
form---in common with the real world.}


\PropositionE{2.023}
{This fixed form consists of the objects.}


\PropositionE{2.0231}
{The substance of the world \emph{can} only determine
a form and not any material properties. For these
are first presented by the propositions---first formed
by the configuration of the objects.}


\PropositionE{2.0232}
{Roughly speaking: objects are colourless.}


\PropositionE{2.0233}
{Two objects of the same logical form are---apart
from their external prop\-er\-ties---only differentiated
from one another in that they are
different.}


\PropositionE{2.02331}
{Either a thing has properties which no other
has, and then one can distinguish it straight away
from the others by a description and refer to it;
or, on the other hand, there are several things
which have the totality of their properties in
% -----File: 037.png---
common, and then it is quite impossible to point
to any one of them.

For if a thing is not distinguished by anything,
I cannot distinguish it---for otherwise it would be
distinguished.}


\PropositionE{2.024}
{Substance is what exists independently of what
is the case.}


\PropositionE{2.025}
{It is form and content.}


\PropositionE{2.0251}
{Space, time and colour (colouredness) are forms
of objects.}


\PropositionE{2.026}
{Only if there are objects can there be a fixed
form of the world.}


\PropositionE{2.027}
{The fixed, the existent and the object are
one.}


\PropositionE{2.0271}
{The object is the fixed, the existent; the configuration
is the changing, the variable.}


\PropositionE{2.0272}
{The configuration of the objects forms the
atomic fact.}


\PropositionE{2.03}
{In the atomic fact objects hang one in another,
like the members of a chain.}


\PropositionE{2.031}
{In the atomic fact the objects are combined in
a definite way.}


\PropositionE{2.032}
{The way in which objects hang together in
the atomic fact is the structure of the atomic
fact.}


\PropositionE{2.033}
{The form is the possibility of the structure.}


\PropositionE{2.034}
{The structure of the fact consists of the structures
of the atomic facts.}


\PropositionE{2.04}
{The totality of existent atomic facts is the
world.}


\PropositionE{2.05}
{The totality of existent atomic facts also determines
which atom\-ic facts do not exist.}


\PropositionE{2.06}
{The existence and non-existence of atomic facts
is the reality.

(The existence of atomic facts we also call
a positive fact, their non-existence a negative
fact.)}


\PropositionE{2.061}
{Atomic facts are independent of one another.}
% -----File: 039.png---


\PropositionE{2.062}
{From the existence or non-existence of an
atomic fact we cannot infer the existence or non-existence
of another.}


\PropositionE{2.063}
{The total reality is the world.}


\PropositionE{2.1}
{We make to ourselves pictures of facts.}


\PropositionE{2.11}
{The picture presents the facts in logical space,
the existence and non-ex\-is\-tence of atomic
facts.}


\PropositionE{2.12}
{The picture is a model of reality.}


\PropositionE{2.13}
{To the objects correspond in the picture the
elements of the picture.}


\PropositionE{2.131}
{The elements of the picture stand, in the picture,
for the objects.}


\PropositionE{2.14}
{The picture consists in the fact that its elements
are combined with one another in a definite way.}


\PropositionE{2.141}
{The picture is a fact.}


\PropositionE{2.15}
{That the elements of the picture are combined
with one another in a definite way, represents that
\enlargethispage{9pt} % enlarge to make the last line fit
the things are so combined with one another.

This connexion of the elements of the picture is
called its structure, and the possibility of this structure
is called the form of representation of the picture.}


\PropositionE{2.151}
{The form of representation is the possibility that
the things are combined with one another as are
the elements of the picture.}


\PropositionE{2.1511}
{Thus the picture is linked with reality; it reaches
up to it.}


\PropositionE{2.1512}
{It is like a scale applied to reality.}


\PropositionE{2.15121}
{Only the outermost points of the dividing lines
\emph{touch} the object to be measured.}


\PropositionE{2.1513}
{According to this view the representing relation
which makes it a picture, also belongs to the
picture.}


\PropositionE{2.1514}
{The representing relation consists of the co-ordinations
of the elements of the picture and the
things.}


\PropositionE{2.1515}
{These co-ordinations are as it were the feelers of
% -----File: 041.png---
its elements with which the picture touches
reality.}


\PropositionE{2.16}
{In order to be a picture a fact must have something
in common with what it pictures.}


\PropositionE{2.161}
{In the picture and the pictured there must be
something identical in order that the one can be a
picture of the other at all.}


\PropositionE{2.17}
{What the picture must have in common with
reality in order to be able to represent it after its
manner---rightly or falsely---is its form of representation.}


\PropositionE{2.171}
{The picture can represent every reality whose
form it has.

The spatial picture, everything spatial, the
coloured, everything coloured, etc.}


\PropositionE{2.172}
{The picture, however, cannot represent its form
of representation; it shows it forth.}


\PropositionE{2.173}
{The picture represents its object from without
(its standpoint is its form of representation), therefore
the picture represents its object rightly or
falsely.}


\PropositionE{2.174}
{But the picture cannot place itself outside of its
form of representation.}


\PropositionE{2.18}
{What every picture, of whatever form, must
have in common with reality in order to be able to
represent it at all---rightly or falsely---is the logical
form, that is, the form of reality.}


\PropositionE{2.181}
{If the form of representation is the logical form,
then the picture is called a logical picture.}


\PropositionE{2.182}
{Every picture is \emph{also} a logical picture. (On the
other hand, for example, not every picture is spatial.)}


\PropositionE{2.19}
{The logical picture can depict the world.}


\PropositionE{2.2}
{The picture has the logical form of representation
in common with what it pictures.}


\PropositionE{2.201}
{The picture depicts reality by representing a
possibility of the existence and non-existence of
atomic facts.}
% -----File: 043.png---


\PropositionE{2.202}
{The picture represents a possible state of affairs
in logical space.}


\PropositionE{2.203}
{The picture contains the possibility of the state
of affairs which it represents.}


\PropositionE{2.21}
{The picture agrees with reality or not; it is
right or wrong, true or false.}


\PropositionE{2.22}
{The picture represents what it represents, independently
of its truth or falsehood, through the
form of representation.}


\PropositionE{2.221}
{What the picture represents is its sense.}


\PropositionE{2.222}
{In the agreement or disagreement of its sense
with reality, its truth or falsity consists.}


\PropositionE{2.223}
{In order to discover whether the picture is true
or false we must compare it with reality.}


\PropositionE{2.224}
{It cannot be discovered from the picture alone
whether it is true or false.}


\PropositionE{2.225}
{There is no picture which is a priori true.}


\PropositionE{3}
{The logical picture of the facts is the
thought.}


\PropositionE{3.001}
{``An atomic fact is thinkable''---means: we can
imagine it.}


\PropositionE{3.01}
{The totality of true thoughts is a picture of the
world.}


\PropositionE{3.02}
{The thought contains the possibility of the state
of affairs which it thinks.
\enlargethispage{3pt} % enlarge to make the last line fit

What is thinkable is also possible.}


\PropositionE{3.03}
{We cannot think anything unlogical, for otherwise
we should have to think unlogically.}


\PropositionE{3.031}
{It used to be said that God could create everything,
except what was contrary to the laws of logic.
The truth is, we could not \emph{say} of an ``unlogical''
world how it would look.}


\PropositionE{3.032}
{To present in language anything which
``contradicts logic'' is as impossible as in
geometry to present by its co-ordinates a figure
which contradicts the laws of space; or to give
% -----File: 045.png---
the co-ordinates of a point which does not
exist.}


\PropositionE{3.0321}
{We could present spatially an atomic fact which
contradicted the laws of physics, but not one which
contradicted the laws of geometry.}


\PropositionE{3.04}
{An a priori true thought would be one whose
possibility guaranteed its truth.}


\PropositionE{3.05}
{We could only know a priori that a thought
is true if its truth was to be recognized from
the thought itself (without an object of comparison).}


\PropositionE{3.1}
{In the proposition the thought is expressed
perceptibly through the senses.}


\PropositionE{3.11}
{We use the sensibly perceptible sign (sound or
written sign, etc.) of the proposition as a projection
of the possible state of affairs.

The method of projection is the thinking of
the sense of the proposition.}


\PropositionE{3.12}
{The sign through which we express the thought
I call the propositional sign. And the proposition
is the propositional sign in its projective
relation to the world.}


\PropositionE{3.13}
{To the proposition belongs everything which
belongs to the projection; but not what is projected.

Therefore the possibility of what is projected but
not this itself.

In the proposition, therefore, its sense is not yet
contained, but the possibility of expressing it.

(``The content of the proposition'' means the
content of the significant proposition.)

In the proposition the form of its sense is
\enlargethispage{15pt} % enlarge to make last line fit
contained, but not its content.}


\PropositionE{3.14}
{The propositional sign consists in the fact that
its elements, the words, are combined in it in a
definite way.

The propositional sign is a fact.}


\PropositionE{3.141}
{The proposition is not a mixture of words
% -----File: 047.png---
(just as the musical theme is not a mixture of
tones).

The proposition is articulate.}


\PropositionE{3.142}
{Only facts can express a sense, a class of names
cannot.}


\PropositionE{3.143}
{That the propositional sign is a fact is concealed
by the ordinary form of expression, written or
printed.

(For in the printed proposition, for example, the
sign of a proposition does not appear essentially
different from a word. Thus it was possible for
Frege to call the proposition a compounded
name.)}


\PropositionE{3.1431}
{The essential nature of the propositional sign
becomes very clear when we imagine it made up
of spatial objects (such as tables, chairs, books)
instead of written signs.

The mutual spatial position of these things then
expresses the sense of the proposition.}


\PropositionE{3.1432}
{We must not say, ``The complex sign `$aRb$'
says `$a$ stands in relation $R$ to $b$'{}''; but we must
say, ``\emph{That} `$a$' stands in a certain relation to `$b$'
says \emph{that $aRb$}''.}


\PropositionE{3.144}
{States of affairs can be described but not
\emph{named}.

(Names resemble points; propositions resemble
arrows, they have sense.)}


\PropositionE{3.2}
{In propositions thoughts can be so expressed
that to the objects of the thoughts correspond the
elements of the propositional sign.}


\PropositionE{3.201}
{These elements I call ``simple signs'' and the
proposition ``completely analysed''.}


\PropositionE{3.202}
{The simple signs employed in propositions are
called names.}


\PropositionE{3.203}
{The name means the object. The object is its
meaning. (``$A$'' is the same sign as ``$A$''.)}


\PropositionE{3.21}
{To the configuration of the simple signs in the
% -----File: 049.png---
propositional sign corresponds the configuration
\enlargethispage{12pt} % enlarge to make the last word fit
of the objects in the state of affairs.}


\PropositionE{3.22}
{In the proposition the name represents the object.}


\PropositionE{3.221}
{Objects I can only \emph{name}. Signs represent them.
I can only speak \emph{of} them. I cannot \emph{assert them}.
A proposition can only say \emph{how} a thing is, not
\emph{what} it is.}


\PropositionE{3.23}
{The postulate of the possibility of the simple
signs is the postulate of the determinateness of
the sense.}


\PropositionE{3.24}
{A proposition about a complex stands in
internal relation to the proposition about its
constituent part.

A complex can only be given by its description,
and this will either be right or wrong. The proposition
in which there is mention of a complex,
if this does not exist, becomes not nonsense but
simply false.

That a propositional element signifies a complex
can be seen from an indeterminateness in the propositions
in which it occurs. We \emph{know} that everything
is not yet determined by this proposition.
(The notation for generality \emph{contains} a prototype.)

The combination of the symbols of a complex
in a simple symbol can be expressed by a definition.}


\PropositionE{3.25}
{There is one and only one complete analysis of
the proposition.}


\PropositionE{3.251}
{The proposition expresses what it expresses in
a definite and clearly specifiable way: the proposition
is articulate.}


\PropositionE{3.26}
{The name cannot be analysed further by any
definition. It is a primitive sign.}


\PropositionE{3.261}
{Every defined sign signifies \emph{via} those signs
by which it is defined, and the definitions show
the way.

Two signs, one a primitive sign, and one
defined by primitive signs, cannot signify in the
% -----File: 051.png---
same way. Names \emph{cannot} be taken to pieces by
definition (nor any sign which alone and independently
has a meaning).}


\PropositionE{3.262}
{What does not get expressed in the sign is
shown by its application. What the signs conceal,
their application declares.}


\PropositionE{3.263}
{The meanings of primitive signs can be
explained by elucidations. Elucidations are propositions
which contain the primitive signs. They
can, therefore, only be understood when the
meanings of these signs are already known.}


\PropositionE{3.3}
{Only the proposition has sense; only in the
context of a proposition has a name meaning.}


\PropositionE{3.31}
{Every part of a proposition which characterizes
its sense I call an expression (a symbol).

(The proposition itself is an expression.)

Expressions are everything---essential for the
sense of the prop\-o\-si\-tion---that propositions can
have in common with one another.

An expression characterizes a form and a
content.}


\PropositionE{3.311}
{An expression presupposes the forms of all
propositions in which it can occur. It is the
common characteristic mark of a class of propositions.}


\PropositionE{3.312}
{It is therefore represented by the general form
of the propositions which it characterizes.

And in this form the expression is \emph{constant} and
everything else \emph{variable}.}


\PropositionE{3.313}
{An expression is thus presented by a variable,
whose values are the propositions which contain
the expression.

(In the limiting case the variables become
constants, the expression a proposition.)

I call such a variable a ``propositional variable''.}


\PropositionE{3.314}
{An expression has meaning only in a proposition.
Every variable can be conceived as a
propositional variable.
% -----File: 053.png---

(Including the variable name.)}


\PropositionE{3.315}
{If we change a constituent part of a proposition
into a variable, there is a class of propositions
which are all the values of the resulting variable
proposition. This class in general still depends
on what, by arbitrary agreement, we mean by
parts of that proposition. But if we change all
those signs, whose meaning was arbitrarily determined,
into variables, there always remains such
a class. But this is now no longer dependent on
any agreement; it depends only on the nature of
the proposition. It corresponds to a logical form,
to a logical prototype.}


\PropositionE{3.316}
{What values the propositional variable can
assume is determined.

The determination of the values \emph{is} the variable.}


\PropositionE{3.317}
{The determination of the values of the propositional
variable is done by \emph{indicating the propositions}
whose common mark the variable is.

The determination is a description of these
propositions.

The determination will therefore deal only with
symbols not with their meaning.

And \emph{only} this is essential to the determination,
\emph{that it is only a description of symbols and asserts
nothing about what is symbolized}.

The way in which we describe the propositions
is not essential.}


\PropositionE{3.318}
{I conceive the proposition---like Frege and
Russell---as a function of the expressions contained
in it.}


\PropositionE{3.32}
{The sign is the part of the symbol perceptible
by the senses.}


\PropositionE{3.321}
{Two different symbols can therefore have the
sign (the written sign or the sound sign) in
common---they then signify in different ways.}
% -----File: 055.png---


\PropositionE{3.322}
{It can never indicate the common characteristic
of two objects that we symbolize them with the
same signs but by different \emph{methods of symbolizing}.
For the sign is arbitrary. We could therefore
equally well choose two different signs and
where then would be what was common in the
symbolization.}


\PropositionE{3.323}
{In the language of everyday life it very often
happens that the same word signifies in two different
ways---and therefore belongs to two different
symbols---or that two words, which signify in
different ways, are apparently applied in the same
way in the proposition.

Thus the word ``is'' appears as the copula,
as the sign of equality, and as the expression of
existence; ``to exist'' as an intransitive verb like
``to go''; ``identical'' as an adjective; we speak
of \emph{something} but also of the fact of \emph{something}
happening.

(In the proposition ``Green is green''---where
the first word is a proper name and the last an
adjective---these words have not merely different
meanings but they are \emph{different symbols}.)}


\PropositionE{3.324}
{Thus there easily arise the most fundamental
confusions (of which the whole of philosophy is
full).}


\PropositionE{3.325}
{In order to avoid these errors, we must employ
a symbolism which excludes them, by not applying
the same sign in different symbols and by
not applying signs in the same way which signify
in different ways. A symbolism, that is to say,
which obeys the rules of \emph{logical} grammar---of logical
syntax.

(The logical symbolism of Frege and Russell
is such a language, which, however, does still not
exclude all errors.)}
% -----File: 057.png---


\PropositionE{3.326}
{In order to recognize the symbol in the sign
we must consider the significant use.}


\PropositionE{3.327}
{The sign determines a logical form only together
with its logical syntactic application.}


\PropositionE{3.328}
{If a sign is \emph{not necessary} then it is meaningless.
That is the meaning of Occam's razor.

(If everything in the symbolism works as
though a sign had meaning, then it has meaning.)}


\PropositionE{3.33}
{In logical syntax the meaning of a sign ought
never to play a rôle; it must admit of being
established without mention being thereby made
of the \emph{meaning} of a sign; it ought to presuppose
\emph{only} the description of the expressions.}


\PropositionE{3.331}
{From this observation we get a further view---into
Russell's \BookTitle{Theory of Types}. Russell's error is
shown by the fact that in drawing up his symbolic
rules he has to speak of the meaning of
the signs.}


\PropositionE{3.332}
{No proposition can say anything about itself,
because the propositional sign cannot be contained
in itself (that is the ``whole theory of types'').}


\PropositionE{3.333}
{A function cannot be its own argument, because
the functional sign already contains the prototype
of its own argument and it cannot contain
itself.

{\verystretchyspace
If, for example, we suppose that the function
$F(fx)$ could be its own argument, then there would
be a proposition ``$F(F(fx))$'', and in this the outer
function $F$ and the inner function $F$ must have
different meanings; for the inner has the form
$\phi(fx)$, the outer the form $\psi(\phi(fx))$. Common to
both functions is only the letter ``$F$'', which by
itself signifies nothing.}

This is at once clear, if instead of ``$F(F(u))$'' we
write ``$(\exists\phi) : F(\phi u) \DotOp \phi u = Fu$''.

Herewith Russell's paradox vanishes.}
% -----File: 059.png---


\PropositionE{3.334}
{The rules of logical syntax must follow of themselves,
if we only know how every single sign
signifies.}


\PropositionE{3.34}
{A proposition possesses essential and accidental
features.

Accidental are the features which are due to a
particular way of producing the propositional sign.
Essential are those which alone enable the proposition
to express its sense.}


\PropositionE{3.341}
{The essential in a proposition is therefore that
which is common to all propositions which can
express the same sense.

And in the same way in general the essential in
a symbol is that which all symbols which can
fulfil the same purpose have in common.}


\PropositionE{3.3411}
{One could therefore say the real name is that
which all symbols, which signify an object, have
in common. It would then follow, step by step,
that no sort of composition was essential for a name.}


\PropositionE{3.342}
{In our notations there is indeed something
arbitrary, but \emph{this} is not arbitrary, namely that
\emph{if} we have determined anything arbitrarily, then
something else \emph{must} be the case. (This results
from the \emph{essence} of the notation.)}


\PropositionE{3.3421}
{A particular method of symbolizing may be
unimportant, but it is always important that this
is a \emph{possible} method of symbolizing. And this
happens as a rule in philosophy: The single
thing proves over and over again to be unimportant,
but the possibility of every single thing reveals
something about the nature of the world.}


\PropositionE{3.343}
{Definitions are rules for the translation of one
language into another. Every correct symbolism
must be translatable into every other according
to such rules. It is \emph{this} which all have in
common.}
\enlargethispage{-9pt} % force the next proposition to the next page


\PropositionE{3.344}
{What signifies in the symbol is what is
common to all those symbols by which it can
% -----File: 061.png---
be replaced according to the rules of logical
syntax.}


\PropositionE{3.3441}
{We can, for example, express what is common to
all notations for the truth-functions as follows: It
is common to them that they all, for example, \emph{can
be replaced} by the notations of ``$\Not{p}$'' (``not $p$'')
and ``$p \lor q$'' (``$p$ or $q$'').

(Herewith is indicated the way in which a special
possible notation can give us general information.)}


\PropositionE{3.3442}
{The sign of the complex is not arbitrarily
resolved in the analysis, in such a way that its
resolution would be different in every propositional
structure.}


\PropositionE{3.4}
{The proposition determines a place in logical
space: the existence of this logical place is guaranteed
by the existence of the constituent parts alone,
by the existence of the significant proposition.}


\PropositionE{3.41}
{The propositional sign and the logical co-ordinates:
that is the logical place.}


\PropositionE{3.411}
{The geometrical and the logical place agree in
that each is the possibility of an existence.}


\PropositionE{3.42}
{Although a proposition may only determine
one place in logical space, the whole logical space
must already be given by it.

(Otherwise denial, the logical sum, the logical
product, etc., would always introduce new elements---in
co-ordination.)

(The logical scaffolding round the picture determines
the logical space. The proposition reaches
through the whole logical space.)}


\PropositionE{3.5}
{The applied, thought, propositional sign is the
thought.}


\PropositionE{4}
{The thought is the significant proposition.}


\PropositionE{4.001}
{The totality of propositions is the language.}


\PropositionE{4.002}
{Man possesses the capacity of constructing
languages, in which every sense can be expressed,
% -----File: 063.png---
without having an idea how and what each word
means---just as one speaks without knowing how
the single sounds are produced.

Colloquial language is a part of the human
organism and is not less complicated than it.

From it it is humanly impossible to gather
immediately the logic of language.

Language disguises the thought; so that from
the external form of the clothes one cannot infer
the form of the thought they clothe, because the
external form of the clothes is constructed with
quite another object than to let the form of the
body be recognized.

The silent adjustments to understand colloquial
language are enormously complicated.}


\PropositionE{4.003}
{Most propositions and questions, that have been
written about philosophical matters, are not false, but
senseless. We cannot, therefore, answer questions
of this kind at all, but only state their senselessness.
Most questions and propositions of the philosophers
result from the fact that we do not understand the
logic of our language.

(They are of the same kind as the question
whether the Good is more or less identical than the
Beautiful.)

And so it is not to be wondered at that the
deepest problems are really \emph{no} problems.}


\PropositionE{4.0031}
{All philosophy is ``Critique of language'' (but
not at all in Mauthner's sense). Russell's merit is
to have shown that the apparent logical form of the
proposition need not be its real form.}


\PropositionE{4.01}
{The proposition is a picture of reality.

The proposition is a model of the reality as we
think it is.}


\PropositionE{4.011}
{At the first glance the proposition---say as it
stands printed on paper---does not seem to be a
% -----File: 065.png---
picture of the reality of which it treats. But nor
does the musical score appear at first sight to be a
picture of a musical piece; nor does our phonetic
spelling (letters) seem to be a picture of our spoken
language. And yet these symbolisms prove to be
pictures---even in the ordinary sense of the word---of
what they represent.}


\PropositionE{4.012}
{It is obvious that we perceive a proposition
of the form $aRb$ as a picture. Here the sign is
obviously a likeness of the signified.}


\PropositionE{4.013}
{And if we penetrate to the essence of this
pictorial nature we see that this is not disturbed
by \emph{apparent irregularities} (like the use of $\sharp$ and $\flat$ in
the score).

For these irregularities also picture what they
are to express; only in another way.}


\PropositionE{4.014}
{The gramophone record, the musical thought,
the score, the waves of sound, all stand to one
another in that pictorial internal relation, which
holds between language and the world. To all of
them the logical structure is common.

(Like the two youths, their two horses and their
lilies in the story. They are all in a certain sense
one.)}


\PropositionE{4.0141}
{In the fact that there is a general rule by which
the musician is able to read the symphony out of
the score, and that there is a rule by which one
could reconstruct the symphony from the line on
a gramophone record and from this again---by
means of the first rule---construct the score, herein
lies the internal similarity between these things
which at first sight seem to be entirely different.
And the rule is the law of projection which projects
the symphony into the language of the musical
score. It is the rule of translation of this language
into the language of the gramophone record.}


\PropositionE{4.015}
{The possibility of all similes, of all the
% -----File: 067.png---
imagery of our language, rests on the logic of
representation.}


\PropositionE{4.016}
{In order to understand the essence of the
proposition, consider hieroglyphic writing, which
pictures the facts it describes.

And from it came the alphabet without the
essence of the representation being lost.}


\PropositionE{4.02}
{This we see from the fact that we understand
the sense of the propositional sign, without having
had it explained to us.}


\PropositionE{4.021}
{The proposition is a picture of reality, for I know
the state of affairs presented by it, if I understand
the proposition. And I understand the proposition,
without its sense having been explained to me.}


\PropositionE{4.022}
{The proposition \emph{shows} its sense.

The proposition \emph{shows} how things stand, \emph{if} it is
true. And it \emph{says}, that they do so stand.}


\PropositionE{4.023}
{The proposition determines reality to this
extent, that one only needs to say ``Yes'' or
``No'' to it to make it agree with reality.

It must therefore be completely described by
the proposition.

A proposition is the description of a fact.

As the description of an object describes it by
its external properties so propositions describe
reality by its internal properties.

The proposition constructs a world with the
help of a logical scaffolding, and therefore one
can actually see in the proposition all the logical
features possessed by reality if it is true. One can
\emph{draw conclusions} from a false proposition.}


\PropositionE{4.024}
{To understand a proposition means to know
what is the case, if it is true.

(One can therefore understand it without
knowing whether it is true or not.)

One understands it if one understands its
constituent parts.}


\PropositionE{4.025}
{The translation of one language into another
% -----File: 069.png---
is not a process of translating each proposition
of the one into a proposition of the other, but
only the constituent parts of propositions are
translated.

(And the dictionary does not only translate
substantives but also adverbs and conjunctions,
etc., and it treats them all alike.)}


\PropositionE{4.026}
{The meanings of the simple signs (the words)
must be explained to us, if we are to understand
them.

By means of propositions we explain ourselves.}


\PropositionE{4.027}
{It is essential to propositions, that they can
communicate a \emph{new} sense to us.}


\PropositionE{4.03}
{A proposition must communicate a new sense
with old words.

The proposition communicates to us a state of
affairs, therefore it must be \emph{essentially} connected
with the state of affairs.

And the connexion is, in fact, that it is its
logical picture.

The proposition only asserts something, in so
far as it is a picture.}


\PropositionE{4.031}
{In the proposition a state of affairs is, as it
were, put together for the sake of experiment.

One can say, instead of, This proposition has
such and such a sense, This proposition represents
such and such a state of affairs.}


\PropositionE{4.0311}
{One name stands for one thing, and another
for another thing, and they are connected together.
And so the whole, like a living picture, presents
the atomic fact.}


\PropositionE{4.0312}
{The possibility of propositions is based upon the
principle of the representation of objects by signs.

My fundamental thought is that the ``logical
constants'' do not represent. That the \emph{logic} of the
facts cannot be represented.}


\PropositionE{4.032}
{The proposition is a picture of its state of
affairs, only in so far as it is logically articulated.
% -----File: 071.png---

(Even the proposition ``ambulo'' is composite,
for its stem gives a different sense with another
termination, or its termination with another
stem.)}


\PropositionE{4.04}
{In the proposition there must be exactly as
many things distinguishable as there are in the
state of affairs, which it represents.

They must both possess the same logical
(mathematical) multiplicity (cf. Hertz's Mechanics,
on Dynamic Models).}


\PropositionE{4.041}
{This mathematical multiplicity naturally cannot
in its turn be represented. One cannot get outside
it in the representation.}


\PropositionE{4.0411}
{If we tried, for example, to express what is
expressed by ``$(x) \DotOp fx$'' by putting an index before
$fx$, like: ``Gen. $fx$'', it would not do, we should
not know what was generalized. If we tried to
show it by an index $g$, like: ``$f(x_{g})$'' it would not
do---we should not know the scope of the generalization.

If we were to try it by introducing a mark
in the argument places, like ``$(G,G) \DotOp F(G,G)$'', it
would not do---we could not determine the identity
of the variables, etc.

All these ways of symbolizing are inadequate
because they have not the necessary mathematical
multiplicity.}


\PropositionE{4.0412}
{For the same reason the idealist explanation of
the seeing of spatial relations through ``spatial
spectacles'' does not do, because it cannot explain
the multiplicity of these relations.}


\PropositionE{4.05}
{Reality is compared with the proposition.}


\PropositionE{4.06}
{Propositions can be true or false only by being
pictures of the reality.}


\PropositionE{4.061}
{If one does not observe that propositions have
a sense independent of the facts, one can easily
believe that true and false are two relations
% -----File: 073.png---
between signs and things signified with equal
rights.

One could then, for example, say that ``$p$''
signifies in the true way what ``$\Not{p}$'' signifies in
the false way, etc.}


\PropositionE{4.062}
{Can we not make ourselves understood by
means of false propositions as hitherto with true
ones, so long as we know that they are meant to
be false? No! For a proposition is true, if
what we assert by means of it is the case; and if
by ``$p$'' we mean $\Not{p}$, and what we mean is the
case, then ``$p$'' in the new conception is true
and not false.}


\PropositionE{4.0621}
{That, however, the signs ``$p$'' and ``$\Not{p}$'' \emph{can}
say the same thing is important, for it shows
that the sign ``$\Not{}$'' corresponds to nothing in
reality.

That negation occurs in a proposition, is no
characteristic of its sense ($\Not{\Not{p = p}}$).

The propositions ``$p$'' and ``$\Not{p}$'' have opposite
senses, but to them corresponds one and
the same reality.}


\PropositionE{4.063}
{An illustration to explain the concept of truth.
A black spot on white paper; the form of the spot
can be described by saying of each point of the
plane whether it is white or black. To the fact
that a point is black corresponds a positive fact;
to the fact that a point is white (not black), a
negative fact. If I indicate a point of the plane
(a truth-value in Frege's terminology), this corresponds
to the assumption proposed for judgment,
etc.\ etc.

But to be able to say that a point is black or
white, I must first know under what conditions a
point is called white or black; in order to be able
to say ``$p$'' is true (or false) I must have determined
under what conditions I call ``$p$'' true,
% -----File: 075.png---
and thereby I determine the sense of the proposition.

The point at which the simile breaks down is
this: we can indicate a point on the paper, without
knowing what white and black are; but to a proposition
without a sense corresponds nothing at
all, for it signifies no thing (truth-value) whose
properties are called ``false'' or ``true''; the verb
of the proposition is not ``is true'' or ``is false''---as
Frege thought---but that which ``is true'' must
already contain the verb.}


\PropositionE{4.064}
{Every proposition must \emph{already} have a sense;
assertion cannot give it a sense, for what it asserts
is the sense itself. And the same holds of
denial, etc.}


\PropositionE{4.0641}
{One could say, the denial is already related to
the logical place determined by the proposition
that is denied.

The denying proposition determines a logical
place \emph{other} than does the proposition denied.

The denying proposition determines a logical
place, with the help of the logical place of the
proposition denied, by saying that it lies outside
the latter place.

That one can deny again the denied proposition,
shows that what is denied is already a proposition
and not merely the preliminary to a
proposition.}


\PropositionE{4.1}
{A proposition presents the existence and non-existence
of atomic facts.}


\PropositionE{4.11}
{The totality of true propositions is the total
natural science (or the totality of the natural
sciences).}


\PropositionE{4.111}
{Philosophy is not one of the natural
sciences.

(The word ``philosophy'' must mean something
% -----File: 077.png---
which stands above or below, but not beside the
natural sciences.)}


\PropositionE{4.112}
{The object of philosophy is the logical clarification
of thoughts.

Philosophy is not a theory but an activity.

A philosophical work consists essentially of
elucidations.

The result of philosophy is not a number of
``philosophical propositions'', but to make propositions
clear.

{\verystretchyspace
Philosophy should make clear and delimit
sharply the thoughts which otherwise are, as it
were, opaque and blurred.}}


\PropositionE{4.1121}
{Psychology is no nearer related to philosophy,
than is any other natural science.

The theory of knowledge is the philosophy of
psychology.

Does not my study of sign-language correspond
to the study of thought processes which philosophers
held to be so essential to the philosophy of logic?
Only they got entangled for the most part in unessential
psychological investigations, and there
is an analogous danger for my method.}


\PropositionE{4.1122}
{The Darwinian theory has no more to do with
philosophy than has any other hypothesis of natural
science.}


\PropositionE{4.113}
{Philosophy limits the disputable sphere of natural
science.}


\PropositionE{4.114}
{It should limit the thinkable and thereby the
unthinkable.

{\stretchyspace
It should limit the unthinkable from within
through the thinkable.}}


\PropositionE{4.115}
{It will mean the unspeakable by clearly displaying
the speakable.}


\PropositionE{4.116}
{Everything that can be thought at all can be
% -----File: 079.png---
thought clearly. Everything that can be said can
be said clearly.}


\PropositionE{4.12}
{Propositions can represent the whole reality,
but they cannot represent what they must have in
common with reality in order to be able to represent
it---the logical form.

To be able to represent the logical form, we
should have to be able to put ourselves with the
propositions outside logic, that is outside the
world.}


\PropositionE{4.121}
{Propositions cannot represent the logical form:
this mirrors itself in the propositions.

That which mirrors itself in language, language
cannot represent.

That which expresses \emph{itself} in language, \emph{we}
cannot express by language.

The propositions \emph{show} the logical form of reality.

They exhibit it.}


\PropositionE{4.1211}
{Thus a proposition ``$fa$'' shows that in its sense
the object $a$ occurs, two propositions ``$fa$'' and
``$ga$'' that they are both about the same object.

If two propositions contradict one another, this
is shown by their structure; similarly if one follows
from another, etc.}


\PropositionE{4.1212}
{What \emph{can} be shown \emph{cannot} be said.}


\PropositionE{4.1213}
{Now we understand our feeling that we are in
possession of the right logical conception, if only
all is right in our symbolism.}


\PropositionE{4.122}
{We can speak in a certain sense of formal
properties of objects and atomic facts, or of properties
of the structure of facts, and in the same
sense of formal relations and relations of
structures.

(Instead of property of the structure I also say
% -----File: 081.png---
``internal property''; instead of relation of structures
``internal relation''.

I introduce these expressions in order to show
the reason for the confusion, very widespread
among philosophers, between internal relations
and proper (external) relations.)

The holding of such internal properties and relations
cannot, however, be asserted by propositions,
but it shows itself in the propositions, which
present the atomic facts and treat of the objects in
question.}


\PropositionE{4.1221}
{An internal property of a fact we also call a
feature of this fact. (In the sense in which we
speak of facial features.)}


\PropositionE{4.123}
{A property is internal if it is unthinkable that
its object does not possess it.

(This blue colour and that stand in the internal
relation of brighter and darker eo ipso. It is
unthinkable that \emph{these} two objects should not stand
in this relation.)

(Here to the shifting use of the words ``property''
and ``relation'' there corresponds the shifting use
of the word ``object''.)}


\PropositionE{4.124}
{The existence of an internal property of a possible
state of affairs is not expressed by a proposition,
but it expresses itself in the proposition which
presents that state of affairs, by an internal property
of this proposition.

It would be as senseless to ascribe a formal
property to a proposition as to deny it the formal
property.}


\PropositionE{4.1241}
{One cannot distinguish forms from one another
by saying that one has this property but the other
that: for this assumes that there is a sense in asserting
either property of either form.}


\PropositionE{4.125}
{The existence of an internal relation between
% -----File: 083.png---
possible states of affairs expresses itself in language
by an internal relation between the propositions
presenting them.}


\PropositionE{4.1251}
{Here the disputed question ``whether all relations
are internal or external'' disappears.}


\PropositionE{4.1252}
{Series which are ordered by \emph{internal} relations I
call formal series.

The series of numbers is ordered not by an
external, but by an internal relation.

Similarly the series of propositions ``$aRb$'',
\[
\begin{array}{l}
``(\exists x) : aRx \DotOp xRb\text{'',}\\
``(\exists x,y) : aRx \DotOp aRy \DotOp yRb\text{'', etc.}
\end{array}
\]

(If $b$ stands in one of these relations to $a$, I call
$b$ a successor of $a$.)}


\PropositionE{4.126}
{In the sense in which we speak of formal
properties we can now speak also of formal
concepts.

(I introduce this expression in order to make
clear the confusion of formal concepts with proper
concepts which runs through the whole of the old
logic.)

That anything falls under a formal concept as
an object belonging to it, cannot be expressed by
a proposition. But it shows itself in the sign of
this object itself. (The name shows that it signifies
an object, the numerical sign that it signifies a
number, etc.)

Formal concepts cannot, like proper concepts,
be presented by a function.

For their characteristics, the formal properties,
are not expressed by the functions.

The expression of a formal property is a feature
of certain symbols.

The sign that signifies the characteristics of a
formal concept is, therefore, a characteristic feature
of all symbols, whose meanings fall under the
concept.
% -----File: 085.png---

The expression of the formal concept is therefore
a propositional variable in which only this
characteristic feature is constant.}


\PropositionE{4.127}
{The propositional variable signifies the formal
concept, and its values signify the objects which
fall under this concept.}


\PropositionE{4.1271}
{Every variable is the sign of a formal
concept.

For every variable presents a constant form,
which all its values possess, and which can
be conceived as a formal property of these
values.}


\PropositionE{4.1272}
{So the variable name ``$x$'' is the proper sign of
the pseudo-concept \emph{object}.

Wherever the word ``object'' (``thing'', ``entity'',
etc.) is rightly used, it is expressed in logical
symbolism by the variable name.

For example in the proposition ``there are two
objects which\ \ldots'', by ``$(\exists x,y)$\ \ldots''.

Wherever it is used otherwise, \idEst\ as a proper
concept word, there arise senseless pseudo-propositions.

So one cannot, \exempliGratia\ say ``There are objects''
as one says ``There are books''. Nor ``There
are 100 objects'' or ``There are $\aleph_0$ objects''. And
it is senseless to speak of the \emph{number of all
objects}.

The same holds of the words ``Complex'',
``Fact'', ``Function'', ``Number'', etc.

They all signify formal concepts and are
presented in logical symbolism by variables, not
by functions or classes (as Frege and Russell
thought).

Expressions like ``1 is a number'', ``there is
only one number nought'', and all like them are
senseless.

(It is as senseless to say, ``there is only one 1''
% -----File: 087.png---
as it would be to say: 2 + 2 is at 3 o'clock equal
to 4.)}


\PropositionE{4.12721}
{The formal concept is already given with an
object, which falls under it. One cannot, therefore,
introduce both, the objects which fall under
a formal concept \emph{and} the formal concept itself,
as primitive ideas. One cannot, therefore, \exempliGratia\ introduce
(as Russell does) the concept of function
and also special functions as primitive ideas; or
the concept of number and definite numbers.}


\PropositionE{4.1273}
{If we want to express in logical symbolism
the general proposition ``$b$ is a successor of $a$''
we need for this an expression for the general
term of the formal series: $aRb$, $(\exists x) : aRx \DotOp xRb$,
$(\exists x,y) : aRx \DotOp xRy \DotOp yRb$,\;\ldots\ The general term of
a formal series can only be expressed by a
variable, for the concept symbolized by ``term of
this formal series'' is a \emph{formal} concept. (This
Frege and Russell overlooked; the way in
which they express general propositions like the
above is, therefore, false; it contains a vicious
circle.)

We can determine the general term of the
formal series by giving its first term and the
general form of the operation, which generates
the following term out of the preceding proposition.}


\PropositionE{4.1274}
{The question about the existence of a formal
concept is senseless. For no proposition can
answer such a question.

(For example, one cannot ask: ``Are there
unanalysable sub\-ject-pre\-di\-cate propositions?'')}


\PropositionE{4.128}
{The logical forms are \emph{anumerical}.

Therefore there are in logic no pre-eminent
numbers, and therefore there is no philosophical
monism or dualism, etc.}


\PropositionE{4.2}
{The sense of a proposition is its agreement
and disagreement with the possibilities of the
% -----File: 089.png---
existence and non-existence of the atomic
facts.}


\PropositionE{4.21}
{The simplest proposition, the elementary proposition,
asserts the existence of an atomic fact.}


\PropositionE{4.211}
{It is a sign of an elementary proposition,
that no elementary proposition can contradict
it.}


\PropositionE{4.22}
{The elementary proposition consists of names.
It is a connexion, a concatenation, of names.}


\PropositionE{4.221}
{It is obvious that in the analysis of propositions
we must come to elementary propositions, which
consist of names in immediate combination.

The question arises here, how the propositional
connexion comes to be.}


\PropositionE{4.2211}
{Even if the world is infinitely complex, so
that every fact consists of an infinite number
of \DPtypo{atomatic}{atomic} facts and every atomic fact is
composed of an infinite number of objects,
even then there must be objects and atomic
facts.}


\PropositionE{4.23}
{The name occurs in the proposition only in
the context of the elementary proposition.}


\PropositionE{4.24}
{The names are the simple symbols, I indicate
them by single letters ($x$, $y$, $z$).

The elementary proposition I write as function
of the names, in the form ``$fx$'', ``$\phi(x,y)$'', etc.

Or I indicate it by the letters $p$, $q$, $r$.}


\PropositionE{4.241}
{If I use two signs with one and the same
meaning, I express this by putting between them
the sign ``=''.

``$a = b$'' means then, that the sign ``$a$'' is
replaceable by the sign ``$b$''.

(If I introduce by an equation a new sign ``$b$'',
by determining that it shall replace a previously
known sign ``$a$'', I write the equation---definition---(like
% -----File: 091.png---
Russell) in the form ``$a = b$ Def.''. A
definition is a symbolic rule.)}


\PropositionE{4.242}
{Expressions of the form ``$a = b$'' are therefore only
expedients in presentation: They assert nothing
about the meaning of the signs ``$a$'' and ``$b$''.}


\PropositionE{4.243}
{Can we understand two names without knowing
whether they signify the same thing or two
different things? Can we understand a proposition
in which two names occur, without knowing if they
mean the same or different things?

If I know the meaning of an English and a
synonymous German word, it is impossible for
me not to know that they are synonymous, it is
impossible for me not to be able to translate them
into one another.

Expressions like ``$a = a$'', or expressions
deduced from these are neither elementary propositions
nor otherwise significant signs. (This
will be shown later.)}


\PropositionE{4.25}
{If the elementary proposition is true, the atomic
fact exists; if it is false the atomic fact does not
exist.}


\PropositionE{4.26}
{The specification of all true elementary propositions
describes the world completely. The
world is completely described by the specification
of all elementary propositions plus the specification,
which of them are true and which false.}


\PropositionE{4.27}
{With regard to the existence of $n$ atomic facts
there are $K_{n} = \sum\limits_{\nu = 0}^n\binom{n}{\nu}$ possibilities.

It is possible for all combinations of atomic
facts to exist, and the others not to exist.}


\PropositionE{4.28}
{To these combinations correspond the same
number of possibilities of the truth---and falsehood---of
$n$ elementary propositions.}


\PropositionE{4.3}
{The truth-possibilities of the elementary propositions
mean the possibilities of the existence
and non-existence of the atomic facts.}
% -----File: 093.png---


\PropositionE{4.31}
{The truth-possibilities can be presented by
schemata of the following kind (``T'' means
``true'', ``F'' ``false''. The rows of T's and F's
under the row of the elementary propositions mean
their truth-possibilities in an easily intelligible
symbolism).

\begin{center}
\begin{tabular}[t]{c|c|c}
p & q & r\\
\hline
\hline
\Strut T & T & T\\
\hline
\Strut F & T & T\\
\hline
\Strut T & F & T\\
\hline
\Strut T & T & F\\
\hline
\Strut F & F & T\\
\hline
\Strut F & T & F\\
\hline
\Strut T & F & F\\
\hline
\Strut F & F & F\\
\hline
\end{tabular}
\hspace{0.5cm}
\begin{tabular}[t]{c|c}
p & q\\
\hline
\hline
\Strut T & T\\
\hline
\Strut F & T\\
\hline
\Strut T & F\\
\hline
\Strut F & F\\
\hline
\end{tabular}
\hspace{0.5cm}
\begin{tabular}[t]{c}
p\\
\hline
\hline
\Strut T\\
\hline
\Strut F\\
\hline
\end{tabular}
\end{center}
}


\PropositionE{4.4}
{A proposition is the expression of agreement
and disagreement with the truth-pos\-si\-bil\-i\-ties of
the elementary propositions.}


\PropositionE{4.41}
{The truth-possibilities of the elementary propositions
are the conditions of the truth and
falsehood of the propositions.}


\PropositionE{4.411}
{It seems probable even at first sight that the
introduction of the elementary propositions is
fundamental for the comprehension of the other
kinds of propositions. Indeed the comprehension
of the general propositions depends \emph{palpably} on
that of the elementary propositions.}


\PropositionE{4.42}
{With regard to the agreement and disagreement
of a proposition with the truth-possibilities
of $n$ elementary propositions there
are $\sum\limits_{\kappa = 0}^{K_n}\binom{K_n}{\kappa} = L_{n}$ possibilities.}


\PropositionE{4.43}
{Agreement with the truth-possibilities can be
% -----File: 095.png---
expressed by co-or\-di\-na\-ting with them in the
schema the mark ``T'' (true).

Absence of this mark means disagreement.}


\PropositionE{4.431}
{The expression of the agreement and disagreement
with the truth-pos\-si\-bil\-i\-ties of the elementary
propositions expresses the truth-conditions of the
proposition.

The proposition is the expression of its truth-conditions.

(Frege has therefore quite rightly put them at
the beginning, as explaining the signs of his
logical symbolism. Only Frege's explanation
of the truth-concept is false: if ``the true'' and
``the false'' were real objects and the arguments
in $\Not{p}$, etc., then the sense of $\Not{p}$ would by no
means be determined by Frege's determination.)}


\PropositionE{4.44}
{The sign which arises from the co-ordination of
that mark ``T'' with the truth-pos\-si\-bil\-i\-ties is a
propositional sign.}


\PropositionE{4.441}
{It is clear that to the complex of the signs ``F''
and ``T'' no object (or complex of objects) corresponds;
any more than to horizontal and vertical
lines or to brackets. There are no ``logical
objects''.

Something analogous holds of course for all
signs, which express the same as the schemata of
``T'' and ``F''.}


\PropositionE{4.442}
{Thus \exempliGratia\\
\phantom{Thus \exempliGratia}
\raisebox{2.4\baselineskip}{``}\begin{tabular}{c|c|c}
p & q &\\
\hline
\hline
\Strut T & T & T\\
\hline
\Strut F & T & T\\
\hline
\Strut T & F &\\
\hline
\Strut F & F & T\\
\hline
\end{tabular}\\
\phantom{Thus \exempliGratia\ ``\begin{tabular}[t]{c|c|c}F&F&T\end{tabular}}
\smash[t]{\raisebox{1.2\baselineskip}{''}}is a propositional sign.

(Frege's assertion sign ``$\vdash$'' is logically altogether
% -----File: 097.png---
meaningless; in Frege (and Russell) it only shows
that these authors hold as true the propositions
marked in this way.

``$\vdash$'' belongs therefore to the propositions no
more than does the number of the proposition. A
proposition cannot possibly assert of itself that it
is true.)

If the sequence of the truth-possibilities in the
schema is once for all determined by a rule of
combination, then the last column is by itself an
expression of the truth-conditions. If we write
this column as a row the propositional sign becomes:
``(TT--T)($p$, $q$)'', or more plainly: ``(TTFT)($p$, $q$)''.

(The number of places in the left-hand bracket
is determined by the number of terms in the right-hand
bracket.)}


\PropositionE{4.45}
{For $n$ elementary propositions there are $L_{n}$
possible groups of truth-con\-di\-tions.

The groups of truth-conditions which belong to
the truth-pos\-si\-bil\-i\-ties of a number of elementary
propositions can be ordered in a series.}


\PropositionE{4.46}
{Among the possible groups of truth-conditions
there are two extreme cases.

In the one case the proposition is true for all the
truth-pos\-si\-bil\-i\-ties of the elementary propositions.
We say that the truth-conditions are \emph{tautological}.

In the second case the proposition is false for all
the truth-pos\-si\-bil\-i\-ties. The truth-conditions are
\emph{self-contradictory}.

In the first case we call the proposition a
tautology, in the second case a contradiction.}


\PropositionE{4.461}
{The proposition shows what it says, the
tautology and the contradiction that they say
nothing.

The tautology has no truth-conditions, for it is
% -----File: 099.png---
unconditionally true; and the contradiction is on
no condition true.

Tautology and contradiction are without sense.

(Like the point from which two arrows go out in
opposite directions.)

(I know, \exempliGratia\ nothing about the weather, when
I know that it rains or does not rain.)}


\PropositionE{4.4611}
{Tautology and contradiction are, however, not
senseless; they are part of the symbolism, in the
same way that ``0'' is part of the symbolism of
Arithmetic.}


\PropositionE{4.462}
{Tautology and contradiction are not pictures of
the reality. They present no possible state of
affairs. For the one allows \emph{every} possible state
of affairs, the other \emph{none}.

In the tautology the conditions of agreement
with the world\AllowBreak---the presenting re\-la\-tions---cancel
one another, so that it stands in no presenting
relation to reality.}


\PropositionE{4.463}
{The truth-conditions determine the range, which
is left to the facts by the proposition.

(The proposition, the picture, the model, are in
a negative sense like a solid body, which restricts
the free movement of another: in a positive sense,
like the space limited by solid substance, in which
a body may be placed.)

Tautology leaves to reality the whole infinite
logical space; contradiction fills the whole logical
space and leaves no point to reality. Neither of
them, therefore, can in any way determine
reality.}


\PropositionE{4.464}
{The truth of tautology is certain, of propositions
possible, of contradiction impossible. (Certain,
possible, impossible: here we have an indication
of that gradation which we need in the theory of
probability.)}


\PropositionE{4.465}
{The logical product of a tautology and a proposition
% -----File: 101.png---
says the same as the proposition. Therefore
that product is identical with the proposition.
For the essence of the symbol cannot be altered
without altering its sense.}


\PropositionE{4.466}
{To a definite logical combination of signs
corresponds a definite logical combination of their
meanings; \emph{every arbitrary} combination only corresponds
to the unconnected signs.

That is, propositions which are true for every
state of affairs cannot be combinations of signs at
all, for otherwise there could only correspond to
them definite combinations of objects.

(And to no logical combination corresponds \emph{no}
combination of the objects.)

Tautology and contradiction are the limiting
cases of the combinations of symbols, namely their
dissolution.}


\PropositionE{4.4661}
{Of course the signs are also combined with one
another in the tautology and contradiction, \idEst\ they
stand in relations to one another, but these
relations are meaningless, unessential to the
\emph{symbol}.}


\PropositionE{4.5}
{Now it appears to be possible to give the
most general form of proposition; \idEst\ to give a
description of the propositions of some one sign
language, so that every possible sense can be
expressed by a symbol, which falls under the
description, and so that every symbol which falls
under the description can express a sense, if
the meanings of the names are chosen accordingly.

It is clear that in the description of the most
general form of proposition \emph{only} what is essential
to it may be described---otherwise it would not be
the most general form.

That there is a general form is proved by the
fact that there cannot be a proposition whose
form could not have been foreseen (\idEst\ constructed).
% -----File: 103.png---
The general form of proposition is: Such and
such is the case.}


\PropositionE{4.51}
{Suppose \emph{all} elementary propositions were given
me: then we can simply ask: what propositions I
can build out of them. And these are \emph{all} propositions
and \emph{so} are they limited.}


\PropositionE{4.52}
{The propositions are everything which follows
from the totality of all elementary propositions (of
course also from the fact that it is the \emph{totality of
them all}). (So, in some sense, one could say, that
\emph{all} propositions are generalizations of the elementary
propositions.)}


\PropositionE{4.53}
{The general propositional form is a variable.}


\PropositionE{5}
{Propositions are truth-functions of elementary
propositions.

(An elementary proposition is a truth-function
of itself.)}


\PropositionE{5.01}
{The elementary propositions are the truth-arguments
of propositions.}


\PropositionE{5.02}
{It is natural to confuse the arguments of
functions with the indices of names. For I
recognize the meaning of the sign containing it
from the argument just as much as from the
index.

In Russell's ``$\DPtypo{+}{+_{c}}$'', for example, ``$c$'' is an
index which indicates that the whole sign is the
addition sign for cardinal numbers. But this way
of symbolizing depends on arbitrary agreement,
and one could choose a simple sign instead of
``$+_{c}$'': but in ``$\Not{p}$'' ``$p$'' is not an index but
an argument; the sense of ``$\Not{p}$'' \emph{cannot} be understood,
unless the sense of ``$p$'' has previously been
understood. (In the name Julius Cæsar, Julius is
an index. The index is always part of a description
of the object to whose name we attach it, \exempliGratia\ \emph{The}
Cæsar of the Julian gens.)

The confusion of argument and index is, if I
am not mistaken, at the root of Frege's theory
% -----File: 105.png---
of the meaning of propositions and functions. For
Frege the propositions of logic were names and
their arguments the indices of these names.}


\PropositionE{5.1}
{The truth-functions can be ordered in
series.

That is the foundation of the theory of probability.}
\enlargethispage{-9pt} % force the next proposition to the next page

\PropositionE{5.101}
{The truth-functions of every number of elementary
propositions can be written in a schema of
the following kind:

\begin{table*}[!h]
\footnotesize\noindent\centering
\begin{tabular}{@{}c@{~}l@{~}l@{}}
(TTTT)($p, q$) & Tautology & (if $p$ then $p$, and if $q$ then $q$) [$p \Implies p \DotOp q \Implies q$]\\
(FTTT)($p, q$) & in words: & Not both $p$ and $q$. [$\Not{(p \DotOp q)}$]\\
(TFTT)($p, q$) & \DittoInWords & If $q$ then $p$. [$q \Implies p$]\\
(TTFT)($p, q$) & \DittoInWords & If $p$ then $q$. [$p \Implies q$]\\
(TTTF)($p, q$) & \DittoInWords & $p$ or $q$. [$p \lor q$]\\
(FFTT)($p, q$) & \DittoInWords & Not $q$. [$\Not{q}$]\\
(FTFT)($p, q$) & \DittoInWords & Not $p$. [$\Not{p}$]\\
(FTTF)($p, q$) & \DittoInWords & $p$ or $q$, but not both. [$p \DotOp \Not{q} : \lor : q \DotOp \Not{p}$]\\
(TFFT)($p, q$) & \DittoInWords & If $p$, then $q$; and if $q$, then $p$. [$p \equiv q$]\\
(TFTF)($p, q$) & \DittoInWords & $p$\\
(TTFF)($p, q$) & \DittoInWords & $q$\\
(FFFT)($p, q$) & \DittoInWords & Neither $p$ nor $q$. [$\Not{p} \DotOp \Not{q}$ or $p \BarOp q$]\\
(FFTF)($p, q$) & \DittoInWords & $p$ and not $q$. [$p \DotOp \Not{q}$]\\
(FTFF)($p, q$) & \DittoInWords & $q$ and not $p$. [$q \DotOp \Not{p}$]\\
(TFFF)($p, q$) & \DittoInWords & $p$ and $q$. [$p \DotOp q$]\\
(FFFF)($p, q$) & Contradiction & ($p$ and not $p$; and $q$ and not $q$.) [$p \DotOp \Not{p} \DotOp q \DotOp \Not{q}$]\\
\end{tabular}
\end{table*}

Those truth-possibilities of its truth-arguments,
which verify the proposition, I shall call its \emph{truth-grounds}.}


\PropositionE{5.11}
{If the truth-grounds which are common to a
number of propositions are all also truth-grounds
of some one proposition, we say that the truth of
this proposition follows from the truth of those
propositions.}


\PropositionE{5.12}
{In particular the truth of a proposition $p$ follows
from that of a proposition $q$, if all the truth-grounds
of the second are truth-grounds of the
first.}
% -----File: 107.png---


\PropositionE{5.121}
{The truth-grounds of $q$ are contained in those
of $p$; $p$ follows from $q$.}


\PropositionE{5.122}
{If $p$ follows from $q$, the sense of ``$p$'' is contained
in that of ``$q$''.}


\PropositionE{5.123}
{If a god creates a world in which certain propositions
are true, he creates thereby also a world
in which all propositions consequent on them are
true. And similarly he could not create a world
in which the proposition ``$p$'' is true without
creating all its objects.}


\PropositionE{5.124}
{A proposition asserts every proposition which
follows from it.}


\PropositionE{5.1241}
{``$p \DotOp q$'' is one of the propositions which assert
``$p$'' and at the same time one of the propositions
which assert ``$q$''.

Two propositions are opposed to one another
if there is no significant proposition which asserts
them both.

Every proposition which contradicts another,
denies it.}


\PropositionE{5.13}
{That the truth of one proposition follows from
the truth of other propositions, we perceive from
the structure of the propositions.}


\PropositionE{5.131}
{If the truth of one proposition follows from the
truth of others, this expresses itself in relations in
which the forms of these propositions stand to one
another, and we do not need to put them in these
relations first by connecting them with one another
in a proposition; for these relations are internal,
and exist as soon as, and by the very fact that,
the propositions exist.}


\PropositionE{5.1311}
{When we conclude from $p \lor q$ and $\Not{p}$ to $q$ the
relation between the forms of the propositions
``$p \lor q$'' and ``$\Not{p}$'' is here concealed by the method
of symbolizing. But if we write, \exempliGratia\ instead of
``$p \lor q$'' ``$p \BarOp q \DotOp \BarOp \DotOp p \BarOp q$'' and instead of ``$\Not{p}$''
``$p \BarOp p$'' ($p \BarOp q$ = neither $p$ nor $q$), then the inner
connexion becomes obvious.
% -----File: 109.png---

(The fact that we can infer $fa$ from $(x) \DotOp fx$ shows
that generality is present also in the symbol
``$(x) \DotOp fx$''.}


\PropositionE{5.132}
{If $p$ follows from $q$, I can conclude from $q$ to $p$;
infer $p$ from $q$.

The method of inference is to be understood
from the two propositions alone.

Only they themselves can justify the inference.

Laws of inference, which---as in Frege and
Russell---are to justify the conclusions, are senseless
and would be superfluous.}


\PropositionE{5.133}
{All inference takes place a priori.}


\PropositionE{5.134}
{From an elementary proposition no other can
be inferred.}


\PropositionE{5.135}
{In no way can an inference be made from the
existence of one state of affairs to the existence of
another entirely different from it.}


\PropositionE{5.136}
{There is no causal nexus which justifies such
an inference.}


\PropositionE{5.1361}
{The events of the future \emph{cannot} be inferred from
those of the present.

Superstition is the belief in the causal
nexus.}


\PropositionE{5.1362}
{The freedom of the will consists in the fact that
future actions cannot be known now. We could
only know them if causality were an \emph{inner} necessity,
like that of logical deduction.---The connexion
of knowledge and what is known is that of logical
necessity.

(``A knows that $p$ is the case'' is senseless if $p$
is a tautology.)}


\PropositionE{5.1363}
{If from the fact that a proposition is obvious
to us it does not \emph{follow} that it is true, then obviousness
is no justification for our belief in its truth.}


\PropositionE{5.14}
{If a proposition follows from another, then the
latter says more than the former, the former less
than the latter.}
% -----File: 111.png---


\PropositionE{5.141}
{If $p$ follows from $q$ and $q$ from $p$ then they are
one and the same proposition.}


\PropositionE{5.142}
{A tautology follows from all propositions: it
says nothing.}


\PropositionE{5.143}
{Contradiction is something shared by propositions,
which \emph{no} proposition has in common with
another. Tautology is that which is shared by
all propositions, which have nothing in common
with one another.

Contradiction vanishes so to speak outside,
tautology inside all propositions.

Contradiction is the external limit of the propositions,
tautology their substanceless centre.}


\PropositionE{5.15}
{If $T_{r}$ is the number of the truth-grounds of the
proposition ``$r$'', $T_{rs}$ the number of those truth-grounds
of the proposition ``$s$'' which are at the
same time truth-grounds of ``$r$'', then we call the
ratio $T_{rs} : T_{r}$ the measure of the \emph{probability} which
the proposition ``$r$'' gives to the proposition ``$s$''.}


\PropositionE{5.151}
{Suppose in a schema like that above in No.~\PropERef{5.101}
$T_{r}$ is the number of the ``T'''s in the proposition
$r$, $T_{rs}$ the number of those ``T'''s in
the proposition $s$, which stand in the same columns
as ``T'''s of the proposition $r$; then the proposition
$r$ gives to the proposition $s$ the probability
$T_{rs} : T_{r}$.}


\PropositionE{5.1511}
{There is no special object peculiar to probability
propositions.}


\PropositionE{5.152}
{Propositions which have no truth-arguments
in common with one another we call independent.
\enlargethispage{-3pt} % force next paragraph to the next page

Independent propositions (\exempliGratia\ any two elementary
propositions) give to one another the probability~$\frac{1}{2}$.

If $p$ follows from $q$, the proposition $q$ gives
to the proposition $p$ the probability~1. The
certainty of logical conclusion is a limiting case
of probability.

(Application to tautology and contradiction.)}


\PropositionE{5.153}
{A proposition is in itself neither probable nor
% -----File: 113.png---
improbable. An event occurs or does not occur,
there is no middle course.}


\PropositionE{5.154}
{In an urn there are equal numbers of white
and black balls (and no others). I draw one
ball after another and put them back in the
urn. Then I can determine by the experiment
that the numbers of the black and white balls
which are drawn approximate as the drawing
continues.

So \emph{this} is not a mathematical fact.

If then, I say, It is equally probable that
I should draw a white and a black ball, this
means, All the circumstances known to me (including
the natural laws hypothetically assumed)
give to the occurrence of the one event no more
probability than to the occurrence of the other.
That is they give---as can easily be understood
from the above explanations---to each the
probability~$\frac{1}{2}$.

What I can verify by the experiment is that
the occurrence of the two events is independent
of the circumstances with which I have no closer
acquaintance.}


\PropositionE{5.155}
{The unit of the probability proposition is: The
circumstances---with which I am not further acquainted---give
to the occurrence of a definite event
such and such a degree of probability.}


\PropositionE{5.156}
{Probability is a generalization.

It involves a general description of a propositional
form. Only in default of certainty do we
need probability.

If we are not completely acquainted with a fact,
but know \emph{something} about its form.

(A proposition can, indeed, be an incomplete
picture of a certain state of affairs, but it is always
\emph{a} complete picture.)
% -----File: 115.png---

The probability proposition is, as it were, an
extract from other propositions.}


\PropositionE{5.2}
{The structures of propositions stand to one
another in internal relations.}


\PropositionE{5.21}
{We can bring out these internal relations in
our manner of expression, by presenting a proposition
as the result of an operation which produces
it from other propositions (the bases of the
operation).}


\PropositionE{5.22}
{The operation is the expression of a relation
between the structures of its result and its
bases.}


\PropositionE{5.23}
{The operation is that which must happen to a
proposition in order to make another out of it.}


\PropositionE{5.231}
{And that will naturally depend on their formal
properties, on the internal similarity of their
forms.}


\PropositionE{5.232}
{The internal relation which orders a series is
equivalent to the operation by which one term
arises from another.}


\PropositionE{5.233}
{The first place in which an operation can occur
is where a proposition arises from another in a
logically significant way; \idEst\ where the logical
construction of the proposition begins.}


\PropositionE{5.234}
{The truth-functions of elementary \DPtypo{proposition}{propositions},
are results of operations which have the elementary
propositions as bases. (I call these
operations, truth-operations.)}


\PropositionE{5.2341}
{The sense of a truth-function of $p$ is a function
of the sense of $p$.

Denial, logical addition, logical multiplication,
etc.\ etc., are operations.

(Denial reverses the sense of a proposition.)}


\PropositionE{5.24}
{An operation shows itself in a variable; it shows
how we can proceed from one form of proposition
to another.

It gives expression to the difference between
the forms.
% -----File: 117.png---

(And that which is common to the bases, and
the result of an operation, is the bases themselves.)}


\PropositionE{5.241}
{The operation does not characterize a form but
only the difference between forms.}


\PropositionE{5.242}
{The same operation which makes ``$q$'' from
``$p$'', makes ``$r$'' from ``$q$'', and so on. This
can only be expressed by the fact that ``$p$'', ``$q$'',
``$r$'', etc., are variables which give general expression
\enlargethispage{10pt} % enlarge to make last line fit
to certain formal relations.}


\PropositionE{5.25}
{The occurrence of an operation does not characterize
the sense of a proposition.

For an operation does not assert anything; only
its result does, and this depends on the bases of
the operation.

(Operation and function must not be confused
with one another.)}


\PropositionE{5.251}
{A function cannot be its own argument, but
the result of an operation can be its own
basis.}


\PropositionE{5.252}
{Only in this way is the progress from term
to term in a formal series possible (from type
to type in the hierarchy of Russell and Whitehead).
(Russell and Whitehead have not admitted
the possibility of this progress but have made use
of it all the same.)}


\PropositionE{5.2521}
{The repeated application of an operation to
its own result I call its successive application
(``$O' O' O' a$'' is the result of the threefold successive
application of ``$O' \xi$'' to ``$a$'').

In a similar sense I speak of the successive
application of \emph{several} operations to a number of
propositions.}


\PropositionE{5.2522}
{The general term of the formal series $a, O' a,
O' O' a$,\;$\fourdots$\ I write thus: ``[$a$, $x$, $O' x$]''. This
expression in brackets is a variable. The first
term of the expression is the beginning of the
% -----File: 119.png---
formal series, the second the form of an arbitrary
term $x$ of the series, and the third the form
of that term of the series which immediately
follows $x$.}


\PropositionE{5.2523}
{The concept of the successive application of
an operation is equivalent to the concept ``and
so on''.}


\PropositionE{5.253}
{One operation can reverse the effect of another.
Operations can cancel one another.}


\PropositionE{5.254}
{Operations can vanish (\exempliGratia\ denial in ``$\Not{\Not{p}}$''\DPtypo{.}{,}
$\Not{\Not{p}} = p$).}


\PropositionE{5.3}
{All propositions are results of truth-operations
on the elementary propositions.

The truth-operation is the way in which a
truth-function arises from elementary propositions.

According to the nature of truth-operations,
in the same way as out of elementary propositions
arise their truth-functions, from truth-func\-tions
arises a new one. Every truth-operation
creates from truth-functions of elementary propositions
another truth-func\-tion of elementary
propositions, \idEst\ a proposition. The result of
every truth-operation on the results of truth-op\-er\-a\-tions
on elementary propositions is also
the result of \emph{one} truth-operation on elementary
propositions.

Every proposition is the result of truth-operations
on elementary propositions.}


\PropositionE{5.31}
{The Schemata No.~\PropERef{4.31} are also significant, if
``$p$'', ``$q$'', ``$r$'', etc.\ are not elementary propositions.

And it is easy to see that the propositional
sign in No.~\DPtypo{\PropERef{4.42}}{\PropERef{4.442}} expresses one truth-function of
elementary propositions even when ``$p$'' and
``$q$'' are truth-functions of elementary propositions.}


\PropositionE{5.32}
{All truth-functions are results of the successive
% -----File: 121.png---
application of a finite number of truth-operations
to elementary propositions.}


\PropositionE{5.4}
{Here it becomes clear that there are no such
things as ``logical objects'' or ``logical constants''
(in the sense of Frege and Russell).}


\PropositionE{5.41}
{For all those results of truth-operations on truth-functions
are identical, which are one and the same
truth-function of elementary propositions.}


\PropositionE{5.42}
{That $\lor$, $\Implies$, etc., are not relations in the sense of
right and left, etc., is obvious.

The possibility of crosswise definition of the
logical ``primitive signs'' of Frege and Russell
shows by itself that these are not primitive signs
and that they signify no relations.

And it is obvious that the ``$\Implies$'' which we define
by means of ``$\Not{}$'' and ``$\lor$'' is identical with that
by which we define ``$\lor$'' with the help of ``$\Not{}$'', and
that this ``$\lor$'' is the same as the first, and
so on.}


\PropositionE{5.43}
{That from a fact $p$ an infinite number of \emph{others}
should follow, namely $\Not{\Not{p}}$, $\Not{\Not{\Not{\Not{p}}}}$, etc., is
indeed hardly to be believed, and it is no less
wonderful that the infinite number of propositions
of logic (of mathematics) should follow from half
a dozen ``primitive propositions''.

But all propositions of logic say the same thing.
That is, nothing.}


\PropositionE{5.44}
{Truth-functions are not material functions.

If \exempliGratia\ an affirmation can be produced by
repeated denial, is the denial---in any sense---contained
in the affirmation?

Does ``$\Not{\Not{p}}$'' deny $\Not{p}$, or does it affirm $p$;
or both?

The proposition ``$\Not{\Not{p}}$'' does not treat of
denial as an object, but the possibility of denial is
already prejudged in affirmation.

And if there was an object called ``$\Not{}$'', then
% -----File: 123.png---
``$\Not{\Not{p}}$'' would have to say something other than
``$p$''. For the one proposition would then treat
of $\Not{}$, the other would not.}


\PropositionE{5.441}
{This disappearance of the apparent logical
constants also occurs if ``$\Not{(\exists x) \DotOp \Not{fx}}$'' says the
same as ``$(x) \DotOp fx$'', or ``$(\exists x) \DotOp fx \DotOp x = a$'' the same
as ``$fa$''.}


\PropositionE{5.442}
{If a proposition is given to us then the results
of all truth-operations which have it as their basis
are given \emph{with} it.}


\PropositionE{5.45}
{If there are logical primitive signs a correct logic
must make clear their position relative to one
another and justify their existence. The construction
of logic \emph{out of} its primitive signs must become
clear.}


\PropositionE{5.451}
{If logic has primitive ideas these must be
independent of one another. If a primitive idea
is introduced it must be introduced in all contexts
in which it occurs at all. One cannot therefore
introduce it for \emph{one} context and then again for
another. For example, if denial is introduced,
we must understand it in propositions of the form
``$\Not{p}$'', just as in propositions like ``$\Not{(p \lor q)}$'',
``$(\exists x) \DotOp \Not{fx}$'' and others. We may not first
introduce it for one class of cases and then for
another, for it would then remain doubtful whether
its meaning in the two cases was the same, and
there would be no reason to use the same way of
symbolizing in the two cases.

(In short, what Frege (``Grundgesetze der
Arithmetik'') has said about the introduction of
signs by definitions holds, mutatis mutandis, for
the introduction of primitive signs also.)}


\PropositionE{5.452}
{The introduction of a new expedient in the
symbolism of logic must always be an event full
of consequences. No new symbol may be introduced
in logic in brackets or in the margin---with,
so to speak, an entirely innocent face.
% -----File: 125.png---

(Thus in the ``Principia Mathematica'' of
Russell and Whitehead there occur definitions
and primitive propositions in words. Why suddenly
words here? This would need a justification.
There was none, and can be none for the
process is actually not allowed.)

But if the introduction of a new expedient has
proved necessary in one place, we must immediately
ask: Where is this expedient \emph{always} to be
used? Its position in logic must be made
clear.}


\PropositionE{5.453}
{All numbers in logic must be capable of
justification.

Or rather it must become plain that there are
no numbers in logic.

There are no pre-eminent numbers.}


\PropositionE{5.454}
{In logic there is no side by side, there can be
no classification.

In logic there cannot be a more general and a
more special.}


\PropositionE{5.4541}
{The solution of logical problems must be simple
for they set the standard of simplicity.

Men have always thought that there must be a
sphere of questions whose answers---a priori---are
symmetrical and united into a closed regular
structure.

A sphere in which the proposition, simplex
sigillum veri, is valid.}


\PropositionE{5.46}
{When we have rightly introduced the logical
signs, the sense of all their combinations has been
already introduced with them: therefore not only
``$p \lor q$'' but also ``$\Not{(p \lor \Not{q})}$'', etc.\ etc. We should
then already have introduced the effect of all
possible combinations of brackets; and it would
then have become clear that the proper general
primitive signs are not ``$p \lor q$'', ``$(\exists x) \DotOp fx$'', etc.,
% -----File: 127.png---
but the most general form of their combinations.}


\PropositionE{5.461}
{The apparently unimportant fact that the apparent
relations like \DPtypo{$v$}{$\lor$} and $\Implies$ need brackets---unlike
real relations is of great importance.

The use of brackets with these apparent primitive
signs shows that these are not the real
primitive signs; and nobody of course would
believe that the brackets have meaning by themselves.}


\PropositionE{5.4611}
{Logical operation signs are punctuations.}


\PropositionE{5.47}
{It is clear that everything which can be said
\emph{beforehand} about the form of \emph{all} propositions at
all can be said \emph{on one occasion}.

For all logical operations are already contained
in the elementary proposition. For ``$fa$'' says
the same as ``$(\exists x) \DotOp fx \DotOp x = a$''.

Where there is composition, there is argument
and function, and where these are, all logical
constants already are.

One could say: the one logical constant is that
which \emph{all} propositions, according to their nature,
have in common with one another.

That however is the general form of proposition.}


\PropositionE{5.471}
{The general form of proposition is the essence
of proposition.}


\PropositionE{5.4711}
{To give the essence of proposition means to
give the essence of all description, therefore the
essence of the world.}


\PropositionE{5.472}
{The description of the most general propositional
form is the description of the one and only
general primitive sign in logic.}


\PropositionE{5.473}
{Logic must take care of itself.

A \emph{possible} sign must also be able to signify.
Everything which is possible in logic is also
permitted. (``Socrates is identical'' means nothing
% -----File: 129.png---
because there is no property which is called
``identical''. The proposition is senseless because
we have not made some arbitrary determination,
not because the symbol is in itself unpermissible.)

In a certain sense we cannot make mistakes in
logic.}


\PropositionE{5.4731}
{Self-evidence, of which Russell has said so
much, can only be discarded in logic by language
itself preventing every logical mistake. That
logic is a priori consists in the fact that we \emph{cannot}
think illogically.}


\PropositionE{5.4732}
{We cannot give a sign the wrong sense.}


\PropositionE{5.47321}
{Occam's razor is, of course, not an arbitrary rule
nor one justified by its practical success. It simply
says that \emph{unnecessary} elements in a symbolism
mean nothing.

Signs which serve \emph{one} purpose are logically
equivalent, signs which serve \emph{no} purpose are
logically meaningless.}


\PropositionE{5.4733}
{Frege says: Every legitimately constructed
proposition must have a sense; and I say: Every
possible proposition is legitimately constructed,
and if it has no sense this can only be because
we have given no \emph{meaning} to some of its constituent
parts.

(Even if we believe that we have done
so.)

Thus ``Socrates is identical'' says nothing,
because we have given \emph{no} meaning to the word
``identical'' as \emph{adjective}. For when it occurs as
the sign of equality it symbolizes in an entirely
different way---the symbolizing relation is another---therefore
the symbol is in the two cases entirely
different; the two symbols have the sign in
common with one another only by accident.}
% -----File: 131.png---


\PropositionE{5.474}
{The number of necessary fundamental operations
depends \emph{only} on our notation.}


\PropositionE{5.475}
{It is only a question of constructing a system
of signs of a definite number of di\-men\-sions---of
a definite mathematical multiplicity.}


\PropositionE{5.476}
{It is clear that we are not concerned here with
a \emph{number of primitive ideas} which must be signified
but with the expression of a rule.}


\PropositionE{5.5}
{Every truth-function is a result of the successive
application of the operation \mbox{(--\;--\;--\;--\;--T)}\AllowBreak($\xi, \fourdots$) to
elementary propositions.

This operation denies all the propositions in
the right-hand bracket and I call it the negation
of these propositions.}


\PropositionE{5.501}
{An expression in brackets whose terms are
propositions I in\-di\-cate---if the order of the terms
in the bracket is indifferent---by a sign of the form
``($\overline{\xi}$)''. ``$\xi$'' is a variable whose values are the
terms of the expression in brackets, and the line
over the variable indicates that it stands for all
its values in the bracket.

(Thus if $\xi$ has the 3 values P, Q, R, then
($\overline{\xi}$) = (P, Q, R).)

The values of the variables must be determined.

{\stretchyspace
The determination is the description of the propositions
which the variable stands for.}

How the description of the terms of the expression
in brackets takes place is unessential.

We may distinguish 3 kinds of description:
1.~Direct enumeration. In this case we can place
simply its constant values instead of the variable.
2.~Giving a function $fx$, whose values for all
values of $x$ are the propositions to be described.
3.~Giving a formal law, according to which those
propositions are constructed. In this case the
% -----File: 133.png---
terms of the expression in brackets are all the
terms of a formal series.}


\PropositionE{5.502}
{Therefore I write instead of \mbox{``(--\;--\;--\;--\;--T)}\AllowBreak($\xi, \fourdots$)'',
``$N(\overline{\xi})$''.

$N(\overline{\xi})$ is the negation of all the values of the
propositional variable $\xi$.}


\PropositionE{5.503}
{As it is obviously easy to express how propositions
can be constructed by means of this operation
and how propositions are not to be constructed by
means of it, this must be capable of exact expression.}


\PropositionE{5.51}
{If $\xi$ has only one value, then $N(\overline{\xi}) = \Not{p}$ (not $p$),
if it has two values then $N(\overline{\xi}) = \Not{p} \DotOp \Not{q}$ (neither
$p$ nor $q$).}


\PropositionE{5.511}
{How can the all-embracing logic which mirrors
the world use such special catches and manipulations?
Only because all these are connected into
an infinitely fine network, to the great mirror.}


\PropositionE{5.512}
{``$\Not{p}$'' is true if ``$p$'' is false. Therefore in the
true proposition ``$\Not{p}$'' ``$p$'' is a false proposition.
How then can the stroke ``$\Not{}$'' bring it into
agreement with reality?

That which denies in ``$\Not{p}$'' is however not
``$\Not{}$'', but that which all signs of this notation,
which deny $p$, have in common.

Hence the common rule according to which
``$\Not{p}$'', ``$\Not{\Not{\Not{p}}}$'', ``${\Not{p}} \lor {\Not{p}}$'', ``$\Not{p} \DotOp \Not{p}$'',
etc.\ etc.\ (to infinity) are constructed. And this
which is common to them all mirrors denial.}


\PropositionE{5.513}
{We could say: What is common to all symbols,
which assert both $p$ and $q$, is the proposition
``$p \DotOp q$''. What is common to all symbols, which
assert either $p$ or $q$, is the proposition ``$p \lor q$''.

And similarly we can say: Two propositions
are opposed to one another when they have
nothing in common with one another; and every
proposition has only one negative, because there
is only one proposition which lies altogether
outside it.
% -----File: 135.png---

Thus even in Russell's notation it is evident
that ``${q : p} \lor {\Not{p}}$'' says the same as ``$q$''; that
``$p \lor {\Not{p}}$'' says nothing.}


\PropositionE{5.514}
{If a notation is fixed, there is in it a rule according
to which all the propositions denying $p$ are
constructed, a rule according to which all the
propositions asserting $p$ are constructed, a rule
according to which all the propositions asserting
$p$ or $q$ are constructed, and so on. These rules
are equivalent to the symbols and in them their
sense is mirrored.}


\PropositionE{5.515}
{It must be recognized in our symbols that what
is connected by ``$\lor$'', ``$\DotOp$'', etc., must be propositions.

And this is the case, for the symbols ``$p$'' and
``$q$'' presuppose ``$\lor$'', ``$\Not{}$'', etc. If the sign
``$p$'' in ``$p \lor q$'' does not stand for a complex sign,
then by itself it cannot have sense; but then also
the signs ``$p \lor p$'', ``$p \DotOp p$'', etc.\ which have the
same sense as ``$p$'' have no sense. If, however,
``$p \lor p$'' has no sense, then also ``$p \lor q$'' can have
no sense.}


\PropositionE{5.5151}
{Must the sign of the negative proposition be
constructed by means of the sign of the positive?
Why should one not be able to express the
negative proposition by means of a negative fact?
(Like: if ``$a$'' does not stand in a certain relation
to ``$b$'', it could express that $aRb$ is not the case.)

{\stretchyspace
But here also the negative proposition is indirectly
constructed with the positive.}

The positive \emph{proposition} must presuppose the
existence of the negative \emph{proposition} and conversely.}


\PropositionE{5.52}
{If the values of $\xi$ are the total values of a function
$fx$ for all values of $x$, then $N(\overline{\xi}) = \Not{(\exists x) \DotOp fx}$.}


\PropositionE{5.521}
{I separate the concept \emph{all} from the truth-function.

Frege and Russell have introduced generality
in connexion with the logical product or the logical
% -----File: 137.png---
sum. Then it would be difficult to understand
the propositions ``$(\exists x) \DotOp fx$'' and ``$(x) \DotOp fx$'' in which
both ideas lie concealed.}


\PropositionE{5.522}
{That which is peculiar to the ``symbolism of
generality'' is firstly, that it refers to a logical
prototype, and secondly, that it makes constants
prominent.}


\PropositionE{5.523}
{The generality symbol occurs as an argument.}


\PropositionE{5.524}
{If the objects are given, therewith are \emph{all} objects
also given.

If the elementary propositions are given, then
therewith \emph{all} elementary propositions are also
given.}


\PropositionE{5.525}
{It is not correct to render the proposition
\enlargethispage{9pt} % enlarge to make the last line fit
``$(\exists x) \DotOp fx$''---as Russell does---in words ``$fx$ is
\emph{possible}''.

Certainty, possibility or impossibility of a state
of affairs are not expressed by a proposition but
by the fact that an expression is a tautology, a
significant proposition or a contradiction.

That precedent to which one would always
appeal, must be present in the symbol itself.}


\PropositionE{5.526}
{One can describe the world completely by
completely generalized propositions, \idEst\ without
from the outset co-ordinating any name with a
definite object.

In order then to arrive at the customary way
of expression we need simply say after an expression
``there is one and only one $x$, which $\fourdots$'':
and this $x$ is $a$.}


\PropositionE{5.5261}
{A completely generalized proposition is like
every other proposition composite. (This is shown
by the fact that in ``$(\exists x, \phi) \DotOp \phi x$'' we must mention
``$\phi$'' and ``$x$'' separately. Both stand independently
in signifying relations to the world
as in the ungeneralized proposition.)
% -----File: 139.png---

A characteristic of a composite symbol: it has
something in common with \emph{other} symbols.}


\PropositionE{5.5262}
{The truth or falsehood of \emph{every} proposition alters
something in the general structure of the world.
And the range which is allowed to its structure by
the totality of elementary propositions is exactly
that which the completely general propositions
delimit.

(If an elementary proposition is true, then, at
any rate, there is one \emph{more} elementary proposition
true.)}


\PropositionE{5.53}
{Identity of the object I express by identity of
the sign and not by means of a sign of identity.
Difference of the objects by difference of the
signs.}


\PropositionE{5.5301}
{That identity is not a relation between objects is
obvious. This becomes very clear if, for example,
one considers the proposition ``$(x) : fx \DotOp \Implies \DotOp x = a$''.
What this proposition says is simply that \emph{only}
$a$ satisfies the function $f$, and not that only such
things satisfy the function $f$ which have a certain
relation to $a$.

One could of course say that in fact \emph{only}
$a$ has this relation to $a$, but in order to express
this we should need the sign of identity itself.}


\PropositionE{5.5302}
{Russell's definition of ``='' won't do; because
according to it one cannot say that two objects
have all their properties in common. (Even if
this proposition is never true, it is nevertheless
\emph{significant}.)}


\PropositionE{5.5303}
{Roughly speaking: to say of \emph{two} things that
they are identical is nonsense, and to say of \emph{one}
thing that it is identical with itself is to say
nothing.}


\PropositionE{5.531}
{I write therefore not ``$f(a,b) \DotOp a = b$'', but ``$f(a,a)$''
(or ``$f(b,b)$''). And not ``$f(a,b) \DotOp \Not{a} = b$'', but
``$f(a,b)$''.}


\PropositionE{5.532}
{And analogously: not ``$(\exists x,y) \DotOp f(x,y) \DotOp x = y$'',
% -----File: 141.png---
but ``$(\exists x) \DotOp f(x,x)$''; and not ``$(\exists x,y) \DotOp f(x,y) \DotOp
\Not{x} = y$'', but ``$(\exists x,y) \DotOp f(x,y)$''.

(Therefore instead of Russell's ``$(\exists x,y) \DotOp f(x,y)$'':
``$(\exists x,y) \DotOp f(x,y) \DotOp \lor \DotOp (\exists x) \DotOp f(x,x)$''.)}


\PropositionE{5.5321}
{Instead of ``$(x) : fx \Implies x = a$'' we therefore write
\exempliGratia\ ``$(\exists x) \DotOp fx \DotOp \Implies \DotOp fa : \Not{(\exists x,y) \DotOp fx \DotOp fy}$''.

And the proposition ``\emph{only} one $x$ satisfies $f()$''
reads: ``$(\exists x) \DotOp fx : \Not{(\exists x,y) \DotOp fx \DotOp fy}$''.}


\PropositionE{5.533}
{The identity sign is therefore not an essential
constituent of logical notation.}


\PropositionE{5.534}
{And we see that apparent propositions like:
``$a = a$'', ``$a = b \DotOp b = c \DotOp \Implies a = c$'', ``$(x) \DotOp x = x$'', ``$(\exists x) \DotOp
x = a$'', etc.\ cannot be written in a correct logical
notation at all.}


\PropositionE{5.535}
{So all problems disappear which are connected
with such pseu\-do-prop\-o\-si\-tions.

This is the place to solve all the problems which
arise through Russell's ``Axiom of Infinity''.

What the axiom of infinity is meant to say
would be expressed in language by the fact that
there is an infinite number of names with different
meanings.}


\PropositionE{5.5351}
{There are certain cases in which one is tempted
to use expressions of the form ``$a = a$'' or ``$p \Implies p$''
and of that kind. And indeed this takes place
when one would like to speak of the archetype
Proposition, Thing, etc. So Russell in the \BookTitle{Principles
of Mathematics} has rendered the nonsense ``$p$
is a proposition'' in symbols by ``$p \Implies p$'' and has
put it as hypothesis before certain propositions to
show that their places for arguments could only
be occupied by propositions.

(It is nonsense to place the hypothesis $p \Implies p$
before a proposition in order to ensure that its
arguments have the right form, because the
hypothesis for a non-proposition as argument
becomes not false but meaningless, and because
the proposition itself becomes senseless for arguments
% -----File: 143.png---
of the wrong kind, and therefore it survives
the wrong arguments no better and no worse
than the senseless hypothesis attached for this
purpose.)}


\PropositionE{5.5352}
{Similarly it was proposed to express ``There are
no things'' by ``$\Not{(\exists x) \DotOp x = x}$''. But even if this
were a proposition---would it not be true if indeed
``There were things'', but these were not identical
with themselves?}


\PropositionE{5.54}
{In the general propositional form, propositions
occur in a proposition only as bases of the truth-operations.}


\PropositionE{5.541}
{At first sight it appears as if there were also a
different way in which one proposition could occur
in another.

Especially in certain propositional forms of
psychology, like ``A thinks, that $p$ is the case'',
or ``A thinks $p$'', etc.

Here it appears superficially as if the proposition
$p$ stood to the object A in a kind of relation.

(And in modern \DPtypo{epistomology}{epistemology} (Russell, Moore,
etc.) those propositions have been conceived in
this way.)}


\PropositionE{5.542}
{But it is clear that ``A believes that $p$'', ``A
thinks $p$'', ``A says $p$'', are of the form ```$p$' says
$p$'': and here we have no co-ordination of a fact
and an object, but a co-ordination of facts by
means of a co-ordination of their objects.}


\PropositionE{5.5421}
{This shows that there is no such thing as the
soul---the subject, etc.---as it is conceived in contemporary
superficial psychology.

A composite soul would not be a soul any
longer.}


\PropositionE{5.5422}
{The correct explanation of the form of the
proposition ``A judges $p$'' must show that it is
impossible to judge a nonsense. (Russell's theory
does not satisfy this condition.)}


\PropositionE{5.5423}
{To perceive a complex means to perceive that
% -----File: 145.png---
its constituents are combined in such and such a
way.

This perhaps explains that the figure
\Illustration{cube}
can be seen in two ways as a cube; and all similar
phenomena. For we really see two different facts.

(If I fix my eyes first on the corners $a$ and only
glance at $b$, $a$ appears in front and $b$ behind, and
vice versa.)}


\PropositionE{5.55}
{We must now answer a priori the question
as to all possible forms of the elementary propositions.

The elementary proposition consists of names.
Since we cannot give the number of names with
different meanings, we cannot give the composition
of the elementary proposition.}


\PropositionE{5.551}
{Our fundamental principle is that every question
which can be decided at all by logic can be decided
without further trouble.

(And if we get into a situation where we need
to answer such a problem by looking at the world,
this shows that we are on a fundamentally wrong
track.)}


\PropositionE{5.552}
{The ``experience'' which we need to understand
logic is not that such and such is the
case, but that something \emph{is}; but that is \emph{no} experience.

Logic \emph{precedes} every experience---that something
is \emph{so}.

It is before the How, not before the What.}
% -----File: 147.png---


\PropositionE{5.5521}
{And if this were not the case, how could
we apply logic? We could say: if there
were a logic, even if there were no world,
how then could there be a logic, since there is a
world?}


\PropositionE{5.553}
{Russell said that there were simple relations
between different numbers of things (individuals).
But between what numbers? And how should
this be decided---by experience?

(There is no pre-eminent number.)}


\PropositionE{5.554}
{The enumeration of any special forms would
be entirely arbitrary.}


\PropositionE{5.5541}
{It should be possible to decide a priori whether,
for example, I can get into a situation in which
I need to symbolize with a sign of a 27-termed
relation.}


\PropositionE{5.5542}
{May we then ask this at all? Can we set out
a sign form and not know whether anything can
correspond to it?

Has the question sense: what must \emph{be} in order
that something can be the case?}


\PropositionE{5.555}
{It is clear that we have a concept of the
elementary proposition apart from its special
logical form.

Where, however, we can build symbols
according to a system, there this system is the
logically important thing and not the single
symbols.

And how would it be possible that I should
have to deal with forms in logic which I can
invent: but I must have to deal with that which
makes it possible for me to invent them.}


\PropositionE{5.556}
{There cannot be a hierarchy of the forms of the
elementary propositions. Only that which we
ourselves construct can we foresee.}


\PropositionE{5.5561}
{Empirical reality is limited by the totality of
objects. The boundary appears again in the
totality of elementary propositions.
% -----File: 149.png---

The hierarchies are and must be independent
of reality.}


\PropositionE{5.5562}
{If we know on purely logical grounds, that
there must be elementary propositions, then this
must be known by everyone who understands the
propositions in their unanalysed form.}


\PropositionE{5.5563}
{All propositions of our colloquial language are
actually, just as they are, logically completely in
order. That most simple thing which we ought to
give here is not a simile of truth but the complete
truth itself.

(Our problems are not abstract but perhaps the
most concrete that there are.)}


\PropositionE{5.557}
{The \emph{application} of logic decides what elementary
propositions there are.

What lies in the application logic cannot
anticipate.

It is clear that logic may not collide with its
application.

But logic must have contact with its application.

Therefore logic and its application may not
overlap one another.}


\PropositionE{5.5571}
{If I cannot give elementary propositions a
priori then it must lead to obvious nonsense to
try to give them.}


\PropositionE{5.6}
{\emph{The limits of my language} mean the limits of my
world.}


\PropositionE{5.61}
{Logic fills the world: the limits of the world
are also its limits.

We cannot therefore say in logic: This and
this there is in the world, that there is not.

For that would apparently presuppose that we
exclude certain possibilities, and this cannot be
the case since otherwise logic must get outside
the limits of the world: that is, if it could
consider these limits from the other side
also.
% -----File: 151.png---

What we cannot think, that we cannot think:
we cannot therefore \emph{say} what we cannot
think.}


\PropositionE{5.62}
{This remark provides a key to the question, to
what extent solipsism is a truth.

In fact what solipsism \emph{means}, is quite correct,
only it cannot be \emph{said}, but it shows itself.

That the world is \emph{my} world, shows itself in the
fact that the limits of the language (the language
which only I understand) mean the limits of \emph{my}
world.}


\PropositionE{5.621}
{The world and life are one.}


\PropositionE{5.63}
{I am my world. (The microcosm.)}


\PropositionE{5.631}
{The thinking, presenting subject; there is no
such thing.

If I wrote a book ``The world as I found it'',
I should also have therein to report on my body
and say which members obey my will and which
do not, etc. This then would be a method of
isolating the subject or rather of showing that in
an important sense there is no subject: that is to
say, of it alone in this book mention could \emph{not} be
made.}


\PropositionE{5.632}
{The subject does not belong to the world but
it is a limit of the world.}


\PropositionE{5.633}
{\emph{Where in} the world is a metaphysical subject to
be noted?

You say that this case is altogether like that of
the eye and the field of sight. But you do \emph{not}
really see the eye.

And from nothing \emph{in the field of sight} can it be
concluded that it is seen from an eye.}


\PropositionE{5.6331}
{For the field of sight has not a form like this:
\Illustration{sight-en}
}
% -----File: 153.png---


\PropositionE{5.634}
{This is connected with the fact that no part of
our experience is also a priori.

Everything we see could also be otherwise.

Everything we can describe at all could also be
otherwise.

There is no order of things a priori.}


\PropositionE{5.64}
{Here we see that solipsism strictly carried out
coincides with pure realism. The I in solipsism
shrinks to an extensionless point and there remains
the reality co-ordinated with it.}


\PropositionE{5.641}
{There is therefore really a sense in which in
philosophy we can talk of a non-psy\-cho\-log\-i\-cal I.

The I occurs in philosophy through the fact
that the ``world is my world''.

The philosophical I is not the man, not the
human body or the human soul of which psychology
treats, but the metaphysical subject, the
limit---not a part of the world.}


\PropositionE{6}
{The general form of truth-function is:
$[\overline{p}, \overline{\xi}, N(\overline{\xi})]$.

This is the general form of proposition.}


\PropositionE{6.001}
{This says nothing else than that every proposition
is the result of successive applications
of the operation $N'(\overline{\xi})$ to the elementary propositions.}


\PropositionE{6.002}
{If we are given the general form of the way in
which a proposition is constructed, then thereby
we are also given the general form of the way in
which by an operation out of one proposition
another can be created.}


\PropositionE{6.01}
{The general form of the operation $\Omega'(\overline{\eta})$ is
therefore: $[\overline{\xi}, N(\overline{\xi})]'${}$(\overline{\eta})$ (= [$\overline{\eta}$, $\overline{\xi}$, $N(\overline{\xi})$]).

This is the most general form of transition from
one proposition to another.}
% -----File: 155.png---


\PropositionE{6.02}
{And thus we come to numbers: I define}
\begin{gather*}
x = \Omega^{0}{}' x \text{ Def.\ and}\\
\Omega'\Omega^{\nu}{}'x = \Omega^{\nu+1}{}'x \text{ Def.}
\end{gather*}

According, then, to these symbolic rules we
write the series $x$, $\Omega'x$, $\Omega'\Omega'x$, $\Omega'\Omega'\Omega'x\fivedots$
\[
\text{as: } \Omega^{0}{}'x, \Omega^{0+1}{}'x, \Omega^{0+1+1}{}'x, \Omega^{0+1+1+1}{}'x\fivedots
\]

Therefore I write in place of ``$[x, \xi, \Omega'\xi]$'',
\[
``[\Omega^{0}{}'x, \Omega^{\nu}{}'x, \Omega^{\nu+1}{}'x]\text{''.}
\]

And I define:
\[
\begin{array}{l}\\
0 + 1 = 1\text{ Def.}\\
0 + 1 + 1 = 2\text{ Def.}\\
0 + 1 + 1 + 1 = 3\text{ Def.}\\
\text{and so on.}
\end{array}
\]

\PropositionE{6.021}
{A number is the exponent of an operation.}


\PropositionE{6.022}
{The concept number is nothing else than that
which is common to all numbers, the general form
of number.

The concept number is the variable number.

And the concept of equality of numbers is the
general form of all special equalities of numbers.}


\PropositionE{6.03}
{The general form of the cardinal number is:
$[0, \xi, \xi + 1]$.}


\PropositionE{6.031}
{The theory of classes is altogether superfluous
in mathematics.

This is connected with the fact that the generality
which we need in mathematics is not the
\emph{accidental} one.}


\PropositionE{6.1}
{The propositions of logic are tautologies.}


\PropositionE{6.11}
{The propositions of logic therefore say nothing.
(They are the analytical propositions.)}


\PropositionE{6.111}
{Theories which make a proposition of logic
appear substantial are always false. One could
\exempliGratia\ believe that the words ``true'' and ``false''
signify two properties among other properties,
and then it would appear as a remarkable fact
% -----File: 157.png---
that every proposition possesses one of these
properties. This now by no means appears self-evident,
no more so than the proposition ``All
roses are either yellow or red'' would sound even
if it were true. Indeed our proposition now gets
quite the character of a proposition of natural
science and this is a certain symptom of its being
falsely understood.}


\PropositionE{6.112}
{The correct explanation of logical propositions
must give them a peculiar position among all
propositions.}


\PropositionE{6.113}
{It is the characteristic mark of logical propositions
that one can perceive in the symbol alone
that they are true; and this fact contains in itself
the whole philosophy of logic. And so also it is
one of the most important facts that the truth or
falsehood of non-logical propositions can \emph{not} be
recognized from the propositions alone.}


\PropositionE{6.12}
{The fact that the propositions of logic are
tautologies \emph{shows} the for\-mal---log\-i\-cal---prop\-er\-ties
of language, of the world.

That its constituent parts connected together \emph{in
this way} give a tautology characterizes the logic of
its constituent parts.

In order that propositions connected together
in a definite way may give a tautology they
must have definite properties of structure. That
they give a tautology when \emph{so} connected shows
therefore that they possess these properties of
structure.}


\PropositionE{6.1201}
{That \exempliGratia\ the propositions ``$p$'' and ``$\Not{p}$'' in
the connexion ``$\Not{(p \DotOp \Not{p})}$'' give a tautology
shows that they contradict one another. That the
propositions ``$p \Implies q$'', ``$p$'' and ``$q$'' connected
together in the form ``$(p \Implies q) \DotOp (p) : \Implies : (q)$'' give a
tautology shows that $q$ follows from $p$ and $p \Implies q$.
% -----File: 159.png---
That ``$(x) \DotOp fx : \Implies : fa$'' is a tautology shows that
$fa$ follows from $(x) \DotOp fx$, etc.\ etc.}


\PropositionE{6.1202}
{It is clear that we could have used for this
purpose contradictions instead of tautologies.}


\PropositionE{6.1203}
{In order to recognize a tautology as such, we
can, in cases in which no sign of generality occurs
in the tautology, make use of the following intuitive
method: I write instead of ``$p$'', ``$q$'', ``$r$'', etc.,
``$TpF$'', ``$TqF$'', ``$TrF$'', etc. The truth-combinations
I express by brackets, \exempliGratia:
\Illustration[0.35\textwidth]{brackets01-en}
and the co-ordination of the truth or falsity of the
whole proposition with the truth-combinations of
the truth-arguments by lines in the following way:
\Illustration[0.4\textwidth]{brackets02-en}

This sign, for example, would therefore present
the proposition $p \Implies q$. Now I will proceed
to inquire whether such a proposition as $\Not{(p \DotOp \Not{p})}$
(The Law of Contradiction) is a tautology. The
form ``$\Not{\xi}$'' is written in our notation
\Illustration[0.1\textwidth]{brackets03-en}
% -----File: 161.png---
the form ``$\xi \DotOp \eta$'' thus:---
\enlargethispage{-29pt} % force the next sentence to the next page
\Illustration[0.4\textwidth]{brackets04-en}

Hence the proposition $\Not{(p \DotOp \Not{q})}$ runs thus:---
\Illustration[0.3\textwidth]{brackets05-en}

If here we put ``$p$'' instead of ``$q$'' and examine
the combination of the outermost T and F with the
innermost, it is seen that the truth of the whole
proposition is co-ordinated with \emph{all} the truth-combinations
of its argument, its falsity with none of
the truth-combinations.}


\PropositionE{6.121}
{The propositions of logic demonstrate the logical
properties of propositions, by combining them into
propositions which say nothing.

This method could be called a zero-method. In
a logical proposition propositions are brought into
equilibrium with one another, and the state of
equilibrium then shows how these propositions
must be logically constructed.}


\PropositionE{6.122}
{Whence it follows that we can get on without
logical propositions, for we can recognize in an
adequate notation the formal properties of the propositions
by mere inspection.}
% -----File: 163.png---


\PropositionE{6.1221}
{If for example two propositions ``$p$'' and ``$q$''
give a tautology in the connexion ``$p \Implies q$'', then
it is clear that $q$ follows from $p$.

\ExempliGratia\ that ``$q$'' follows from ``$p \Implies q \DotOp p$'' we see from
these two propositions themselves, but we can also
show it by combining them to ``$p \Implies q \DotOp p : \Implies : q$'' and
then showing that this is a tautology.}


\PropositionE{6.1222}
{This throws light on the question why logical
propositions can no more be empirically established
than they can be empirically refuted. Not only
must a proposition of logic be incapable of being
contradicted by any possible experience, but it
must also be incapable of being established by any
such.}


\PropositionE{6.1223}
{It now becomes clear why we often feel as though
``logical truths'' must be ``\emph{postulated}'' by us. We
can in fact postulate them in so far as we can
postulate an adequate notation.}


\PropositionE{6.1224}
{It also becomes clear why logic has been called
the theory of forms and of inference.}


\PropositionE{6.123}
{It is clear that the laws of logic cannot themselves
obey further logical laws.

(There is not, as Russell supposed, for every
``type'' a special law of contradiction; but one is
sufficient, since it is not applied to itself.)}


\PropositionE{6.1231}
{The mark of logical propositions is not their
general validity.

To be general is only to be accidentally valid
for all things. An ungeneralized proposition can
be tautologous just as well as a generalized
one.}


\PropositionE{6.1232}
{Logical general validity, we could call essential
as opposed to accidental general validity, \exempliGratia\ of the
proposition ``all men are mortal''. Propositions
like Russell's ``axiom of reducibility'' are not
% -----File: 165.png---
logical propositions, and this explains our feeling
that, if true, they can only be true by a happy
chance.}


\PropositionE{6.1233}
{We can imagine a world in which the axiom of
reducibility is not valid. But it is clear that logic
has nothing to do with the question whether our
world is really of this kind or not.}


\PropositionE{6.124}
{The logical propositions describe the scaffolding
of the world, or rather they present it. They
``treat'' of nothing. They presuppose that names
have meaning, and that elementary propositions
have sense. And this is their connexion with the
world. It is clear that it must show something
about the world that certain combinations of symbols---which
essentially have a definite character---are
tautologies. Herein lies the decisive point. We
said that in the symbols which we use much is
arbitrary, much not. In logic only this expresses:
but this means that in logic it is not \emph{we} who express,
by means of signs, what we want, but in logic the
nature of the essentially necessary signs itself
asserts. That is to say, if we know the logical
syntax of any sign language, then all the propositions
of logic are already given.}


\PropositionE{6.125}
{It is possible, even in the old logic, to give
at the outset a description of all ``true'' logical
propositions.}


\PropositionE{6.1251}
{Hence there can \emph{never} be surprises in logic.}


\PropositionE{6.126}
{Whether a proposition belongs to logic can be
determined by determining the logical properties
of the \emph{symbol}.

And this we do when we prove a logical proposition.
For without troubling ourselves about
a sense and a meaning, we form the logical
propositions out of others by mere \emph{symbolic
rules}.
% -----File: 167.png---

We prove a logical proposition by creating it
out of other logical propositions by applying in
succession certain operations, which again generate
tautologies out of the first. (And from a tautology
only tautologies \emph{follow}.)

Naturally this way of showing that its propositions
are tautologies is quite unessential to
logic. Because the propositions, from which the
proof starts, must show without proof that they
are tautologies.}


\PropositionE{6.1261}
{In logic process and result are equivalent.
(Therefore no surprises.)}


\PropositionE{6.1262}
{Proof in logic is only a mechanical expedient
to facilitate the recognition of tautology, where
it is complicated.}


\PropositionE{6.1263}
{It would be too remarkable, if one could prove
a significant proposition \emph{logically} from another, and
a logical proposition \emph{also}. It is clear from the
beginning that the logical proof of a significant
proposition and the proof \emph{in} logic must be two
quite different things.}


\PropositionE{6.1264}
{{\stretchyspace
The significant proposition asserts something,
and its proof shows that it is so; in logic every
proposition is the form of a proof.}

Every proposition of logic is a modus ponens
presented in signs. (And the modus ponens can
not be expressed by a proposition.)}


\PropositionE{6.1265}
{Logic can always be conceived to be such that
every proposition is its own proof.}


\PropositionE{6.127}
{All propositions of logic are of equal rank;
there are not some which are essentially primitive
and others deduced from these.

Every tautology itself shows that it is a
tautology.}


\PropositionE{6.1271}
{It is clear that the number of ``primitive propositions
of logic'' is arbitrary, for we could deduce
% -----File: 169.png---
logic from one primitive proposition by simply
forming, for example, the logical product of Frege's
primitive propositions. (Frege would perhaps say
that this would no longer be immediately self-evident.
But it is remarkable that so exact a
thinker as Frege should have appealed to the
degree of self-evidence as the criterion of a
logical proposition.)}


\PropositionE{6.13}
{Logic is not a theory but a reflexion of the
world.

Logic is transcendental.}


\PropositionE{6.2}
{Mathematics is a logical method.

The propositions of mathematics are equations,
and therefore pseudo-prop\-o\-si\-tions.}


\PropositionE{6.21}
{Mathematical propositions express no thoughts.}


\PropositionE{6.211}
{In life it is never a mathematical proposition
which we need, but we use mathematical propositions
\emph{only} in order to infer from propositions
which do not belong to mathematics to others
which equally do not belong to mathematics.

(In philosophy the question ``Why do we really
use that word, that proposition?'' constantly leads
to valuable results.)}


\PropositionE{6.22}
{The logic of the world which the propositions
of logic show in tautologies, mathematics shows
in equations.}


\PropositionE{6.23}
{{\stretchyspace
If two expressions are connected by the sign of
equality, this means that they can be substituted
for one another. But whether this is the case
must show itself in the two expressions themselves.}

It characterizes the logical form of two expressions,
that they can be substituted for one
another.}


\PropositionE{6.231}
{It is a property of affirmation that it can be
conceived as double denial.

It is a property of ``$1 + 1 + 1 + 1$'' that it can be
conceived as ``$(1 + 1) + (1 + 1)$''.}
% -----File: 171.png---


\PropositionE{6.232}
{Frege says that these expressions have the same
meaning but different senses.

But what is essential about equation is that it
is not necessary in order to show that both expressions,
which are connected by the sign of
equality, have the same meaning: for this can be
perceived from the two expressions themselves.}


\PropositionE{6.2321}
{And, that the propositions of mathematics can
be proved means nothing else than that their
correctness can be seen without our having to
compare what they express with the facts as regards
correctness.}


\PropositionE{6.2322}
{The identity of the meaning of two expressions
cannot be \emph{asserted}. For in order to be able to
assert anything about their meaning, I must know
their meaning, and if I know their meaning, I
know whether they mean the same or something
different.}


\PropositionE{6.2323}
{The equation characterizes only the standpoint
from which I consider the two expressions, that
is to say the standpoint of their equality of
meaning.}


\PropositionE{6.233}
{To the question whether we need intuition for
the solution of mathematical problems it must be
answered that language itself here supplies the
necessary intuition.}


\PropositionE{6.2331}
{The process of calculation brings about just
this intuition.

Calculation is not an experiment.}


\PropositionE{6.234}
{Mathematics is a method of logic.}


\PropositionE{6.2341}
{The essential of mathematical method is working
with equations. On this method depends the
fact that every proposition of mathematics must
be self-intelligible.}


\PropositionE{6.24}
{The method by which mathematics arrives at
its equations is the method of substitution.

For equations express the substitutability of
two expressions, and we proceed from a number
% -----File: 173.png---
of equations to new equations, replacing expressions
by others in accordance with the
equations.}


\PropositionE{6.241}
{Thus the proof of the proposition $2 \times 2 = 4$ runs:
\begin{gather*}
(\Omega^{\nu})^{\mu}{}'x = \Omega^{\nu \times \mu}{}'x \text{ Def.}\\
\begin{split}
\Omega^{2 \times 2}{}'x = (\Omega^{2})^{2}{}'x = (\Omega^{2})^{1 + 1}{}'x = \Omega^{2}{}'\Omega^{2}{}'x = \Omega^{1 + 1}{}'\Omega^{1 + 1}{}'x\\
= (\Omega'\Omega)'(\Omega'\Omega)'x = \Omega'\Omega'\Omega'\Omega'x = \Omega^{1 + 1 + 1 + 1}{}'x = \Omega^{4}{}'x.
\end{split}
\end{gather*}}


\PropositionE{6.3}
{Logical research means the investigation of \emph{all
\enlargethispage{2pt} % enlarge to make the last line fit
regularity}. And outside logic all is accident.}


\PropositionE{6.31}
{The so-called law of induction cannot in any
case be a logical law, for it is obviously a significant
proposition.---And therefore it cannot be
a law a priori either.}


\PropositionE{6.32}
{The law of causality is not a law but the form
of a law.\footnote{\IdEst\ not the form of one particular law, but of any law of a certain
sort (B.\;R.).}}


\PropositionE{6.321}
{``Law of Causality'' is a class name. And as
in mechanics there are, for instance, minimum-laws,
such as that of least action, so in physics
there are causal laws, laws of the causality
form.}


\PropositionE{6.3211}
{Men had indeed an idea that there must be
\emph{a} ``law of least action'', before they knew
exactly how it ran. (Here, as always, the a
priori certain proves to be something purely
logical.)}


\PropositionE{6.33}
{We do not \emph{believe} a priori in a law of conservation,
but we \emph{know} a priori the possibility of
a logical form.}


\PropositionE{6.34}
{All propositions, such as the law of causation,
the law of continuity in nature, the law of least
expenditure in nature, etc.\ etc., all these are
a priori intuitions of possible forms of the propositions
of science.}


\PropositionE{6.341}
{Newtonian mechanics, for example, brings the
description of the universe to a unified form.
% -----File: 175.png---
Let us imagine a white surface with irregular
black spots. We now say: Whatever kind of
picture these make I can always get as near as
I like to its description, if I cover the surface
with a sufficiently fine square network and now
say of every square that it is white or black.
In this way I shall have brought the description
of the surface to a unified form. This form is
arbitrary, because I could have applied with equal
success a net with a triangular or hexagonal
mesh. It can happen that the description would
have been simpler with the aid of a triangular
mesh; that is to say we might have described
the surface more accurately with a triangular,
and coarser, than with the finer square mesh, or
vice versa, and so on. To the different networks
correspond different systems of describing the
world. Mechanics determine a form of description
by saying: All propositions in the description
of the world must be obtained in a given
way from a number of given propositions---the
mechanical axioms. It thus provides the bricks
for building the edifice of science, and says:
Whatever building thou wouldst erect, thou shalt
construct it in some manner with these bricks
and these alone.

(As with the system of numbers one must be
able to write down any arbitrary number, so
with the system of mechanics one must be able
to write down any arbitrary physical proposition.)}


\PropositionE{6.342}
{And now we see the relative position of logic
and mechanics. (We could construct the network
out of figures of different kinds, as out of
triangles and hexagons together.) That a picture
like that instanced above can be described by a
network of a given form asserts \emph{nothing} about
% -----File: 177.png---
the picture. (For this holds of every picture of
this kind.) But \emph{this} does characterize the picture,
the fact, namely, that it can be \emph{completely} described
by a definite net of \emph{definite} fineness.

So too the fact that it can be described by
Newtonian mechanics asserts nothing about the
world; but \emph{this} asserts something, namely, that
it can be described in that particular way in which
it is described, as is indeed the case. The fact,
too, that it can be described more simply by one
system of mechanics than by another says something
about the world.}


\PropositionE{6.343}
{Mechanics is an attempt to construct according
to a single plan all \emph{true} propositions which we
need for the description of the world.}


\PropositionE{6.3431}
{Through the whole apparatus of logic the
physical laws still speak of the objects of the
world.}


\PropositionE{6.3432}
{We must not forget that the description of the
world by mechanics is always quite general.
There is, for example, never any mention of
\emph{particular} material points in it, but always only
of \emph{some points or other}.}


\PropositionE{6.35}
{Although the spots in our picture are geometrical
figures, geometry can obviously say nothing
about their actual form and position. But the
network is \emph{purely} geometrical, and all its properties
can be given a priori.

Laws, like the law of causation, etc., treat
of the network and not of what the network
described.}


\PropositionE{6.36}
{If there were a law of causality, it might run:
``There are natural laws''.

But that can clearly not be said: it shows
itself.}


\PropositionE{6.361}
{In the terminology of Hertz we might say:
Only \emph{uniform} connexions are \emph{thinkable}.}
% -----File: 179.png---


\PropositionE{6.3611}
{We cannot compare any process with the
``passage of time''---there is no such thing---but
only with another process (say, with the movement
of the chronometer).

Hence the description of the temporal sequence
of events is only possible if we support ourselves
on another process.

It is exactly analogous for space. When, for
example, we say that neither of two events (which
mutually exclude one another) can occur, because
there is \emph{no cause} why the one should occur rather
than the other, it is really a matter of our being
unable to describe \emph{one} of the two events unless
there is some sort of asymmetry. And if there \emph{is}
such an asymmetry, we can regard this as the
\emph{cause} of the occurrence of the one and of the non-occurrence
of the other.}


\PropositionE{6.36111}
{The Kantian problem of the right and left hand
which cannot be made to cover one another already
exists in the plane, and even in one-di\-men\-sio\-nal
space; where the two congruent figures $a$ and $b$
cannot be made to cover one another without
moving them out of this space. The right and
left hand are in fact completely congruent. And
the fact that they cannot be made to cover one
another has nothing to do with it.
\Illustration[0.45\textwidth]{space}

A right-hand glove could be put on a left hand
if it could be turned round in four-dimensional
space.}


\PropositionE{6.362}
{What can be described can happen too, and
what is excluded by the law of causality cannot be
described.}


\PropositionE{6.363}
{The process of induction is the process of
% -----File: 181.png---
assuming the \emph{simplest} law that can be made to
harmonize with our experience.}


\PropositionE{6.3631}
{This process, however, has no logical foundation
but only a psychological one.

It is clear that there are no grounds for believing
that the simplest course of events will really
happen.}


\PropositionE{6.36311}
{That the sun will rise to-morrow, is an hypothesis;
and that means that we do not \emph{know} whether
it will rise.}


\PropositionE{6.37}
{A necessity for one thing to happen because
another has happened does not exist. There is
only \emph{logical} necessity.}


\PropositionE{6.371}
{At the basis of the whole modern view of
the world lies the illusion that the so-called
laws of nature are the explanations of natural
phenomena.}


\PropositionE{6.372}
{So people stop short at natural laws as at something
unassailable, as did the ancients at God
and Fate.

And they both are right and wrong. But the
ancients were clearer, in so far as they recognized
one clear conclusion, whereas in the modern
system it should appear as though \emph{everything} were
explained.}


\PropositionE{6.373}
{The world is independent of my will.}


\PropositionE{6.374}
{Even if everything we wished were to happen,
this would only be, so to speak, a favour of
fate, for there is no \emph{logical} connexion between will
and world, which would guarantee this, and the
assumed physical connexion itself we could not
again will.}


\PropositionE{6.375}
{As there is only a \emph{logical} necessity, so there is
only a \emph{logical} impossibility.}


\PropositionE{6.3751}
{For two colours, \exempliGratia\ to be at one place in the
visual field, is impossible, logically impossible,
for it is excluded by the logical structure of
colour.
% -----File: 183.png---

Let us consider how this contradiction presents
itself in physics. Somewhat as follows: That a
particle cannot at the same time have two velocities,
\idEst\ that at the same time it cannot be in
two places, \idEst\ that particles in different places
at the same time cannot be identical.

(It is clear that the logical product of two
elementary propositions can neither be a tautology
nor a contradiction. The assertion that a point
in the visual field has two different colours at the
same time, is a contradiction.)}


\PropositionE{6.4}
{All propositions are of equal value.}


\PropositionE{6.41}
{The sense of the world must lie outside the
world. In the world everything is as it is and
happens as it does happen. \emph{In} it there is no value---and
\enlargethispage{11pt} % enlarge to make the last line fit
if there were, it would be of no value.

If there is a value which is of value, it must
lie outside all happening and being-so. For all
happening and being-so is accidental.

What makes it non-accidental cannot lie \emph{in}
the world, for otherwise this would again be accidental.

It must lie outside the world.}


\PropositionE{6.42}
{Hence also there can be no ethical propositions.

Propositions cannot express anything higher.}


\PropositionE{6.421}
{It is clear that ethics cannot be expressed.

Ethics are transcendental.

(Ethics and æsthetics are one.)}


\PropositionE{6.422}
{The first thought in setting up an ethical law
of the form ``thou shalt\;\ldots'' is: And what
if I do not do it. But it is clear that ethics has
nothing to do with punishment and reward in the
ordinary sense. This question as to the \emph{consequences}
of an action must therefore be irrelevant.
At least these consequences will not be events.
For there must be something right in that formulation
of the question. There must be some sort
% -----File: 185.png---
of ethical reward and ethical punishment, but this
must lie in the action itself.

(And this is clear also that the reward must be
something acceptable, and the punishment something
unacceptable.)}


\PropositionE{6.423}
{Of the will as the bearer of the ethical we cannot
speak.

And the will as a phenomenon is only of interest
to psychology.}


\PropositionE{6.43}
{If good or bad willing changes the world, it
can only change the limits of the world, not the
facts; not the things that can be expressed in
language.

In brief, the world must thereby become quite
another. It must so to speak wax or wane as a
whole.

The world of the happy is quite another than
that of the unhappy.}


\PropositionE{6.431}
{As in death, too, the world does not change,
but ceases.}


\PropositionE{6.4311}
{Death is not an event of life. Death is not lived
through.

If by eternity is understood not endless temporal
duration but timelessness, then he lives eternally
who lives in the present.

Our life is endless in the way that our visual
field is without limit.}


\PropositionE{6.4312}
{The temporal immortality of the soul of man,
that is to say, its eternal survival also after
death, is not only in no way guaranteed, but
this assumption in the first place will not do
for us what we always tried to make it do.
Is a riddle solved by the fact that I survive for
ever? Is this eternal life not as enigmatic as
our present one? The solution of the riddle of
life in space and time lies \emph{outside} space and
time.
% -----File: 187.png---

{\stretchyspace
(It is not problems of natural science which have
to be solved.)}}


\PropositionE{6.432}
{\emph{How} the world is, is completely indifferent for
what is higher. God does not reveal himself \emph{in} the
world.}


\PropositionE{6.4321}
{The facts all belong only to the task and not to
its performance.}


\PropositionE{6.44}
{Not \emph{how} the world is, is the mystical, but \emph{that}
it is.}


\PropositionE{6.45}
{The contemplation of the world sub specie aeterni
is its contemplation as a limited whole.

The feeling of the world as a limited whole is
the mystical feeling.}


\PropositionE{6.5}
{For an answer which cannot be expressed the
question too cannot be expressed.

\emph{The riddle} does not exist.

If a question can be put at all, then it \emph{can} also
be answered.}


\PropositionE{6.51}
{Scepticism is \emph{not} irrefutable, but palpably senseless,
if it would doubt where a question cannot be
asked.

For doubt can only exist where there is a
question; a question only where there is an answer,
and this only where something \emph{can} be \emph{said}.}


\PropositionE{6.52}
{We feel that even if \emph{all possible} scientific
questions be answered, the problems of life have
still not been touched at all. Of course there is
then no question left, and just this is the
answer.}


\PropositionE{6.521}
{The solution of the problem of life is seen in the
vanishing of this problem.

(Is not this the reason why men to whom
after long doubting the sense of life became
clear, could not then say wherein this sense
consisted?)}


\PropositionE{6.522}
{There is indeed the inexpressible. This \emph{shows}
itself; it is the mystical.}


\PropositionE{6.53}
{The right method of philosophy would be this.
% -----File: 189.png---
To say nothing except what can be said, \idEst\ the
propositions of natural science, \idEst\ something that
has nothing to do with philosophy: and then
always, when someone else wished to say something
metaphysical, to demonstrate to him that he
had given no meaning to certain signs in his
propositions. This method would be unsatisfying
to the other---he would not have the feeling that
we were teaching him philosophy---but it would be
the only strictly correct method.}


\PropositionE{6.54}
{My propositions are elucidatory in this way:
he who understands me finally recognizes them as
senseless, when he has climbed out through them,
on them, over them. (He must so to speak throw
away the ladder, after he has climbed up on it.)

He must surmount these propositions; then he
sees the world rightly.}


\PropositionE{7}
{Whereof one cannot speak, thereof one must be
silent.}
\end{propositions}
% -----File: 026.png---




\selectlanguage{german}
\Preface{Logisch-Philosophische Abhandlung}{Vorwort}


Dieses Buch wird vielleicht nur der verstehen, der
die Gedanken, die darin ausgedrückt sind---oder doch
ähnliche Gedanken---schon selbst einmal gedacht hat.---Es
ist also kein Lehrbuch.---Sein Zweck wäre erreicht,
wenn es Einem, der es mit Verständnis liest Vergnügen
bereitete.

Das Buch behandelt die philosophischen Probleme und
zeigt---wie ich glau\-be---dass die Fragestellung dieser Probleme
auf dem Missverständnis der Logik unserer Sprache
beruht. Man könnte den ganzen Sinn des Buches etwa in
die Worte fassen: Was sich überhaupt sagen lässt, lässt
sich klar sagen; und wovon man nicht reden kann, darüber
muss man schweigen.

Das Buch will also dem Denken eine Grenze ziehen, oder
viel\-mehr---nicht dem Denken, sondern dem Ausdruck der
Gedanken: Denn um dem Denken eine Grenze zu ziehen,
müssten wir beide Seiten dieser Grenze denken können (wir
müssten also denken können, was sich nicht denken
lässt).

Die Grenze wird also nur in der Sprache gezogen werden
können und was jenseits der Grenze liegt, wird einfach
Unsinn sein.

Wieweit meine Bestrebungen mit denen anderer Philosophen
zusammenfallen, will ich nicht beurteilen. Ja, was
ich hier geschrieben habe macht im Einzelnen überhaupt
nicht den Anspruch auf Neuheit; und darum gebe ich auch
keine Quellen an, weil es mir gleichgültig ist, ob das was
ich gedacht habe, vor mir schon ein anderer gedacht hat.
% -----File: 028.png---

Nur das will ich erwähnen, dass ich den grossartigen
Werken Freges und den Arbeiten meines Freundes Herrn
Bertrand Russell einen grossen Teil der Anregung zu
meinen Gedanken schulde.

Wenn diese Arbeit einen Wert hat, so besteht er in
Zweierlei. Erstens darin, dass in ihr Gedanken ausgedrückt
sind, und dieser Wert wird umso grösser sein, je
besser die Gedanken ausgedrückt sind. Je mehr der
Nagel auf den Kopf getroffen ist.---Hier bin ich mir
bewusst, weit hinter dem Möglichen zurückgeblieben zu
sein. Einfach darum, weil meine Kraft zur Bewältigung
der Aufgabe zu gering ist.---Mögen andere kommen und
es besser machen.

Dagegen scheint mir die \Emph{Wahrheit} der hier mitgeteilten
Gedanken unantastbar und definitiv. Ich bin
also der Meinung, die Probleme im Wesentlichen
endgültig gelöst zu haben. Und wenn ich mich hierin
nicht irre, so besteht nun der Wert dieser Arbeit zweitens
darin, dass sie zeigt, wie wenig damit getan ist, dass
diese Probleme gelöst sind.

\begin{minipage}{0.9\textwidth}
\vspace{3.5ex}
\begin{flushright}
L.~W.
\end{flushright}

\textit{\footnotesize Wien, 1918.}
\end{minipage}
% -----File: 030.png---


\MainMatter{Logisch-philosophische Abhandlung}

\begin{propositions}
\PropositionG{1}
{Die Welt ist alles, was der Fall ist.\footnote{Die Decimalzahlen als Nummern der einzelnen Sätze deuten das logische
Gewicht der Sätze an, den Nachdruck, der auf ihnen in meiner Darstellung liegt\DPtypo{,}{.}
Die Sätze \textit{n}.1, \textit{n}.2, \textit{n}.3, etc., sind Bemerkungen zum \DPtypo{Sätze}{Satze} No.\;\textit{n}; die Sätze \textit{n}.\textit{m}1,
\textit{n}.\textit{m}2, etc.\ Bemerkungen zum Satze No.\;\textit{n}.\textit{m}; und so weiter.}}


\PropositionG{1.1}
{Die Welt ist die Gesamtheit der Tatsachen,
nicht der Dinge.}


\PropositionG{1.11}
{Die Welt ist durch die Tatsachen bestimmt und
dadurch, dass es \Emph{alle} Tatsachen sind.}


\PropositionG{1.12}
{Denn, die Gesamtheit der Tatsachen bestimmt,
was der Fall ist und auch, was alles nicht der Fall ist.}


\PropositionG{1.13}
{Die Tatsachen im logischen Raum sind die Welt.}


\PropositionG{1.2}
{Die Welt zerfällt in Tatsachen.}


\PropositionG{1.21}
{Eines kann der Fall sein oder nicht der Fall sein
und alles übrige gleich bleiben.}


\PropositionG{2}
{Was der Fall ist, die Tatsache, ist das Bestehen
von Sachverhalten.}


\PropositionG{2.01}
{Der Sachverhalt ist eine Verbindung von
Gegenständen. (Sachen, Dingen.)}


\PropositionG{2.011}
{Es ist dem Ding wesentlich, der Bestandteil
eines Sachverhaltes sein zu können.}


\PropositionG{2.012}
{In der Logik ist nichts zufällig: Wenn das Ding
im Sachverhalt vorkommen \Emph{kann}, so muss die
Möglichkeit des Sachverhaltes im Ding bereits
präjudiziert sein.}


\PropositionG{2.0121}
{Es erschiene gleichsam als Zufall, wenn dem
Ding, das allein für sich bestehen könnte, nachträglich
eine Sachlage passen würde.

Wenn die Dinge in Sachverhalten vorkommen
können, so muss dies schon in ihnen liegen.

(Etwas Logisches kann nicht nur-möglich sein.
Die Logik handelt von jeder Möglichkeit und alle
Möglichkeiten sind ihre Tatsachen.)
% -----File: 032.png---

Wie wir uns räumliche Gegenstände überhaupt
nicht ausserhalb des Raumes, zeitliche nicht ausserhalb
der Zeit denken können, so können wir uns
\Emph{keinen} Gegenstand ausserhalb der Möglichkeit
seiner Verbindung mit anderen denken.

Wenn ich mir den Gegenstand im Verbande
des Sachverhalts denken kann, so kann ich ihn
nicht ausserhalb der \Emph{Möglichkeit} dieses Verbandes
denken.}


\PropositionG{2.0122}
{Das Ding ist selbständig, insofern es in allen
\Emph{möglichen} Sachlagen vorkommen kann, aber
diese Form der Selbständigkeit ist eine Form des
Zusammenhangs mit dem Sachverhalt, eine Form
der Unselbständigkeit. (Es ist unmöglich, dass
Worte in zwei verschiedenen Weisen auftreten,
allein und im Satz.)}


\PropositionG{2.0123}
{Wenn ich den Gegenstand kenne, so kenne ich
auch sämtliche Möglichkeiten seines Vorkommens
in Sachverhalten.

(Jede solche Möglichkeit muss in der Natur des
Gegenstandes liegen.)

Es kann nicht nachträglich eine neue Möglichkeit
gefunden werden.}


\PropositionG{2.01231}
{Um einen Gegenstand zu kennen, muss ich zwar
nicht seine externen---aber ich muss alle seine
internen Eigenschaften kennen.}


\PropositionG{2.0124}
{Sind alle Gegenstände gegeben, so sind damit
auch alle \Emph{möglichen} Sachverhalte gegeben.}


\PropositionG{2.013}
{Jedes Ding ist, gleichsam, in einem Raume
möglicher Sachverhalte. Diesen Raum kann ich
mir leer denken, nicht aber das Ding ohne den
Raum.}


\PropositionG{2.0131}
{Der räumliche Gegenstand muss im unendlichen
Raume liegen. (Der Raumpunkt ist eine Argumentstelle.)

Der Fleck im Gesichtsfeld muss zwar nicht rot
sein, aber eine Farbe muss er haben: er hat sozusagen
den Farbenraum um sich. Der Ton muss
% -----File: 034.png---
\Emph{eine} Höhe haben, der Gegenstand des Tastsinnes
\Emph{eine} Härte usw.}


\PropositionG{2.014}
{Die Gegenstände enthalten die Möglichkeit aller
Sachlagen.}


\PropositionG{2.0141}
{Die Möglichkeit seines Vorkommens in Sachverhalten,
ist die Form des Gegenstandes.}


\PropositionG{2.02}
{Der Gegenstand ist einfach.}


\PropositionG{2.0201}
{Jede Aussage über Komplexe lässt sich in eine
Aussage über deren Bestandteile und in diejenigen
Sätze zerlegen, welche die Komplexe vollständig
beschreiben.}


\PropositionG{2.021}
{Die Gegenstände bilden die Substanz der Welt.
Darum können sie nicht zusammengesetzt sein.}


\PropositionG{2.0211}
{Hätte die Welt keine Substanz, so würde, ob ein
Satz Sinn hat, davon abhängen, ob ein anderer Satz
wahr ist.}


\PropositionG{2.0212}
{Es wäre dann unmöglich, ein Bild der Welt
(wahr oder falsch) zu entwerfen.}


\PropositionG{2.022}
{Es ist offenbar, dass auch eine von der wirklichen
noch so verschieden gedachte Welt Etwas---eine
Form---mit der wirklichen gemein haben muss.}


\PropositionG{2.023}
{Diese feste Form besteht eben aus den Gegenständen.}


\PropositionG{2.0231}
{Die Substanz der Welt \Emph{kann} nur eine Form und
keine materiellen Eigenschaften bestimmen. Denn
diese werden erst durch die Sätze dar\-ge\-stellt---erst
durch die Konfiguration der Gegenstände gebildet.}


\PropositionG{2.0232}
{Beiläufig gesprochen: Die Gegenstände sind
farblos.}


\PropositionG{2.0233}
{Zwei Gegenstände von der gleichen logischen
Form sind---ab\-ge\-se\-hen von ihren externen Eigenschaften---von
einander nur dadurch unterschieden,
dass sie verschieden sind.}


\PropositionG{2.02331}
{Entweder ein Ding hat Eigenschaften, die kein
anderes hat, dann kann man es ohneweiteres durch
eine Beschreibung aus den anderen herausheben,
und darauf hinweisen; oder aber, es gibt mehrere
Dinge, die ihre sämtlichen Eigenschaften gemeinsam
% -----File: 036.png---
haben, dann ist es überhaupt unmöglich auf
eines von ihnen zu zeigen.

Denn, ist das Ding durch nichts hervorgehoben,
so kann ich es nicht hervorheben, denn sonst ist
es eben hervorgehoben.}


\PropositionG{2.024}
{Die Substanz ist das, was unabhängig von dem
was der Fall ist, besteht.}


\PropositionG{2.025}
{Sie ist Form und Inhalt.}


\PropositionG{2.0251}
{Raum, Zeit und Farbe (Färbigkeit) sind Formen
der Gegenstände.}


\PropositionG{2.026}
{Nur wenn es Gegenstände gibt, kann es eine
feste Form der Welt geben.}


\PropositionG{2.027}
{Das Feste, das Bestehende und der Gegenstand
sind Eins.}


\PropositionG{2.0271}
{Der Gegenstand ist das Feste, Bestehende; die
Konfiguration ist das Wechselnde, Unbeständige.}


\PropositionG{2.0272}
{Die Konfiguration der Gegenstände bildet den
Sachverhalt.}


\PropositionG{2.03}
{Im Sachverhalt hängen die Gegenstände ineinander,
wie die Glieder einer Kette.}


\PropositionG{2.031}
{Im Sachverhalt verhalten sich die Gegenstände
in bestimmter Art und Weise zueinander.}


\PropositionG{2.032}
{Die Art und Weise, wie die Gegenstände im
Sachverhalt zusammenhängen, ist die Struktur
des Sachverhaltes.}


\PropositionG{2.033}
{Die Form ist die Möglichkeit der Struktur.}


\PropositionG{2.034}
{Die Struktur der Tatsache besteht aus den
Strukturen der Sachverhalte.}


\PropositionG{2.04}
{Die Gesamtheit der bestehenden Sachverhalte
ist die Welt.}


\PropositionG{2.05}
{Die Gesamtheit der bestehenden Sachverhalte
bestimmt auch, welche Sachverhalte nicht bestehen.}


\PropositionG{2.06}
{Das Bestehen und Nichtbestehen von Sachverhalten
ist die Wirklichkeit.

(Das Bestehen von Sachverhalten nennen wir
auch eine positive, das Nichtbestehen eine negative
Tatsache.)}


\PropositionG{2.061}
{Die Sachverhalte sind von einander unabhängig.}
% -----File: 038.png---


\PropositionG{2.062}
{Aus dem Bestehen oder Nichtbestehen eines
Sachverhaltes kann nicht auf das Bestehen oder
Nichtbestehen eines anderen geschlossen werden.}


\PropositionG{2.063}
{Die gesamte Wirklichkeit ist die Welt.}


\PropositionG{2.1}
{Wir machen uns Bilder der Tatsachen.}


\PropositionG{2.11}
{Das Bild stellt die Sachlage im logischen Raume,
das Bestehen und Nichtbestehen von Sachverhalten
vor.}


\PropositionG{2.12}
{Das Bild ist ein Modell der Wirklichkeit.}


\PropositionG{2.13}
{Den Gegenständen entsprechen im Bilde die
Elemente des Bildes.}


\PropositionG{2.131}
{Die Elemente des Bildes vertreten im Bild die
Gegenstände.}


\PropositionG{2.14}
{Das Bild besteht darin, dass sich seine Elemente
in bestimmter Art und Weise zu einander verhalten.}


\PropositionG{2.141}
{Das Bild ist eine Tatsache.}


\PropositionG{2.15}
{Dass sich die Elemente des Bildes in bestimmter
Art und Weise zu einander verhalten stellt vor,
dass sich die Sachen so zu einander verhalten.

Dieser Zusammenhang der Elemente des Bildes
heisse seine Struktur und ihre Möglichkeit seine
Form der Abbildung.}


\PropositionG{2.151}
{Die Form der Abbildung ist die Möglichkeit,
dass sich die Dinge so zu einander verhalten, wie
die Elemente des Bildes.}


\PropositionG{2.1511}
{Das Bild ist \Emph{so} mit der Wirklichkeit verknüpft;
es reicht bis zu ihr.}


\PropositionG{2.1512}
{Es ist wie ein \DPtypo{Masstab}{Massstab} an die Wirklichkeit
angelegt.}


\PropositionG{2.15121}
{Nur die äussersten Punkte der Teilstriche
\Emph{berühren} den zu messenden Gegenstand.}


\PropositionG{2.1513}
{Nach dieser Auffassung gehört also zum Bilde
auch noch die abbildende Beziehung, die es zum
Bild macht.}


\PropositionG{2.1514}
{Die abbildende Beziehung besteht aus den
Zuordnungen der Elemente des Bildes und der
Sachen.}


\PropositionG{2.1515}
{Diese Zuordnungen sind gleichsam die Fühler
% -----File: 040.png---
der \DPtypo{Bildelmente}{Bildelemente}, mit denen das Bild die Wirklichkeit
berührt.}


\PropositionG{2.16}
{Die Tatsache muss um Bild zu sein, etwas mit
dem Abgebildeten gemeinsam haben.}


\PropositionG{2.161}
{In Bild und Abgebildetem muss etwas identisch
sein, damit das eine überhaupt ein Bild des anderen
sein kann.}


\PropositionG{2.17}
{Was das Bild mit der Wirklichkeit gemein
haben muss, um sie auf seine Art und Weise---richtig
oder falsch---abbilden zu können, ist seine
Form der Abbildung.}


\PropositionG{2.171}
{Das Bild kann jede Wirklichkeit abbilden,
deren Form es hat.

Das räumliche Bild alles Räumliche, das farbige
alles Farbige, etc.}


\PropositionG{2.172}
{Seine Form der Abbildung aber, kann das Bild
nicht abbilden; es weist sie auf.}


\PropositionG{2.173}
{Das Bild stellt sein Objekt von ausserhalb dar
(sein Standpunkt ist seine Form der Darstellung),
darum stellt das Bild sein Objekt richtig oder
falsch dar.}


\PropositionG{2.174}
{Das Bild kann sich aber nicht ausserhalb seiner
Form der Darstellung stellen.}


\PropositionG{2.18}
{Was jedes Bild, welcher Form immer, mit der
Wirklichkeit gemein haben muss, um sie überhaupt---richtig
oder falsch---ab\-bil\-den zu können,
ist die logische Form, das ist, die Form der
Wirklichkeit.}


\PropositionG{2.181}
{Ist die Form der Abbildung die logische Form,
so heisst das Bild das logische Bild.}


\PropositionG{2.182}
{Jedes Bild ist \Emph{auch} ein logisches. (Dagegen
ist \zumBeispiel\ nicht jedes Bild ein räumliches.)}


\PropositionG{2.19}
{Das logische Bild kann die Welt abbilden.}


\PropositionG{2.2}
{Das Bild hat mit dem Abgebildeten die logische
Form der Abbildung gemein.}


\PropositionG{2.201}
{Das Bild bildet die Wirklichkeit ab, indem es
eine Möglichkeit des Bestehens und Nichtbestehens
von Sachverhalten darstellt.}
% -----File: 042.png---


\PropositionG{2.202}
{Das Bild stellt eine mögliche Sachlage im
logischen Raume dar.}


\PropositionG{2.203}
{Das Bild enthält die Möglichkeit der Sachlage,
die es darstellt.}


\PropositionG{2.21}
{Das Bild stimmt mit der Wirklichkeit überein
oder nicht; es ist richtig oder unrichtig, wahr
oder falsch.}


\PropositionG{2.22}
{{\stretchyspace
Das Bild stellt dar, was es darstellt, unabhängig
von seiner Wahr- oder Falschheit, durch die Form
der Abbildung.}}


\PropositionG{2.221}
{Was das Bild darstellt, ist sein Sinn.}


\PropositionG{2.222}
{In der Übereinstimmung oder Nichtübereinstimmung
seines Sinnes mit der Wirklichkeit,
besteht seine Wahrheit oder Falschheit.}


\PropositionG{2.223}
{Um zu erkennen, ob das Bild wahr oder falsch
ist, müssen wir es mit der Wirklichkeit vergleichen.}


\PropositionG{2.224}
{Aus dem Bild allein ist nicht zu erkennen, ob
es wahr oder falsch ist.}


\PropositionG{2.225}
{Ein a priori wahres Bild gibt es nicht.}


\PropositionG{3}
{Das logische Bild der Tatsachen ist der
Gedanke.}


\PropositionG{3.001}
{\glqq{}Ein Sachverhalt ist denkbar\grqq{} heisst: Wir
können uns ein Bild von ihm machen.}


\PropositionG{3.01}
{Die Gesamtheit der wahren Gedanken sind
ein Bild der Welt.}


\PropositionG{3.02}
{Der Gedanke enthält die Möglichkeit der
Sachlage die er denkt. Was denkbar ist, ist
auch möglich.}


\PropositionG{3.03}
{Wir können nichts Unlogisches denken, weil
wir sonst unlogisch denken müssten.}


\PropositionG{3.031}
{Man sagte einmal, dass Gott alles schaffen
könne, nur nichts, was den logischen Gesetzen
zuwider wäre.---Wir könnten nämlich von einer
\glqq{}unlogischen\grqq{} Welt nicht \Emph{sagen}, wie sie aussähe.}


\PropositionG{3.032}
{Etwas \glqq{}der Logik widersprechendes\grqq{} in der
Sprache darstellen, kann man ebensowenig, wie
in der Geometrie eine den Gesetzen des Raumes
widersprechende Figur durch ihre Koordinaten
% -----File: 044.png---
darstellen; oder die Koordinaten eines Punktes
angeben, welcher nicht existiert.}


\PropositionG{3.0321}
{Wohl können wir einen Sachverhalt räumlich
darstellen, welcher den Gesetzen der Physik,
aber keinen, der den Gesetzen der Geometrie
zuwiderliefe.}


\PropositionG{3.04}
{Ein a priori richtiger Gedanke wäre ein solcher,
dessen Möglichkeit seine Wahrheit bedingte.}


\PropositionG{3.05}
{Nur so könnten wir a priori wissen, dass ein
Gedanke wahr ist, wenn aus dem Gedanken
selbst (ohne Vergleichsobjekt) seine Wahrheit
zu erkennen wäre.}


\PropositionG{3.1}
{Im Satz drückt sich der Gedanke sinnlich
wahrnehmbar aus.}


\PropositionG{3.11}
{{\stretchyspace
Wir benützen das sinnlich wahrnehmbare
Zeichen (Laut- oder Schriftzeichen etc.) des Satzes
als Projektion der möglichen Sachlage.}

Die Projektionsmethode ist das Denken des
Satz-Sinnes.}


\PropositionG{3.12}
{Das Zeichen, durch welches wir den Gedanken
ausdrücken, nenne ich das Satzzeichen. Und der
Satz ist das Satzzeichen in seiner projektiven
Beziehung zur Welt.}


\PropositionG{3.13}
{Zum Satz gehört alles, was zur Projektion
gehört; aber nicht das Projizierte.

Also die Möglichkeit des Projizierten, aber nicht
dieses selbst.

Im Satz ist also sein Sinn noch nicht enthalten,
wohl aber die Möglichkeit ihn auszudrücken.

(\glqq{}Der Inhalt des Satzes\grqq{} heisst der Inhalt des
sinnvollen Satzes.)

Im Satz ist die Form seines Sinnes enthalten,
aber nicht dessen Inhalt.}


\PropositionG{3.14}
{Das Satzzeichen besteht darin, dass sich seine
Elemente, die Wörter, in ihm auf bestimmte Art
und Weise zu einander verhalten.

Das Satzzeichen ist eine Tatsache.}


\PropositionG{3.141}
{Der Satz ist kein Wörtergemisch.---(Wie
% -----File: 046.png---
das musikalische Thema kein Gemisch von
Tönen.)

Der Satz ist artikuliert.}


\PropositionG{3.142}
{Nur Tatsachen können einen Sinn ausdrücken,
eine Klasse von Namen kann es nicht.}


\PropositionG{3.143}
{Dass das Satzzeichen eine Tatsache ist, wird
durch die gewöhnliche Ausdrucksform der Schrift
oder des Druckes verschleiert.

Denn im gedruckten Satz \zumBeispiel\ sieht das Satzzeichen
nicht wesentlich verschieden aus vom
Wort.

(So war es möglich, dass Frege den Satz einen
zusammengesetzten Namen nannte.)}


\PropositionG{3.1431}
{Sehr klar wird das Wesen des Satzzeichens,
wenn wir es uns, statt aus Schriftzeichen, aus
räumlichen Gegenständen (etwa Tischen, Stühlen,
Büchern) zusammengesetzt denken.

Die gegenseitige räumliche Lage dieser Dinge
drückt dann den Sinn des Satzes aus.}


\PropositionG{3.1432}
{Nicht: \glqq{}Das komplexe Zeichen \glq{}$aRb$\grq{} sagt,
dass $a$ in der Beziehung $R$ zu $b$ steht\grqq{}, sondern:
\Emph{Dass} \glqq{}$a$\grqq{} in einer gewissen Beziehung zu \glqq{}$b$\grqq{}
steht, sagt, \Emph{dass} $aRb$.}


\PropositionG{3.144}
{Sachlagen kann man beschreiben, nicht \Emph{benennen}.

(Namen gleichen Punkten, Sätze Pfeilen, sie
haben Sinn.)}


\PropositionG{3.2}
{Im Satze kann der Gedanke so ausgedrückt sein,
dass den Gegenständen des Gedankens Elemente
des Satzzeichens entsprechen.}


\PropositionG{3.201}
{Diese Elemente nenne ich \glqq{}einfache Zeichen\grqq{}
und den Satz \glqq{}vollständig analysiert\grqq{}.}


\PropositionG{3.202}
{Die im Satze angewandten einfachen Zeichen
heissen Namen.}


\PropositionG{3.203}
{Der Name bedeutet den Gegenstand. Der
Gegenstand ist seine Bedeutung. (\glqq{}$A$\grqq{} ist dasselbe
Zeichen wie \glqq{}$A$\grqq{}.)}


\PropositionG{3.21}
{Der Konfiguration der einfachen Zeichen im
% -----File: 048.png---
Satzzeichen entspricht die Konfiguration der Gegenstände
in der Sachlage.}


\PropositionG{3.22}
{Der Name vertritt im Satz den Gegenstand.}


\PropositionG{3.221}
{Die Gegenstände kann ich nur \Emph{nennen}. Zeichen
vertreten sie. Ich kann nur \Emph{von} ihnen sprechen,
\Emph{sie aussprechen} kann ich nicht. Ein Satz
kann nur sagen, \Emph{wie} ein Ding ist, nicht \Emph{was} es ist.}


\PropositionG{3.23}
{Die Forderung der Möglichkeit der einfachen
Zeichen ist die Forderung der Bestimmtheit des
Sinnes.}


\PropositionG{3.24}
{Der Satz, welcher vom Komplex handelt, steht
in interner Beziehung zum Satze, der von dessen
Bestandteil handelt.

Der Komplex kann nur durch seine Beschreibung
gegeben sein, und diese wird stimmen oder
nicht stimmen. Der Satz, in welchem von einem
Komplex die Rede ist, wird, wenn dieser nicht
existiert, nicht unsinnig, sondern einfach falsch sein.

Dass ein Satzelement einen Komplex bezeichnet,
kann man aus einer Unbestimmtheit in den Sätzen
sehen, worin es vorkommt. Wir \Emph{wissen}, durch
diesen Satz ist noch nicht alles bestimmt. (Die
Allgemeinheitsbezeichnung \Emph{enthält} ja ein Urbild.)

Die Zusammenfassung des Symbols eines Komplexes
in ein einfaches Symbol kann durch eine
\enlargethispage{4pt} % enlarge to make the last word fit
Definition ausgedrückt werden.}


\PropositionG{3.25}
{Es gibt eine und nur eine vollständige Analyse
des Satzes.}


\PropositionG{3.251}
{Der Satz drückt auf bestimmte, klar angebbare
Weise aus, was er ausdrückt: Der Satz ist artikuliert.}


\PropositionG{3.26}
{Der \DPtypo{name}{Name} ist durch keine Definition weiter zu
zergliedern: er ist ein Urzeichen.}


\PropositionG{3.261}
{Jedes definierte Zeichen bezeichnet \Emph{über} jene
Zeichen, durch welche es definiert wurde; und die
Definitionen weisen den Weg.

Zwei Zeichen, ein Urzeichen, und ein durch
Urzeichen definiertes, können nicht auf dieselbe
% -----File: 050.png---
Art und Weise bezeichnen. Namen \Emph{kann} man
nicht durch Definitionen auseinanderlegen. (Kein
Zeichen, welches allein, selbständig eine Bedeutung
hat.)}


\PropositionG{3.262}
{Was in den Zeichen nicht zum Ausdruck kommt,
das zeigt ihre Anwendung. Was die Zeichen
verschlucken, das spricht ihre Anwendung aus.}


\PropositionG{3.263}
{Die Bedeutungen von Urzeichen können durch
Erläuterungen erklärt werden. Erläuterungen
sind Sätze, welche die Urzeichen enthalten. Sie
können also nur verstanden werden, wenn die
Bedeutungen dieser Zeichen bereits bekannt sind.}


\PropositionG{3.3}
{Nur der Satz hat Sinn; nur im Zusammenhange
des Satzes hat ein Name Bedeutung.}


\PropositionG{3.31}
{Jeden Teil des Satzes, der seinen Sinn charakterisiert,
nenne ich einen Ausdruck (ein Symbol).

(Der Satz selbst ist ein Ausdruck.)

Ausdruck ist alles, für den Sinn des Satzes
wesentliche, was Sätze miteinander gemein haben
können.

Der Ausdruck kennzeichnet eine Form und
einen Inhalt.}


\PropositionG{3.311}
{Der Ausdruck setzt die Formen aller Sätze
voraus, in welchen er vorkommen kann. Er ist
das gemeinsame charakteristische Merkmal einer
Klasse von Sätzen.}


\PropositionG{3.312}
{Er wird also dargestellt durch die allgemeine
Form der Sätze, die er charakterisiert.

Und zwar wird in dieser Form der Ausdruck
\Emph{konstant} und alles übrige \Emph{variabel} sein.}


\PropositionG{3.313}
{Der Ausdruck wird also durch eine Variable
\enlargethispage{6pt} % enlarge to make the last line fit
dargestellt, deren Werte die Sätze sind, die den
Ausdruck enthalten.

(Im Grenzfall wird die Variable zur Konstanten,
der Ausdruck zum Satz.)

Ich nenne eine solche Variable \glqq{}Satzvariable\grqq{}.}


\PropositionG{3.314}
{Der Ausdruck hat nur im Satz Bedeutung.
Jede Variable lässt sich als Satzvariable auffassen.
% -----File: 052.png---

(Auch der variable Name.)}


\PropositionG{3.315}
{Verwandeln wir einen Bestandteil eines Satzes
in eine Variable, so gibt es eine Klasse von Sätzen,
welche sämtlich Werte des so entstandenen variablen
Satzes sind. Diese Klasse hängt im allgemeinen
noch davon ab, was wir, nach willkürlicher
Übereinkunft, mit Teilen jenes Satzes meinen.
Verwandeln wir aber alle jene Zeichen, deren
Bedeutung willkürlich bestimmt wurde, in Variable,
so gibt es nun noch immer eine solche Klasse.
Diese aber ist nun von keiner Übereinkunft
abhängig, sondern nur noch von der Natur des
Satzes. Sie entspricht einer logischen Form---einem
logischen Urbild.}


\PropositionG{3.316}
{Welche Werte die Satzvariable annehmen darf,
wird festgesetzt.

Die Festsetzung der Werte \Emph{ist} die Variable.}


\PropositionG{3.317}
{Die Festsetzung der Werte der Satzvariablen
ist die \Emph{Angabe der Sätze}, deren gemeinsames
Merkmal die Variable ist.

Die Festsetzung ist eine Beschreibung dieser
Sätze.

Die Festsetzung wird also nur von Symbolen,
nicht von deren Bedeutung handeln.

Und \Emph{nur} dies ist der Festsetzung wesentlich,
\Emph{dass sie nur eine Beschreibung von
Symbolen ist und nichts über das Bezeichnete
aussagt}.

Wie die Beschreibung der Sätze geschieht, ist
unwesentlich.}


\PropositionG{3.318}
{Den Satz fasse ich---wie Frege und Russell---als
Funktion der in ihm enthaltenen Ausdrücke auf.}


\PropositionG{3.32}
{Das Zeichen ist das sinnlich Wahrnehmbare am
Symbol.}


\PropositionG{3.321}
{Zwei verschiedene Symbole können also das
Zeichen (Schriftzeichen oder Lautzeichen etc.)
miteinander gemein haben---sie bezeichnen dann
auf verschiedene Art und Weise.}
% -----File: 054.png---


\PropositionG{3.322}
{Es kann nie das gemeinsame Merkmal zweier
Gegenstände anzeigen, dass wir sie mit demselben
Zeichen, aber durch zwei verschiedene \Emph{Bezeichnungsweisen}
bezeichnen. Denn das Zeichen
ist ja willkürlich. Man könnte also auch zwei verschiedene
Zeichen wählen, und wo bliebe dann das
Gemeinsame in der Bezeichnung.}


\PropositionG{3.323}
{In der Umgangssprache kommt es ungemein
häufig vor, dass dasselbe Wort auf verschiedene
Art und Weise bezeichnet---also verschiedenen
Symbolen an\-ge\-hört---, oder, dass zwei Wörter,
die auf verschiedene Art und Weise bezeichnen,
äusserlich in der gleichen Weise im Satze angewandt
werden.

So erscheint das Wort \glqq{}ist\grqq{} als Kopula, als
Gleichheitszeichen und als Ausdruck der Existenz;
\glqq{}existieren\grqq{} als intransitives Zeitwort wie \glqq{}gehen\grqq{};
\glqq{}identisch\grqq{} als Eigenschaftswort; wir reden von
\Emph{Etwas}, aber auch davon, dass \Emph{etwas} geschieht.

(Im Satze \glqq{}Grün ist grün\grqq{}---wo das erste Wort
ein Personenname, das letzte ein Eigenschaftswort
ist---haben diese Worte nicht einfach verschiedene
Bedeutung, sondern es sind \Emph{verschiedene
Symbole}.)}


\PropositionG{3.324}
{So entstehen leicht die fundamentalsten Verwechslungen
(deren die ganze Philosophie voll
ist).}


\PropositionG{3.325}
{Um diesen Irrtümern zu entgehen, müssen
wir eine Zeichensprache verwenden, welche sie
ausschliesst, indem sie nicht das gleiche Zeichen
in verschiedenen Symbolen, und Zeichen, welche
auf verschiedene Art bezeichnen, nicht äusserlich
auf die gleiche Art verwendet. Eine Zeichensprache
also, die der \Emph{logischen} Grammatik---der logischen
Syntax---gehorcht.

(Die Begriffsschrift Frege's und Russell's ist
eine solche Sprache, die allerdings noch nicht alle
Fehler ausschliesst.)}
% -----File: 056.png---


\PropositionG{3.326}
{Um das Symbol am Zeichen zu erkennen, muss
man auf den sinnvollen Gebrauch achten.}


\PropositionG{3.327}
{Das Zeichen bestimmt erst mit seiner logisch-syntaktischen
Verwendung zusammen eine logische
Form.}


\PropositionG{3.328}
{Wird ein Zeichen \Emph{nicht gebraucht}, so ist
es bedeutungslos. Das ist der Sinn der Devise
Occams.

(Wenn sich alles so verhält als hätte ein Zeichen
Bedeutung, dann hat es auch Bedeutung.)}


\PropositionG{3.33}
{In der logischen Syntax darf nie die Bedeutung
eines Zeichens eine Rolle spielen; sie muss sich
aufstellen lassen, ohne dass dabei von der \Emph{Bedeutung}
eines Zeichens die Rede wäre, sie darf \Emph{nur}
die Beschreibung der Ausdrücke voraussetzen.}


\PropositionG{3.331}
{Von dieser Bemerkung sehen wir in Russell's
\glqq{}Theory of types\grqq{} hinüber: Der Irrtum Russell's
zeigt sich darin, dass er bei der Aufstellung der
Zeichenregeln von der Bedeutung der Zeichen
reden musste.}


\PropositionG{3.332}
{Kein Satz kann etwas über sich selbst aussagen,
weil das Satzzeichen nicht in sich selbst enthalten
sein kann, (das ist die ganze \glqq{}Theory of types\grqq{}).}


\PropositionG{3.333}
{Eine Funktion kann darum nicht ihr eigenes
Argument sein, weil das Funktionszeichen bereits
das Urbild seines Arguments enthält und es sich
nicht selbst enthalten kann.

Nehmen wir nämlich an, die Funktion $F (fx)$
könnte ihr eigenes Argument sein; dann gäbe es
also einen Satz: \glqq{}$F(F(fx))$\grqq{} und in diesem müssen
die äussere Funktion $F$ und die innere Funktion $F$
verschiedene Bedeutungen haben, denn die innere
hat die Form $\phi(fx)$, die äussere, die Form $\psi(\phi(fx))$.
Gemeinsam ist den beiden Funktionen nur der
Buchstabe \glqq{}$F$\grqq{}, der aber allein nichts bezeichnet.

Dies wird sofort klar, wenn wir statt \glqq{}$F(F(u))$\grqq{}
schreiben \glqq{}$(\exists\phi) : F(\phi u) \DotOp \phi u = Fu$\grqq{}.

Hiermit erledigt sich Russell's Paradox.}
% -----File: 058.png---


\PropositionG{3.334}
{Die Regeln der logischen Syntax müssen sich
von selbst verstehen, wenn man nur weiss, wie
ein jedes Zeichen bezeichnet.}


\PropositionG{3.34}
{Der Satz besitzt wesentliche und zufällige Züge.

Zufällig sind die Züge, die von der besonderen
Art der Hervorbringung des Satzzeichens herrühren.
Wesentlich diejenigen, welche allein den Satz befähigen,
seinen Sinn auszudrücken.}


\PropositionG{3.341}
{Das Wesentliche am Satz ist also das, was allen
Sätzen, welche den gleichen Sinn ausdrücken
können, gemeinsam ist.

Und ebenso ist allgemein das Wesentliche am
Symbol das, was alle Symbole, die denselben
Zweck erfüllen können, gemeinsam haben.}


\PropositionG{3.3411}
{Man könnte also sagen: Der eigentliche Name
ist das, was alle Symbole, die den Gegenstand
bezeichnen, gemeinsam haben. Es würde sich so
successive ergeben, dass keinerlei Zusammensetzung
für den Namen wesentlich ist.}


\PropositionG{3.342}
{An unseren Notationen ist zwar etwas willkürlich,
aber \Emph{das} ist nicht willkürlich: Dass, \Emph{wenn} wir
etwas willkürlich bestimmt haben, dann etwas
anderes der Fall sein muss. (Dies hängt von dem
\Emph{Wesen} der Notation ab.)}


\PropositionG{3.3421}
{{\stretchyspace
Eine besondere Bezeichnungsweise mag unwichtig
sein, aber wichtig ist es immer, dass diese
eine \Emph{mögliche} Bezeichnungsweise ist. Und so
verhält es sich in der Philosophie überhaupt: Das
Einzelne erweist sich immer wieder als unwichtig,
aber die Möglichkeit jedes Einzelnen gibt uns
einen Aufschluss über das Wesen der Welt.}}


\PropositionG{3.343}
{Definitionen sind Regeln der Übersetzung von
einer Sprache in eine andere. Jede richtige Zeichensprache
muss sich in jede andere nach solchen
Regeln übersetzen lassen: \Emph{Dies} ist, was sie alle
gemeinsam haben.}


\PropositionG{3.344}
{Das, was am Symbol bezeichnet, ist das Gemeinsame
aller jener Symbole, durch die das erste den
% -----File: 060.png---
Regeln der logischen Syntax zufolge ersetzt werden
kann.}


\PropositionG{3.3441}
{Man kann \zumBeispiel\ das Gemeinsame aller Notationen
für die Wahrheitsfunktionen so ausdrücken: Es ist
ihnen gemeinsam, dass sich alle---\zumBeispiel---durch die
Notation von \glqq{}$\Not{p}$\grqq{} (\glqq{}nicht $p$\grqq{}) und \glqq{}$p \lor q$\grqq{} (\glqq{}$p$ oder $q$\grqq{})
\Emph{ersetzen lassen}.

{\stretchyspace
(Hiermit ist die Art und Weise gekennzeichnet,
wie eine spezielle mögliche Notation uns allgemeine
Aufschlüsse geben kann.)}}


\PropositionG{3.3442}
{Das Zeichen des Komplexes löst sich auch bei
der Analyse nicht willkürlich auf, so dass etwa seine
Auflösung in jedem Satzgefüge eine andere wäre.}


\PropositionG{3.4}
{Der Satz bestimmt einen Ort im logischen Raum.
Die Existenz dieses logischen Ortes ist durch die
Existenz der Bestandteile allein verbürgt, durch die
Existenz des sinnvollen Satzes.}


\PropositionG{3.41}
{Das Satzzeichen und die logischen Koordinaten:
\enlargethispage{1pt} % enlarge to make the last line fit
Das ist der logische Ort.}


\PropositionG{3.411}
{Der geometrische und der logische Ort stimmen
darin überein, dass beide die Möglichkeit einer
Existenz sind.}


\PropositionG{3.42}
{Obwohl der Satz nur einen Ort des logischen
Raumes bestimmen darf, so muss doch durch
ihn schon der ganze logische Raum gegeben
sein.

(Sonst würden durch die Verneinung, die logische
Summe, das logische Produkt, etc.\ immer neue
Elemente---in Ko\-or\-di\-na\-ti\-on---eingeführt.)

(Das logische Gerüst um das Bild herum bestimmt
den logischen Raum. Der Satz durchgreift den
ganzen logischen Raum.)}


\PropositionG{3.5}
{Das angewandte, gedachte, Satzzeichen ist der
Gedanke.}


\PropositionG{4}
{Der Gedanke ist der sinnvolle Satz.}


\PropositionG{4.001}
{Die Gesamtheit der Sätze ist die Sprache.}


\PropositionG{4.002}
{Der Mensch besitzt die Fähigkeit Sprachen zu
bauen, womit sich jeder Sinn ausdrücken lässt,
% -----File: 062.png---
ohne eine Ahnung davon zu haben, wie und was
jedes Wort bedeutet.---Wie man auch spricht, ohne
zu wissen, wie die einzelnen Laute hervorgebracht
werden.

Die Umgangssprache ist ein Teil des menschlichen
Organismus und nicht weniger kompliziert als
dieser.

Es ist menschenunmöglich, die Sprachlogik aus
ihr unmittelbar zu entnehmen.

Die Sprache verkleidet den Gedanken. Und
zwar so, dass man nach der äusseren Form des
Kleides, nicht auf die Form des bekleideten Gedankens
schliessen kann; weil die äussere Form des
Kleides nach ganz anderen Zwecken gebildet ist, als
danach, die Form des Körpers erkennen zu lassen.

{\stretchyspace
Die stillschweigenden Abmachungen zum Verständnis
der Umgangssprache sind enorm kompliziert.}}


\PropositionG{4.003}
{Die meisten Sätze und Fragen, welche über
philosophische Dinge geschrieben worden sind, sind
nicht falsch, sondern unsinnig. Wir können daher
Fragen dieser Art überhaupt nicht beantworten,
sondern nur ihre Unsinnigkeit feststellen. Die
meisten Fragen und Sätze der Philosophen beruhen
darauf, \DPtypo{das}{dass} wir unsere Sprachlogik nicht verstehen.

(Sie sind von der Art der Frage, ob das Gute
\enlargethispage{1pt} % enlarge to make the last line fit
mehr oder weniger identisch sei als das Schöne.)

Und es ist nicht verwunderlich, dass die tiefsten
Probleme eigentlich \Emph{keine} Probleme sind.}


\PropositionG{4.0031}
{Alle Philosophie ist \glqq{}Sprachkritik\grqq{}. (Allerdings
nicht im Sinne Mauthners.) Russell's Verdienst ist
es, gezeigt zu haben, dass die scheinbare logische
Form des Satzes nicht seine wirkliche sein muss.}


\PropositionG{4.01}
{Der Satz ist ein Bild der Wirklichkeit.

Der Satz ist ein Modell der Wirklichkeit, so wie
wir sie uns denken.}


\PropositionG{4.011}
{Auf den ersten Blick scheint der Satz---wie er
etwa auf dem Papier gedruckt steht---kein Bild der
% -----File: 064.png---
Wirklichkeit zu sein, von der er handelt. Aber
auch die Notenschrift scheint auf den ers\-ten Blick
kein Bild der Musik zu sein, und unsere Lautzeichen-\mbox{(Buchstaben-)}\AllowBreak{}Schrift
kein Bild unserer Lautsprache.

Und doch erweisen sich diese Zeichensprachen
auch im gewöhnlichen Sinne als Bilder dessen, was
sie darstellen.}


\PropositionG{4.012}
{Offenbar ist, dass wir einen Satz von der Form
\glqq{}$aRb$\grqq{} als Bild empfinden. Hier ist das Zeichen
offenbar ein Gleichnis des Bezeichneten.}


\PropositionG{4.013}
{Und wenn wir in das Wesentliche dieser Bildhaftigkeit
eindringen, so sehen wir, dass dieselbe
durch \Emph{scheinbare Unregelmässigkeiten}
(wie die Verwendung der $\sharp$ und $\flat$ in der Notenschrift)
\Emph{nicht} gestört wird.

Denn auch diese Unregelmässigkeiten bilden
das ab, was sie ausdrücken sollen; nur auf eine
andere Art und Weise.}


\PropositionG{4.014}
{Die Grammophonplatte, der musikalische Gedanke,
die Notenschrift, die Schallwellen, stehen
alle in jener abbildenden internen Beziehung zu
einander, die zwischen Sprache und Welt besteht.

Ihnen allen ist der logische Bau gemeinsam.

(Wie im Märchen die zwei Jünglinge, ihre zwei
Pferde und ihre Lilien. Sie sind alle in gewissem
Sinne Eins.)}


\PropositionG{4.0141}
{Dass es eine allgemeine Regel gibt, durch die
der Musiker aus der Partitur die Symphonie
entnehmen kann, durch welche man aus der Linie
auf der Grammophonplatte die Symphonie und
nach der ersten Regel wieder die Partitur ableiten
kann, darin besteht eben die innere Ähnlichkeit
dieser scheinbar so ganz verschiedenen Gebilde.
Und jene Regel ist das Gesetz der Projektion,
welches die Symphonie in die Notensprache projiziert.
Sie ist die Regel der Übersetzung der
Notensprache in die Sprache der Grammophonplatte.}


\PropositionG{4.015}
{Die Möglichkeit aller Gleichnisse, der ganzen
% -----File: 066.png---
Bildhaftigkeit unserer Ausdrucksweise, ruht in der
Logik der Abbildung.}


\PropositionG{4.016}
{Um das Wesen des Satzes zu verstehen, denken
wir an die Hieroglyphenschrift, welche die Tatsachen
die sie beschreibt abbildet.

Und aus ihr wurde die Buchstabenschrift, ohne
das Wesentliche der Abbildung zu verlieren.}


\PropositionG{4.02}
{Dies sehen wir daraus, dass wir den Sinn des
Satzzeichens verstehen, ohne dass er uns erklärt
wurde.}


\PropositionG{4.021}
{Der Satz ist ein Bild der Wirklichkeit: Denn
ich kenne die von ihm dargestellte Sachlage, wenn
ich den Satz verstehe. Und den Satz verstehe ich,
ohne dass mir sein Sinn erklärt wurde.}


\PropositionG{4.022}
{Der Satz \Emph{zeigt} seinen Sinn.

Der Satz \Emph{zeigt}, wie es sich verhält, \Emph{wenn} er
wahr ist. Und er \Emph{sagt}, \Emph{dass} es sich so verhält.}


\PropositionG{4.023}
{\DPtypo{Der}{Die} Wirklichkeit muss durch den Satz auf ja
oder nein fixiert sein.

Dazu muss sie durch ihn vollständig beschrieben
werden.

Der Satz ist die Beschreibung eines Sachverhaltes.

Wie die Beschreibung einen Gegenstand nach
seinen externen Eigenschaften, so beschreibt der
Satz die Wirklichkeit nach ihren internen Eigenschaften.

Der Satz konstruiert eine Welt mit Hilfe eines
logischen Gerüstes und darum kann man am Satz
auch sehen, wie sich alles Logische verhält, \Emph{wenn}
er wahr ist. Man kann aus einem falschen Satz
\Emph{Schlüsse ziehen}.}


\PropositionG{4.024}
{Einen Satz verstehen, heisst, wissen was der
Fall ist, wenn er wahr ist.

(Man kann ihn also verstehen, ohne zu wissen,
ob er wahr ist.)

Man versteht ihn, wenn man seine Bestandteile
versteht.}


\PropositionG{4.025}
{Die Übersetzung einer Sprache in eine andere
% -----File: 068.png---
geht nicht so vor sich, dass man jeden \Emph{Satz} der
einen in einen \Emph{Satz} der anderen übersetzt, sondern
nur die Satzbestandteile werden übersetzt.

(Und das Wörterbuch übersetzt nicht nur
Substantiva, sondern auch \mbox{Zeit-,} Eigenschafts- und
Bindewörter etc.; und es behandelt sie alle gleich.)}


\PropositionG{4.026}
{Die Bedeutungen der einfachen Zeichen (der
Wörter) müssen uns erklärt werden, dass wir sie
verstehen.

Mit den Sätzen aber verständigen wir uns.}


\PropositionG{4.027}
{Es liegt im Wesen des Satzes, dass er uns einen
\Emph{neuen} Sinn mitteilen kann.}


\PropositionG{4.03}
{Ein Satz muss mit alten Ausdrücken einen
neuen Sinn mitteilen.

Der Satz teilt uns eine Sachlage mit, also
muss er \Emph{wesentlich} mit der Sachlage zusammenhängen.

Und der Zusammenhang ist eben, dass er ihr
logisches Bild ist.

Der Satz sagt nur insoweit etwas aus, als er ein
Bild ist.}


\PropositionG{4.031}
{Im Satz wird gleichsam eine Sachlage probeweise
zusammengestellt.

Man kann geradezu sagen: statt, dieser Satz
hat diesen und diesen Sinn; dieser Satz stellt diese
und diese Sachlage dar.}


\PropositionG{4.0311}
{Ein Name steht für ein Ding, ein anderer für
ein anderes Ding und untereinander sind sie
verbunden, so stellt das Ganze---wie ein lebendes
Bild---den Sachverhalt vor.}


\PropositionG{4.0312}
{Die Möglichkeit des Satzes beruht auf dem
Prinzip der Vertretung von Gegenständen durch
Zeichen.

{\stretchyspace
Mein Grundgedanke ist, dass die \glqq{}logischen
Konstanten\grqq{} nicht vertreten. Dass sich die \Emph{Logik}
der Tatsachen nicht vertreten lässt.}}


\PropositionG{4.032}
{Nur insoweit ist der Satz ein Bild einer Sachlage,
als er logisch gegliedert ist.
% -----File: 070.png---

(Auch der Satz \glqq{}ambulo\grqq{} ist zusammengesetzt,
denn sein Stamm ergibt mit einer anderen Endung
und seine Endung mit einem anderen Stamm, einen
anderen Sinn.)}


\PropositionG{4.04}
{Am Satz muss gerade soviel zu unterscheiden
sein, als an der Sachlage die er darstellt.

{\stretchyspace
Die beiden müssen die gleiche logische (mathematische)
Mannigfaltigkeit besitzen. (Vergleiche
Hertz's Mechanik, über Dynamische Modelle.)}}


\PropositionG{4.041}
{Diese mathematische Mannigfaltigkeit kann
man natürlich nicht selbst wieder abbilden. Aus
ihr kann man beim Abbilden nicht heraus.}


\PropositionG{4.0411}
{Wollten wir \zumBeispiel\ das, was wir durch \glqq{}$(x) fx$\grqq{}
ausdrücken, durch Vorsetzen eines Indexes vor
\glqq{}$fx$\grqq{} ausdrücken---etwa so: \glqq{}Alg. $fx$\grqq{}, es würde
nicht genügen---wir wüssten nicht, was verallgemeinert
wurde. Wollten wir es durch einen
Index \glqq{}$a$\grqq{} anzeigen---etwa so: \glqq{}$f(x_{a}$)\grqq{}---es würde
auch nicht genügen---wir wüssten nicht den
Bereich der Allgemeinheitsbezeichnung.

Wollten wir es durch Einführung einer Marke
in die Argumentstellen ver\-su\-chen---etwa so:
\glqq{}$(A, A) \DotOp F (A, A)$\grqq{}---es würde nicht ge\-nü\-gen---wir
könnten die Identität der Variablen nicht feststellen.
U.s.w.

Alle diese Bezeichnungsweisen genügen nicht,
weil sie nicht die notwendige mathematische
Mannigfaltigkeit haben.}


\PropositionG{4.0412}
{{\stretchyspace
Aus demselben Grunde genügt die idealistische
Erklärung des Sehens der räumlichen Beziehungen
durch die \glqq{}Raumbrille\grqq{} nicht, weil sie nicht die
Mannigfaltigkeit dieser Beziehungen erklären kann.}}


\PropositionG{4.05}
{Die Wirklichkeit wird mit dem Satz verglichen.}


\PropositionG{4.06}
{Nur dadurch kann der Satz wahr oder falsch
sein, indem er ein Bild der Wirklichkeit ist.}


\PropositionG{4.061}
{Beachtet man nicht, dass der Satz einen von
den Tatsachen unabhängigen Sinn hat, so kann
man leicht glauben, dass wahr und falsch gleichberechtigte
% -----File: 072.png---
Beziehungen von Zeichen und Bezeichnetem
sind.

Man könnte dann \zumBeispiel\ sagen, dass \glqq{}$p$\grqq{} auf die
wahre Art bezeichnet, was \glqq{}$\Not{p}$\grqq{} auf die falsche
Art, etc.}


\PropositionG{4.062}
{Kann man sich nicht mit falschen Sätzen, wie
bisher mit wahren, verständigen? Solange man
nur weiss, dass sie falsch gemeint sind. Nein!
Denn, wahr ist ein Satz, wenn es sich so verhält,
wie wir es durch ihn sagen; und wenn wir mit
\glqq{}$p$\grqq{} $\Not{p}$ meinen, und es sich so verhält wie wir es
meinen, so ist \glqq{}$p$\grqq{} in der neuen Auffassung wahr
und nicht falsch.}


\PropositionG{4.0621}
{Dass aber die Zeichen \glqq{}$p$\grqq{} und \glqq{}$\Not{p}$\grqq{} das gleiche
sagen \Emph{können}, ist wichtig. Denn es zeigt, dass
dem Zeichen \glqq{}$\Not{}$\grqq{} in der Wirklichkeit nichts
entspricht.

Dass in einem Satz die Verneinung vorkommt,
ist noch kein Merkmal seines Sinnes ($\Not{\Not{p}} = p$).

Die Sätze \glqq{}$p$\grqq{} und \glqq{}$\Not{p}$\grqq{} haben entgegengesetzten
Sinn, aber es entspricht ihnen eine und
dieselbe Wirklichkeit.}


\PropositionG{4.063}
{Ein Bild zur Erklärung des Wahrheitsbegriffes:
Schwarzer Fleck auf weissem Papier; die Form
des Fleckes kann man beschreiben, indem man
für jeden Punkt der Fläche angibt, ob er weiss
oder schwarz ist. Der Tatsache, dass ein Punkt
schwarz ist, entspricht eine positive---der, dass
ein Punkt weiss (nicht schwarz) ist, eine negative
Tatsache. Bezeichne ich einen Punkt der Fläche
(einen Frege'schen Wahrheitswert), so entspricht
dies der Annahme, die zur Beurteilung aufgestellt
wird, etc.\ etc.

Um aber sagen zu können, ein Punkt sei
schwarz oder weiss, muss ich vorerst wissen,
wann man einen Punkt schwarz und wann
man ihn weiss nennt; um sagen zu können:
\glqq{}$p$\grqq{} ist wahr (oder falsch), muss ich bestimmt
% -----File: 074.png---
haben, unter welchen Umständen ich \glqq{}$p$\grqq{} wahr
nenne, und damit bestimme ich den Sinn des
Satzes.

Der Punkt an dem das Gleichnis hinkt ist
nun der: Wir können auf einen Punkt des Papiers
zeigen, auch ohne zu wissen, was weiss und
schwarz ist; einem Satz ohne Sinn aber entspricht
gar nichts, denn er bezeichnet kein Ding (Wahrheitswert)
dessen Eigenschaften etwa \glqq{}falsch\grqq{} oder
\glqq{}wahr\grqq{} hiessen; das Verbum eines Satzes ist nicht
\glqq{}ist wahr\grqq{} oder \glqq{}ist falsch\grqq{}---wie Frege glaubte---,
sondern das, was \glqq{}wahr ist\grqq{} muss das Verbum
schon enthalten.}


\PropositionG{4.064}
{Jeder Satz muss \Emph{schon} einen Sinn haben;
die Bejahung kann ihn ihm nicht geben, denn
sie bejaht ja gerade den Sinn. Und dasselbe gilt
von der Verneinung, etc.}


\PropositionG{4.0641}
{Man könnte sagen: Die Verneinung bezieht
sich schon auf den logischen Ort, den der verneinte
Satz bestimmt.

Der verneinende Satz bestimmt einen \Emph{anderen}
logischen Ort als der verneinte.

Der verneinende Satz bestimmt einen logischen
Ort mit Hilfe des logischen Ortes des verneinten
Satzes, indem er jenen ausserhalb diesem liegend
beschreibt.

Dass man den verneinten Satz wieder verneinen
kann, zeigt schon, dass das, was verneint wird,
schon ein Satz und nicht erst die Vorbereitung
zu einem Satze ist.}


\PropositionG{4.1}
{Der Satz stellt das Bestehen und Nichtbestehen
der Sachverhalte dar.}


\PropositionG{4.11}
{Die Gesamtheit der wahren Sätze ist die
gesamte Naturwissenschaft (oder die Gesamtheit
der Naturwissenschaften).}


\PropositionG{4.111}
{Die Philosophie ist keine der Naturwissenschaften.

(Das Wort \glqq{}Philosophie\grqq{} muss etwas bedeuten,
% -----File: 076.png---
was über oder unter, aber nicht neben den Naturwissenschaften
steht.)}


\PropositionG{4.112}
{Der Zweck der Philosophie ist die logische
Klärung der Gedanken.

Die Philosophie ist keine Lehre, sondern eine
Tätigkeit.

Ein philosophisches Werk besteht wesentlich
aus Erläuterungen.

Das Resultat der Philosophie sind nicht \glqq{}philosophische
Sätze\grqq{}, sondern das Klarwerden von
Sätzen.

Die Philosophie soll die Gedanken, die sonst,
gleichsam, trübe und verschwommen sind, klar
machen und scharf abgrenzen.}


\PropositionG{4.1121}
{Die Psychologie ist der Philosophie nicht verwandter
als irgend eine andere Naturwissenschaft.

Erkenntnistheorie ist die Philosophie der
Psychologie.

Entspricht nicht mein Studium der Zeichensprache
dem Studium der Denkprozesse, welches
die Philosophen für die Philosophie der Logik für
so wesentlich hielten? Nur verwickelten sie sich
meistens in unwesentliche psychologische Untersuchungen
und eine analoge Gefahr gibt es auch
bei meiner Methode.}


\PropositionG{4.1122}
{Die Darwinsche Theorie hat mit der Philosophie
nicht mehr zu schaffen, als irgend eine andere
Hypothese der Naturwissenschaft.}


\PropositionG{4.113}
{Die Philosophie begrenzt das bestreitbare
Gebiet der Naturwissenschaft.}


\PropositionG{4.114}
{Sie soll das Denkbare abgrenzen und damit das
Undenkbare.

Sie soll das Undenkbare von innen durch das
Denkbare begrenzen.}


\PropositionG{4.115}
{Sie wird das Unsagbare bedeuten, indem sie
das Sagbare klar darstellt.}


\PropositionG{4.116}
{Alles was überhaupt gedacht werden kann,
% -----File: 078.png---
kann klar gedacht werden. Alles was sich aussprechen
lässt, lässt sich klar aussprechen.}


\PropositionG{4.12}
{Der Satz kann die gesamte Wirklichkeit darstellen,
aber er kann nicht das darstellen, was er
mit der Wirklichkeit gemein haben muss, um sie
darstellen zu können---die logische Form.

Um die logische Form darstellen zu können,
müssten wir uns mit dem Satze ausserhalb der
Logik aufstellen können, das heisst ausserhalb der
Welt.}


\PropositionG{4.121}
{Der Satz kann die logische Form nicht darstellen,
sie spiegelt sich in ihm.

Was sich in der Sprache spiegelt, kann sie
nicht darstellen.

Was \Emph{sich} in der Sprache ausdrückt, können
\Emph{wir} nicht durch sie ausdrücken.

Der Satz \Emph{zeigt} die logische Form der Wirklichkeit.

Er weist sie auf.}


\PropositionG{4.1211}
{So zeigt ein Satz \glqq{}$fa$\grqq{}, dass in seinem Sinn der
Gegenstand $a$ vorkommt, zwei Sätze \glqq{}$fa$\grqq{} und \glqq{}$ga$\grqq{},
dass in ihnen beiden von demselben Gegenstand
die Rede ist.

Wenn zwei Sätze einander widersprechen, so
zeigt dies ihre Struktur; ebenso, wenn einer aus
dem anderen folgt. U.s.w.}


\PropositionG{4.1212}
{Was gezeigt werden \Emph{kann}, \Emph{kann} nicht gesagt
werden.}


\PropositionG{4.1213}
{Jetzt verstehen wir auch unser Gefühl: dass wir
im Besitze einer richtigen logischen Auffassung
seien, wenn nur einmal alles in unserer Zeichensprache
stimmt.}


\PropositionG{4.122}
{{\stretchyspace
Wir können in gewissem Sinne von formalen
Eigenschaften der Gegenstände und Sachverhalte
bezw.\ von Eigenschaften der Struktur der Tatsachen
reden und in demselben Sinne von formalen
Relationen und Relationen von Strukturen.}

(Statt Eigenschaft der Struktur sage ich auch
% -----File: 080.png---
\glqq{}interne Eigenschaft\grqq{}; statt Relation der Strukturen
\glqq{}interne Relation\grqq{}.

Ich führe diese Ausdrücke ein, um den Grund
der, bei den Philosophen sehr verbreiteten Verwechslung
zwischen den internen Relationen und
den eigentlichen (externen) Relationen zu zeigen.)

Das Bestehen solcher interner Eigenschaften
und Relationen kann aber nicht durch Sätze
behauptet werden, sondern es zeigt sich in den
Sätzen, welche jene Sachverhalte darstellen und
von jenen Gegenständen handeln.}


\PropositionG{4.1221}
{Eine interne Eigenschaft einer Tatsache können
wir auch einen Zug dieser Tatsache nennen. (In
dem Sinn, in welchem wir etwa von Gesichtszügen
sprechen.)}


\PropositionG{4.123}
{Eine Eigenschaft ist intern, wenn es undenkbar
ist, dass ihr Gegenstand sie nicht besitzt.

(Diese blaue Farbe und jene stehen in der
internen Relation von heller und dunkler eo ipso.
Es ist undenkbar, dass \Emph{diese} beiden Gegenstände
nicht in dieser Relation stünden.)

(Hier entspricht dem schwankenden Gebrauch
der Worte \glqq{}Eigenschaft\grqq{} und \glqq{}Relation\grqq{} der
schwankende Gebrauch des Wortes \glqq{}Gegenstand\grqq{}.)}


\PropositionG{4.124}
{Das Bestehen einer internen Eigenschaft einer
möglichen Sachlage wird nicht durch einen Satz
ausgedrückt, sondern es drückt sich in dem sie
darstellenden Satz, durch eine interne Eigenschaft
dieses Satzes aus.

Es wäre ebenso unsinnig, dem Satze eine
formale Eigenschaft zuzusprechen, als sie ihm
abzusprechen.}


\PropositionG{4.1241}
{Formen kann man nicht dadurch von einander
unterscheiden, dass man sagt, die eine habe diese,
die andere aber jene Eigenschaft; denn dies setzt
voraus, dass es einen Sinn habe, beide Eigenschaften
von beiden Formen auszusagen.}


\PropositionG{4.125}
{Das Bestehen einer internen Relation zwischen
% -----File: 082.png---
möglichen Sachlagen drückt sich sprachlich durch
eine interne Relation zwischen den sie darstellenden
Sätzen aus.}


\PropositionG{4.1251}
{Hier erledigt sich nun die Streitfrage \glqq{}ob alle
Relationen intern oder extern\grqq{} seien.}


\PropositionG{4.1252}
{Reihen, welche durch \Emph{interne} Relationen
geordnet sind, nenne ich Formenreihen.

Die Zahlenreihe ist nicht nach einer externen,
sondern nach einer internen Relation geordnet.

{\stretchyspace
Ebenso die Reihe der Sätze \glqq{}$aRb$\grqq{},
\glqq{}$(\exists x): aRx \DotOp xRb$\grqq{},
\glqq{}$(\exists x,y): aRx \DotOp xRy \DotOp yRb$\grqq{}, \undSoFort}

(Steht $b$ in einer dieser Beziehungen zu $a$, so
nenne ich $b$ einen Nachfolger von $a$.)}


\PropositionG{4.126}
{In dem Sinne, in welchem wir von formalen
Eigenschaften sprechen, können wir nun auch
von formalen Begriffen reden.

(Ich führe diesen Ausdruck ein, um den Grund
der Verwechslung der formalen Begriffe mit den
eigentlichen Begriffen, welche die ganze alte Logik
durchzieht, klar zu machen.)

Dass etwas unter einen formalen Begriff als
dessen Gegenstand fällt, kann nicht durch einen
Satz ausgedrückt werden. Sondern es zeigt sich
an dem Zeichen dieses Gegenstandes selbst. (Der
Name zeigt, dass er einen Gegenstand bezeichnet,
das Zahlenzeichen, dass es eine Zahl bezeichnet etc.)

Die formalen Begriffe können ja nicht, wie
die eigentlichen Begriffe, durch eine Funktion
dargestellt werden.

Denn ihre Merkmale, die formalen Eigenschaften,
werden nicht durch Funktionen ausgedrückt.

Der Ausdruck der formalen Eigenschaft ist ein
Zug gewisser Symbole.

Das Zeichen der Merkmale eines formalen
Begriffes ist also ein charakteristischer Zug aller
Symbole, deren Bedeutungen unter den Begriff
fallen.
% -----File: 084.png---

Der Ausdruck des formalen Begriffes also, eine
Satzvariable, in welcher nur dieser charakteristische
Zug konstant ist.}


\PropositionG{4.127}
{Die Satzvariable bezeichnet den formalen
Begriff und ihre Werte die Gegenstände, welche
unter diesen Begriff fallen.}


\PropositionG{4.1271}
{Jede Variable ist das Zeichen eines formalen
Begriffes.

Denn jede Variable stellt eine konstante Form
dar, welche alle ihre Werte besitzen, und die als
\enlargethispage{-4pt} % force a line to the next page
formale Eigenschaft dieser Werte aufgefasst werden
kann.}


\PropositionG{4.1272}
{So ist der variable Name \glqq{}$x$\grqq{} das eigentliche
Zeichen des Scheinbegriffes \Emph{Gegenstand}.

Wo immer das Wort \glqq{}Gegenstand\grqq{} (\glqq{}Ding\grqq{},
\glqq{}Sache\grqq{}, etc.) richtig gebraucht wird, wird es in
der Begriffsschrift durch den variablen Namen
ausgedrückt.

Zum Beispiel in dem Satz \glqq{}es gibt 2 Gegenstände,
welche\ \ldots\grqq{} durch \glqq{}$(\exists x, y)$ $\ldots$\grqq{}.

Wo immer es anders, also als eigentliches
Begriffswort gebraucht wird, entstehen unsinnige
Scheinsätze.

So kann man \zumBeispiel\ nicht sagen \glqq{}Es gibt Gegenstände\grqq{},
wie man etwa sagt \glqq{}Es gibt Bücher\grqq{}.
Und ebenso wenig \glqq{}Es gibt 100 Gegenstände\grqq{},
oder \glqq{}Es gibt $\aleph_0$ Gegenstände\grqq{}.

Und es ist unsinnig, von der \Emph{Anzahl aller
Gegenstände} zu sprechen.

Dasselbe gilt von den Worten \glqq{}Komplex\grqq{},
\glqq{}Tatsache\grqq{}, \glqq{}Funktion\grqq{}, \glqq{}Zahl\grqq{}, etc.

Sie alle bezeichnen formale Begriffe und werden
in der Begriffsschrift durch Variable, nicht durch
Funktionen oder Klassen dargestellt. (Wie Frege
und Russell glaubten.)

Ausdrücke wie \glqq{}1 ist eine Zahl\grqq{}, \glqq{}es gibt nur
Eine Null\grqq{} und alle ähnlichen sind unsinnig.

(Es ist ebenso unsinnig zu sagen \glqq{}es gibt nur
% -----File: 086.png---
eine 1\grqq{}, als es unsinnig wäre, zu sagen: $2 + 2$ ist
um 3 Uhr gleich 4.)}


\PropositionG{4.12721}
{Der formale Begriff ist mit einem Gegenstand,
der unter ihn fällt, bereits gegeben. Man kann
also nicht Gegenstände eines formalen Begriffes
\Emph{und} den formalen Begriff selbst als Grundbegriffe
einführen. Man kann also \zumBeispiel\ nicht den Begriff
der Funktion, und auch spezielle Funktionen (wie
Russell) als Grundbegriffe einführen; oder den
Begriff der Zahl und bestimmte Zahlen.}


\PropositionG{4.1273}
{Wollen wir den allgemeinen Satz: \glqq{}$b$ ist ein
Nachfolger von $a$\grqq{} in der Begriffsschrift ausdrücken,
so brauchen wir hierzu einen Ausdruck
für das allgemeine Glied der Formenreihe: $aRb$,
$(\exists x) : aRx \DotOp xRb$, $(\exists x,y) : aRx \DotOp xRy \DotOp yRb$, \ldots{} Das
allgemeine Glied einer Formenreihe kann man nur
durch eine Variable ausdrücken, denn der Begriff:
Glied dieser Formenreihe, ist ein \Emph{formaler}
Begriff. (Dies haben Frege und Russell übersehen;
die Art und Weise wie sie allgemeine
Sätze, wie den obigen ausdrücken wollen ist daher
falsch; sie enthält einen circulus vitiosus.)

Wir können das allgemeine Glied der Formenreihe
bestimmen, indem wir ihr erstes Glied
angeben und die allgemeine Form der Operation,
welche das folgende Glied aus dem vorhergehenden
Satz erzeugt.}


\PropositionG{4.1274}
{Die Frage nach der Existenz eines formalen
Begriffes ist unsinnig. Denn kein Satz kann eine
solche Frage beantworten.

(Man kann also \zumBeispiel\ nicht fragen: \glqq{}Gibt es
unanalysierbare Sub\-jekt-Prä\-di\-kat\-sät\-ze?\grqq{})}


\PropositionG{4.128}
{Die logischen Formen sind zahl\EmphPart{los}.

Darum gibt es in der Logik keine ausgezeichneten
Zahlen und darum gibt es keinen philosophischen
Monismus oder Dualismus, etc.}


\PropositionG{4.2}
{Der Sinn des Satzes ist seine Übereinstimmung,
und Nichtübereinstimmung mit den Möglichkeiten
% -----File: 088.png---
des Bestehens und Nichtbestehens der
Sachverhalte.}


\PropositionG{4.21}
{Der einfachste Satz, der Elementarsatz, behauptet
das Bestehen eines Sachverhaltes.}


\PropositionG{4.211}
{Ein Zeichen des Elementarsatzes ist es, dass
kein Elementarsatz mit ihm in Widerspruch stehen
kann.}


\PropositionG{4.22}
{Der Elementarsatz besteht aus Namen. Er ist
ein Zusammenhang, eine Verkettung, von Namen.}


\PropositionG{4.221}
{Es ist offenbar, dass wir bei der Analyse der
Sätze auf Elementarsätze kommen müssen, die aus
Namen in unmittelbarer Verbindung bestehen.

Es frägt sich hier, wie kommt der Satzverband
zustande.}


\PropositionG{4.2211}
{Auch wenn die Welt unendlich komplex ist,
so dass jede Tatsache aus unendlich vielen Sachverhalten
besteht und jeder Sachverhalt aus unendlich
vielen Gegenständen zusammengesetzt ist,
auch dann müsste es Gegenstände und Sachverhalte
geben.}


\PropositionG{4.23}
{Der Name kommt im Satz nur im Zusammenhange
des Elementarsatzes vor.}


\PropositionG{4.24}
{Die Namen sind die einfachen Symbole, ich
deute sie durch einzelne Buchstaben (\glqq{}$x$\grqq{}, \glqq{}$y$\grqq{}, \glqq{}$z$\grqq{})
an.

Den Elementarsatz schreibe ich als Funktion
der Namen in der Form: \glqq{}$fx$\grqq{}, \glqq{}$\phi(x,y\DPtypo{,}{})$\grqq{}, etc.

Oder ich deute ihn durch die Buchstaben $p$, $q$,
$r$ an.}


\PropositionG{4.241}
{Gebrauche ich zwei Zeichen in ein und derselben
Bedeutung, so drücke ich dies aus, indem
ich zwischen beide das Zeichen \glqq{}$=$\grqq{} setze.

\glqq{}$a = b$\grqq{} heisst also: das Zeichen \glqq{}$a$\grqq{} ist durch
das Zeichen \glqq{}$b$\grqq{} ersetzbar.

(Führe ich durch eine Gleichung ein neues
Zeichen \glqq{}$b$\grqq{} ein, indem ich bestimme, es solle ein
bereits bekanntes Zeichen \glqq{}$a$\grqq{} ersetzen, so schreibe
ich die Gleichung---Definition---(wie Russell) in
% -----File: 090.png---
der Form \glqq{}$a = b$ Def.\grqq{}. Die Definition ist eine
Zeichenregel.)}


\PropositionG{4.242}
{Ausdrücke von der Form \glqq{}$a = b$\grqq{} sind also nur
Behelfe der Darstellung; sie sagen nichts über die
Bedeutung der Zeichen \glqq{}$a$\grqq{}, \glqq{}$b$\grqq{} aus.}


\PropositionG{4.243}
{Können wir zwei Namen verstehen, ohne zu
wissen, ob sie dasselbe Ding oder zwei verschiedene
Dinge bezeichnen?---Können wir einen Satz,
worin zwei Namen vorkommen, verstehen, ohne
zu wissen, ob sie Dasselbe oder Verschiedenes
bedeuten?

Kenne ich etwa die Bedeutung eines englischen
und eines gleichbedeutenden deutschen Wortes, so
ist es unmöglich, dass ich nicht weiss, dass die
beiden gleichbedeutend sind; es ist unmöglich,
dass ich sie nicht ineinander übersetzen kann.

Ausdrücke wie \glqq{}$a = a$\grqq{}, oder von diesen abgeleitete,
sind weder Elementarsätze, noch sonst sinnvolle
Zeichen. (Dies wird sich später zeigen.)}


\PropositionG{4.25}
{Ist der Elementarsatz wahr, so besteht der
Sachverhalt; ist der Elementarsatz falsch, so besteht
der Sachverhalt nicht.}


\PropositionG{4.26}
{Die Angabe aller wahren Elementarsätze beschreibt
die Welt vollständig. Die Welt ist
vollständig beschrieben durch die Angaben aller
Elementarsätze plus der Angabe, welche von ihnen
wahr und welche falsch sind.}


\PropositionG{4.27}
{Bezüglich des Bestehens und Nichtbestehens von
\enlargethispage{9pt} % enlarge to make the last line fit
$n$ Sachverhalten gibt es $K_{n} = \sum\limits_{\nu = 0}^n\binom{n}{\nu}$ Möglichkeiten.

Es können alle Kombinationen der Sachverhalte
bestehen, die andern nicht bestehen.}


\PropositionG{4.28}
{Diesen Kombinationen entsprechen ebenso viele
Möglichkeiten der Wahr\-heit---und Falschheit---von
$n$ Elementarsätzen.}


\PropositionG{4.3}
{Die Wahrheitsmöglichkeiten der Elementarsätze
bedeuten die Möglichkeiten des Bestehens und
Nichtbestehens der Sachverhalte.}
% -----File: 092.png---


\PropositionG{4.31}
{Die Wahrheitsmöglichkeiten können wir durch
Schemata folgender Art darstellen (\glqq{}W\grqq{} bedeutet
\glqq{}wahr\grqq{}, \glqq{}F\grqq{}, \glqq{}falsch\grqq{}. Die Reihen der \glqq{}W\grqq{} und
\glqq{}F\grqq{} unter der Reihe der Elementarsätze bedeuten
in leichtverständlicher Symbolik deren Wahrheitsmöglichkeiten):

\begin{center}
\begin{tabular}[t]{c|c|c}
p & q & r\\
\hline
\hline
\Strut W & W & W\\
\hline
\Strut F & W & W\\
\hline
\Strut W & F & W\\
\hline
\Strut W & W & F\\
\hline
\Strut F & F & W\\
\hline
\Strut F & W & F\\
\hline
\Strut W & F & F\\
\hline
\Strut F & F & F\\
\hline
\end{tabular}
\hspace{0.5cm}
\begin{tabular}[t]{c|c}
p & q\\
\hline
\hline
\Strut W & W\\
\hline
\Strut F & W\\
\hline
\Strut W & F\\
\hline
\Strut F & F\\
\hline
\end{tabular}
\hspace{0.5cm}
\begin{tabular}[t]{c}
p\\
\hline
\hline
\Strut W\\
\hline
\Strut F\\
\hline
\end{tabular}
\end{center}
}


\PropositionG{4.4}
{Der Satz ist der Ausdruck der Übereinstimmung
und Nichtübereinstimmung mit den Wahrheitsmöglichkeiten
der Elementarsätze.}


\PropositionG{4.41}
{Die Wahrheitsmöglichkeiten der Elementarsätze
sind die Bedingungen der Wahrheit und Falschheit
der Sätze.}


\PropositionG{4.411}
{{\stretchyspace
Es ist von vornherein wahrscheinlich, dass die
Einführung der Elementarsätze für das Verständnis
aller anderen Satzarten grundlegend ist. Ja, das
Verständnis der allgemeinen Sätze hängt \Emph{fühlbar}
von dem der Elementarsätze ab.}}


\PropositionG{4.42}
{Bezüglich der Übereinstimmung und Nichtübereinstimmung
eines Satzes mit den Wahrheitsmöglichkeiten
von $n$ Elementarsätzen gibt es
$\sum\limits_{\kappa = 0}^{K_n}\binom{K_n}{\kappa} = L_{n}$ Möglichkeiten.}


\PropositionG{4.43}
{Die Übereinstimmung mit den Wahrheitsmöglichkeiten
% -----File: 094.png---
können wir dadurch ausdrücken, indem
wir ihnen im Schema etwa das Abzeichen \glqq{}W\grqq{}
(wahr) zuordnen.

Das Fehlen dieses Abzeichens bedeutet die
Nichtübereinstimmung.}


\PropositionG{4.431}
{Der Ausdruck der Übereinstimmung und Nichtübereinstimmung
mit den Wahrheitsmöglichkeiten
der Elementarsätze drückt die Wahrheitsbedingungen
des Satzes aus.

Der Satz ist der Ausdruck seiner Wahrheitsbedingungen.

(Frege hat sie daher ganz richtig als Erklärung
der Zeichen seiner Begriffsschrift vorausgeschickt.
Nur ist die Erklärung des Wahrheitsbegriffes bei
Frege falsch: Wären \glqq{}das Wahre\grqq{} und \glqq{}das Falsche\grqq{}
wirklich Gegenstände und die Argumente in $\Not{p}$
etc.\ dann wäre nach Frege's Bestimmung der Sinn
von \glqq{}$\Not{p}$\grqq{} keineswegs bestimmt.)}


\PropositionG{4.44}
{Das Zeichen, welches durch die Zuordnung
jener Abzeichen \glqq{}W\grqq{} und der Wahrheitsmöglichkeiten
entsteht, ist ein Satzzeichen.}


\PropositionG{4.441}
{Es ist klar, dass dem Komplex der Zeichen
\glqq{}F\grqq{} und \glqq{}W\grqq{} kein Gegenstand (oder Komplex von
Gegenständen) entspricht; so wenig, wie den horizontalen
und vertikalen Strichen oder den Klammern.---\glqq{}Logische
Gegenstände\grqq{} gibt es nicht.

Analoges gilt natürlich für alle Zeichen, die dasselbe
ausdrücken wie die Schemata der \glqq{}W\grqq{} und \glqq{}F\grqq{}.}


\PropositionG{4.442}
{Es ist \zumBeispiel:\\
\phantom{Es ist \zumBeispiel:}
\raisebox{-2.3\baselineskip}{\glqq{}}\begin{tabular}{c|c|c}
p & q &\\
\hline
\hline
\Strut W & W & W\\
\hline
\Strut F & W & W\\
\hline
\Strut W & F &\\
\hline
\Strut F & F & W\\
\hline
\end{tabular}\\
\phantom{Es ist \zumBeispiel: \glqq{}\begin{tabular}{c|c|c}W&W&W\end{tabular}}
\smash[t]{\raisebox{5.8\baselineskip}{\grqq{}}} ein Satzzeichen.

(Frege's \glqq{}\DPtypo{Urteilstrich}{Urteilsstrich}\grqq{} \glqq{}$\vdash$\grqq{} ist logisch ganz
% -----File: 096.png---
bedeutungslos; er zeigt bei Frege (und Russell)
nur an, dass diese Autoren die so bezeichneten
Sätze für wahr halten. \glqq{}$\vdash$\grqq{} gehört daher ebenso
wenig zum Satzgefüge, wie etwa die Nummer des
Satzes. Ein Satz kann unmöglich von sich selbst
aussagen, dass er wahr ist.)

Ist die Reihenfolge der Wahrheitsmöglichkeiten
im Schema durch eine Kombinationsregel ein für
allemal festgesetzt, dann ist die letzte Kolonne
allein schon ein Ausdruck der Wahrheitsbedingungen.
Schreiben wir diese Kolonne als Reihe
hin, so wird das Satzzeichen zu:

\glqq{}(WW--W)($p$, $q$)\grqq{} oder deutlicher \glqq{}(WWFW)($p$, $q$)\grqq{}.

(Die Anzahl der Stellen in der linken Klammer
ist durch die Anzahl der Glieder in der rechten
bestimmt.)}


\PropositionG{4.45}
{Für $n$ Elementarsätze gibt es $L_{n}$ mögliche Gruppen
von Wahrheitsbedingungen.

{\stretchyspace
Die Gruppen von Wahrheitsbedingungen,
welche zu den Wahrheitsmöglichkeiten einer
Anzahl von Elementarsätzen gehören, lassen sich
in eine Reihe ordnen.}}


\PropositionG{4.46}
{Unter den möglichen Gruppen von Wahrheitsbedingungen
gibt es zwei extreme Fälle.

In dem einen Fall ist der Satz für sämtliche
Wahrheitsmöglichkeiten der Elementarsätze wahr.
Wir sagen, die Wahrheitsbedingungen sind
\Emph{tautologisch}.

Im zweiten Fall ist der Satz für sämtliche
Wahrheitsmöglichkeiten falsch: Die Wahrheitsbedingungen
sind \Emph{kontradiktorisch}.

Im ersten Fall nennen wir den Satz eine
Tautologie, im zweiten Fall eine Kontradiktion.}


\PropositionG{4.461}
{Der Satz zeigt was er sagt, die Tautologie und
die Kontradiktion, dass sie nichts sagen.

Die Tautologie hat keine Wahrheitsbedingungen,
denn sie ist bedingungslos wahr; und
% -----File: 098.png---
die Kontradiktion ist unter keiner Bedingung
wahr.

Tautologie und Kontradiktion sind sinnlos.

(Wie der Punkt von dem zwei Pfeile in
entgegengesetzter Richtung auseinandergehen.)

(Ich weiss \zumBeispiel\ nichts über das Wetter, wenn
ich weiss, dass es regnet oder nicht regnet.)}


\PropositionG{4.4611}
{Tautologie und Kontradiktion sind aber nicht
unsinnig; sie gehören zum Symbolismus, und
zwar ähnlich wie die \glqq{}0\grqq{} zum Symbolismus der
Arithmetik.}


\PropositionG{4.462}
{Tautologie und Kontradiktion sind nicht Bilder
der Wirklichkeit. Sie stellen keine mögliche
Sachlage dar. Denn jene lässt \Emph{jede} mögliche
Sachlage zu, diese \Emph{keine}.

In der Tautologie heben die Bedingungen der
Übereinstimmung mit der Welt---die darstellenden
Beziehungen---einander auf, so dass sie in keiner
darstellenden Beziehung zur Wirklichkeit steht.}


\PropositionG{4.463}
{Die Wahrheitsbedingungen bestimmen den
Spielraum, der den Tatsachen durch den Satz
gelassen wird.

(Der Satz, das Bild, das Modell, sind im
negativen Sinne wie ein fester Körper, der die
Bewegungsfreiheit der anderen beschränkt; im
positiven Sinne, wie der von fester Substanz
begrenzte Raum, worin ein Körper Platz hat.)

Die Tautologie lässt der Wirklichkeit den gan\-zen---un\-end\-li\-chen---lo\-gi\-schen
Raum; die Kontradiktion
erfüllt den ganzen logischen Raum und lässt
der Wirklichkeit keinen Punkt. Keine von beiden
kann daher die Wirklichkeit irgendwie bestimmen.}


\PropositionG{4.464}
{Die Wahrheit der Tautologie ist gewiss, des
Satzes möglich, der Kontradiktion unmöglich.

(Gewiss, möglich, unmöglich: Hier haben wir
das Anzeichen jener Gradation, die wir in der
Wahrscheinlichkeitslehre brauchen.)}


\PropositionG{4.465}
{Das logische Produkt einer Tautologie und
% -----File: 100.png---
eines Satzes sagt dasselbe, wie der Satz. Also ist
jenes Produkt identisch mit dem Satz. Denn man
kann das Wesentliche des Symbols nicht ändern,
ohne seinen Sinn zu ändern.}


\PropositionG{4.466}
{Einer bestimmten logischen Verbindung von
Zeichen entspricht eine bestimmte logische Verbindung
ihrer Bedeutungen; \Emph{jede beliebige}
Verbindung entspricht nur den unverbundenen
Zeichen.

Das heisst, Sätze die für jede Sachlage wahr
sind, können überhaupt keine Zeichenverbindungen
sein, denn sonst könnten ihnen nur bestimmte
Verbindungen von Gegenständen entsprechen.

(Und keiner logischen Verbindung entspricht
\Emph{keine} Verbindung der Gegenstände.)

Tautologie und Kontradiktion sind die Grenzfälle
der Zeichenverbindung, nämlich ihre Auflösung.}


\PropositionG{4.4661}
{Freilich sind auch in der Tautologie und Kontradiktion
die Zeichen noch mit einander verbunden,
\dasHeiszt\ sie stehen in Beziehungen zu einander,
aber diese Beziehungen sind bedeutungslos, dem
\Emph{Symbol} unwesentlich.}


\PropositionG{4.5}
{Nun scheint es möglich zu sein, die allgemeinste
Satzform anzugeben: das heisst, eine Beschreibung
der Sätze \Emph{irgend einer} Zeichensprache zu geben,
so dass jeder mögliche Sinn durch ein Symbol,
auf welches die Beschreibung passt, ausgedrückt
werden kann, und dass jedes Symbol, worauf die
Beschreibung passt, einen Sinn ausdrücken kann,
wenn die Bedeutungen der Namen entsprechend
gewählt werden.

Es ist klar, dass bei der Beschreibung der
allgemeinsten Satzform \Emph{nur} ihr Wesentliches
beschrieben werden darf,---sonst wäre sie nämlich
nicht die allgemeinste.

Dass es eine allgemeine Satzform gibt, wird
dadurch bewiesen, dass es keinen Satz geben darf,
dessen Form man nicht hätte voraussehen (\dasHeiszt\ konstruieren)
% -----File: 102.png---
können. Die allgemeine Form des
Satzes ist: Es verhält sich so und so.}


\PropositionG{4.51}
{Angenommen, mir wären \Emph{alle} Elementarsätze
gegeben: Dann lässt sich einfach fragen: welche
Sätze kann ich aus ihnen bilden. Und das sind
\Emph{alle} Sätze und \Emph{so} sind sie begrenzt.}


\PropositionG{4.52}
{Die Sätze sind Alles, was aus der Gesamtheit
aller Elementarsätze folgt (natürlich auch daraus,
dass es die \Emph{Gesamtheit aller} ist). (So könnte
man in gewissem Sinne sagen, dass \Emph{alle} Sätze
Verallgemeinerungen der Elementarsätze sind.)}


\PropositionG{4.53}
{Die allgemeine Satzform ist eine Variable.}


\PropositionG{5}
{Der Satz ist eine Wahrheitsfunktion der Elementarsätze.

(Der Elementarsatz ist eine Wahrheitsfunktion
seiner selbst.)}


\PropositionG{5.01}
{Die Elementarsätze sind die Wahrheitsargumente
des Satzes.}


\PropositionG{5.02}
{Es liegt nahe, die Argumente von Funktionen
mit den Indices von Namen zu verwechseln. Ich
erkenne nämlich sowohl am Argument wie am
Index die Bedeutung des sie enthaltenden Zeichens.

In Russell's \glqq{}$+_{c}$\grqq{} ist \zumBeispiel\ \glqq{}$c$\grqq{} ein Index, der darauf
hinweist, dass das ganze Zeichen das Additionszeichen
für Kardinalzahlen ist. Aber diese Bezeichnung
beruht auf willkürlicher Übereinkunft und
man könnte statt \glqq{}$+_{c}$\grqq{} auch ein einfaches Zeichen
wählen; in \glqq{}$\Not{p}$\grqq{} aber ist \glqq{}$p$\grqq{} kein Index, sondern
ein Argument: der Sinn von \glqq{}$\Not{p}$\grqq{} \Emph{kann nicht}
verstanden werden, ohne dass vorher der Sinn von
\glqq{}$p$\grqq{} verstanden worden wäre. (Im Namen Julius
Cäsar ist \glqq{}Julius\grqq{} ein Index. Der Index ist immer
ein Teil einer Beschreibung des Gegenstandes,
dessen Namen wir ihn anhängen. \ZumBeispiel\ \Emph{Der}
Cäsar aus dem Geschlechte der Julier.)

Die Verwechslung von Argument und Index
liegt, wenn ich mich nicht irre, der Theorie Frege's
von der Bedeutung der Sätze und Funktionen
% -----File: 104.png---
zugrunde. Für Frege waren die Sätze der Logik
Namen, und deren Argumente die Indices dieser
Namen.}


\PropositionG{5.1}
{Die Wahrheitsfunktionen lassen sich in Reihen
ordnen.

Das ist die Grundlage der Wahrscheinlichkeitslehre.}


\PropositionG{5.101}
{Die Wahrheitsfunktionen jeder Anzahl von
Elementarsätzen lassen sich in einem Schema
folgender Art hinschreiben:

\begin{table*}[!h]
\footnotesize\noindent\centering
\begin{tabular}{@{}c@{~}l@{~}l@{}}
(\Wahr\Wahr\Wahr\Wahr)($p, q$) & Tautologie & (Wenn $p$, so $p$; und wenn $q$, so $q$.) ($p \Implies p \DotOp q \Implies q$)\\
(\False\Wahr\Wahr\Wahr)($p, q$) & in Worten: & Nicht beides $p$ und $q$. ($\Not{(p \DotOp q)}$)\\
(\Wahr\False\Wahr\Wahr)($p, q$) & \DittoInWorten & Wenn $q$, so $p$. ($q \Implies p$)\\
(\Wahr\Wahr\False\Wahr)($p, q$) & \DittoInWorten & Wenn $p$, so $q$. ($p \Implies q$)\\
(\Wahr\Wahr\Wahr\False)($p, q$) & \DittoInWorten & $p$ oder $q$. ($p \lor q$)\\
(\False\False\Wahr\Wahr)($p, q$) & \DittoInWorten & Nicht $q$. ($\Not{q}$)\\
(\False\Wahr\False\Wahr)($p, q$) & \DittoInWorten & Nicht $p$. ($\Not{p}$)\\
(\False\Wahr\Wahr\False)($p, q$) & \DittoInWorten & $p$, oder $q$, aber nicht beide. ($p \DotOp \Not{q} : \lor : q \DotOp \Not{p}$)\\
(\Wahr\False\False\Wahr)($p, q$) & \DittoInWorten & Wenn $p$, so $q$; und wenn $q$, so $p$. ($p \equiv q$)\\
(\Wahr\False\Wahr\False)($p, q$) & \DittoInWorten & $p$\\
(\Wahr\Wahr\False\False)($p, q$) & \DittoInWorten & $q$\\
(\False\False\False\Wahr)($p, q$) & \DittoInWorten & Weder $p$ noch $q$. ($\Not{p} \DotOp \Not{q}$) oder ($p \BarOp q$)\\
(\False\False\Wahr\False)($p, q$) & \DittoInWorten & $p$ und nicht $q$. ($p \DotOp \Not{q}$)\\
(\False\Wahr\False\False)($p, q$) & \DittoInWorten & $q$ und nicht $p$. ($q \DotOp \Not{p}$)\\
(\Wahr\False\False\False)($p, q$) & \DittoInWorten & $q$ und $p$. ($q \DotOp p$)\\
(\False\False\False\False)($p, q$) & Kontradiktion & ($p$ und nicht $p$; und $q$ und nicht $q$.) ($p \DotOp \Not{p} \DotOp q \DotOp \Not{q}$)\\
\end{tabular}
\end{table*}

{\verystretchyspace
Diejenigen Wahrheitsmöglichkeiten seiner
Wahrheitsargumente, welche den Satz bewahrheiten,
will ich seine \Emph{Wahrheitsgründe}
nennen.}}


\PropositionG{5.11}
{Sind die Wahrheitsgründe, die einer Anzahl
von Sätzen gemeinsam sind, sämtlich auch Wahrheitsgründe
eines bestimmten Satzes, so sagen
wir, die Wahrheit dieses Satzes folge aus der
Wahrheit jener Sätze.}


\PropositionG{5.12}
{Insbesondere folgt die Wahrheit eines Satzes
\glqq{}$p$\grqq{} aus der Wahrheit eines anderen \glqq{}$q$\grqq{}, wenn
alle Wahrheitsgründe des zweiten Wahrheitsgründe
des ersten sind.}
% -----File: 106.png---


\PropositionG{5.121}
{Die Wahrheitsgründe des einen sind in denen
des anderen enthalten; $p$ folgt aus $q$.}


\PropositionG{5.122}
{Folgt $p$ aus $q$, so ist der Sinn von \glqq{}$p$\grqq{} im
Sinne von \glqq{}$q$\grqq{} enthalten.}


\PropositionG{5.123}
{Wenn ein Gott eine Welt erschafft, worin
gewisse Sätze wahr sind, so schafft er damit auch
schon eine Welt, in welcher alle ihre Folgesätze
stimmen. Und ähnlich könnte er keine Welt
schaffen, worin der Satz \glqq{}$p$\grqq{} wahr ist, ohne seine
sämtlichen Gegenstände zu schaffen.}


\PropositionG{5.124}
{Der Satz bejaht jeden Satz der aus ihm
folgt.}


\PropositionG{5.1241}
{\glqq{}$p \DotOp q$\grqq{} ist einer der Sätze, welche \glqq{}$p$\grqq{} bejahen
und zugleich einer der Sätze, welche \glqq{}$q$\grqq{}
bejahen.

Zwei Sätze sind einander entgegengesetzt, wenn
es keinen sinnvollen Satz gibt, der sie beide
bejaht.

Jeder Satz der einem anderen widerspricht,
verneint ihn.}


\PropositionG{5.13}
{Dass die Wahrheit eines Satzes aus der Wahrheit
anderer Sätze folgt, ersehen wir aus der
Struktur der Sätze.}


\PropositionG{5.131}
{Folgt die Wahrheit eines Satzes aus der Wahrheit
anderer, so drückt sich dies durch Beziehungen
aus, in welchen die Formen jener Sätze zu
einander stehen; und zwar brauchen wir sie nicht
erst in jene Beziehungen zu setzen, indem wir
sie in einem Satze miteinander verbinden, sondern
diese Beziehungen sind intern und bestehen, sobald,
und dadurch dass, jene Sätze bestehen.}


\PropositionG{5.1311}
{Wenn wir von $p \lor q$ und $\Not{p}$ auf $q$ schliessen,
so ist hier durch die Bezeichnungsweise die Beziehung
der Satzformen von \glqq{}$p \lor q$\grqq{} und \glqq{}$\Not{p}$\grqq{} verhüllt.
Schreiben wir aber \zumBeispiel\ statt \glqq{}$p \lor q$\grqq{}
\glqq{}$p \BarOp q \DotOp \BarOp \DotOp p \BarOp q$\grqq{} und statt \glqq{}$\Not{p}$\grqq{} \glqq{}$p \BarOp p$\grqq{} ($p \BarOp q$ = weder
$p$, noch $q$), so wird der innere Zusammenhang
offenbar.
% -----File: 108.png---

(Dass man aus $(x) \DotOp fx$ auf $fa$ schliessen kann,
das zeigt, dass die Allgemeinheit auch im Symbol
\glqq{}$(x) \DotOp fx$\grqq{} vorhanden ist.)}


\PropositionG{5.132}
{Folgt $p$ aus $q$, so kann ich von $q$ auf $p$ schliessen;
$p$ aus $q$ folgern.

Die Art des Schlusses ist allein aus den beiden
Sätzen zu entnehmen.

Nur sie selbst können den Schluss rechtfertigen.

{\stretchyspace
\glqq{}Schlussgesetze\grqq{}, welche---wie bei Frege und
Russell---die Schlüsse rechtfertigen sollen, sind
sinnlos, und wären überflüssig.}}


\PropositionG{5.133}
{Alles Folgern geschieht a priori.}


\PropositionG{5.134}
{Aus einem Elementarsatz lässt sich kein anderer
folgern.}


\PropositionG{5.135}
{Auf keine Weise kann aus dem Bestehen irgend
einer Sachlage auf das Bestehen einer, von ihr gänzlich
verschiedenen Sachlage geschlossen werden.}


\PropositionG{5.136}
{Einen Kausalnexus, der einen solchen Schluss
rechtfertigte, gibt es nicht.}


\PropositionG{5.1361}
{Die Ereignisse der Zukunft \Emph{können} wir nicht
aus den gegenwärtigen erschliessen.

Der Glaube an den Kausalnexus ist der \Emph{Aberglaube}.}


\PropositionG{5.1362}
{Die Willensfreiheit besteht darin, dass zukünftige
Handlungen jetzt nicht gewusst werden können.
Nur dann könnten wir sie wissen, wenn die Kausalität
eine \Emph{innere} Notwendigkeit wäre, wie die
des logischen Schlusses.---Der Zusammenhang
von Wissen und Gewusstem, ist der der logischen
Notwendigkeit.

(\glqq{}A weiss, dass $p$ der Fall ist\grqq{} ist sinnlos, wenn
$p$ eine Tautologie ist.)}


\PropositionG{5.1363}
{Wenn daraus, dass ein Satz uns einleuchtet,
nicht \Emph{folgt}, dass er wahr ist, so ist das Einleuchten
auch keine Rechtfertigung für unseren
Glauben an seine Wahrheit.}


\PropositionG{5.14}
{Folgt ein Satz aus einem anderen, so sagt
dieser mehr als jener, jener weniger als dieser.}
% -----File: 110.png---


\PropositionG{5.141}
{Folgt $p$ aus $q$ und $q$ aus $p$, so sind sie ein und
derselbe Satz.}


\PropositionG{5.142}
{Die Tautologie folgt aus allen Sätzen: sie sagt
Nichts.}


\PropositionG{5.143}
{Die Kontradiktion ist das Gemeinsame der
Sätze, was \Emph{kein} Satz mit einem anderen gemein
hat. Die Tautologie ist das Gemeinsame aller
Sätze, welche nichts miteinander gemein haben.

Die Kontradiktion verschwindet sozusagen
ausserhalb, die Tautologie innerhalb aller Sätze.

Die Kontradiktion ist die äussere Grenze der
Sätze, die Tautologie ihr substanzloser Mittelpunkt.}


\PropositionG{5.15}
{Ist $W_{r}$ die Anzahl der Wahrheitsgründe des
Satzes \glqq{}$r$\grqq{}, $W_{rs}$ die Anzahl derjenigen Wahrheitsgründe
des Satzes \glqq{}$s$\grqq{}, die zugleich Wahrheitsgründe
von \glqq{}$r$\grqq{} sind, dann nennen wir das Verhältnis: $W_{rs} :
W_{r}$ das Mass der \Emph{Wahrscheinlichkeit}, welche
der Satz \glqq{}$r$\grqq{} dem Satz \glqq{}$s$\grqq{} gibt.}


\PropositionG{5.151}
{Sei in einem Schema wie dem obigen in No.~\PropGRef{5.101}
$W_{r}$ die Anzahl der \glqq{}$W$\grqq{} im Satze $r$; $W_{rs}$ die
Anzahl derjenigen \glqq{}$W$\grqq{} im Satze $s$, die in gleichen
Kolonnen mit \glqq{}$W$\grqq{} des Satzes $r$ stehen. Der Satz
$r$ gibt dann dem Satze $s$ die Wahrscheinlichkeit:
$W_{rs} : W_{r}$.}


\PropositionG{5.1511}
{Es gibt keinen besonderen Gegenstand, der den
Wahrscheinlichkeitssätzen eigen wäre.}


\PropositionG{5.152}
{Sätze, welche keine Wahrheitsargumente mit
einander gemein haben, nennen wir von einander
unabhängig.

Von einander unabhängige Sätze (\zumBeispiel\ irgend
zwei Elementarsätze) geben einander die Wahrscheinlichkeit~$\frac{1}{2}$.

Folgt $p$ aus $q$, so gibt der Satz \glqq{}$q$\grqq{} dem Satz
\glqq{}$p$\grqq{} die Wahrscheinlichkeit~1. Die Gewissheit
des logischen Schlusses ist ein Grenzfall der
Wahrscheinlichkeit.

(Anwendung auf Tautologie und Kontradiktion.)}


\PropositionG{5.153}
{Ein Satz ist an sich weder wahrscheinlich noch
% -----File: 112.png---
unwahrscheinlich. Ein Ereignis trifft ein, oder
es trifft nicht ein, ein Mittelding gibt es nicht.}


\PropositionG{5.154}
{In einer Urne seien gleichviel weisse und
schwarze Kugeln (und keine anderen). Ich ziehe
eine Kugel nach der anderen und lege sie wieder
in die Urne zurück. Dann kann ich durch den
Versuch feststellen, dass sich die Zahlen der
gezogenen schwarzen und weissen Kugeln bei
fortgesetztem Ziehen einander nähern.

\Emph{Das} ist also kein mathematisches Faktum.

Wenn ich nun sage: Es ist gleich wahrscheinlich,
dass ich eine weisse Kugel wie eine
schwarze ziehen werde, so heisst das: Alle mir
bekannten Umstände (die hypothetisch angenommenen
Naturgesetze mitinbegriffen) geben dem
Eintreffen des einen Ereignisses nicht \Emph{mehr}
Wahrscheinlichkeit als dem Eintreffen des anderen.
Das heisst, sie geben---wie aus den obigen Erklärungen
leicht zu entnehmen ist---jedem die
Wahrscheinlichkeit~$\frac{1}{2}$.

Was ich durch den Versuch bestätige ist, dass
das Eintreffen der beiden Ereignisse von den Umständen,
die ich nicht näher kenne, unabhängig ist.}


\PropositionG{5.155}
{Die Einheit des Wahrscheinlichkeitssatzes ist:
Die Umstände---die ich sonst nicht weiter kenne---geben
dem Eintreffen eines bestimmten Ereignisses
den und den Grad der Wahrscheinlichkeit.}


\PropositionG{5.156}
{So ist die Wahrscheinlichkeit eine Verallgemeinerung.

Sie involviert eine allgemeine Beschreibung
einer Satzform.

Nur in Ermanglung der Gewissheit gebrauchen
wir die Wahr\-schein\-lich\-keit.---Wenn wir zwar eine
Tatsache nicht vollkommen kennen, wohl aber
\Emph{etwas} über ihre Form wissen.

(Ein Satz kann zwar ein unvollständiges Bild
einer gewissen Sachlage sein, aber er ist immer
\Emph{ein} vollständiges Bild.)
% -----File: 114.png---

Der Wahrscheinlichkeitssatz ist gleichsam ein
Auszug aus anderen Sätzen.}


\PropositionG{5.2}
{Die Strukturen der Sätze stehen in internen
Beziehungen zu einander.}


\PropositionG{5.21}
{Wir können diese internen Beziehungen
dadurch in unserer Ausdrucksweise hervorheben,
dass wir einen Satz als Resultat einer Operation
darstellen, die ihn aus anderen Sätzen (den Basen
der Operation) hervorbringt.}


\PropositionG{5.22}
{Die Operation ist der Ausdruck einer Beziehung
zwischen den Strukturen ihres Resultats und ihrer
Basen.}


\PropositionG{5.23}
{Die Operation ist das, was mit dem einen Satz
geschehen muss, um aus ihm den anderen zu machen.}


\PropositionG{5.231}
{Und das wird natürlich von ihren formalen
Eigenschaften, von der internen Ähnlichkeit ihrer
Formen abhängen.}


\PropositionG{5.232}
{Die interne Relation, die eine Reihe ordnet, ist
äquivalent mit der Operation, durch welche ein
Glied aus dem anderen entsteht.}


\PropositionG{5.233}
{Die Operation kann erst dort auftreten, wo ein
Satz auf logisch bedeutungsvolle Weise aus einem
anderen entsteht. Also dort, wo die logische
Konstruktion des Satzes anfängt.}


\PropositionG{5.234}
{Die Wahrheitsfunktionen der Elementarsätze
sind Resultate von Operationen, die die Elementarsätze
als Basen haben. (Ich nenne diese Operationen
Wahrheitsoperationen.)}


\PropositionG{5.2341}
{Der Sinn einer Wahrheitsfunktion von $p$ ist
eine Funktion des Sinnes von $p$.

Verneinung, logische Addition, logische Multiplikation,
etc., etc.\ sind Operationen.

(Die Verneinung verkehrt den Sinn des Satzes.)}


\PropositionG{5.24}
{Die Operation zeigt sich in einer Variablen;
sie zeigt, wie man von einer Form von Sätzen zu
einer anderen gelangen kann.

Sie bringt den Unterschied der Formen zum
Ausdruck.
% -----File: 116.png---

(Und das Gemeinsame zwischen den Basen
und dem Resultat der Operation sind eben die
Basen.)}


\PropositionG{5.241}
{Die Operation kennzeichnet keine Form, sondern
nur den Unterschied der Formen.}


\PropositionG{5.242}
{Dieselbe Operation, die \glqq{}$q$\grqq{} aus \glqq{}$p$\grqq{} macht,
macht aus \glqq{}$q$\grqq{} \glqq{}$r$\grqq{} \undSoFort{} Dies kann nur darin
ausgedrückt sein, dass \glqq{}$p$\grqq{}, \glqq{}$q$\grqq{}, \glqq{}$r$\grqq{}, etc.\ Variable
sind, die gewisse formale Relationen allgemein
zum Ausdruck bringen.}


\PropositionG{5.25}
{Das Vorkommen der Operation charakterisiert
den Sinn des Satzes nicht.

Die Operation sagt ja nichts aus, nur ihr Resultat,
und dies hängt von den Basen der Operation
ab.

(Operation und Funktion dürfen nicht miteinander
verwechselt werden.)}


\PropositionG{5.251}
{Eine Funktion kann nicht ihr eigenes Argument
sein, wohl aber kann das Resultat einer Operation
ihre eigene Basis werden.}


\PropositionG{5.252}
{Nur so ist das Fortschreiten von Glied zu Glied
in einer Formenreihe (von Type zu Type in den
Hierarchien Russells und Whiteheads) möglich.
(Russell und Whitehead haben die Möglichkeit
dieses Fortschreitens nicht zugegeben, aber immer
wieder von ihr Gebrauch gemacht.)}


\PropositionG{5.2521}
{Die fortgesetzte Anwendung einer Operation
\enlargethispage{2pt} % enlarge to make one more line fit
auf ihr eigenes Resultat nenne ich ihre successive
Anwendung (\glqq{}$O' O' O' a$\grqq{} ist das Resultat der
dreimaligen successiven Anwendung von \glqq{}$O' \xi$\grqq{}
auf \glqq{}$a$\grqq{}).

In einem ähnlichen Sinne rede ich von der
successiven Anwendung \Emph{mehrerer} Operationen
auf eine Anzahl von Sätzen.}


\PropositionG{5.2522}
{{\stretchyspace
Das allgemeine Glied einer Formenreihe $a$, $O' a$,
$O' O' a$, $\fourdots$ schreibe ich daher so: \glqq{}$[a, x, O' x]$\grqq{}.
Dieser Klammerausdruck ist eine Variable. Das
erste Glied des Klammerausdruckes ist der Anfang
% -----File: 118.png---
der Formenreihe, das zweite die Form eines
beliebigen Gliedes $x$ der Reihe und das dritte
die Form desjenigen Gliedes der Reihe, welches
auf $x$ unmittelbar folgt.}}


\PropositionG{5.2523}
{Der Begriff der successiven Anwendung der
Operation ist äquivalent mit dem Begriff \glqq{}und so
weiter\grqq{}.}


\PropositionG{5.253}
{Eine Operation kann die Wirkung einer anderen
rückgängig machen. Operationen können einander
aufheben.}


\PropositionG{5.254}
{\stretchyspace
{Die Operation kann verschwinden (\zumBeispiel\ die
Verneinung in \glqq{}$\Not{\Not{p}}$\grqq{}\DPtypo{}{,} $\Not{\Not{p}}$ $= p$).}}


\PropositionG{5.3}
{Alle Sätze sind Resultate von Wahrheitsoperationen
mit den Elementarsätzen.

Die Wahrheitsoperation ist die Art und Weise,
wie aus den Elementarsätzen die Wahrheitsfunktion
entsteht.

{\verystretchyspace
Nach dem Wesen der Wahrheitsoperation wird
auf die gleiche Weise, wie aus den Elementarsätzen
ihre Wahrheitsfunktion, aus Wahrheitsfunktionen
eine Neue. Jede Wahrheitsoperation erzeugt aus
Wahrheitsfunktionen von Elementarsätzen wieder
eine Wahrheitsfunktion von Elementarsätzen, einen
Satz. Das Resultat jeder Wahrheitsoperation mit
den Resultaten von Wahrheitsoperationen mit
Elementarsätzen ist wieder das Resultat \Emph{Einer}
Wahrheitsoperation mit Elementarsätzen.}

Jeder Satz ist das Resultat von Wahrheitsoperationen
mit Elementarsätzen.}


\PropositionG{5.31}
{Die Schemata No.~\PropGRef{4.31} haben auch dann eine
Bedeutung, wenn \glqq{}$p$\grqq{}, \glqq{}$q$\grqq{}, \glqq{}$r$\grqq{}, etc.\ nicht Elementarsätze
sind.

{\verystretchyspace
Und es ist leicht zu sehen, dass das Satzzeichen
in No.~\DPtypo{\PropGRef{4.42}}{\PropGRef{4.442}}, auch wenn \glqq{}$p$\grqq{} und \glqq{}$q$\grqq{} Wahrheitsfunktionen
von Elementarsätzen sind, Eine
Wahrheitsfunktion von Elementarsätzen ausdrückt.}}


\PropositionG{5.32}
{Alle Wahrheitsfunktionen sind Resultate der
% -----File: 120.png---
successiven Anwendung einer endlichen Anzahl
von Wahrheitsoperationen auf die Elementarsätze.}


\PropositionG{5.4}
{Hier zeigt es sich, dass es \glqq{}logische Gegenstände\grqq{},
\glqq{}logische Konstante\grqq{} (im Sinne Freges
und Russells) nicht gibt.}


\PropositionG{5.41}
{Denn: Alle Resultate von Wahrheitsoperationen
mit Wahrheitsfunktionen sind identisch,
welche eine und dieselbe Wahrheitsfunktion von
Elementarsätzen sind.}


\PropositionG{5.42}
{Dass $\lor$, $\Implies$, etc.\ nicht Beziehungen im Sinne von
rechts und links etc.\ sind, leuchtet ein.

Die Möglichkeit des kreuzweisen Definierens der
logischen \glqq{}Urzeichen\grqq{} Freges und Russells zeigt
schon, dass dies keine Urzeichen sind, und schon
erst recht, dass sie keine Relationen bezeichnen.

Und es ist offenbar, dass das \glqq{}$\Implies$\grqq{}, welches wir
durch \glqq{}$\Not{}$\grqq{} und \glqq{}$\lor$\grqq{} definieren, identisch ist mit dem,
durch welches wir \glqq{}$\lor$\grqq{} mit \glqq{}$\Not{}$\grqq{} definieren und dass
dieses \glqq{}$\lor$\grqq{} mit dem ersten identisch ist. \UndSoWeiter}


\PropositionG{5.43}
{Dass aus einer Tatsache $p$ unendlich viele
\Emph{andere} folgen sollten, nämlich $\Not{\Not{p}}$, $\Not{\Not{\Not{\Not{p}}}}$,
etc., ist doch von vornherein kaum zu glauben.
Und nicht weniger merkwürdig ist, dass die unendliche
Anzahl der Sätze der Logik (der Mathematik)
aus einem halben Dutzend \glqq{}Grundgesetzen\grqq{} folgen.

Alle Sätze der Logik sagen aber dasselbe. Nämlich
Nichts.}


\PropositionG{5.44}
{Die Wahrheitsfunktionen sind keine materiellen
Funktionen.

Wenn man \zumBeispiel\ eine Bejahung durch doppelte
Verneinung erzeugen kann, ist dann die Verneinung---in
irgend einem Sinn\AllowBreak---in der Bejahung enthalten?
Verneint \glqq{}$\Not{\Not{p}}$\grqq{} $\Not{p}$, oder bejaht es $p$; oder beides?

Der Satz \glqq{}$\Not{\Not{p}}$\grqq{} handelt nicht von der Verneinung
wie von einem Gegenstand; wohl aber ist
die Möglichkeit der Verneinung in der Bejahung
bereits präjudiziert.

Und gäbe es einen Gegenstand, der \glqq{}$\Not{}$\grqq{} hiesse,
% -----File: 122.png---
so müsste \glqq{}$\Not{\Not{p}}$\grqq{} etwas anderes sagen als \glqq{}$p$\grqq{}.
Denn der eine Satz würde dann eben von $\Not{}$
handeln, der andere nicht.}


\PropositionG{5.441}
{Dieses Verschwinden der scheinbaren logischen
Konstanten tritt auch ein, wenn \glqq{}$\Not{(\exists x) \DotOp \Not{fx}}$\grqq{}
dasselbe sagt wie \glqq{}$(x) \DotOp fx$\grqq{}, oder \glqq{}$(\exists x) \DotOp fx \DotOp x = a$\grqq{}
dasselbe wie \glqq{}$fa$\grqq{}.}


\PropositionG{5.442}
{Wenn uns ein Satz gegeben ist, so sind \Emph{mit
ihm} auch schon die Resultate aller Wahrheitsoperationen,
die ihn zur Basis haben, gegeben.}


\PropositionG{5.45}
{Gibt es logische Urzeichen, so muss eine richtige
Logik ihre Stellung zueinander klar machen und
ihr Dasein rechtfertigen. Der Bau der Logik \Emph{aus}
ihren Urzeichen muss klar werden.}


\PropositionG{5.451}
{Hat die Logik Grundbegriffe, so müssen sie von
einander unabhängig sein. Ist ein Grundbegriff
eingeführt, so muss er in allen Verbindungen
eingeführt sein, worin er überhaupt vorkommt. Man
kann ihn also nicht zuerst für \Emph{eine} Verbindung,
dann noch einmal für eine andere einführen.
\ZumBeispiel: Ist die Verneinung eingeführt, so müssen
wir sie jetzt in Sätzen von der Form \glqq{}$\Not{p}$\grqq{} ebenso
verstehen, wie in Sätzen wie \glqq{}$\Not{(p \lor q)}$\grqq{}, \glqq{}$(\exists x) \DotOp \Not{fx}$\grqq{}~\undAndere\
Wir dürfen sie nicht erst für die eine Klasse
von Fällen, dann für die andere einführen, denn es
bliebe dann zweifelhaft, ob ihre Bedeutung in beiden
Fällen die gleiche wäre und es wäre kein Grund
vorhanden, in beiden Fällen dieselbe Art der
Zeichenverbindung zu benützen.

(Kurz, für die Einführung der Urzeichen gilt,
mutatis mutandis, dasselbe, was Frege (\glqq{}Grundgesetze
der Arithmetik\grqq{}) für die Einführung von
Zeichen durch Definitionen gesagt hat.)}


\PropositionG{5.452}
{Die Einführung eines neuen Behelfes in den Symbolismus
der Logik muss immer ein folgenschweres
Ereignis sein. Kein neuer Behelf darf in die Logik---sozusagen,
mit ganz unschuldiger Miene---in Klammern
oder unter dem Striche eingeführt werden.
% -----File: 124.png---

(So kommen in den \glqq{}Principia Mathematica\grqq{}
von Russell und Whitehead Definitionen und
Grundgesetze in Worten vor. Warum hier plötzlich
Worte? Dies bedürfte einer Rechtfertigung.
Sie fehlt und muss fehlen, da das Vorgehen tatsächlich
unerlaubt ist.)

Hat sich aber die Einführung eines neuen
Behelfes an einer Stelle als nötig erwiesen, so muss
man sich nun sofort fragen: Wo muss dieser
Behelf nun \Emph{immer} angewandt werden? Seine
Stellung in der Logik muss nun erklärt werden.}


\PropositionG{5.453}
{Alle Zahlen der Logik müssen sich rechtfertigen
lassen.

Oder vielmehr: Es muss sich herausstellen,
dass es in der Logik keine Zahlen gibt.

Es gibt keine ausgezeichneten Zahlen.}


\PropositionG{5.454}
{In der Logik gibt es kein Nebeneinander, kann
es keine Klassifikation geben.

In der Logik kann es nicht Allgemeineres und
Spezielleres geben.}


\PropositionG{5.4541}
{Die Lösungen der logischen Probleme müssen
einfach sein, denn sie setzen den Standard der
Einfachheit.

Die Menschen haben immer geahnt, dass es ein
Gebiet von Fragen geben müsse, deren Antworten---a
priori---symmetrisch, und zu einem abgeschlossenen,
regelmässigen Gebilde vereintliegen.

Ein Gebiet, in dem der Satz gilt: simplex
sigillum veri.}


\PropositionG{5.46}
{Wenn man die logischen Zeichen richtig
einführte, so hätte man damit auch schon den Sinn
aller ihrer Kombinationen eingeführt; also nicht
nur \glqq{}$p \lor q$\grqq{} sondern auch schon \glqq{}$\Not{(p \lor \Not{q})}$\grqq{} etc.\ etc.
Man hätte damit auch schon die Wirkung
aller nur möglichen Kombinationen von Klammern
eingeführt. Und damit wäre es klar geworden,
dass die eigentlichen allgemeinen Urzeichen nicht
% -----File: 126.png---
die \glqq{}$p \lor q$\grqq{}, \glqq{}$(\exists x) \DotOp fx$\grqq{}, etc.\ sind, sondern die allgemeinste
Form ihrer Kombinationen.}


\PropositionG{5.461}
{Bedeutungsvoll ist die scheinbar unwichtige
Tatsache, dass die logischen Scheinbeziehungen,
wie $\lor$ und $\Implies$, der Klammern be\-dür\-fen---im Gegensatz
zu den wirklichen Beziehungen.

Die Benützung der Klammern mit jenen scheinbaren
Urzeichen deutet ja schon darauf hin, dass
diese nicht die wirklichen Urzeichen sind. Und
es wird doch wohl niemand glauben, dass die
Klammern eine selbständige Bedeutung haben.}


\PropositionG{5.4611}
{Die logischen Operationszeichen sind Interpunktionen.}


\PropositionG{5.47}
{Es ist klar, dass alles was sich überhaupt \Emph{von
vornherein} über die Form aller Sätze sagen
lässt, sich \Emph{aufeinmal} sagen lassen muss.

Sind ja schon im Elementarsatze alle logischen
\enlargethispage{1pt} % enlarge to make the last line fit
Operationen enthalten. Denn \glqq{}$fa$\grqq{} sagt dasselbe
wie \glqq{}$(\exists x) \DotOp fx \DotOp x = a$\grqq{}.

Wo Zusammengesetztheit ist, da ist Argument
und Funktion, und wo diese sind, sind bereits alle
logischen Konstanten.

Man könnte sagen: Die Eine logische Konstante
ist das, was \Emph{alle} Sätze, ihrer Natur nach, mit
einander gemein haben.

Das aber ist die allgemeine Satzform.}


\PropositionG{5.471}
{Die allgemeine Satzform ist das Wesen des
Satzes.}


\PropositionG{5.4711}
{Das Wesen des Satzes angeben, heisst, das
Wesen aller Beschreibung angeben, also das
Wesen der Welt.}


\PropositionG{5.472}
{Die Beschreibung der allgemeinsten Satzform
ist die Beschreibung des einen und einzigen
allgemeinen Urzeichens der Logik.}


\PropositionG{5.473}
{Die Logik muss für sich selber sorgen.

Ein \Emph{mögliches} Zeichen muss auch bezeichnen
können. Alles was in der Logik möglich ist, ist
auch erlaubt. (\glqq{}Sokrates ist identisch\grqq{} heisst darum
% -----File: 128.png---
nichts, weil es keine Eigenschaft gibt, die
\glqq{}identisch\grqq{} heisst. Der Satz ist unsinnig, weil
wir eine willkürliche Bestimmung nicht getroffen
haben, aber nicht darum, weil das Symbol an und
für sich unerlaubt wäre.)

Wir können uns, in gewissem Sinne, nicht in
der Logik irren.}


\PropositionG{5.4731}
{Das Einleuchten, von dem Russell so viel
sprach, kann nur dadurch in der Logik entbehrlich
werden, dass die Sprache selbst jeden logischen
Fehler ver\-hin\-dert.---Dass die Logik a priori ist,
besteht darin, dass nicht unlogisch gedacht werden
\Emph{kann}.}


\PropositionG{5.4732}
{Wir können einem Zeichen nicht den unrechten
Sinn geben.}


\PropositionG{5.47321}
{Occams Devise ist natürlich keine willkürliche,
oder durch ihren praktischen Erfolg gerechtfertigte,
Regel: Sie besagt, dass \Emph{unnötige} Zeicheneinheiten
nichts bedeuten.

Zeichen, die \Emph{Einen} Zweck erfüllen, sind logisch
äquivalent, Zeichen, die \Emph{keinen} Zweck erfüllen,
logisch bedeutungslos.}


\PropositionG{5.4733}
{Frege sagt: Jeder rechtmässig gebildete Satz
muss einen Sinn haben; und ich sage: Jeder
mögliche Satz ist rechtmässig gebildet, und wenn er
keinen Sinn hat, so kann das nur daran liegen, dass
wir einigen seiner Bestandteile keine \Emph{Bedeutung}
gegeben haben.

(Wenn wir auch glauben, es getan zu haben.)

So sagt \glqq{}Sokrates ist identisch\grqq{} darum nichts,
weil wir dem Wort \glqq{}identisch\grqq{} als \Emph{Eigenschaftswort}
\Emph{keine} Bedeutung gegeben haben. Denn,
wenn es als Gleichheitszeichen auftritt, so symbolisiert
es auf ganz andere Art und Weise---die
bezeichnende Beziehung ist eine an\-de\-re,---also ist
auch das Symbol in beiden Fällen ganz verschieden;
die beiden Symbole haben nur das Zeichen zufällig
miteinander gemein.}
% -----File: 130.png---


\PropositionG{5.474}
{Die Anzahl der nötigen Grundoperationen hängt
\Emph{nur} von unserer Notation ab.}


\PropositionG{5.475}
{Es kommt nur darauf an, ein Zeichensystem von
einer bestimmten Anzahl von Dimensionen---von
einer bestimmten mathematischen Man\-nig\-fal\-tig\-keit---zu
bilden.}


\PropositionG{5.476}
{Es ist klar, dass es sich hier nicht um eine
\Emph{Anzahl von Grundbegriffen} handelt, die
bezeichnet werden müssen, sondern um den
Ausdruck einer Regel.}


\PropositionG{5.5}
{Jede Wahrheitsfunktion ist ein Resultat der
successiven Anwendung der Operation \mbox{(--\;--\;--\;--\;--W)}\AllowBreak($\xi, \fourdots$)
auf Elementarsätze.

Diese Operation verneint sämtliche Sätze in der
rechten Klammer und ich nenne sie die Negation
dieser Sätze.}


\PropositionG{5.501}
{Einen Klammerausdruck, dessen Glieder Sätze
sind, deute ich\AllowBreak---wenn die Reihenfolge der Glieder in
der Klammer gleichgültig ist---durch ein Zeichen von
der Form \glqq{}$(\overline{\xi})$\grqq{} an. \glqq{}$\xi$\grqq{} ist eine Variable, deren Werte
die Glieder des Klammerausdruckes sind; und der
Strich über der Variablen deutet an, dass sie ihre
sämtlichen Werte in der Klammer vertritt.

(Hat also $\xi$ etwa die 3 Werte P, Q, R, so ist
($\overline{\xi}$) = (P, Q, R).)

Die Werte der Variablen werden festgesetzt.

Die Festsetzung ist die Beschreibung der Sätze,
welche die Variable vertritt.

Wie die Beschreibung der Glieder des Klammerausdruckes
geschieht, ist unwesentlich.

Wir \Emph{können} drei Arten der Beschreibung
unterscheiden: 1.~Die direkte Aufzählung. In
diesem Fall können wir statt der Variablen einfach
ihre konstanten Werte setzen. 2.~Die Angabe
einer Funktion $fx$, deren Werte für alle Werte von
$x$ die zu beschreibenden Sätze sind. 3.~Die Angabe
eines formalen Gesetzes, nach welchem jene Sätze
gebildet sind. In diesem Falle sind die Glieder des
% -----File: 132.png---
Klammerausdrucks sämtliche Glieder einer Formenreihe.}


\PropositionG{5.502}
{Ich schreibe also statt \mbox{\glqq{}(--\;--\;--\;--\;--W)}\AllowBreak($\xi, \fourdots$)\grqq{}
\glqq{}$N(\overline{\xi})$\grqq{}.

$N(\overline{\xi})$ ist die Negation sämtlicher Werte der
Satzvariablen $\xi$.}


\PropositionG{5.503}
{Da sich offenbar leicht ausdrücken lässt, wie mit
dieser Operation Sätze gebildet werden können und
wie Sätze mit ihr nicht zu bilden sind, so muss
dies auch einen exakten Ausdruck finden können.}


\PropositionG{5.51}
{Hat $\xi$ nur einen Wert, so ist $N(\overline{\xi}) = \Not{p}$ (nicht $p$),
hat es zwei Werte, so ist $N(\overline{\xi}) = \Not{p} \DotOp \Not{q}$ (weder
$p$ noch $q$).}


\PropositionG{5.511}
{Wie kann die allumfassende, weltspiegelnde
Logik so spezielle Haken und Manipulationen
gebrauchen? Nur, indem sich alle diese zu einem
unendlich feinen Netzwerk, zu dem grossen Spiegel,
verknüpfen.}


\PropositionG{5.512}
{\glqq{}$\Not{p}$\grqq{} ist wahr, wenn \glqq{}$p$\grqq{} falsch ist. Also in
dem wahren Satz \glqq{}$\Not{p}$\grqq{} ist \glqq{}$p$\grqq{} ein falscher Satz.
Wie kann ihn nun der Strich \glqq{}$\Not{}$\grqq{} mit der Wirklichkeit
zum Stimmen bringen?

Das, was in \glqq{}$\Not{p}$\grqq{} verneint, ist aber nicht das
\glqq{}$\Not{}$\grqq{}, sondern dasjenige, was allen Zeichen dieser
Notation, welche $p$ verneinen, gemeinsam ist.

Also die gemeinsame Regel, nach welcher
\glqq{}$\Not{p}$\grqq{}, \glqq{}$\Not{\Not{\Not{p}}}$\grqq{}, \glqq{}$\Not{p} \lor \Not{p}$\grqq{}, \glqq{}$\Not{p} \DotOp \Not{p}$\grqq{}, etc.\ etc.\ (ad
inf.) gebildet werden. Und dies Gemeinsame
spiegelt die Verneinung \DPtypo{wieder}{wider}.}


\PropositionG{5.513}
{Man könnte sagen: Das Gemeinsame aller Symbole,
die sowohl $p$ als $q$ bejahen, ist der Satz
\glqq{}$p \DotOp q$\grqq{}. Das Gemeinsame aller Symbole, die
entweder $p$ oder $q$ bejahen, ist der Satz \glqq{}$p \lor q$\grqq{}.

Und so kann man sagen: Zwei Sätze sind
einander entgegengesetzt, wenn sie nichts miteinander
gemein haben, und: Jeder Satz hat nur ein
Negativ, weil es nur einen Satz gibt, der ganz
ausserhalb seiner liegt.
% -----File: 134.png---

Es zeigt sich so auch in Russells Notation, dass
\glqq{}$q : p \lor \Not{p}$\grqq{} dasselbe sagt wie \glqq{}$q$\grqq{}; dass \glqq{}$p \lor \Not{p}$\grqq{}
\DPtypo{nichtssagt}{nichts sagt}.}


\PropositionG{5.514}
{Ist eine Notation festgelegt, so gibt es in ihr eine
Regel, nach der alle $p$ verneinenden \DPtypo{Sätz}{Sätze} gebildet
werden, eine Regel, nach der alle $p$ bejahenden
Sätze gebildet werden, eine Regel, nach der alle
$p$ oder $q$ bejahenden Sätze gebildet werden, \undSoFort{}
Diese Regeln sind den Symbolen äquivalent
und in ihnen spiegelt sich ihr Sinn \DPtypo{wieder}{wider}.}


\PropositionG{5.515}
{Es muss sich an unseren Symbolen zeigen, dass
das, was durch \glqq{}$\lor$\grqq{}, \glqq{}$\DotOp$\grqq{}, etc.\ miteinander verbunden
ist, Sätze sein müssen.

Und dies ist auch der Fall, denn das Symbol \glqq{}$p$\grqq{}
und \glqq{}$q$\grqq{} setzt ja selbst das \glqq{}$\lor$\grqq{}, \glqq{}$\Not{}$\grqq{}, etc.\ voraus.
Wenn das Zeichen \glqq{}$p$\grqq{} in \glqq{}$p \lor q$\grqq{} nicht für ein komplexes
Zeichen steht, dann kann es allein nicht
Sinn haben; dann können aber auch die mit \glqq{}$p$\grqq{}
gleichsinnigen Zeichen \glqq{}$p \lor p$\grqq{}, \glqq{}$p \DotOp p$\grqq{}, etc.\ keinen
Sinn haben. Wenn aber \glqq{}$p \lor p$\grqq{} keinen Sinn hat,
dann kann auch \glqq{}$p \lor q$\grqq{} keinen Sinn haben.}


\PropositionG{5.5151}
{Muss das Zeichen des negativen Satzes mit dem
Zeichen des positiven gebildet werden? Warum
sollte man den negativen Satz nicht durch eine negative
Tatsache ausdrücken können. (Etwa: Wenn
\glqq{}$a$\grqq{} nicht in einer bestimmten Beziehung zu \glqq{}$b$\grqq{} steht,
könnte das ausdrücken, dass $aRb$ nicht der Fall ist.)

Aber auch hier ist ja der negative Satz indirekt
durch den positiven gebildet.

Der positive \Emph{Satz} muss die Existenz des negativen
\Emph{Satzes} voraussetzen und umgekehrt.}


\PropositionG{5.52}
{Sind die Werte von $\xi$ sämtliche Werte einer
Funktion $fx$ für alle Werte von $x$, so wird
$N(\overline{\xi}) = \Not{(\exists x) \DotOp fx}$.}


\PropositionG{5.521}
{Ich trenne den Begriff \Emph{Alle} von der Wahrheitsfunktion.

Frege und Russell haben die Allgemeinheit in
Verbindung mit dem logischen Produkt oder der
% -----File: 136.png---
logischen Summe eingeführt. So wurde es schwer,
die Sätze \glqq{}$(\exists x) \DotOp fx$\grqq{} und \glqq{}$(x) \DotOp fx$\grqq{}, in welchen beide
Ideen beschlossen liegen, zu verstehen.}


\PropositionG{5.522}
{Das Eigentümliche der Allgemeinheitsbezeichnung
ist erstens, dass sie auf ein logisches Urbild
hinweist, und zweitens, dass sie Konstante
hervorhebt.}


\PropositionG{5.523}
{Die Allgemeinheitsbezeichnung tritt als Argument
auf.}


\PropositionG{5.524}
{Wenn die Gegenstände gegeben sind, so sind
uns damit auch schon \Emph{alle} Gegenstände gegeben.

Wenn die Elementarsätze gegeben sind, so sind
damit auch \Emph{alle} Elementarsätze gegeben.}


\PropositionG{5.525}
{Es ist unrichtig, den Satz \glqq{}$(\exists x) \DotOp fx$\grqq{}---wie
Russell dies tut---in Worten durch \glqq{}$fx$ ist \Emph{möglich}\grqq{}
wiederzugeben.

Gewissheit, Möglichkeit oder Unmöglichkeit
einer Sachlage wird nicht durch einen Satz ausgedrückt,
sondern dadurch, dass ein Ausdruck eine
Tautologie, ein sinnvoller Satz, oder eine Kontradiktion
ist.

Jener Präzedenzfall, auf den man sich immer
berufen möchte, muss schon im Symbol selber
liegen.}


\PropositionG{5.526}
{Man kann die Welt vollständig durch vollkommen
verallgemeinerte Sätze beschreiben, das
heisst also, ohne irgend einen Namen von vornherein
einem bestimmten Gegenstand zuzuordnen.

Um dann auf die gewöhnliche Ausdrucksweise
zu kommen, muss man einfach nach einem Ausdruck
\glqq{}es gibt ein und nur ein $x$, welches~$\fourdots$\grqq{} sagen:
Und dies $x$ ist $a$.}


\PropositionG{5.5261}
{Ein vollkommen verallgemeinerter Satz ist, wie
jeder andere Satz zusammengesetzt. (Dies zeigt
sich daran, dass wir in \glqq{}$(\exists x, \phi) \DotOp \phi x$\grqq{} \glqq{}$\phi$\grqq{} und \glqq{}$x$\grqq{}
getrennt erwähnen müssen. Beide stehen unabhängig
in bezeichnenden Beziehungen zur Welt,
wie im unverallgemeinerten Satz.)
% -----File: 138.png---

Kennzeichen des zusammengesetzten Symbols:
Es hat etwas mit \Emph{anderen} Symbolen gemeinsam.}


\PropositionG{5.5262}
{Es verändert ja die Wahr- oder Falschheit \Emph{jedes}
Satzes etwas am allgemeinen Bau der Welt. Und
der Spielraum, welcher ihrem Bau durch die
Gesamtheit der Elementarsätze gelassen wird, ist
eben derjenige, welchen die ganz allgemeinen
Sätze begrenzen.

(Wenn ein Elementarsatz wahr ist, so ist damit
doch jedenfalls Ein Elementarsatz \Emph{mehr} wahr.)}


\PropositionG{5.53}
{{\verystretchyspace
Gleichheit des Gegenstandes drücke ich durch
Gleichheit des Zeichens aus, und nicht mit Hilfe
eines Gleichheitszeichens. Verschiedenheit der
\enlargethispage{7pt} % enlarge to make the last word fit
Gegenstände durch Verschiedenheit der Zeichen.}}


\PropositionG{5.5301}
{Dass die Identität keine Relation zwischen Gegenständen
ist, leuchtet ein. Dies wird sehr klar,
wenn man \zumBeispiel\ den Satz \glqq{}$(x) : fx \DotOp \Implies \DotOp x = a$\grqq{}
betrachtet. Was dieser Satz sagt, ist einfach,
dass \Emph{nur} $a$ der Funktion $f$ genügt, und nicht,
dass nur solche Dinge der Funktion $f$ genügen,
welche eine gewisse Beziehung zu $a$ haben.

Man könnte nun freilich sagen, dass eben \Emph{nur}
$a$ diese Beziehung zu $a$ habe, aber um dies auszudrücken,
brauchten wir das Gleichheitszeichen
selber.}


\PropositionG{5.5302}
{Russells Definition von \glqq{}$=$\grqq{} genügt nicht; weil
man nach ihr nicht sagen kann, dass zwei Gegenstände
alle Eigenschaften gemeinsam haben.
(Selbst wenn dieser Satz nie richtig ist, hat er
doch \Emph{Sinn}.)}


\PropositionG{5.5303}
{Beiläufig gesprochen: Von \Emph{zwei} Dingen zu
sagen, sie seien identisch, ist ein Unsinn, und von
\Emph{Einem} zu sagen, es sei identisch mit sich selbst,
sagt gar nichts.}


\PropositionG{5.531}
{Ich schreibe also nicht \glqq{}$f(a,b) \DotOp a = b$\grqq{}, sondern
\glqq{}$f(a,a)$\grqq{} (oder \glqq{}$f(b,b)$\grqq{}). Und nicht \glqq{}$f(a,b) \DotOp \Not{a = b}$\grqq{},
sondern \glqq{}$f(a,b)$\grqq{}.}


\PropositionG{5.532}
{Und analog: Nicht \glqq{}$(\exists x,y) \DotOp f(x,y) \DotOp x = y$\grqq{},
% -----File: 140.png---
sondern \glqq{}$(\exists x) \DotOp f(x,x)$\grqq{}; und nicht \glqq{}$(\exists x,y) \DotOp f(x,y) \DotOp
\Not{x = y}$\grqq{}, sondern \glqq{}$(\exists x,y) \DotOp f(x,y)$\grqq{}.

(Also statt des Russell'schen \glqq{}$(\exists x,y) \DotOp f(x,y)$\grqq{}:
\glqq{}$(\exists x,y) \DotOp f(x,y) \DotOp \lor \DotOp (\exists x) \DotOp f(x,x)$\grqq{}.)}


\PropositionG{5.5321}
{Statt \glqq{}$(x) : fx \Implies x = a$\grqq{} schreiben wir also \zumBeispiel\ \glqq{}$(\exists
x) \DotOp fx \DotOp \Implies \DotOp fa : \Not{(\exists x,y) \DotOp fx \DotOp fy}$\grqq{}.

Und der Satz \glqq{}\Emph{nur} Ein $x$ befriedigt $f()$\grqq{} lautet:
\glqq{}$(\exists x) \DotOp fx : \Not{(\exists x,y) \DotOp fx \DotOp fy}$\grqq{}.}


\PropositionG{5.533}
{Das Gleichheitszeichen ist also kein wesentlicher
Bestandteil der Begriffsschrift.}


\PropositionG{5.534}
{Und nun sehen wir, dass Scheinsätze wie:
\glqq{}$a = a$\grqq{}, \glqq{}$a = b \DotOp b = c \DotOp \Implies a = c$\grqq{}, \glqq{}$(x) \DotOp x = x$\grqq{}, \glqq{}$(\exists x) \DotOp
x = a$\grqq{}, etc.\ sich in einer richtigen Begriffsschrift gar
nicht hinschreiben lassen.}


\PropositionG{5.535}
{Damit erledigen sich auch alle Probleme, die
an solche Scheinsätze geknüpft waren.

Alle Probleme, die Russells \glqq{}Axiom of Infinity\grqq{}
\enlargethispage{7pt} % enlarge to make the last line fit
mit sich bringt, sind schon hier zu lösen.

Das, was das Axiom of infinity sagen soll, würde
sich in der Sprache dadurch ausdrücken, dass es
unendlich viele Namen mit verschiedener Bedeutung
gäbe.}


\PropositionG{5.5351}
{Es gibt gewisse Fälle, wo man in Versuchung
gerät, Ausdrücke von der Form \glqq{}$a = a$\grqq{} oder \glqq{}$p \Implies p$\grqq{}
u.~dgl.\ zu benützen. Und zwar geschieht dies,
wenn man von dem Urbild: Satz, Ding, etc.\ reden
möchte. So hat Russell in den \glqq{}Principles of
Mathematics\grqq{} den Unsinn \glqq{}$p$ ist ein Satz\grqq{} in Symbolen
durch \glqq{}$p \Implies p$\grqq{} wiedergegeben und als Hypothese
vor gewisse Sätze gestellt, damit deren
Argumentstellen nur von Sätzen besetzt werden
könnten.

(Es ist schon darum Unsinn, die Hypothese
$p \Implies p$ vor einen Satz zu stellen, um ihm Argumente
der richtigen Form zu sichern, weil die Hypothese
für einen Nicht-Satz als Argument nicht falsch,
sondern unsinnig wird, und weil der Satz selbst
durch die unrichtige Gattung von Argumenten
% -----File: 142.png---
unsinnig wird, also sich selbst ebenso gut, oder so
schlecht, vor den unrechten Argumenten bewahrt,
wie die zu diesem Zweck angehängte sinnlose
Hypothese.)}


\PropositionG{5.5352}
{Ebenso wollte man \glqq{}Es gibt keine \Emph{Dinge}\grqq{} ausdrücken
durch \glqq{}$\Not{(\exists x) \DotOp x = x}$\grqq{}. Aber selbst wenn
dies ein Satz wäre,---wäre er nicht auch wahr, wenn
es zwar \glqq{}Dinge gäbe\grqq{}, aber diese nicht mit sich
selbst identisch wären?}


\PropositionG{5.54}
{In der allgemeinen Satzform kommt der Satz im
Satze nur als Basis der Wahrheitsoperationen vor.}


\PropositionG{5.541}
{Auf den ersten Blick scheint es, als könne ein Satz
in einem anderen auch auf andere Weise vorkommen.

Besonders in gewissen Satzformen der Psychologie,
wie \glqq{}A glaubt, dass $p$ der Fall ist\grqq{}, oder
\glqq{}A denkt $p$\grqq{}, etc.

Hier scheint es nämlich oberflächlich, als stünde
der Satz $p$ zu einem Gegenstand A in einer Art
von Relation.

(Und in der modernen Erkenntnistheorie (Russell,
Moore, etc.) sind jene Sätze auch so aufgefasst
worden.)}


\PropositionG{5.542}
{Es ist aber klar, dass \glqq{}A glaubt, dass $p$\grqq{}, \glqq{}A
denkt $p$\grqq{}, \glqq{}A sagt $p$\grqq{} von der Form \glqq{}\glq{}$p$\grq{} sagt $p$\grqq{} sind:
Und hier handelt es sich nicht um eine Zuordnung
von einer Tatsache und einem Gegenstand, sondern
um die Zuordnung von Tatsachen durch Zuordnung
ihrer Gegenstände.}


\PropositionG{5.5421}
{Dies zeigt auch, dass die Seele---das Subjekt,
etc.---wie sie in der heutigen oberflächlichen Psychologie
aufgefasst wird, ein Unding ist.

{\verystretchyspace
Eine zusammengesetzte Seele wäre nämlich
keine Seele mehr.}}


\PropositionG{5.5422}
{Die richtige Erklärung der Form des Satzes \glqq{}A
urteilt $p$\grqq{} muss zeigen, dass es unmöglich ist, einen
Unsinn zu urteilen. (Russells Theorie genügt
dieser Bedingung nicht.)}


\PropositionG{5.5423}
{Einen Komplex wahrnehmen, heisst, wahrnehmen,
% -----File: 144.png---
dass sich seine Bestandteile so und so zu einander
verhalten.

Dies erklärt wohl auch, dass man die Figur
\Illustration{cube}
auf zweierlei Art als Würfel sehen kann; und alle
ähnlichen Erscheinungen. Denn wir sehen eben
wirklich zwei verschiedene Tatsachen.

(Sehe ich erst auf die Ecken $a$ und nur flüchtig
auf $b$, so erscheint $a$ vorne; und umgekehrt.)}


\PropositionG{5.55}
{Wir müssen nun die Frage nach allen möglichen
Formen der Elementarsätze a priori beantworten.

Der Elementarsatz besteht aus Namen. Da wir
aber die Anzahl der Namen von verschiedener
Bedeutung nicht angeben können, so können wir
auch nicht die Zusammensetzung des Elementarsatzes
angeben.}


\PropositionG{5.551}
{Unser Grundsatz ist, dass jede Frage, die sich
überhaupt durch die Logik entscheiden lässt, sich
ohne weiteres entscheiden lassen muss.

(Und wenn wir in die Lage kommen, ein solches
Problem durch Ansehen der Welt beantworten zu
müssen, so zeigt dies, dass wir auf grundfalscher
Fährte sind.)}


\PropositionG{5.552}
{Die \glqq{}Erfahrung\grqq{}, die wir zum Verstehen der
Logik brauchen, ist nicht die, dass sich etwas so
und so verhält, sondern, dass etwas \Emph{ist}: aber das
ist eben \Emph{keine} Erfahrung.

Die Logik ist \Emph{vor} jeder Erfahrung---dass etwas
\Emph{so} ist.

Sie ist vor dem Wie, nicht vor dem Was.}
% -----File: 146.png---


\PropositionG{5.5521}
{Und wenn dies nicht so wäre, wie könnten wir
die Logik anwenden? Man könnte sagen: Wenn
es eine Logik gäbe, auch wenn es keine Welt gäbe,
wie könnte es dann eine Logik geben, da es eine
Welt gibt.}


\PropositionG{5.553}
{Russell sagte, es gäbe einfache Relationen
zwischen verschiedenen Anzahlen von Dingen
(Individuals). Aber zwischen welchen Anzahlen?
Und wie soll sich das entscheiden?---Durch die
Erfahrung?

(Eine ausgezeichnete Zahl gibt es nicht.)}


\PropositionG{5.554}
{Die Angabe jeder speziellen Form wäre vollkommen
willkürlich.}


\PropositionG{5.5541}
{Es soll sich a priori angeben lassen, ob ich \zumBeispiel\ in
die Lage kommen kann, etwas mit dem
Zeichen einer 27-stelligen Relation bezeichnen zu
müssen.}


\PropositionG{5.5542}
{Dürfen wir denn aber überhaupt so fragen?
Können wir eine Zeichenform aufstellen und nicht
wissen, ob ihr etwas entsprechen könne?

Hat die Frage einen Sinn: Was muss \Emph{sein},
damit etwas der-Fall-sein kann?}


\PropositionG{5.555}
{Es ist klar, wir haben vom Elementarsatz einen
Begriff, abgesehen von seiner besonderen logischen
Form.

Wo man aber Symbole nach einem System
bilden kann, dort ist dieses System das logisch
wichtige und nicht die einzelnen Symbole.

Und wie wäre es auch möglich, dass ich es in
der Logik mit Formen zu tun hätte, die ich erfinden
kann; sondern mit dem muss ich es zu tun haben,
was es mir möglich macht, sie zu erfinden.}


\PropositionG{5.556}
{Eine Hierarchie der Formen der Elementarsätze
kann es nicht geben. Nur was wir selbst
konstruieren, können wir voraussehen.}


\PropositionG{5.5561}
{Die empirische Realität ist begrenzt durch die
Gesamtheit der Gegenstände. Die Grenze zeigt
sich wieder in der Gesamtheit der Elementarsätze.
% -----File: 148.png---

Die Hierarchien sind, und müssen unabhängig
von der Realität sein.}


\PropositionG{5.5562}
{Wissen wir aus rein logischen Gründen, dass
es Elementarsätze geben muss, dann muss es jeder
wissen, der die Sätze in ihrer unanalysierten Form
versteht.}


\PropositionG{5.5563}
{Alle Sätze unserer Umgangssprache sind tatsächlich,
so wie sie sind, logisch vollkommen geordnet.---Jenes
Einfachste, was wir hier angeben sollen,
ist nicht ein Gleichnis der Wahrheit, sondern die
volle Wahrheit selbst.

(Unsere Probleme sind nicht abstrakt, sondern
vielleicht die konkretesten, die es gibt.)}


\PropositionG{5.557}
{Die \Emph{Anwendung} der Logik entscheidet
darüber, welche Elementarsätze es gibt.

Was in der Anwendung liegt, kann die Logik
nicht vorausnehmen.

Das ist klar: Die Logik darf mit ihrer Anwendung
nicht kollidieren.

Aber die Logik muss sich mit ihrer Anwendung
berühren.

Also dürfen die Logik und ihre Anwendung
einander nicht übergreifen.}


\PropositionG{5.5571}
{Wenn ich die Elementarsätze nicht a priori
angeben kann, dann muss es zu offenbarem Unsinn
führen, sie angeben zu wollen.}


\PropositionG{5.6}
{\Emph{Die Grenzen meiner Sprache} bedeuten
die Grenzen meiner Welt.}


\PropositionG{5.61}
{Die Logik erfüllt die Welt; die Grenzen der
Welt sind auch ihre Grenzen.

Wir können also in der Logik nicht sagen: Das
und das gibt es in der Welt, jenes nicht.

Das würde nämlich scheinbar voraussetzen, dass
\enlargethispage{9pt} % enlarge to make one more line fit
wir gewisse Möglichkeiten ausschliessen und dies
kann nicht der Fall sein, da sonst die Logik
über die Grenzen der Welt hinaus müsste; wenn
sie nämlich diese Grenzen auch von der anderen
Seite betrachten könnte.
% -----File: 150.png---

Was wir nicht denken können, das können wir
nicht denken; wir können also auch nicht \Emph{sagen},
was wir nicht denken können.}


\PropositionG{5.62}
{Diese Bemerkung gibt den Schlüssel zur
Entscheidung der Frage, inwieweit der Solipsismus
eine Wahrheit ist.

Was der Solipsismus nämlich \Emph{meint}, ist ganz
richtig, nur lässt es sich nicht \Emph{sagen}, sondern es
zeigt sich.

Dass die Welt \Emph{meine} Welt ist, das zeigt sich darin,
dass die Grenzen \Emph{der} Sprache (der Sprache, die allein
ich verstehe) die Grenzen \Emph{meiner} Welt bedeuten.}


\PropositionG{5.621}
{Die Welt und das Leben sind Eins.}


\PropositionG{5.63}
{Ich bin meine Welt. (Der Mikrokosmos.)}


\PropositionG{5.631}
{Das denkende, vorstellende, Subjekt gibt es nicht.

Wenn ich ein Buch schriebe \glqq{}Die Welt, wie ich
sie vorfand\grqq{}, so wäre darin auch über meinen Leib
zu berichten und zu sagen, welche Glieder meinem
Willen unterstehen und welche nicht etc., dies ist
nämlich eine Methode, das Subjekt zu isolieren,
oder vielmehr zu zeigen, dass es in einem wichtigen
Sinne kein Subjekt gibt: Von ihm allein nämlich
könnte in diesem Buche \Emph{nicht} die Rede sein.---}


\PropositionG{5.632}
{Das Subjekt gehört nicht zur Welt, sondern es
ist eine Grenze der Welt.}


\PropositionG{5.633}
{Wo in der Welt ist ein \DPtypo{methaphysisches}{metaphysisches} Subjekt
zu merken?

Du sagst, es verhält sich hier ganz, wie mit Auge
und Gesichtsfeld. Aber das Auge siehst du wirklich
\Emph{nicht}.

Und nichts \Emph{am Gesichtsfeld} lässt darauf
schliessen, dass es von einem Auge gesehen wird.}


\PropositionG{5.6331}
{Das Gesichtsfeld hat nämlich nicht etwa eine
solche Form:
\Illustration{sight-de}
}
% -----File: 152.png---


\PropositionG{5.634}
{Das hängt damit zusammen, dass kein Teil
unserer Erfahrung auch a priori ist.

Alles, was wir sehen, könnte auch anders
sein.

Alles, was wir überhaupt beschreiben können,
könnte auch anders sein.

Es gibt keine Ordnung der Dinge a priori.}


\PropositionG{5.64}
{Hier sieht man, dass der Solipsismus, streng
durchgeführt, mit dem reinen Realismus zusammenfällt.
Das Ich des Solipsismus schrumpft zum
ausdehnungslosen Punkt zusammen, und es bleibt
die ihm koordinierte Realität.}


\PropositionG{5.641}
{Es gibt also wirklich einen Sinn, in welchem in
der Philosophie nicht-psy\-cho\-lo\-gisch vom Ich die
Rede sein kann.

Das Ich tritt in die Philosophie dadurch ein,
dass die \glqq{}Welt meine Welt ist\grqq{}.

{\verystretchyspace
Das philosophische Ich ist nicht der Mensch,
nicht der menschliche Körper, oder die menschliche
Seele, von der die Psychologie handelt, sondern das
metaphysische Subjekt, die Grenze---nicht ein Teil
der Welt.}}


\PropositionG{6}
{Die allgemeine Form der Wahrheitsfunktion ist:
$[\overline{p}, \overline{\xi}, N(\overline{\xi})]$.

Dies ist die allgemeine Form des Satzes.}


\PropositionG{6.001}
{Dies sagt nichts anderes, als dass jeder Satz ein
Resultat der successiven Anwendung der Operation
$N'(\overline{\xi})$ auf die Elementarsätze ist.}


\PropositionG{6.002}
{Ist die allgemeine Form gegeben, wie ein Satz
gebaut ist, so ist damit auch schon die allgemeine
Form davon gegeben, wie aus einem Satz durch
eine Operation ein anderer erzeugt werden
kann.}


\PropositionG{6.01}
{Die allgemeine Form der Operation $\Omega'(\overline{\eta})$ ist
also: $[\overline{\xi}, N(\overline{\xi})]'${}$(\overline{\eta})$ (=~[$\overline{\eta}$, $\overline{\xi}$, $N(\overline{\xi})$]).

Das ist die allgemeinste Form des Überganges
von einem Satz zum anderen.}
% -----File: 154.png---


\PropositionG{6.02}
{Und so kommen wir zu den Zahlen: Ich definiere
\begin{gather*}
x = \Omega^{0}{}' x \text{ Def.\ und}\\
\Omega'\Omega^{\nu}{}'x = \Omega^{\nu+1}{}'x \text{ Def.}
\end{gather*}

Nach diesen Zeichenregeln schreiben wir also
die Reihe $x$, $\Omega'x$, $\Omega'\Omega'x$, $\Omega'\Omega'\Omega'x$, $\fivedots$
\[
\text{so: }\Omega^{0}{}'x, \Omega^{0+1}{}'x, \Omega^{0+1+1}{}'x, \Omega^{0+1+1+1}{}'x, \fivedots
\]

Also schreibe ich statt \glqq{}$[x, \xi, \Omega'\xi]$\grqq{}:
\[
\text{\quotedblbase} [\Omega^{0}{}'x, \Omega^{\nu}{}'x, \Omega^{\nu+1}{}'x]\text{\grqq{}.}
\]

Und definiere:
\[
\begin{array}{l}\\
0 + 1 = 1\text{ Def.}\\
0 + 1 + 1 = 2\text{ Def.}\\
0 + 1 + 1 + 1 = 3\text{ Def.}\\
\text{(\undSoFort)}
\end{array}
\]
}


\PropositionG{6.021}
{Die Zahl ist der Exponent einer Operation.}


\PropositionG{6.022}
{Der Zahlbegriff ist nichts anderes, als das
Gemeinsame aller Zahlen, die allgemeine Form
der Zahl.

Der Zahlbegriff ist die variable Zahl.

Und der Begriff der Zahlengleichheit ist die
allgemeine Form aller speziellen Zahlengleichheiten.}


\PropositionG{6.03}
{Die allgemeine Form der ganzen Zahl ist:
$[0, \xi, \xi + 1]$.}


\PropositionG{6.031}
{Die Theorie der Klassen ist in der Mathematik
ganz überflüssig.

Dies hängt damit zusammen, dass die Allgemeinheit,
welche wir in der Mathematik brauchen,
nicht die \Emph{zufällige} ist.}


\PropositionG{6.1}
{Die Sätze der Logik sind Tautologien.}


\PropositionG{6.11}
{Die Sätze der Logik sagen also Nichts. (Sie
sind die analytischen Sätze.)}


\PropositionG{6.111}
{Theorien, die einen Satz der Logik gehaltvoll
erscheinen lassen, sind immer falsch. Man könnte
\zumBeispiel\ glauben, dass die Worte \glqq{}wahr\grqq{} und \glqq{}falsch\grqq{}
zwei Eigenschaften unter anderen Eigenschaften
bezeichnen, und da erschiene es als eine merkwürdige
% -----File: 156.png---
Tatsache, dass jeder Satz eine dieser
Eigenschaften besitzt. Das scheint nun nichts
weniger als selbstverständlich zu sein, ebensowenig
selbstverständlich, wie etwa der Satz, \glqq{}alle Rosen
sind entweder gelb oder rot\grqq{} klänge, auch wenn er
wahr wäre. Ja, jener Satz bekommt nun ganz
den Charakter eines naturwissenschaftlichen Satzes
und dies ist das sichere Anzeichen dafür, dass er
falsch aufgefasst wurde.}


\PropositionG{6.112}
{Die richtige Erklärung der logischen Sätze
muss ihnen eine einzigartige Stellung unter allen
Sätzen geben.}


\PropositionG{6.113}
{Es ist das besondere Merkmal der logischen
Sätze, dass man am Symbol allein erkennen kann,
dass sie wahr sind, und diese Tatsache schliesst
die ganze Philosophie der Logik in sich. Und
so ist es auch eine der wichtigsten Tatsachen, dass
sich die Wahrheit oder Falschheit der nicht-logischen
Sätze \Emph{nicht} am Satz allein erkennen
lässt.}


\PropositionG{6.12}
{Dass die Sätze der Logik Tautologien sind,
das \Emph{zeigt} die for\-ma\-len---lo\-gi\-schen---Ei\-gen\-schaf\-ten
der Sprache, der Welt.

Dass ihre Bestandteile \Emph{so} verknüpft eine Tautologie
ergeben, das charakterisiert die Logik ihrer
Bestandteile.

Damit Sätze, auf bestimmte Art und Weise
verknüpft, eine Tautologie ergeben, dazu müssen
sie bestimmte Eigenschaften der Struktur haben.
Dass sie \Emph{so} verbunden eine Tautologie ergeben,
zeigt also, dass sie diese Eigenschaften der Struktur
besitzen.}


\PropositionG{6.1201}
{Dass \zumBeispiel\ die Sätze \glqq{}$p$\grqq{} und \glqq{}$\Not{p}$\grqq{} in der
Verbindung \glqq{}$\Not{(p \DotOp \Not{p})}$\grqq{} eine Tautologie ergeben,
zeigt, dass sie einander widersprechen. Dass
die Sätze \glqq{}$p \Implies q$\grqq{}, \glqq{}$p$\grqq{} und \glqq{}$q$\grqq{} in der Form
\glqq{}$(p \Implies q) \DotOp (p) : \Implies : (q)$\grqq{} miteinander verbunden eine
Tautologie ergeben, zeigt, dass $q$ aus $p$ und $p \Implies q$
% -----File: 158.png---
folgt. Dass \glqq{}$(x) \DotOp fx : \Implies : fa$\grqq{} eine Tautologie ist,
dass $fa$ aus $(x) \DotOp fx$ folgt.{} etc.\ etc.}


\PropositionG{6.1202}
{Es ist klar, dass man zu demselben Zweck statt
der Tautologien auch die Kontradiktionen verwenden
könnte.}


\PropositionG{6.1203}
{{\verystretchyspace
Um eine Tautologie als solche zu erkennen,
kann man sich, in den Fällen, in welchen in der
Tautologie keine Allgemeinheitsbezeichnung vorkommt,
folgender anschaulichen Methode bedienen:
Ich schreibe statt \glqq{}$p$\grqq{}, \glqq{}$q$\grqq{}, \glqq{}$r$\grqq{} etc.\ \glqq{}W$p$F\grqq{},
\glqq{}W$q$F\grqq{}, \glqq{}W$r$F\grqq{} etc. Die Wahrheitskombinationen
drücke ich durch Klammern aus.
\zumBeispiel:}
\Illustration[0.35\textwidth]{brackets01-de}
und die Zuordnung der Wahr- oder Falschheit des
ganzen Satzes und der Wahrheitskombinationen
der Wahrheitsargumente durch Striche auf
folgende Weise:
\Illustration[0.4\textwidth]{brackets02-de}
Dies Zeichen würde also \zumBeispiel\ den Satz $p \Implies q$
darstellen. Nun will ich \zumBeispiel\ den Satz $\Not{(p \DotOp \Not{p})}$
(Gesetz des Widerspruchs) daraufhin untersuchen,
ob er eine Tautologie ist. Die Form \glqq{}$\Not{\xi}$\grqq{} wird
in unserer Notation
\Illustration[0.1\textwidth]{brackets03-de}
% -----File: 160.png---
geschrieben; die Form \glqq{}$\xi \DotOp \eta$\grqq{} so:
\Illustration[0.4\textwidth]{brackets04-de}
Daher lautet der Satz $\Not{(p \DotOp \Not{q})}$ so:
\Illustration{brackets05-de}
Setzen wir hier statt \glqq{}$q$\grqq{} \glqq{}$p$\grqq{} ein und untersuchen
die Verbindung der äussersten W und F mit den
innersten, so ergibt sich, dass die Wahrheit des
ganzen Satzes \Emph{allen} Wahrheitskombinationen
seines Argumentes, seine Falschheit keiner der
Wahrheitskombinationen zugeordnet ist.}


\PropositionG{6.121}
{Die Sätze der Logik demonstrieren die logischen
Eigenschaften der Sätze, indem sie sie zu nichtssagenden
Sätzen verbinden.

Diese Methode könnte man auch eine Nullmethode
nennen. Im logischen Satz werden Sätze
miteinander ins Gleichgewicht gebracht und der
Zustand des Gleichgewichts zeigt dann an, wie
diese Sätze logisch beschaffen sein müssen.}


\PropositionG{6.122}
{Daraus ergibt sich, dass wir auch ohne die
logischen Sätze auskommen können, da wir ja in
einer entsprechenden Notation die formalen Eigenschaften
der Sätze durch das blosse Ansehen dieser
Sätze erkennen können.}
% -----File: 162.png---


\PropositionG{6.1221}
{Ergeben \zumBeispiel\ zwei Sätze \glqq{}$p$\grqq{} und \glqq{}$q$\grqq{} in der
Verbindung \glqq{}$p \Implies q$\grqq{} eine Tautologie, so ist \DPtypo{kar}{klar},
dass $q$ aus $p$ folgt.

Dass \zumBeispiel\ \glqq{}$q$\grqq{} aus \glqq{}$p \Implies q \DotOp p$\grqq{} folgt, ersehen wir
aus diesen beiden Sätzen selbst, aber wir können
es auch \Emph{so} zeigen, indem wir sie zu \glqq{}$p \Implies q \DotOp p : \Implies : q$\grqq{}
verbinden und nun zeigen, dass dies eine Tautologie
ist.}


\PropositionG{6.1222}
{Dies wirft ein Licht auf die Frage, warum die
logischen Sätze nicht durch die Erfahrung bestätigt
werden können, ebenso wenig, wie sie durch die
Erfahrung widerlegt werden können. Nicht nur
muss ein Satz der Logik durch keine mögliche Erfahrung
widerlegt werden können, sondern er darf auch
nicht durch eine solche bestätigt werden können.}


\PropositionG{6.1223}
{Nun wird klar, warum man oft fühlte, als wären
die \glqq{}logischen Wahrheiten\grqq{} von uns zu \glqq{}\Emph{fordern}\grqq{}:
Wir können sie nämlich insofern fordern, als wir
eine genügende Notation fordern können.}


\PropositionG{6.1224}
{Es wird jetzt auch klar, warum die Logik die
Lehre von den Formen und vom Schliessen genannt
wurde.}


\PropositionG{6.123}
{Es ist klar: Die logischen Gesetze dürfen nicht
selbst wieder logischen Gesetzen unterstehen.

(Es gibt nicht, wie Russell meinte, für jede
\glqq{}Type\grqq{} ein eigenes Gesetz des Widerspruches,
sondern Eines genügt, da es auf sich selbst nicht
angewendet wird.)}


\PropositionG{6.1231}
{Das Anzeichen des logischen Satzes ist \Emph{nicht}
die Allgemeingültigkeit.

Allgemein sein, heisst ja nur: Zufälligerweise
für alle Dinge gelten. Ein unverallgemeinerter
Satz kann ja ebensowohl tautologisch sein, als ein
verallgemeinerter.}


\PropositionG{6.1232}
{Die logische Allgemeingültigkeit könnte man
wesentlich nennen, im Gegensatz zu jener zufälligen,
etwa des Satzes \glqq{}alle Menschen sind sterblich\grqq{}.
Sätze, wie Russells \glqq{}Axiom of reducibility\grqq{} sind
% -----File: 164.png---
nicht logische Sätze, und dies erklärt unser Gefühl:
Dass sie, wenn wahr, so doch nur durch einen
günstigen Zufall wahr sein könnten.}


\PropositionG{6.1233}
{Es lässt sich eine Welt denken, in der das
Axiom of reducibility nicht gilt. Es ist aber klar,
dass die Logik nichts mit der Frage zu schaffen
hat, ob unsere Welt wirklich so ist oder nicht.}


\PropositionG{6.124}
{Die logischen Sätze beschreiben das Gerüst der
Welt, oder vielmehr, sie stellen es dar. Sie
\glqq{}handeln\grqq{} von nichts. Sie setzen voraus, dass
Namen Bedeutung, und Elementarsätze Sinn
haben: Und dies ist ihre Verbindung mit der
Welt. Es ist klar, dass es etwas über die Welt
anzeigen muss, dass gewisse Verbindungen von
Symbolen---welche wesentlich einen bestimmten
Charakter haben---Tautologien sind. Hierin liegt
das Entscheidende. Wir sagten, manches an
den Symbolen, die wir gebrauchen, wäre willkürlich,
manches nicht. In der Logik drückt nur
dieses aus: Dass heisst aber, in der Logik drücken
nicht \Emph{wir} mit Hilfe der Zeichen aus, was wir
wollen, sondern in der Logik sagt die Natur der
naturnotwendigen Zeichen selbst aus: Wenn wir die
logische Syntax irgend einer Zeichensprache kennen,
dann sind bereits alle Sätze der Logik gegeben.}


\PropositionG{6.125}
{Es ist möglich, und zwar auch nach der alten
Auffassung der Logik, von vornherein eine Beschreibung
aller \glqq{}wahren\grqq{} logischen Sätze zu geben.}


\PropositionG{6.1251}
{Darum kann es in der Logik auch \Emph{nie} Überraschungen
geben.}


\PropositionG{6.126}
{Ob ein Satz der Logik angehört, kann man
berechnen, indem man die logischen Eigenschaften
des \Emph{Symbols} berechnet.

Und dies tun wir, wenn wir einen logischen
Satz \glqq{}beweisen\grqq{}. Denn, ohne uns um einen Sinn
und eine Bedeutung zu kümmern, bilden wir den
logischen Satz aus anderen nach blossen \Emph{Zeichenregeln}.
% -----File: 166.png---

Der Beweis der logischen Sätze besteht darin,
dass wir sie aus anderen logischen Sätzen durch
successive Anwendung gewisser Operationen entstehen
lassen, die aus den ersten immer wieder
Tautologien erzeugen. (Und zwar \Emph{folgen} aus
einer Tautologie nur Tautologien.)

Natürlich ist diese Art zu zeigen, dass ihre
Sätze Tautologien sind, der Logik durchaus unwesentlich.
Schon darum, weil die Sätze, von
welchen der Beweis ausgeht, ja ohne Beweis zeigen
müssen, dass sie Tautologien sind.}


\PropositionG{6.1261}
{In der Logik sind Prozess und Resultat äquivalent.
(Darum keine Überraschung.)}


\PropositionG{6.1262}
{Der Beweis in der Logik ist nur ein mechanisches
Hilfsmittel zum leichteren Erkennen der
Tautologie, wo sie kompliziert ist.}


\PropositionG{6.1263}
{Es wäre ja auch zu merkwürdig, wenn man
einen sinnvollen Satz \Emph{logisch} aus anderen beweisen
könnte, und einen logischen Satz \Emph{auch}.
Es ist von vornherein klar, dass der logische
Beweis eines sinnvollen Satzes und der Beweis \Emph{in}
der Logik zwei ganz verschiedene Dinge sein
müssen.}


\PropositionG{6.1264}
{Der sinnvolle Satz sagt etwas aus, und sein
Beweis zeigt, dass es so ist; in der Logik ist jeder
Satz die Form eines Beweises.

Jeder Satz der Logik ist ein in Zeichen dargestellter
modus ponens. (Und den modus ponens
kann man nicht durch einen Satz ausdrücken.)}


\PropositionG{6.1265}
{Immer kann man die Logik so auffassen, dass
jeder Satz sein eigener Beweis ist.}


\PropositionG{6.127}
{Alle Sätze der Logik sind gleichberechtigt, es
gibt unter ihnen nicht wesentlich Grundgesetze
und abgeleitete Sätze.

Jede Tautologie zeigt selbst, dass sie eine
Tautologie ist.}


\PropositionG{6.1271}
{Es ist klar, dass die Anzahl der \glqq{}logischen
Grundgesetze\grqq{} willkürlich ist, denn man könnte
% -----File: 168.png---
die Logik ja aus Einem Grundgesetz ableiten,
indem man einfach \zumBeispiel\ aus Freges Grundgesetzen
das logische Produkt bildet. (Frege würde
vielleicht sagen, dass dieses Grundgesetz nun
nicht mehr unmittelbar einleuchte. Aber es ist
merkwürdig, dass ein so exakter Denker wie
Frege sich auf den Grad des Einleuchtens als
Kriterium des logischen Satzes berufen hat.)}


\PropositionG{6.13}
{Die Logik ist keine Lehre, sondern ein Spiegelbild
der Welt.

Die Logik ist transcendental.}


\PropositionG{6.2}
{Die Mathematik ist eine logische Methode.

Die Sätze der Mathematik sind Gleichungen
also Scheinsätze.}


\PropositionG{6.21}
{Der Satz der Mathematik drückt keinen Gedanken
aus.}


\PropositionG{6.211}
{Im Leben ist es ja nie der mathematische Satz,
den wir brauchen, sondern wir benützen den
mathematischen Satz \Emph{nur}, um aus Sätzen, welche
nicht der Mathematik angehören, auf andere zu
schliessen, welche gleichfalls nicht der Mathematik
angehören.

(In der Philosophie führt die Frage \glqq{}wozu
gebrauchen wir eigentlich jenes Wort, jenen Satz\grqq{}
immer wieder zu wertvollen Einsichten.)}


\PropositionG{6.22}
{Die Logik der Welt, die die Sätze der Logik in
den Tautologien zeigen, zeigt die Mathematik in
den Gleichungen.}


\PropositionG{6.23}
{Wenn zwei Ausdrücke durch das Gleichheitszeichen
verbunden werden, so heisst das, sie sind
durch einander ersetzbar. Ob dies aber der Fall ist
muss sich an den beiden Ausdrücken selbst zeigen.

Es charakterisiert die logische Form zweier Ausdrücke,
dass sie durch einander ersetzbar sind.}


\PropositionG{6.231}
{Es ist eine Eigenschaft der Bejahung, dass man
sie als doppelte Verneinung auffassen kann.

Es ist eine Eigenschaft von \glqq{}$1 + 1 + 1 + 1$\grqq{}, dass
man es als \glqq{}$(1 + 1) + (1 + 1)$\grqq{} auffassen kann.}
% -----File: 170.png---


\PropositionG{6.232}
{Frege sagt, die beiden Ausdrücke haben dieselbe
Bedeutung, aber verschiedenen Sinn.

Das Wesentliche an der Gleichung ist aber, dass
sie nicht notwendig ist, um zu zeigen, dass die beiden
Ausdrücke, die das Gleichheitszeichen verbindet,
dieselbe Bedeutung haben, da sich dies aus den
beiden Ausdrücken selbst ersehen lässt.}


\PropositionG{6.2321}
{Und, dass die Sätze der Mathematik bewiesen
werden können, heisst ja nichts anderes, als dass
ihre Richtigkeit einzusehen ist, ohne dass das, was
sie ausdrücken, selbst mit den Tatsachen auf seine
Richtigkeit hin verglichen werden muss.}


\PropositionG{6.2322}
{Die Identität der Bedeutung zweier Ausdrücke
lässt sich nicht \Emph{behaupten}. Denn um etwas von
ihrer Bedeutung behaupten zu können, muss ich
ihre Bedeutung kennen: und indem ich ihre Bedeutung
kenne, weiss ich, ob sie dasselbe oder
verschiedenes bedeuten.}


\PropositionG{6.2323}
{Die Gleichung kennzeichnet nur den Standpunkt,
von welchem ich die beiden Ausdrücke
betrachte, nämlich vom Standpunkte ihrer Bedeutungsgleichheit.}


\PropositionG{6.233}
{Die Frage, ob man zur Lösung der mathematischen
Probleme die Anschauung brauche, muss
dahin beantwortet werden, dass eben die Sprache
hier die nötige Anschauung liefert.}


\PropositionG{6.2331}
{Der Vorgang des \Emph{Rechnens} vermittelt eben
diese Anschauung.

Die Rechnung ist kein Experiment.}


\PropositionG{6.234}
{Die Mathematik ist eine Methode der Logik.}


\PropositionG{6.2341}
{Das Wesentliche der mathematischen Methode
ist es, mit Gleichungen zu arbeiten. Auf dieser
Methode beruht es nämlich, dass jeder Satz der
Mathematik sich von selbst verstehen muss.}


\PropositionG{6.24}
{Die Methode der Mathematik, zu ihren Gleichungen
zu kommen, ist die Substitutionsmethode.

{\verystretchyspace
Denn die Gleichungen drücken die Ersetzbarkeit
zweier Ausdrücke aus und wir schreiten von einer
% -----File: 172.png---
Anzahl von Gleichungen zu neuen Gleichungen
vor, indem wir, den Gleichungen entsprechend,
Ausdrücke durch andere ersetzen.}}


\PropositionG{6.241}
{So lautet der Beweis des Satzes $2 \times 2 = 4$:
\begin{gather*}
(\Omega^{\nu})^{\mu}{}'x = \Omega^{\nu \times \mu}{}'x \text{ Def.}\\
\begin{split}
\Omega^{2 \times 2}{}'x = (\Omega^{2})^{2}{}'x = (\Omega^{2})^{1 + 1}{}'x = \Omega^{2}{}'\Omega^{2}{}'x = \Omega^{1 + 1}{}'\Omega^{1 + 1}{}'x\\
= (\Omega'\Omega)'(\Omega'\Omega)'x = \Omega'\Omega'\Omega'\Omega'x = \Omega^{1 + 1 + 1 + 1}{}'x = \Omega^{4}{}'x.
\end{split}
\end{gather*}
}


\PropositionG{6.3}
{Die Erforschung der Logik bedeutet die Erforschung
\Emph{aller Gesetzmässigkeit}. Und ausserhalb
der Logik ist alles Zufall.}


\PropositionG{6.31}
{Das sogenannte Gesetz der Induktion kann
jedenfalls kein logisches Gesetz sein, denn es ist
offenbar ein sinnvoller Satz.---Und darum kann es
auch kein Gesetz a priori sein.}


\PropositionG{6.32}
{Das Kausalitätsgesetz ist kein Gesetz, sondern
die Form eines Gesetzes.}


\PropositionG{6.321}
{\glqq{}Kausalitätsgesetz\grqq{}, das ist ein Gattungsname.
Und wie es in der Mechanik, sagen wir, Minimum-Gesetze
gibt,---etwa der kleinsten Wir\-kung---so
gibt es in der Physik Kausalitätsgesetze, Gesetze
von der Kausalitätsform.}


\PropositionG{6.3211}
{Man hat ja auch davon eine Ahnung gehabt, dass
es \Emph{ein} \glqq{}Gesetz der kleinsten Wirkung\grqq{} geben müsse,
ehe man genau wuss\-te, wie es lautete. (Hier, wie
immer, stellt sich das a priori Gewisse als etwas
rein Logisches heraus.)}


\PropositionG{6.33}
{Wir \Emph{glauben} nicht a priori an ein Erhaltungsgesetz,
sondern wir \Emph{wissen} a priori die
Möglichkeit einer logischen Form.}


\PropositionG{6.34}
{Alle jene Sätze, wie der Satz vom Grunde, von
der Kontinuität in der Natur, vom kleinsten Aufwande
in der Natur etc.\ etc., alle diese sind Einsichten
a priori über die mögliche Formgebung der
Sätze der Wissenschaft.}


\PropositionG{6.341}
{Die Newtonsche Mechanik \zumBeispiel\ bringt die Weltbeschreibung
auf eine einheitliche Form. Denken
% -----File: 174.png---
wir uns eine weisse Fläche, auf der unregelmässige
schwarze Flecken wären. Wir sagen nun: Was für
ein Bild immer hierdurch entsteht, immer kann ich
seiner Beschreibung beliebig nahe kommen, indem
ich die Fläche mit einem entsprechend feinen quadratischen
Netzwerk bedecke und nun von jedem
Quadrat sage, dass es weiss oder schwarz ist. Ich
werde auf diese Weise die Beschreibung der Fläche
auf eine einheitliche Form gebracht haben. Diese
Form ist beliebig, denn ich hätte mit dem gleichen
Erfolge ein Netz aus dreieckigen oder sechseckigen
Maschen verwenden können. Es kann sein, dass
die Beschreibung mit Hilfe eines Dreiecks-Netzes
einfacher geworden wäre; das heisst, dass wir die
Fläche mit einem gröberen Dreiecks-Netz genauer
beschreiben könnten, als mit einem feineren quadratischen
(oder umgekehrt) usw. Den verschiedenen
Netzen entsprechen verschiedene Systeme der
Weltbeschreibung. Die Mechanik bestimmt eine
Form der Weltbeschreibung, indem sie sagt:
Alle Sätze der Weltbeschreibung müssen aus einer
Anzahl gegebener Sätze---den mechanischen Axiomen---auf
eine gegebene Art und Weise erhalten
werden. Hierdurch liefert sie die Bausteine zum
Bau des wissenschaftlichen Gebäudes und sagt:
Welches Gebäude immer du aufführen willst, jedes
musst du irgendwie mit diesen und nur diesen
Bausteinen zusammenbringen.

(Wie man mit dem Zahlensystem jede beliebige
Anzahl, so muss man mit dem System der
Mechanik jeden beliebigen Satz der Physik
hinschreiben können.)}


\PropositionG{6.342}
{Und nun sehen wir die gegenseitige Stellung
von Logik und Mechanik. (Man könnte das Netz
auch aus verschiedenartigen Figuren etwa aus
Dreie\discretionary{k-}{}{c}ken und Sechsecken bestehen lassen.) Dass
sich ein Bild, wie das vorhin erwähnte, durch ein
Netz von gegebener Form beschreiben lässt, sagt
% -----File: 176.png---
über das Bild \Emph{nichts} aus. (Denn dies gilt für
jedes Bild dieser Art.) \Emph{Das} aber charakterisiert
das Bild, dass es sich durch ein bestimmtes Netz
von \Emph{bestimmter} Feinheit \Emph{vollständig} beschreiben
lässt.

So auch sagt es nichts über die Welt aus, dass
sie sich durch die Newtonsche Mechanik beschreiben
lässt; wohl aber, dass sie sich \Emph{so} durch
jene beschreiben lässt, wie dies eben der Fall ist.
Auch das sagt etwas über die Welt, dass sie sich
durch die eine Mechanik einfacher beschreiben
lässt, als durch die andere.}


\PropositionG{6.343}
{Die Mechanik ist ein Versuch, alle \Emph{wahren}
Sätze, die wir zur Weltbeschreibung brauchen,
nach Einem Plane zu konstruieren.}


\PropositionG{6.3431}
{Durch den ganzen logischen Apparat hindurch
sprechen die physikalischen Gesetze doch von den
Gegenständen der Welt.}


\PropositionG{6.3432}
{Wir dürfen nicht vergessen, dass die Weltbeschreibung
durch die Mechanik immer die ganz
allgemeine ist. Es ist in ihr \zumBeispiel\ nie von
\Emph{bestimmten} materiellen Punkten die Rede,
sondern immer nur von \Emph{irgend welchen}.}


\PropositionG{6.35}
{Obwohl die Flecke in unserem Bild geometrische
Figuren sind, so kann doch selbstverständlich
die Geometrie gar nichts über ihre
tatsächliche Form und Lage sagen. Das Netz
aber ist \Emph{rein} geometrisch, alle seine Eigenschaften
können a priori angegeben werden.

Gesetze, wie der Satz vom Grunde, etc., handeln
vom Netz, nicht von dem, was das Netz beschreibt.}


\PropositionG{6.36}
{Wenn es ein Kausalitätsgesetz gäbe, so könnte
es lauten: \glqq{}Es gibt Naturgesetze\grqq{}.

Aber freilich kann man das nicht sagen: es
zeigt sich.}


\PropositionG{6.361}
{In der Ausdrucksweise Hertz's könnte man
sagen: Nur \Emph{gesetzmässige} Zusammenhänge
sind \Emph{denkbar}.}
% -----File: 178.png---


\PropositionG{6.3611}
{Wir können keinen Vorgang mit dem \glqq{}Ablauf
der Zeit\grqq{} ver\-glei\-chen---diesen gibt es nicht---,
sondern nur mit einem anderen Vorgang (etwa
mit dem Gang des Chronometers).

Daher ist die Beschreibung des zeitlichen
Verlaufs nur so möglich, dass wir uns auf einen
anderen Vorgang stützen.

Ganz Analoges gilt für den Raum. Wo man
\zumBeispiel\ sagt, es könne keines von zwei Ereignissen
(die sich gegenseitig aus\-schlies\-sen) eintreten, weil
\Emph{keine Ursache} vorhanden sei, warum das eine
eher als das andere eintreten solle, da handelt es
sich in Wirklichkeit darum, dass man gar nicht
\Emph{eines} der beiden Ereignisse beschreiben kann,
wenn nicht irgend eine Asymmetrie vorhanden ist.
Und \Emph{wenn} eine solche Asymmetrie vorhanden \Emph{ist},
so können wir diese als \Emph{Ursache} des Eintreffens
des einen und Nicht-Eintreffens des anderen
auffassen.}


\PropositionG{6.36111}
{Das Kant'sche Problem von der rechten und
linken Hand, die man nicht zur Deckung bringen
kann, besteht schon in der Ebene, ja im eindimensionalen
Raum, wo die beiden kongruenten
Figuren $a$ und $b$ auch nicht zur Deckung gebracht
werden können, ohne aus diesem Raum
herausbewegt zu werden. Rechte und linke Hand
sind tatsächlich vollkommen kongruent. Und
dass man sie nicht zur Deckung bringen kann,
hat damit nichts zu tun.

\Illustration[0.45\textwidth]{space}

Den rechten Handschuh könnte man an die
linke Hand ziehen, wenn man ihn im vierdimensionalen
Raum umdrehen könnte.}


\PropositionG{6.362}
{Was sich beschreiben lässt, das kann auch
geschehen, und was das Kausalitätsgesetz ausschliessen
soll, das lässt sich auch nicht beschreiben.}


\PropositionG{6.363}
{Der Vorgang der Induktion besteht darin, dass
% -----File: 180.png---
wir das \Emph{einfachste} Gesetz annehmen, das mit
unseren Erfahrungen in Einklang zu bringen ist.}


\PropositionG{6.3631}
{Dieser Vorgang hat aber keine logische, sondern
nur eine psychologische Begründung.

Es ist klar, dass kein Grund vorhanden ist, zu
glauben, es werde nun auch wirklich der einfachste
Fall eintreten.}


\PropositionG{6.36311}
{Dass die Sonne morgen aufgehen wird, ist eine
Hypothese; und das heisst: wir \Emph{wissen} nicht, ob
sie aufgehen wird.}


\PropositionG{6.37}
{Einen Zwang, nach dem Eines geschehen müsste,
weil etwas anderes geschehen ist, gibt es nicht. Es
gibt nur eine \Emph{logische} Notwendigkeit.}


\PropositionG{6.371}
{Der ganzen modernen Weltanschauung liegt die
Täuschung zugrunde, dass die sogenannten Naturgesetze
die Erklärungen der Naturerscheinungen
seien.}


\PropositionG{6.372}
{So bleiben sie bei den Naturgesetzen als bei
etwas Unantastbarem stehen, wie die älteren bei
Gott und dem Schicksal.

Und sie haben ja beide Recht, und Unrecht. Die
Alten sind allerdings insofern klarer, als sie einen
klaren Abschluss anerkennen, während es bei dem
neuen System scheinen soll, als sei \Emph{alles} erklärt.}


\PropositionG{6.373}
{Die Welt ist unabhängig von meinem Willen.}


\PropositionG{6.374}
{Auch wenn alles, was wir wünschen, geschähe,
so wäre dies doch nur, sozusagen, eine Gnade des
Schicksals, denn es ist kein \Emph{logischer} Zusammenhang
zwischen Willen und Welt, der dies
verbürgte, und den angenommenen physikalischen
Zusammenhang könnten wir doch nicht selbst
wieder wollen.}


\PropositionG{6.375}
{Wie es nur eine \Emph{logische} Notwendigkeit gibt,
so gibt es auch nur eine \Emph{logische} Unmöglichkeit.}


\PropositionG{6.3751}
{Dass \zumBeispiel\ zwei Farben zugleich an einem Ort
des Gesichtsfeldes sind, ist unmöglich und zwar
logisch unmöglich, denn es ist durch die logische
Struktur der Farbe ausgeschlossen.
% -----File: 182.png---

Denken wir daran, wie sich dieser Widerspruch
in der Physik darstellt: Ungefähr so, dass ein
Teilchen nicht zu gleicher Zeit zwei Geschwindigkeiten
haben kann; das heisst, dass es nicht zu
gleicher Zeit an zwei Orten sein kann; das heisst,
dass Teilchen an verschiedenen Orten zu Einer Zeit
nicht identisch sein können.

(Es ist klar, dass das logische Produkt zweier
Elementarsätze weder eine Tautologie noch eine
Kontradiktion sein kann. Die Aussage, dass ein
Punkt des Gesichtsfeldes zu gleicher Zeit zwei
verschiedene Farben hat, ist eine Kontradiktion.)}


\PropositionG{6.4}
{Alle Sätze sind gleichwertig.}


\PropositionG{6.41}
{Der Sinn der Welt muss ausserhalb ihrer liegen.
In der Welt ist alles wie es ist und geschieht alles
wie es geschieht; es gibt \Emph{in} ihr keinen Wert---und
wenn es ihn gäbe, so hätte er keinen Wert.

Wenn es einen Wert gibt, der Wert hat, so muss
er ausserhalb alles Geschehens und So-Seins liegen.
Denn alles Geschehen und So-Sein ist zufällig.

Was es nicht-zufällig macht, kann nicht \Emph{in} der
Welt liegen, denn sonst wäre dies wieder zufällig.

Es muss ausserhalb der Welt liegen.}


\PropositionG{6.42}
{Darum kann es auch keine Sätze der Ethik geben.

Sätze können nichts Höheres ausdrücken.}


\PropositionG{6.421}
{Es ist klar, dass sich die Ethik nicht aussprechen
lässt.

Die Ethik ist \DPtypo{transscendental}{transcendental}.

(Ethik und Aesthetik sind Eins.)}


\PropositionG{6.422}
{Der erste Gedanke bei der Aufstellung eines
ethischen Gesetzes von der Form \glqq{}du sollst $\fourdots$\grqq{}
ist: Und was dann, wenn ich es nicht tue? Es ist
aber klar, dass die Ethik nichts mit Strafe und
Lohn im gewöhnlichen Sinne zu tun hat. Also
muss diese Frage nach den \Emph{Folgen} einer Handlung
belanglos sein.---Zum Mindesten dürfen diese
Folgen nicht Ereignisse sein. Denn etwas muss
doch an jener Fragestellung richtig sein. Es muss
% -----File: 184.png---
zwar eine Art von ethischem Lohn und ethischer
Strafe geben, aber diese müssen in der Handlung
selbst liegen.

(Und das ist auch klar, dass der Lohn etwas
Angenehmes, die Strafe etwas Unangenehmes sein
muss.)}


\PropositionG{6.423}
{Vom Willen als dem Träger des Ethischen kann
nicht gesprochen werden.

Und der Wille als Phänomen interessiert nur
die Psychologie.}


\PropositionG{6.43}
{Wenn das gute oder böse Wollen die Welt
ändert, so kann es nur die Grenzen der Welt ändern,
nicht die Tatsachen; nicht das, was durch die
Sprache ausgedrückt werden kann.

Kurz, die Welt muss dann dadurch überhaupt
eine andere werden. Sie muss sozusagen als
Ganzes abnehmen oder zunehmen.

Die Welt des Glücklichen ist eine andere als die
des Unglücklichen.}


\PropositionG{6.431}
{Wie auch beim Tod die Welt sich nicht ändert,
sondern aufhört.}


\PropositionG{6.4311}
{Der Tod ist kein Ereignis des Lebens. Den
Tod erlebt man nicht.

Wenn man unter Ewigkeit nicht unendliche
Zeitdauer, sondern Unzeitlichkeit versteht, dann
lebt der ewig, der in der Gegenwart lebt.

Unser Leben ist ebenso endlos, wie unser
Gesichtsfeld grenzenlos ist.}


\PropositionG{6.4312}
{Die zeitliche Unsterblichkeit der Seele des
Menschen, das heisst also ihr ewiges Fortleben
auch nach dem Tode, ist nicht nur auf keine Weise
verbürgt, sondern vor allem leistet diese Annahme
gar nicht das, was man immer mit ihr erreichen
wollte. Wird denn dadurch ein Rätsel gelöst, dass
ich ewig fortlebe? Ist denn dieses ewige Leben
dann nicht ebenso rätselhaft wie das gegenwärtige?
Die Lösung des Rätsels des Lebens in Raum und
Zeit liegt \Emph{ausserhalb} von Raum und Zeit.
% -----File: 186.png---

(Nicht Probleme der Naturwissenschaft sind ja
zu lösen.)}


\PropositionG{6.432}
{\Emph{Wie} die Welt ist, ist für das Höhere vollkommen
\enlargethispage{1pt} % enlarge to make the last line fit
gleichgültig. Gott offenbart sich nicht \Emph{in}
der Welt.}


\PropositionG{6.4321}
{Die Tatsachen gehören alle nur zur Aufgabe,
nicht zur Lösung.}


\PropositionG{6.44}
{Nicht \Emph{wie} die Welt ist, ist das Mystische,
sondern \Emph{dass} sie ist.}


\PropositionG{6.45}
{Die Anschauung der Welt sub specie aeterni
ist ihre Anschauung als---be\-grenz\-tes---Gan\-zes.

Das Gefühl der Welt als begrenztes Ganzes ist
das mystische.}


\PropositionG{6.5}
{Zu einer Antwort, die man nicht aussprechen
kann, kann man auch die Frage nicht aussprechen.

\Emph{Das Rätsel} gibt es nicht.

Wenn sich eine Frage überhaupt stellen lässt,
so \Emph{kann} sie auch beantwortet werden.}


\PropositionG{6.51}
{Skeptizismus ist \Emph{nicht} unwiderleglich, sondern
offenbar unsinnig, wenn er bezweifeln will, wo
nicht gefragt werden kann.

Denn Zweifel kann nur bestehen, wo eine Frage
besteht; eine Frage nur, wo eine Antwort besteht,
und diese nur, wo etwas \Emph{gesagt} werden \Emph{kann}.}


\PropositionG{6.52}
{Wir fühlen, dass selbst, wenn alle \Emph{möglichen}
wissenschaftlichen Fragen beantwortet sind, unsere
Lebensprobleme noch gar nicht berührt sind.
Freilich bleibt dann eben keine Frage mehr; und
eben dies ist die Antwort.}


\PropositionG{6.521}
{Die Lösung des Problems des Lebens merkt
man am Verschwinden dieses Problems.

(Ist nicht dies der Grund, warum Menschen,
denen der Sinn des Lebens nach langen Zweifeln
klar wurde, warum diese dann nicht sagen konnten,
worin dieser Sinn bestand.)}


\PropositionG{6.522}
{Es gibt allerdings Unaussprechliches. Dies
\Emph{zeigt} sich, es ist das Mystische.}


\PropositionG{6.53}
{{\verystretchyspace
Die richtige Methode der Philosophie wäre
% -----File: 188.png---
eigentlich die: Nichts zu sagen, als was sich sagen
lässt, also Sätze der Na\-tur\-wis\-sen\-schaft---also etwas,
was mit Philosophie nichts zu tun hat---, und dann
immer, wenn ein anderer etwas \DPtypo{Methaphysisches}{Metaphysisches}
sagen wollte, ihm nachzuweisen, dass er gewissen
Zeichen in seinen Sätzen keine Bedeutung gegeben
hat. Diese Methode wäre für den anderen un\-be\-frie\-di\-gend---er
hätte nicht das Gefühl, dass wir
ihn Philosophie lehrten---aber \Emph{sie} wäre die einzig
streng richtige.}}


\PropositionG{6.54}
{Meine Sätze erläutern dadurch, dass sie der,
welcher mich versteht, am Ende als unsinnig
erkennt, wenn er durch sie---auf ihnen---über sie
hinausgestiegen ist. (Er muss sozusagen die Leiter
wegwerfen, nachdem er auf ihr hinaufgestiegen ist.)

Er muss diese Sätze überwinden, dann sieht er
die Welt richtig.}


\PropositionG{7}
{Wovon man nicht sprechen kann, darüber muss
man schweigen.}
\end{propositions}
% -----File: 190.png---

\newpage
\pagestyle{empty}\null\vfill
\begin{center}
{\footnotesize PRINTED IN GREAT BRITAIN BY THE EDINBURGH PRESS,\\
9 AND 11 YOUNG STREET, EDINBURGH.}
\end{center}
\vfill

%%%%%%%%%%%%%%%%%%%%%%%%% GUTENBERG LICENCE %%%%%%%%%%%%%%%%%%%%%%%%%%

\Licence

\pagestyle{plain}
\begin{PGtext}
End of the Project Gutenberg EBook of Tractatus Logico-Philosophicus, by 
Ludwig Wittgenstein

*** END OF THIS PROJECT GUTENBERG EBOOK TRACTATUS LOGICO-PHILOSOPHICUS ***

***** This file should be named 5740-pdf.pdf or 5740-pdf.zip *****
This and all associated files of various formats will be found in:
        http://www.gutenberg.org/5/7/4/5740/

Produced by Jana Srna, Norbert H. Langkau, and the Online
Distributed Proofreading Team at http://www.pgdp.net


Updated editions will replace the previous one--the old editions
will be renamed.

Creating the works from public domain print editions means that no
one owns a United States copyright in these works, so the Foundation
(and you!) can copy and distribute it in the United States without
permission and without paying copyright royalties.  Special rules,
set forth in the General Terms of Use part of this license, apply to
copying and distributing Project Gutenberg-tm electronic works to
protect the PROJECT GUTENBERG-tm concept and trademark.  Project
Gutenberg is a registered trademark, and may not be used if you
charge for the eBooks, unless you receive specific permission.  If you
do not charge anything for copies of this eBook, complying with the
rules is very easy.  You may use this eBook for nearly any purpose
such as creation of derivative works, reports, performances and
research.  They may be modified and printed and given away--you may do
practically ANYTHING with public domain eBooks.  Redistribution is
subject to the trademark license, especially commercial
redistribution.



*** START: FULL LICENSE ***

THE FULL PROJECT GUTENBERG LICENSE
PLEASE READ THIS BEFORE YOU DISTRIBUTE OR USE THIS WORK

To protect the Project Gutenberg-tm mission of promoting the free
distribution of electronic works, by using or distributing this work
(or any other work associated in any way with the phrase "Project
Gutenberg"), you agree to comply with all the terms of the Full Project
Gutenberg-tm License (available with this file or online at
http://gutenberg.org/license).


Section 1.  General Terms of Use and Redistributing Project Gutenberg-tm
electronic works

1.A.  By reading or using any part of this Project Gutenberg-tm
electronic work, you indicate that you have read, understand, agree to
and accept all the terms of this license and intellectual property
(trademark/copyright) agreement.  If you do not agree to abide by all
the terms of this agreement, you must cease using and return or destroy
all copies of Project Gutenberg-tm electronic works in your possession.
If you paid a fee for obtaining a copy of or access to a Project
Gutenberg-tm electronic work and you do not agree to be bound by the
terms of this agreement, you may obtain a refund from the person or
entity to whom you paid the fee as set forth in paragraph 1.E.8.

1.B.  "Project Gutenberg" is a registered trademark.  It may only be
used on or associated in any way with an electronic work by people who
agree to be bound by the terms of this agreement.  There are a few
things that you can do with most Project Gutenberg-tm electronic works
even without complying with the full terms of this agreement.  See
paragraph 1.C below.  There are a lot of things you can do with Project
Gutenberg-tm electronic works if you follow the terms of this agreement
and help preserve free future access to Project Gutenberg-tm electronic
works.  See paragraph 1.E below.

1.C.  The Project Gutenberg Literary Archive Foundation ("the Foundation"
or PGLAF), owns a compilation copyright in the collection of Project
Gutenberg-tm electronic works.  Nearly all the individual works in the
collection are in the public domain in the United States.  If an
individual work is in the public domain in the United States and you are
located in the United States, we do not claim a right to prevent you from
copying, distributing, performing, displaying or creating derivative
works based on the work as long as all references to Project Gutenberg
are removed.  Of course, we hope that you will support the Project
Gutenberg-tm mission of promoting free access to electronic works by
freely sharing Project Gutenberg-tm works in compliance with the terms of
this agreement for keeping the Project Gutenberg-tm name associated with
the work.  You can easily comply with the terms of this agreement by
keeping this work in the same format with its attached full Project
Gutenberg-tm License when you share it without charge with others.

1.D.  The copyright laws of the place where you are located also govern
what you can do with this work.  Copyright laws in most countries are in
a constant state of change.  If you are outside the United States, check
the laws of your country in addition to the terms of this agreement
before downloading, copying, displaying, performing, distributing or
creating derivative works based on this work or any other Project
Gutenberg-tm work.  The Foundation makes no representations concerning
the copyright status of any work in any country outside the United
States.

1.E.  Unless you have removed all references to Project Gutenberg:

1.E.1.  The following sentence, with active links to, or other immediate
access to, the full Project Gutenberg-tm License must appear prominently
whenever any copy of a Project Gutenberg-tm work (any work on which the
phrase "Project Gutenberg" appears, or with which the phrase "Project
Gutenberg" is associated) is accessed, displayed, performed, viewed,
copied or distributed:

This eBook is for the use of anyone anywhere at no cost and with
almost no restrictions whatsoever.  You may copy it, give it away or
re-use it under the terms of the Project Gutenberg License included
with this eBook or online at www.gutenberg.org

1.E.2.  If an individual Project Gutenberg-tm electronic work is derived
from the public domain (does not contain a notice indicating that it is
posted with permission of the copyright holder), the work can be copied
and distributed to anyone in the United States without paying any fees
or charges.  If you are redistributing or providing access to a work
with the phrase "Project Gutenberg" associated with or appearing on the
work, you must comply either with the requirements of paragraphs 1.E.1
through 1.E.7 or obtain permission for the use of the work and the
Project Gutenberg-tm trademark as set forth in paragraphs 1.E.8 or
1.E.9.

1.E.3.  If an individual Project Gutenberg-tm electronic work is posted
with the permission of the copyright holder, your use and distribution
must comply with both paragraphs 1.E.1 through 1.E.7 and any additional
terms imposed by the copyright holder.  Additional terms will be linked
to the Project Gutenberg-tm License for all works posted with the
permission of the copyright holder found at the beginning of this work.

1.E.4.  Do not unlink or detach or remove the full Project Gutenberg-tm
License terms from this work, or any files containing a part of this
work or any other work associated with Project Gutenberg-tm.

1.E.5.  Do not copy, display, perform, distribute or redistribute this
electronic work, or any part of this electronic work, without
prominently displaying the sentence set forth in paragraph 1.E.1 with
active links or immediate access to the full terms of the Project
Gutenberg-tm License.

1.E.6.  You may convert to and distribute this work in any binary,
compressed, marked up, nonproprietary or proprietary form, including any
word processing or hypertext form.  However, if you provide access to or
distribute copies of a Project Gutenberg-tm work in a format other than
"Plain Vanilla ASCII" or other format used in the official version
posted on the official Project Gutenberg-tm web site (www.gutenberg.org),
you must, at no additional cost, fee or expense to the user, provide a
copy, a means of exporting a copy, or a means of obtaining a copy upon
request, of the work in its original "Plain Vanilla ASCII" or other
form.  Any alternate format must include the full Project Gutenberg-tm
License as specified in paragraph 1.E.1.

1.E.7.  Do not charge a fee for access to, viewing, displaying,
performing, copying or distributing any Project Gutenberg-tm works
unless you comply with paragraph 1.E.8 or 1.E.9.

1.E.8.  You may charge a reasonable fee for copies of or providing
access to or distributing Project Gutenberg-tm electronic works provided
that

- You pay a royalty fee of 20% of the gross profits you derive from
     the use of Project Gutenberg-tm works calculated using the method
     you already use to calculate your applicable taxes.  The fee is
     owed to the owner of the Project Gutenberg-tm trademark, but he
     has agreed to donate royalties under this paragraph to the
     Project Gutenberg Literary Archive Foundation.  Royalty payments
     must be paid within 60 days following each date on which you
     prepare (or are legally required to prepare) your periodic tax
     returns.  Royalty payments should be clearly marked as such and
     sent to the Project Gutenberg Literary Archive Foundation at the
     address specified in Section 4, "Information about donations to
     the Project Gutenberg Literary Archive Foundation."

- You provide a full refund of any money paid by a user who notifies
     you in writing (or by e-mail) within 30 days of receipt that s/he
     does not agree to the terms of the full Project Gutenberg-tm
     License.  You must require such a user to return or
     destroy all copies of the works possessed in a physical medium
     and discontinue all use of and all access to other copies of
     Project Gutenberg-tm works.

- You provide, in accordance with paragraph 1.F.3, a full refund of any
     money paid for a work or a replacement copy, if a defect in the
     electronic work is discovered and reported to you within 90 days
     of receipt of the work.

- You comply with all other terms of this agreement for free
     distribution of Project Gutenberg-tm works.

1.E.9.  If you wish to charge a fee or distribute a Project Gutenberg-tm
electronic work or group of works on different terms than are set
forth in this agreement, you must obtain permission in writing from
both the Project Gutenberg Literary Archive Foundation and Michael
Hart, the owner of the Project Gutenberg-tm trademark.  Contact the
Foundation as set forth in Section 3 below.

1.F.

1.F.1.  Project Gutenberg volunteers and employees expend considerable
effort to identify, do copyright research on, transcribe and proofread
public domain works in creating the Project Gutenberg-tm
collection.  Despite these efforts, Project Gutenberg-tm electronic
works, and the medium on which they may be stored, may contain
"Defects," such as, but not limited to, incomplete, inaccurate or
corrupt data, transcription errors, a copyright or other intellectual
property infringement, a defective or damaged disk or other medium, a
computer virus, or computer codes that damage or cannot be read by
your equipment.

1.F.2.  LIMITED WARRANTY, DISCLAIMER OF DAMAGES - Except for the "Right
of Replacement or Refund" described in paragraph 1.F.3, the Project
Gutenberg Literary Archive Foundation, the owner of the Project
Gutenberg-tm trademark, and any other party distributing a Project
Gutenberg-tm electronic work under this agreement, disclaim all
liability to you for damages, costs and expenses, including legal
fees.  YOU AGREE THAT YOU HAVE NO REMEDIES FOR NEGLIGENCE, STRICT
LIABILITY, BREACH OF WARRANTY OR BREACH OF CONTRACT EXCEPT THOSE
PROVIDED IN PARAGRAPH 1.F.3.  YOU AGREE THAT THE FOUNDATION, THE
TRADEMARK OWNER, AND ANY DISTRIBUTOR UNDER THIS AGREEMENT WILL NOT BE
LIABLE TO YOU FOR ACTUAL, DIRECT, INDIRECT, CONSEQUENTIAL, PUNITIVE OR
INCIDENTAL DAMAGES EVEN IF YOU GIVE NOTICE OF THE POSSIBILITY OF SUCH
DAMAGE.

1.F.3.  LIMITED RIGHT OF REPLACEMENT OR REFUND - If you discover a
defect in this electronic work within 90 days of receiving it, you can
receive a refund of the money (if any) you paid for it by sending a
written explanation to the person you received the work from.  If you
received the work on a physical medium, you must return the medium with
your written explanation.  The person or entity that provided you with
the defective work may elect to provide a replacement copy in lieu of a
refund.  If you received the work electronically, the person or entity
providing it to you may choose to give you a second opportunity to
receive the work electronically in lieu of a refund.  If the second copy
is also defective, you may demand a refund in writing without further
opportunities to fix the problem.

1.F.4.  Except for the limited right of replacement or refund set forth
in paragraph 1.F.3, this work is provided to you 'AS-IS' WITH NO OTHER
WARRANTIES OF ANY KIND, EXPRESS OR IMPLIED, INCLUDING BUT NOT LIMITED TO
WARRANTIES OF MERCHANTIBILITY OR FITNESS FOR ANY PURPOSE.

1.F.5.  Some states do not allow disclaimers of certain implied
warranties or the exclusion or limitation of certain types of damages.
If any disclaimer or limitation set forth in this agreement violates the
law of the state applicable to this agreement, the agreement shall be
interpreted to make the maximum disclaimer or limitation permitted by
the applicable state law.  The invalidity or unenforceability of any
provision of this agreement shall not void the remaining provisions.

1.F.6.  INDEMNITY - You agree to indemnify and hold the Foundation, the
trademark owner, any agent or employee of the Foundation, anyone
providing copies of Project Gutenberg-tm electronic works in accordance
with this agreement, and any volunteers associated with the production,
promotion and distribution of Project Gutenberg-tm electronic works,
harmless from all liability, costs and expenses, including legal fees,
that arise directly or indirectly from any of the following which you do
or cause to occur: (a) distribution of this or any Project Gutenberg-tm
work, (b) alteration, modification, or additions or deletions to any
Project Gutenberg-tm work, and (c) any Defect you cause.


Section  2.  Information about the Mission of Project Gutenberg-tm

Project Gutenberg-tm is synonymous with the free distribution of
electronic works in formats readable by the widest variety of computers
including obsolete, old, middle-aged and new computers.  It exists
because of the efforts of hundreds of volunteers and donations from
people in all walks of life.

Volunteers and financial support to provide volunteers with the
assistance they need, are critical to reaching Project Gutenberg-tm's
goals and ensuring that the Project Gutenberg-tm collection will
remain freely available for generations to come.  In 2001, the Project
Gutenberg Literary Archive Foundation was created to provide a secure
and permanent future for Project Gutenberg-tm and future generations.
To learn more about the Project Gutenberg Literary Archive Foundation
and how your efforts and donations can help, see Sections 3 and 4
and the Foundation web page at http://www.pglaf.org.


Section 3.  Information about the Project Gutenberg Literary Archive
Foundation

The Project Gutenberg Literary Archive Foundation is a non profit
501(c)(3) educational corporation organized under the laws of the
state of Mississippi and granted tax exempt status by the Internal
Revenue Service.  The Foundation's EIN or federal tax identification
number is 64-6221541.  Its 501(c)(3) letter is posted at
http://pglaf.org/fundraising.  Contributions to the Project Gutenberg
Literary Archive Foundation are tax deductible to the full extent
permitted by U.S. federal laws and your state's laws.

The Foundation's principal office is located at 4557 Melan Dr. S.
Fairbanks, AK, 99712., but its volunteers and employees are scattered
throughout numerous locations.  Its business office is located at
809 North 1500 West, Salt Lake City, UT 84116, (801) 596-1887, email
business@pglaf.org.  Email contact links and up to date contact
information can be found at the Foundation's web site and official
page at http://pglaf.org

For additional contact information:
     Dr. Gregory B. Newby
     Chief Executive and Director
     gbnewby@pglaf.org


Section 4.  Information about Donations to the Project Gutenberg
Literary Archive Foundation

Project Gutenberg-tm depends upon and cannot survive without wide
spread public support and donations to carry out its mission of
increasing the number of public domain and licensed works that can be
freely distributed in machine readable form accessible by the widest
array of equipment including outdated equipment.  Many small donations
($1 to $5,000) are particularly important to maintaining tax exempt
status with the IRS.

The Foundation is committed to complying with the laws regulating
charities and charitable donations in all 50 states of the United
States.  Compliance requirements are not uniform and it takes a
considerable effort, much paperwork and many fees to meet and keep up
with these requirements.  We do not solicit donations in locations
where we have not received written confirmation of compliance.  To
SEND DONATIONS or determine the status of compliance for any
particular state visit http://pglaf.org

While we cannot and do not solicit contributions from states where we
have not met the solicitation requirements, we know of no prohibition
against accepting unsolicited donations from donors in such states who
approach us with offers to donate.

International donations are gratefully accepted, but we cannot make
any statements concerning tax treatment of donations received from
outside the United States.  U.S. laws alone swamp our small staff.

Please check the Project Gutenberg Web pages for current donation
methods and addresses.  Donations are accepted in a number of other
ways including checks, online payments and credit card donations.
To donate, please visit: http://pglaf.org/donate


Section 5.  General Information About Project Gutenberg-tm electronic
works.

Professor Michael S. Hart is the originator of the Project Gutenberg-tm
concept of a library of electronic works that could be freely shared
with anyone.  For thirty years, he produced and distributed Project
Gutenberg-tm eBooks with only a loose network of volunteer support.


Project Gutenberg-tm eBooks are often created from several printed
editions, all of which are confirmed as Public Domain in the U.S.
unless a copyright notice is included.  Thus, we do not necessarily
keep eBooks in compliance with any particular paper edition.


Most people start at our Web site which has the main PG search facility:

     http://www.gutenberg.org

This Web site includes information about Project Gutenberg-tm,
including how to make donations to the Project Gutenberg Literary
Archive Foundation, how to help produce our new eBooks, and how to
subscribe to our email newsletter to hear about new eBooks.
\end{PGtext}

% %%%%%%%%%%%%%%%%%%%%%%%%%%%%%%%%%%%%%%%%%%%%%%%%%%%%%%%%%%%%%%%%%%%%%%% %
%                                                                         %
% End of the Project Gutenberg EBook of Tractatus Logico-Philosophicus, by 
% Ludwig Wittgenstein                                                     %
%                                                                         %
% *** END OF THIS PROJECT GUTENBERG EBOOK TRACTATUS LOGICO-PHILOSOPHICUS ***
%                                                                         %
% ***** This file should be named 5740-t.tex or 5740-t.zip *****          %
% This and all associated files of various formats will be found in:      %
%         http://www.gutenberg.org/5/7/4/5740/                            %
%                                                                         %
% %%%%%%%%%%%%%%%%%%%%%%%%%%%%%%%%%%%%%%%%%%%%%%%%%%%%%%%%%%%%%%%%%%%%%%% %

\end{document}

###

###
This is pdfTeXk, Version 3.141592-1.40.3 (Web2C 7.5.6) (format=pdflatex 2010.5.6)  22 OCT 2010 13:47
entering extended mode
 %&-line parsing enabled.
**5740-t.tex
(./5740-t.tex
LaTeX2e <2005/12/01>
Babel <v3.8h> and hyphenation patterns for english, usenglishmax, dumylang, noh
yphenation, arabic, farsi, croatian, ukrainian, russian, bulgarian, czech, slov
ak, danish, dutch, finnish, basque, french, german, ngerman, ibycus, greek, mon
ogreek, ancientgreek, hungarian, italian, latin, mongolian, norsk, icelandic, i
nterlingua, turkish, coptic, romanian, welsh, serbian, slovenian, estonian, esp
eranto, uppersorbian, indonesian, polish, portuguese, spanish, catalan, galicia
n, swedish, ukenglish, pinyin, loaded.
(/usr/share/texmf-texlive/tex/latex/base/book.cls
Document Class: book 2005/09/16 v1.4f Standard LaTeX document class
(/usr/share/texmf-texlive/tex/latex/base/bk12.clo
File: bk12.clo 2005/09/16 v1.4f Standard LaTeX file (size option)
)
\c@part=\count79
\c@chapter=\count80
\c@section=\count81
\c@subsection=\count82
\c@subsubsection=\count83
\c@paragraph=\count84
\c@subparagraph=\count85
\c@figure=\count86
\c@table=\count87
\abovecaptionskip=\skip41
\belowcaptionskip=\skip42
\bibindent=\dimen102
)

LaTeX Warning: You have requested, on input line 86, version
               `2007/10/19' of document class book,
               but only version
               `2005/09/16 v1.4f Standard LaTeX document class'
               is available.

(/usr/share/texmf-texlive/tex/latex/base/inputenc.sty
Package: inputenc 2006/05/05 v1.1b Input encoding file
\inpenc@prehook=\toks14
\inpenc@posthook=\toks15
(/usr/share/texmf-texlive/tex/latex/base/latin1.def
File: latin1.def 2006/05/05 v1.1b Input encoding file
))

LaTeX Warning: You have requested, on input line 88, version
               `2008/03/30' of package inputenc,
               but only version
               `2006/05/05 v1.1b Input encoding file'
               is available.

(/usr/share/texmf-texlive/tex/latex/base/fontenc.sty
Package: fontenc 2005/09/27 v1.99g Standard LaTeX package
(/usr/share/texmf-texlive/tex/latex/base/t1enc.def
File: t1enc.def 2005/09/27 v1.99g Standard LaTeX file
LaTeX Font Info:    Redeclaring font encoding T1 on input line 43.
)) (/usr/share/texmf-texlive/tex/generic/babel/babel.sty
Package: babel 2005/11/23 v3.8h The Babel package
(/usr/share/texmf-texlive/tex/generic/babel/germanb.ldf
Language: germanb 2004/02/19 v2.6k German support from the babel system
(/usr/share/texmf-texlive/tex/generic/babel/babel.def
File: babel.def 2005/11/23 v3.8h Babel common definitions
\babel@savecnt=\count88
\U@D=\dimen103
)
\l@austrian = a dialect from \language\l@german 
Package babel Info: Making " an active character on input line 91.
) (/usr/share/texmf-texlive/tex/generic/babel/english.ldf
Language: english 2005/03/30 v3.3o English support from the babel system
\l@british = a dialect from \language\l@english 
\l@UKenglish = a dialect from \language\l@english 
\l@canadian = a dialect from \language\l@american 
\l@australian = a dialect from \language\l@british 
\l@newzealand = a dialect from \language\l@british 
))

LaTeX Warning: You have requested, on input line 90, version
               `2008/07/06' of package babel,
               but only version
               `2005/11/23 v3.8h The Babel package'
               is available.

(/usr/share/texmf-texlive/tex/latex/base/ifthen.sty
Package: ifthen 2001/05/26 v1.1c Standard LaTeX ifthen package (DPC)
) (/usr/share/texmf-texlive/tex/latex/amsmath/amsmath.sty
Package: amsmath 2000/07/18 v2.13 AMS math features
\@mathmargin=\skip43
For additional information on amsmath, use the `?' option.
(/usr/share/texmf-texlive/tex/latex/amsmath/amstext.sty
Package: amstext 2000/06/29 v2.01
(/usr/share/texmf-texlive/tex/latex/amsmath/amsgen.sty
File: amsgen.sty 1999/11/30 v2.0
\@emptytoks=\toks16
\ex@=\dimen104
)) (/usr/share/texmf-texlive/tex/latex/amsmath/amsbsy.sty
Package: amsbsy 1999/11/29 v1.2d
\pmbraise@=\dimen105
) (/usr/share/texmf-texlive/tex/latex/amsmath/amsopn.sty
Package: amsopn 1999/12/14 v2.01 operator names
)
\inf@bad=\count89
LaTeX Info: Redefining \frac on input line 211.
\uproot@=\count90
\leftroot@=\count91
LaTeX Info: Redefining \overline on input line 307.
\classnum@=\count92
\DOTSCASE@=\count93
LaTeX Info: Redefining \ldots on input line 379.
LaTeX Info: Redefining \dots on input line 382.
LaTeX Info: Redefining \cdots on input line 467.
\Mathstrutbox@=\box26
\strutbox@=\box27
\big@size=\dimen106
LaTeX Font Info:    Redeclaring font encoding OML on input line 567.
LaTeX Font Info:    Redeclaring font encoding OMS on input line 568.
\macc@depth=\count94
\c@MaxMatrixCols=\count95
\dotsspace@=\muskip10
\c@parentequation=\count96
\dspbrk@lvl=\count97
\tag@help=\toks17
\row@=\count98
\column@=\count99
\maxfields@=\count100
\andhelp@=\toks18
\eqnshift@=\dimen107
\alignsep@=\dimen108
\tagshift@=\dimen109
\tagwidth@=\dimen110
\totwidth@=\dimen111
\lineht@=\dimen112
\@envbody=\toks19
\multlinegap=\skip44
\multlinetaggap=\skip45
\mathdisplay@stack=\toks20
LaTeX Info: Redefining \[ on input line 2666.
LaTeX Info: Redefining \] on input line 2667.
) (/usr/share/texmf-texlive/tex/latex/amsfonts/amssymb.sty
Package: amssymb 2002/01/22 v2.2d
(/usr/share/texmf-texlive/tex/latex/amsfonts/amsfonts.sty
Package: amsfonts 2001/10/25 v2.2f
\symAMSa=\mathgroup4
\symAMSb=\mathgroup5
LaTeX Font Info:    Overwriting math alphabet `\mathfrak' in version `bold'
(Font)                  U/euf/m/n --> U/euf/b/n on input line 132.
))

LaTeX Warning: You have requested, on input line 95, version
               `2009/06/22' of package amssymb,
               but only version
               `2002/01/22 v2.2d'
               is available.

(/usr/share/texmf-texlive/tex/latex/footmisc/footmisc.sty
Package: footmisc 2005/03/17 v5.3d a miscellany of footnote facilities
\FN@temptoken=\toks21
\footnotemargin=\dimen113
\c@pp@next@reset=\count101
\c@@fnserial=\count102
Package footmisc Info: Declaring symbol style bringhurst on input line 817.
Package footmisc Info: Declaring symbol style chicago on input line 818.
Package footmisc Info: Declaring symbol style wiley on input line 819.
Package footmisc Info: Declaring symbol style lamport-robust on input line 823.

Package footmisc Info: Declaring symbol style lamport* on input line 831.
Package footmisc Info: Declaring symbol style lamport*-robust on input line 840
.
)

LaTeX Warning: You have requested, on input line 97, version
               `2009/09/15' of package footmisc,
               but only version
               `2005/03/17 v5.3d a miscellany of footnote facilities'
               is available.

(/usr/share/texmf-texlive/tex/latex/fancyhdr/fancyhdr.sty
\fancy@headwidth=\skip46
\f@ncyO@elh=\skip47
\f@ncyO@erh=\skip48
\f@ncyO@olh=\skip49
\f@ncyO@orh=\skip50
\f@ncyO@elf=\skip51
\f@ncyO@erf=\skip52
\f@ncyO@olf=\skip53
\f@ncyO@orf=\skip54
) (/usr/share/texmf-texlive/tex/latex/graphics/graphicx.sty
Package: graphicx 1999/02/16 v1.0f Enhanced LaTeX Graphics (DPC,SPQR)
(/usr/share/texmf-texlive/tex/latex/graphics/keyval.sty
Package: keyval 1999/03/16 v1.13 key=value parser (DPC)
\KV@toks@=\toks22
) (/usr/share/texmf-texlive/tex/latex/graphics/graphics.sty
Package: graphics 2006/02/20 v1.0o Standard LaTeX Graphics (DPC,SPQR)
(/usr/share/texmf-texlive/tex/latex/graphics/trig.sty
Package: trig 1999/03/16 v1.09 sin cos tan (DPC)
) (/etc/texmf/tex/latex/config/graphics.cfg
File: graphics.cfg 2007/01/18 v1.5 graphics configuration of teTeX/TeXLive
)
Package graphics Info: Driver file: pdftex.def on input line 90.
(/usr/share/texmf-texlive/tex/latex/pdftex-def/pdftex.def
File: pdftex.def 2007/01/08 v0.04d Graphics/color for pdfTeX
\Gread@gobject=\count103
))
\Gin@req@height=\dimen114
\Gin@req@width=\dimen115
) (/usr/share/texmf-texlive/tex/latex/base/alltt.sty
Package: alltt 1997/06/16 v2.0g defines alltt environment
) (./enumitem.sty
Package: enumitem 2009/05/18 v2.2 Customized lists
\enit@toks=\toks23
\labelindent=\skip55
\enit@outerparindent=\dimen116
\enitdp@desc=\count104
) (/usr/share/texmf-texlive/tex/latex/soul/soul.sty
Package: soul 2003/11/17 v2.4 letterspacing/underlining (mf)
\SOUL@word=\toks24
\SOUL@lasttoken=\toks25
\SOUL@cmds=\toks26
\SOUL@buffer=\toks27
\SOUL@token=\toks28
\SOUL@spaceskip=\skip56
\SOUL@ttwidth=\dimen117
\SOUL@uldp=\dimen118
\SOUL@ulht=\dimen119
) (/usr/share/texmf-texlive/tex/latex/geometry/geometry.sty
Package: geometry 2002/07/08 v3.2 Page Geometry
\Gm@cnth=\count105
\Gm@cntv=\count106
\c@Gm@tempcnt=\count107
\Gm@bindingoffset=\dimen120
\Gm@wd@mp=\dimen121
\Gm@odd@mp=\dimen122
\Gm@even@mp=\dimen123
\Gm@dimlist=\toks29
(/usr/share/texmf-texlive/tex/xelatex/xetexconfig/geometry.cfg)) (/usr/share/te
xmf-texlive/tex/latex/hyperref/hyperref.sty
Package: hyperref 2007/02/07 v6.75r Hypertext links for LaTeX
\@linkdim=\dimen124
\Hy@linkcounter=\count108
\Hy@pagecounter=\count109
(/usr/share/texmf-texlive/tex/latex/hyperref/pd1enc.def
File: pd1enc.def 2007/02/07 v6.75r Hyperref: PDFDocEncoding definition (HO)
) (/etc/texmf/tex/latex/config/hyperref.cfg
File: hyperref.cfg 2002/06/06 v1.2 hyperref configuration of TeXLive
) (/usr/share/texmf-texlive/tex/latex/oberdiek/kvoptions.sty
Package: kvoptions 2006/08/22 v2.4 Connects package keyval with LaTeX options (
HO)
)
Package hyperref Info: Option `hyperfootnotes' set `false' on input line 2238.
Package hyperref Info: Option `bookmarks' set `true' on input line 2238.
Package hyperref Info: Option `linktocpage' set `true' on input line 2238.
Package hyperref Info: Option `pdfdisplaydoctitle' set `true' on input line 223
8.
Package hyperref Info: Option `pdfpagelabels' set `true' on input line 2238.
Package hyperref Info: Option `bookmarksopen' set `true' on input line 2238.
Package hyperref Info: Option `colorlinks' set `true' on input line 2238.
Package hyperref Info: Hyper figures OFF on input line 2288.
Package hyperref Info: Link nesting OFF on input line 2293.
Package hyperref Info: Hyper index ON on input line 2296.
Package hyperref Info: Plain pages OFF on input line 2303.
Package hyperref Info: Backreferencing OFF on input line 2308.
Implicit mode ON; LaTeX internals redefined
Package hyperref Info: Bookmarks ON on input line 2444.
(/usr/share/texmf-texlive/tex/latex/ltxmisc/url.sty
\Urlmuskip=\muskip11
Package: url 2005/06/27  ver 3.2  Verb mode for urls, etc.
)
LaTeX Info: Redefining \url on input line 2599.
\Fld@menulength=\count110
\Field@Width=\dimen125
\Fld@charsize=\dimen126
\Choice@toks=\toks30
\Field@toks=\toks31
Package hyperref Info: Hyper figures OFF on input line 3102.
Package hyperref Info: Link nesting OFF on input line 3107.
Package hyperref Info: Hyper index ON on input line 3110.
Package hyperref Info: backreferencing OFF on input line 3117.
Package hyperref Info: Link coloring ON on input line 3120.
\Hy@abspage=\count111
\c@Item=\count112
)
*hyperref using driver hpdftex*
(/usr/share/texmf-texlive/tex/latex/hyperref/hpdftex.def
File: hpdftex.def 2007/02/07 v6.75r Hyperref driver for pdfTeX
\Fld@listcount=\count113
)

LaTeX Warning: You have requested, on input line 143, version
               `2009/10/09' of package hyperref,
               but only version
               `2007/02/07 v6.75r Hypertext links for LaTeX'
               is available.

\DittoLen=\skip57
\WLen=\skip58
(./5740-t.aux)
\openout1 = `5740-t.aux'.

LaTeX Font Info:    Checking defaults for OML/cmm/m/it on input line 350.
LaTeX Font Info:    ... okay on input line 350.
LaTeX Font Info:    Checking defaults for T1/cmr/m/n on input line 350.
LaTeX Font Info:    ... okay on input line 350.
LaTeX Font Info:    Checking defaults for OT1/cmr/m/n on input line 350.
LaTeX Font Info:    ... okay on input line 350.
LaTeX Font Info:    Checking defaults for OMS/cmsy/m/n on input line 350.
LaTeX Font Info:    ... okay on input line 350.
LaTeX Font Info:    Checking defaults for OMX/cmex/m/n on input line 350.
LaTeX Font Info:    ... okay on input line 350.
LaTeX Font Info:    Checking defaults for U/cmr/m/n on input line 350.
LaTeX Font Info:    ... okay on input line 350.
LaTeX Font Info:    Checking defaults for PD1/pdf/m/n on input line 350.
LaTeX Font Info:    ... okay on input line 350.
(/usr/share/texmf/tex/context/base/supp-pdf.tex
[Loading MPS to PDF converter (version 2006.09.02).]
\scratchcounter=\count114
\scratchdimen=\dimen127
\scratchbox=\box28
\nofMPsegments=\count115
\nofMParguments=\count116
\everyMPshowfont=\toks32
\MPscratchCnt=\count117
\MPscratchDim=\dimen128
\MPnumerator=\count118
\everyMPtoPDFconversion=\toks33
)
-------------------- Geometry parameters
paper: class default
landscape: --
twocolumn: --
twoside: true
asymmetric: true
h-parts: 99.58466pt, 369.88582pt, 99.58466pt
v-parts: 110.39664pt, 569.05511pt, 165.595pt
hmarginratio: 1:1
vmarginratio: 2:3
lines: --
heightrounded: --
bindingoffset: 28.45274pt
truedimen: --
includehead: --
includefoot: --
includemp: --
driver: pdftex
-------------------- Page layout dimensions and switches
\paperwidth  597.50787pt
\paperheight 845.04675pt
\textwidth  369.88582pt
\textheight 569.05511pt
\oddsidemargin  55.76741pt
\evensidemargin 27.31467pt
\topmargin  6.25285pt
\headheight 15.0pt
\headsep    19.8738pt
\footskip   30.0pt
\marginparwidth 47.0pt
\marginparsep   7.0pt
\columnsep  10.0pt
\skip\footins  10.8pt plus 4.0pt minus 2.0pt
\hoffset 0.0pt
\voffset 0.0pt
\mag 1000
\@twosidetrue 
(1in=72.27pt, 1cm=28.45pt)
-----------------------
(/usr/share/texmf-texlive/tex/latex/graphics/color.sty
Package: color 2005/11/14 v1.0j Standard LaTeX Color (DPC)
(/etc/texmf/tex/latex/config/color.cfg
File: color.cfg 2007/01/18 v1.5 color configuration of teTeX/TeXLive
)
Package color Info: Driver file: pdftex.def on input line 130.
)
Package hyperref Info: Link coloring ON on input line 350.
(/usr/share/texmf-texlive/tex/latex/hyperref/nameref.sty
Package: nameref 2006/12/27 v2.28 Cross-referencing by name of section
(/usr/share/texmf-texlive/tex/latex/oberdiek/refcount.sty
Package: refcount 2006/02/20 v3.0 Data extraction from references (HO)
)
\c@section@level=\count119
)
LaTeX Info: Redefining \ref on input line 350.
LaTeX Info: Redefining \pageref on input line 350.
(./5740-t.out) (./5740-t.out)
\@outlinefile=\write3
\openout3 = `5740-t.out'.

\c@propositionsi=\count120
\enitdp@propositions=\count121
LaTeX Font Info:    Try loading font information for T1+cmtt on input line 369.

(/usr/share/texmf-texlive/tex/latex/base/t1cmtt.fd
File: t1cmtt.fd 1999/05/25 v2.5h Standard LaTeX font definitions
)
LaTeX Font Info:    Try loading font information for U+msa on input line 394.
(/usr/share/texmf-texlive/tex/latex/amsfonts/umsa.fd
File: umsa.fd 2002/01/19 v2.2g AMS font definitions
)
LaTeX Font Info:    Try loading font information for U+msb on input line 394.
(/usr/share/texmf-texlive/tex/latex/amsfonts/umsb.fd
File: umsb.fd 2002/01/19 v2.2g AMS font definitions
) [1

{/var/lib/texmf/fonts/map/pdftex/updmap/pdftex.map}] [2

] [1

] [2

] [3] [4] [5] [6

] [7] [8] [9] [10] [11] [12] [13] [14] [15] [16] [17] [18] [19] [20

] [21] [22

] [23] [24] [25

] [26] [27] [28] [29] [30] [31] [32] [33] [34] [35] [36] [37] [38] [39] [40] [4
1] [42] [43] [44] [45] [46] [47] [48] [49] [50] [51] [52] [53] [54] [55] [56] [
57] [58] [59] [60] [61] [62] [63] [64] [65] [66] [67] [68] [69] [70] [71] <imag
es/cube.pdf, id=1587, 469.755pt x 506.89375pt>
File: images/cube.pdf Graphic file (type pdf)
<use images/cube.pdf> [72 <./images/cube.pdf>] [73] [74] <images/sight-en.pdf, 
id=1673, 469.755pt x 219.82124pt>
File: images/sight-en.pdf Graphic file (type pdf)
<use images/sight-en.pdf> [75 <./images/sight-en.pdf>] [76] [77] <images/bracke
ts01-en.pdf, id=1756, 469.755pt x 178.6675pt>
File: images/brackets01-en.pdf Graphic file (type pdf)
<use images/brackets01-en.pdf> <images/brackets02-en.pdf, id=1758, 469.755pt x 
230.8625pt>
File: images/brackets02-en.pdf Graphic file (type pdf)
<use images/brackets02-en.pdf> <images/brackets03-en.pdf, id=1760, 469.755pt x 
626.34pt>
File: images/brackets03-en.pdf Graphic file (type pdf)
<use images/brackets03-en.pdf> <images/brackets04-en.pdf, id=1762, 469.755pt x 
241.90375pt>
File: images/brackets04-en.pdf Graphic file (type pdf)
<use images/brackets04-en.pdf> <images/brackets05-en.pdf, id=1764, 470.75874pt 
x 523.9575pt>
File: images/brackets05-en.pdf Graphic file (type pdf)
<use images/brackets05-en.pdf> [78 <./images/brackets01-en.pdf> <./images/brack
ets02-en.pdf> <./images/brackets03-en.pdf> <./images/brackets04-en.pdf>] [79 <.
/images/brackets05-en.pdf>] [80] [81] [82] [83] [84] [85] <images/space.pdf, id
=1980, 469.755pt x 56.21pt>
File: images/space.pdf Graphic file (type pdf)
<use images/space.pdf> [86 <./images/space.pdf>] [87] [88] [89] [90] [91

] [92] [93

] [94] [95] [96] [97] [98] [99] [100] [101] [102] [103] [104] [105] [106] [107]
[108] [109] [110] [111] [112] [113] [114] [115] [116] [117] [118] [119] [120] [
121] [122] [123] [124] [125] [126] [127] [128] [129] [130] [131] [132] [133] [1
34] [135] [136] [137] [138] [139] [140] [141]
File: images/cube.pdf Graphic file (type pdf)
<use images/cube.pdf> [142] [143] [144] <images/sight-de.pdf, id=2760, 469.755p
t x 210.7875pt>
File: images/sight-de.pdf Graphic file (type pdf)
<use images/sight-de.pdf> [145 <./images/sight-de.pdf>] [146] [147] <images/bra
ckets01-de.pdf, id=2799, 469.755pt x 167.62625pt>
File: images/brackets01-de.pdf Graphic file (type pdf)
<use images/brackets01-de.pdf> [148] <images/brackets02-de.pdf, id=2805, 469.75
5pt x 223.83624pt>
File: images/brackets02-de.pdf Graphic file (type pdf)
<use images/brackets02-de.pdf> <images/brackets03-de.pdf, id=2807, 463.7325pt x
 650.43pt>
File: images/brackets03-de.pdf Graphic file (type pdf)
<use images/brackets03-de.pdf> <images/brackets04-de.pdf, id=2809, 469.755pt x 
234.8775pt>
File: images/brackets04-de.pdf Graphic file (type pdf)
<use images/brackets04-de.pdf> <images/brackets05-de.pdf, id=2811, 469.755pt x 
529.98pt>
File: images/brackets05-de.pdf Graphic file (type pdf)
<use images/brackets05-de.pdf> [149 <./images/brackets01-de.pdf> <./images/brac
kets02-de.pdf> <./images/brackets03-de.pdf> <./images/brackets04-de.pdf> <./ima
ges/brackets05-de.pdf>] [150] [151] [152] [153] [154] [155] [156]
File: images/space.pdf Graphic file (type pdf)
<use images/space.pdf> [157] [158] [159] [160] [161] [162] [163] [164

] [1] [2] [3] [4] [5] [6] [7] (./5740-t.aux)

 *File List*
    book.cls    2005/09/16 v1.4f Standard LaTeX document class
    bk12.clo    2005/09/16 v1.4f Standard LaTeX file (size option)
inputenc.sty    2006/05/05 v1.1b Input encoding file
  latin1.def    2006/05/05 v1.1b Input encoding file
 fontenc.sty
   t1enc.def    2005/09/27 v1.99g Standard LaTeX file
   babel.sty    2005/11/23 v3.8h The Babel package
 germanb.ldf    2004/02/19 v2.6k German support from the babel system
 english.ldf    2005/03/30 v3.3o English support from the babel system
  ifthen.sty    2001/05/26 v1.1c Standard LaTeX ifthen package (DPC)
 amsmath.sty    2000/07/18 v2.13 AMS math features
 amstext.sty    2000/06/29 v2.01
  amsgen.sty    1999/11/30 v2.0
  amsbsy.sty    1999/11/29 v1.2d
  amsopn.sty    1999/12/14 v2.01 operator names
 amssymb.sty    2002/01/22 v2.2d
amsfonts.sty    2001/10/25 v2.2f
footmisc.sty    2005/03/17 v5.3d a miscellany of footnote facilities
fancyhdr.sty    
graphicx.sty    1999/02/16 v1.0f Enhanced LaTeX Graphics (DPC,SPQR)
  keyval.sty    1999/03/16 v1.13 key=value parser (DPC)
graphics.sty    2006/02/20 v1.0o Standard LaTeX Graphics (DPC,SPQR)
    trig.sty    1999/03/16 v1.09 sin cos tan (DPC)
graphics.cfg    2007/01/18 v1.5 graphics configuration of teTeX/TeXLive
  pdftex.def    2007/01/08 v0.04d Graphics/color for pdfTeX
   alltt.sty    1997/06/16 v2.0g defines alltt environment
enumitem.sty    2009/05/18 v2.2 Customized lists
    soul.sty    2003/11/17 v2.4 letterspacing/underlining (mf)
geometry.sty    2002/07/08 v3.2 Page Geometry
geometry.cfg
hyperref.sty    2007/02/07 v6.75r Hypertext links for LaTeX
  pd1enc.def    2007/02/07 v6.75r Hyperref: PDFDocEncoding definition (HO)
hyperref.cfg    2002/06/06 v1.2 hyperref configuration of TeXLive
kvoptions.sty    2006/08/22 v2.4 Connects package keyval with LaTeX options (HO
)
     url.sty    2005/06/27  ver 3.2  Verb mode for urls, etc.
 hpdftex.def    2007/02/07 v6.75r Hyperref driver for pdfTeX
supp-pdf.tex
   color.sty    2005/11/14 v1.0j Standard LaTeX Color (DPC)
   color.cfg    2007/01/18 v1.5 color configuration of teTeX/TeXLive
 nameref.sty    2006/12/27 v2.28 Cross-referencing by name of section
refcount.sty    2006/02/20 v3.0 Data extraction from references (HO)
  5740-t.out
  5740-t.out
  t1cmtt.fd    1999/05/25 v2.5h Standard LaTeX font definitions
    umsa.fd    2002/01/19 v2.2g AMS font definitions
    umsb.fd    2002/01/19 v2.2g AMS font definitions
images/cube.pdf
images/sight-en.pdf
images/brackets01-en.pdf
images/brackets02-en.pdf
images/brackets03-en.pdf
images/brackets04-en.pdf
images/brackets05-en.pdf
images/space.pdf
images/cube.pdf
images/sight-de.pdf
images/brackets01-de.pdf
images/brackets02-de.pdf
images/brackets03-de.pdf
images/brackets04-de.pdf
images/brackets05-de.pdf
images/space.pdf
 ***********

 ) 
Here is how much of TeX's memory you used:
 7747 strings out of 94074
 99898 string characters out of 1165154
 167674 words of memory out of 1500000
 9649 multiletter control sequences out of 10000+50000
 26295 words of font info for 68 fonts, out of 1200000 for 2000
 666 hyphenation exceptions out of 8191
 29i,14n,43p,293b,586s stack positions out of 5000i,500n,6000p,200000b,5000s
{/usr/share/texmf/fonts/enc/dvips/cm-super/cm-super-t1.enc}</usr/share/texmf-
texlive/fonts/type1/bluesky/cm/cmex10.pfb></usr/share/texmf-texlive/fonts/type1
/bluesky/cm/cmmi10.pfb></usr/share/texmf-texlive/fonts/type1/bluesky/cm/cmmi12.
pfb></usr/share/texmf-texlive/fonts/type1/bluesky/cm/cmmi6.pfb></usr/share/texm
f-texlive/fonts/type1/bluesky/cm/cmmi8.pfb></usr/share/texmf-texlive/fonts/type
1/bluesky/cm/cmr10.pfb></usr/share/texmf-texlive/fonts/type1/bluesky/cm/cmr12.p
fb></usr/share/texmf-texlive/fonts/type1/bluesky/cm/cmr8.pfb></usr/share/texmf-
texlive/fonts/type1/bluesky/cm/cmsy10.pfb></usr/share/texmf-texlive/fonts/type1
/bluesky/cm/cmsy7.pfb></usr/share/texmf-texlive/fonts/type1/bluesky/cm/cmsy8.pf
b></usr/share/texmf-texlive/fonts/type1/bluesky/ams/msbm10.pfb></usr/share/texm
f/fonts/type1/public/cm-super/sfbx1200.pfb></usr/share/texmf/fonts/type1/public
/cm-super/sfbx2074.pfb></usr/share/texmf/fonts/type1/public/cm-super/sfbx2488.p
fb></usr/share/texmf/fonts/type1/public/cm-super/sfcc1200.pfb></usr/share/texmf
/fonts/type1/public/cm-super/sfrm0800.pfb></usr/share/texmf/fonts/type1/public/
cm-super/sfrm1000.pfb></usr/share/texmf/fonts/type1/public/cm-super/sfrm1095.pf
b></usr/share/texmf/fonts/type1/public/cm-super/sfrm1200.pfb></usr/share/texmf/
fonts/type1/public/cm-super/sfrm1440.pfb></usr/share/texmf/fonts/type1/public/c
m-super/sfrm1728.pfb></usr/share/texmf/fonts/type1/public/cm-super/sfrm2074.pfb
></usr/share/texmf/fonts/type1/public/cm-super/sfti1000.pfb></usr/share/texmf/f
onts/type1/public/cm-super/sfti1095.pfb></usr/share/texmf/fonts/type1/public/cm
-super/sfti1200.pfb></usr/share/texmf/fonts/type1/public/cm-super/sftt0900.pfb>
Output written on 5740-t.pdf (173 pages, 1315243 bytes).
PDF statistics:
 3362 PDF objects out of 3580 (max. 8388607)
 1248 named destinations out of 1440 (max. 131072)
 271 words of extra memory for PDF output out of 10000 (max. 10000000)

